
\section{Improved Work Stealing Implementation}
\label{sec:rts_work_stealing2}

\status{The discussion of the benchmarks is new,
other parts have had a round of reviewing already.}

In this section we take the opportunity to improve on our work stealing
implementation from section \ref{sec:rts_work_stealing}.
While we made these improvements,
we also found it useful to change how idle engines behave.
Although these too changes are conceptually distinct,
they were made together and their implementations are interlinked.
Therefore we will present and benchmark them together as one set of changes.

\plan{Describe problems with associating stacks with contexts}
Sparks were originally stored on context local deques.
This has the following significant problems.

\begin{description}

\item[There is a dynamic number of spark deques.]

The number of contexts changes during a program's execution,
therefore the number of spark deques also changes.
This means that we must have code to manage this changing number of
deques.
This code makes the runtime system more complicated than necessary,
both when stealing a spark and when creating or destroying a context.

\item[Locking is required.]

The management of contexts must be thread safe so that the set of
spark deques is not corrupted.
We store spark deques in a global array protected by a lock.
It may be possible to replace the array with a lock free data structure.
However it is better to remove the need for thread safety by using a
constant set of deques.

\item[A large number of contexts makes work stealing slower.]

In prior sections
we have shown that the number of contexts in use can often be very high,
much higher than the number of Mercury engines.
If there are $N$ engines and $M$ contexts,
then there can be at most $N$ contexts running and
at least $M-N$ contexts suspended (blocked and waiting to run).
A context can become suspended in one of two ways:
by blocking on a future's value in a call to \wait,
or by blocking on an incomplete conjunct in a call to \joinandcontinue.
We attempt to minimise the former case (see chapter \ref{chap:overlap})
and the latter case cannot occur
if the context has a spark on its local spark deque (it would run the
spark rather than block).
Therefore the large majority of the suspended contexts will not have
sparks on their deques,
and the probability of selecting a deque at random with a spark on its
stack is low.
Furthermore the value of $M$ can be very high in pathological cases.
If an engine does not successfully steal a spark from a deque,
it will continue by trying to steal a spark from a different deque.
An engine can exhaust all its attempts, even when there is work
available:
a spark may be placed on a deque after the engine has already attempted
to steal from it.
Upon exhausting all its attempts or all the deques,
the engine will sleep before making another round of attempts
(see \trystealspark in algorithm \ref{alg:try_steal_spark_initial} on
page \pageref{alg:try_steal_spark_initial}).
Each round of attempts (regardless of success or failure) has a
complexity of $O(M)$.

\end{description}

\plan{We associate stacks with engines}
The approach we have taken to solving these problems is based on associating
spark deques with engines rather than with contexts.
A running Mercury program has a constant number of engines,
and therefore the number of spark deques will not vary.
This allows us to remove the code used to resize the deque array.
This makes context creation and destruction much simpler.
It also removes the need for the lock protecting the global array of spark
deques.
We can also remove the locking code in \trystealspark
(whose original version was shown in algorithm
\ref{alg:try_steal_spark_initial})
which was used to
ensure that the array is not changed while a thief is trying to steal a
spark.
The cost of work stealing also becomes linear in the number of engines
rather than the number of contexts.

\begin{algorithm}[tbp]
\begin{algorithmic}[1]
\Procedure{MR\_try\_steal\_spark}{}
  \If{$\left(
          \begin{array}{l}
          (current\_context \neq \text{NULL})\, \vee \\
          (MR\_num\_outstanding\_contexts < MR\_max\_contexts) \\
          \end{array}
          \right)$}
    \For{$attempt = 0$ to $MR\_num\_engines - 1$}
      \State $victim\_index \gets
          (MR\_engine.victim\_counter + attempt) \bmod MR\_num\_engines$
      \If{MR\_steal\_spark($MR\_spark\_deques$[$victim\_index$],
          \text{\code{\&}}$spark$)}
        \State $MR\_engine.victim\_counter \gets victim\_index$
        \State MR\_prepare\_engine\_for\_spark($spark$, NULL)
        \State \Return $spark.resume$
      \EndIf
    \EndFor
  \EndIf
  \State \Return NULL
\EndProcedure
\end{algorithmic}
\caption{MR\_try\_steal\_spark}
\label{alg:try_steal_spark_revised}
\end{algorithm}

\begin{algorithm}[tbp]
\begin{algorithmic}
\Procedure{MR\_prepare\_engine\_for\_spark}{$spark$, $join\_label$}
    \If{$current\_context = \text{NULL}$}
        \State $ctxt \gets$ MR\_get\_free\_context()
        \If{$ctxt = \text{NULL}$}
            \State $ctxt \gets$ MR\_create\_context()
        \EndIf
        \State MR\_load\_context($ctxt$)
    \EndIf
    \State $MR\_parent\_sp \gets spark.parent\_sp$
    \State $ctxt.thread\_local\_mutables \gets
      spark.thread\_local\_mutables$
    \State \Return $spark.resume$
\EndProcedure
\end{algorithmic}
\caption{MR\_prepare\_engine\_for\_spark()}
\label{alg:prepare_engine_for_spark}
\end{algorithm}

\plan{Show new \trystealspark}
We have made some other changes to \trystealspark,
whose new code is shown in algorithm \ref{alg:try_steal_spark_revised}.
Previously \trystealspark required its caller to check that stealing a spark
would not exceed the context limit;
this check has been moved into \trystealspark (line 2).
Another change is that \trystealspark
will now call \prepareengineforspark (line 7)
which will load a context into the engine if it does not already have one
(algorithm \ref{alg:prepare_engine_for_spark}).
Finally \trystealspark will return the address of the spark's code;
the caller will jump to this address.
If \trystealspark could not find a spark it will return NULL,
in this case its caller will look for other work.

In the previous version,
the lock was also used to protect the victim counter;
it ensured that round-robin selection of the victim was maintained.
Now each engine independently performs its own round-robin selection of the
victim using a new field \code{victim\_counter} in the engine structure.
If a spark is found, \trystealspark will save the current victim index to
this field, as this deque may contain more sparks which it can try to
steal later.
We have not evaluated if this policy is an improvement.
However when compared with the improvement due to removing the lock and
potentially large number of spark deques,
any difference in performance due to the selection of victims will be
negligible.
The two other changes are minor:
we have removed the configurable limit of the number of work stealing
attempts per round,
and the test for null array slots;
both are now unnecessary.

\begin{algorithm}[tbp]
\begin{algorithmic}
    \Procedure{MR\_idle}{}
        \State $code\_ptr \gets$ MR\_try\_run\_context()
        \If{$code\_ptr \neq$ NULL}
            \Goto $code\_ptr$
        \EndIf
        \State $code\_ptr \gets$ MR\_try\_run\_local\_spark($NULL$)
        \If{$code\_ptr \neq$ NULL}
            \Goto $code\_ptr$
        \EndIf
        \State $code\_ptr \gets$ MR\_try\_steal\_spark()
        \If{$code\_ptr \neq$ NULL}
            \Goto $code\_ptr$
        \EndIf
        \Goto MR\_sleep
    \EndProcedure
\end{algorithmic}
\caption{New \idle code}
\label{alg:idle_entry_point}
\end{algorithm}

\begin{algorithm}[tbp]
\begin{algorithmic}
\Procedure{MR\_try\_run\_context}{}
    \State MR\_acquire\_lock($MR\_runqueue\_lock$)
    \State $ctxt \gets$ MR\_get\_runnable\_context()
    \State MR\_release\_lock($MR\_runqueue\_lock$)
    \If{$ctxt \neq$ NULL}
        \If{$current\_context \neq$ NULL}
            \State MR\_release\_context($current\_context$)
        \EndIf
        \State MR\_load\_context($ctxt$)
        \State $resume \gets ctxt.resume$
        \State $ctxt.resume \gets$ NULL
        \State \Return $resume$
    \Else
        \State \Return NULL
    \EndIf
\EndProcedure
\end{algorithmic}
\caption{MR\_try\_run\_context()}
\label{alg:try_run_context}
\end{algorithm}

\begin{algorithm}[tbp]
\begin{algorithmic}
\Procedure{MR\_try\_run\_local\_spark}{$join\_label$}
    \State $result \gets$ MR\_pop\_spark($this\_engine.spark\_deque$)
    \If{$result =$ NULL}
        \State \Return NULL
    \EndIf
    \If{$\left(
        \begin{array}{l}
            (current\_context \neq \text{NULL}) \wedge \\
            (join\_label \neq \text{NULL}) \wedge \\
            (spark.sync\_term.orig\_context \neq current\_context)
        \end{array}\right)$}
        \State MR\_push\_spark($this\_engine.spark\_deque$, $spark$)
        \State \Return NULL
    \EndIf
    \State MR\_prepare\_engine\_for\_spark($spark$, $join\_label$)
    \State \Return $spark.resume$
\EndProcedure
\end{algorithmic}
\caption{MR\_try\_run\_local\_spark()}
\label{alg:try_run_local_spark}
\end{algorithm}

We have also made changes to the code that \idle engines execute.
An idle engine with a context that is free (it can be used for any spark) or
without a context will call \idle to acquire new work.
\idle has changed significantly;
the new version is shown in algorithm \ref{alg:idle_entry_point}.
Three new C functions are now used to implement the details of \idle, these
are:
\tryruncontext, \tryrunlocalspark and \trystealspark.
They are separate C functions so that we can use them from procedures other
than \idle, such as \joinandcontinue below.
\idle always tries to find a runnable context before a local spark.
A context that has already been created usually represents work that is
\emph{to the left of}
any work that is still a spark (section \ref{sec:rts_work_stealing}).
Therefore, we attempt to resume the execution of a context first
as a context whose execution has already begun
may make more parallel work available by creating sparks or signalling
futures.
Also if the context does not block, then when it finishes its work,
it can be used to execute a spark from the engine's local spark queue.
Checking for runnable contexts first also helps to avoid the creation of new
contexts before existing contexts' computations are finished.
This partially avoids deadlocks that can occur when the context limit is
exceeded:
executing a waiting context rather than attempting to create a new context
for a spark does not increase the number of contexts that exist.

\begin{figure}
\begin{center}
\begin{verbatim}
nested_par(...) ;-
    (
        p(..., X),
    &
        (
            q(X, ...)
        & 
            r(..., Y)
        ),
        s(...)
    &
        t(Y, ...)
    ).
\end{verbatim}
\end{center}
\caption{Nested dependent parallelism.}
\label{fig:nested_dep_par}
\end{figure}

If \idle cannot find a context to resume then it
will call \tryrunlocalspark.
If there is a spark to execute and the engine does not have a context,
a new one needs to be created.
We considered only creating a context and executing the spark only if the
context limit had not been reached,
however this can create a problem, which we explain using an example.
Consider figure \ref{fig:nested_dep_par} which shows two nested
parallel conjunctions.
There are also two futures that create dependencies between the parallel
computations:
\code{X} creates a dependency between \code{p} and \code{q},
and \code{Y} creates a dependency between \code{r} and \code{t}.
When an engine, called E1,
executes this procedure it pushes the spark for the second and third
conjuncts of the outer conjunction onto its local spark deque.
E1 then begins the execution of \code{p}.
A second engine, called E2,
steals the spark from E1's spark deque and immediately creates two new
sparks, the first is for \code{t} and the second for \code{r}.
It pushes the sparks onto its deque,
therefore the spark for \code{t} is on the bottom (cold end).
E2 then begins the execution of \code{q}.
A third engine, E3,
wakes up and steals the spark for \code{t} spark from E2.
Next,
E2 attempts to read the value of \code{X} in \code{q} but it has not yet
been produced,
therefore E2 suspends its context and calls \idle, it cannot find a context
to execute and calls \tryrunlocalspark,
which finds the spark for \code{r}.
E2 needs a new context to execute \code{r}, if the context limit is exceeded
it E2 cannot do any work.
And if E2 does not execute \code{r} then E3 cannot complete the execution of
\code{t} and free up its context.
The system will not deadlock, as the execution of \code{p} by E1 can still
continue, and will eventually allow the execution of \code{q} to continue.
However, we still wish to avoid such situations,
therefore \tryrunlocalspark does not check the context limit.
By doing this E2 is able to create a context and execute \code{r},
this can never create more than a handful of extra contexts:
while the limit is exceeded engines will not steal work and will therefore
execute sparks in a left-to-right order only where dependencies and barriers
cannot cause more contexts to become blocked.
The two choices we have made,
executing local sparks without checking the context limit and
attempting to resume contexts before attempting to execute sparks,
work together to prevent deadlocks that could be caused by
the context limit and dependencies.
Even though the context limit is a work-around to prevent some kinds of
worst case behaviour,
it is still important to use scheduling policies such as these that keep the
number of contexts low:
contexts consume significant amounts of memory and
the garbage collector spends time scanning that memory.
By reducing the number of contexts we can improve memory and time usage.

\idle is C code that uses the Mercury calling convention,
this convention passes all its arguments and results in the Mercury abstract
machine registers, including the return address.
Therefore, it does not need to create a stack frame on any of the C or
Mercury stacks;
any local C variables are stored on the stack frame shared between all
Mercury procedures (and must be saved by the caller).
Unlike other Mercury procedures, \idle never returns control to its caller.
Instead it continues execution by jumping to the code address of the next
thing to execute,
such as the resume address of a context,
or the entry point of a spark.
If one of its callees such as \trystealspark where to make this jump rather
than returning and then letting \idle jump,
then this would leak the stack frames created for \trystealspark.
Therefore, each of the functions \tryrunlocalspark, \tryruncontext and
\trystealspark must return the address of the next thing to execute and
allow \idle to jump to this code or jump to \sleep.

\tryruncontext is shown in algorithm \ref{alg:try_run_context}.
It attempts to take a context from the context runqueue while holding the
context runqueue lock.
If \tryruncontext finds a context,
it first checks for and unloads the engine's current context,
and then loads the new context.
It clears the context's resume field and then returns the field's previous
value, which \idle will jump to.
The context's resume field must be cleared here as it may later be
synchronised on by \joinandcontinue.
Previously the runqueue's lock protected all of \idle,
this critical section has been made smaller which might be more efficient.

\tryrunlocalspark is shown in algorithm \ref{alg:try_run_local_spark};
It attempts to pop a spark of the engine's deque.
If the spark deque is empty then \tryrunlocalspark
will return NULL.
If it was not empty then
we prepare the engine to execute the popped spark using
\prepareengineforspark
(algorithm \ref{alg:prepare_engine_for_spark}).
All this entails is making sure that the engine has a context,
the context is loaded and
the engine has the correct value in its parent stack pointer register.
After this \tryrunlocalspark returns the spark's entry point to \idle which
will jump to it.

Before executing a spark,
\tryrunlocalspark
will check if the engine's current context is incompatible with the spark,
if so it puts the spark back on the deque and returns NULL.
However when \idle calls \tryrunlocalspark the spark will always be
compatible.
\tryrunlocalspark can also be called by \joinandcontinue,
in this case the engine's context may be incompatible with the spark.
We will discuss this later including the test for incompatibility and why we
place the spark back onto the deque.

\paul{Move later}
A context is incompatible if it contains stack frames of a computation that
is not a parent of the computation the spark represents.
We test this using three conditions:
first, is that there is a current context, a non-existent context cannot
be incompatible;
second, resume\_label is non-NULL if a current context represents an
existing computation;
third, if the current context does have a computation, then that computation
may be incompatible if the spark's conjunction's original context is not the
current context.

\begin{algorithm}[tbp]
\begin{minipage}{\textwidth}
\begin{verbatim}
struct MR_engine_sleep_sync {
    sem_t                               sleep_sem;
    lock                                lock;
    volatile unsigned                   state;
    volatile unsigned                   action;
    union MR_engine_wake_action_data    action_data;
};

union MR_engine_wake_action_data {
    MR_EngineId     worksteal_engine;
    MR_Context      *context;
};
\end{verbatim}

\begin{algorithmic}
    \Procedure{MR\_sleep}{}
        \Loop
            \State $eng\_data \gets MR\_engine\_sleep\_data$[$engine\_id$]
            \State $eng\_data.state \gets$ SLEEPING
            \State $sem\_wait(eng\_data.sleep_sem)$
            \Switch{$eng\_data.action$}
              \Case{ACTION\_SHUTDOWN}
                \State $\cdots$
              \EndCase
              \Case{ACTION\_RUN\_CONTEXT}
                \State $ctxt \gets eng\_data.action\_data.context$
                \State MR\_load\_context($ctxt$)
                \Goto $ctxt.resume\_label$
              \EndCase
              \Case{ACTION\_STEAL\_SPARK}
                \State $MR\_engine.victim\_counter \gets
                    eng\_data.action\_data.worksteal\_engine$
                \State $code\_ptr \gets$ MR\_try\_steal\_spark()
                \If{$code\_ptr \neq$ NULL}
                    \Goto $code\_ptr$
                \EndIf
                \State $code\_ptr \gets$ MR\_try\_run\_context()
                \If{$code\_ptr \neq$ NULL}
                    \Goto $code\_ptr$
                \EndIf
                \State \Break
              \EndCase
              \Case{ACTION\_NOT\_SPECIFIED}
                \State $code\_ptr \gets$ MR\_try\_run\_context()
                \If{$code\_ptr \neq$ NULL}
                    \Goto $code\_ptr$
                \EndIf
                \State $code\_ptr \gets$ MR\_try\_steal\_spark()
                \If{$code\_ptr \neq$ NULL}
                    \Goto $code\_ptr$
                \EndIf
                \State \Break
              \EndCase
            \EndSwitch
        \EndLoop
    \EndProcedure
\end{algorithmic}

\end{minipage}
\caption{The \sleep code}
\end{algorithm}

\sleep is another hand written Mercury procedure.
It uses an
\enginesleepsync structure to manage the state of each engine.
These structures are organised into a global array indexed by engine ids,
making them accessible to other engines.
each structure contains a semaphore and a lock.
The engine owning the structure waits on \code{sleep\_sem} in order to
sleep,
and \code{lock} is used by other engines to ensure mutual exclusion for
their accesses to the \enginesleepsync structure.
There are three other fields in each structure:
\code{state} is used by the engine that owns the structure to communicate
its state to other engines,
while \code{action} and \code{action\_data} are used by other threads when
communicating to this engine.

Upon jumping into \idle,
an engine sets its state to \code{SLEEPING} and waits on \code{sleep\_sem}.
If another engine posts to \code{sleep\_sem} then the engine is woken up,
and retrieves the value of the \code{action} field.
The engine switches on this value to determine what to do,
such as:
shut down, 
execute a context, or
steal and execute a spark.
If it is instructed to run a context,
it will be passed the context directly
using the \code{action\_data} field.
Similarly, if it instructed to steal a spark,
then \code{action\_data} will tell
it whose spark stack it should begin its round-robin search from.
If thread stealing fails,
the engine will check the global context queue for
a runnable context;
if that fails,
then \idle will loop and put the engine back to sleep.
If no action is specified
(the value of the \code{action} field is \code{ACTION\_NOT\_SPECIFIED}),
then \sleep will try to run a context from the global context queue before
attempting to steal a spark.

\begin{algorithm}[tbp]
\begin{algorithmic}
\Procedure{MR\_wake\_engine}{$engine\_id, action, action\_data, states$}
    \State MR\_acquire\_lock($MR\_engine\_sleep\_data$[$engine\_id$]$.lock$)
    \If{$eng\_data.state \in states$}
        \State $MR\_engine\_sleep\_data$[$engine\_id$]$.action \gets action$
        \State $MR\_engine\_sleep\_data$[$engine\_id$]$.action\_data \gets action\_data$
        \State $MR\_engine\_sleep\_data$[$engine\_id$]$.state \gets$ WOKEN
        \State MR\_sem\_post($MR\_engine\_sleep\_data$[$engine\_id$]$.sleep\_sem$)
        \State $result \gets$ true
    \Else
        \State $result \gets$ false
    \EndIf
    \State MR\_release\_lock($MR\_engine\_sleep\_data$[$engine\_id$]$.lock$)
    \State \Return $result$
\EndProcedure
\end{algorithmic}
\caption{\wakeengine}
\label{alg:wake_engine}
\end{algorithm}

Engines may be woken with a call to \wakeengine,
whose code is shown in algorithm \ref{alg:wake_engine}.
\wakeengine takes as parameters the id of the engine to awaken,
the action and its data that the engine should perform upon awakening,
and the set of states from which we may awaken the engine from (usually
\code{SLEEPING}).
The implementation of \wakeengine is mostly straightforward,
one important detail is that we must only wake the engine if it has not
been woken already, this is done by checking the value of
\code{eng\_data.state} after acquiring the lock,
and setting the new state while holding the lock.
The action and action data must also be written to the
\enginesleepsync structure before posting to the target engine's sleep
semaphore.
The parameter \code{states} is a bit field, with each state represented by
its own bit.
This is flexible, it allows us to message an engine regardless of its
current state.
For example,
we shut down an engine by calling \wakeengine with the shut down action and
all bits set in the \code{states} bit field.
Compared to the previous system which used a lock and a condition variable,
this system allows us to wake engines selectively.
This is useful in the case when a context must be executed on a particular
engine because that engine's C stack contains a frame that made a call to
some Mercury code and now must be returned into.
In the future, this feature could be used to schedule computations on
\emph{nearby} processors,
processors that may share a second or third level cache with the current
processor.

\begin{algorithm}[tbp]
\begin{algorithmic}[1]
\Procedure{MR\_join\_and\_continue}{$ST, ContLabel$}
  \State $finished \gets$ MR\_atomic\_dec\_and\_is\_zero($ST.num\_outstanding$)
  \If{$finished$}
    \If{$ST.orig\_context \neq current\_context$}
      \While{$ST.orig\_context.resume\_label \neq ContLabel$}
        \State CPU\_relax
      \EndWhile
      \State MR\_release\_context($current\_context$)
      \State $current\_context \gets$ NULL
      \State MR\_load\_context($ST.parent$)
    \EndIf
    \Goto{$ContLabel$}
  \Else
    \If{$ST.orig\_context = current\_context$}
      \State $MR\_r1 \gets ContLabel$
      \Goto{MR\_idle\_orig\_context}
    \Else
      \Goto{MR\_idle}
    \EndIf
  \EndIf
\EndProcedure
\end{algorithmic}
\caption{\joinandcontinue}
\label{alg:join_and_continue_ws2}
\end{algorithm}

Moving spark deques from contexts to engines made it necessary to change
\idle.
Similarly we must also change the \joinandcontinue barrier.
The new version of \joinandcontinue is shown in algorithm
\ref{alg:join_and_continue_ws2}.
The most significant change is on line 15,
\joinandcontinue no longer attempts to run a spark from the local deque,
it relies on \idle to do this.
The second change is that \joinandcontinue no longer suspends the original
context, it executes \idle while holding the original context.
This would invalidate \idle's precondition that if an engine has a context,
that the context is otherwise free to execute \emph{any} spark,
but an original context may only execute sparks that were created by the
parallel conjunction that it is currently executing.
The new \idle code allows us to create any number of entry points,
we have created the \idleorigcontext entry point which expects to be called
with such a context.
However, since we do not set the context's \code{resume\_label} field until
the engine has given up the context,
we must pass the value of the continuation label to \idleorigcontext.
We do this by using \code{MR\_r1}, the first Mercury abstract machine
register;
this is the normal calling convention for Mercury procedures.

The other change is an optimisation on line 10.
In the old version of this algorithm the engine would schedule the original
context,
by placing it on the run queue,
and then call \idle,
which would most likely take the same context off the run queue and then
execute it.
We are able to take advantage of the knowledge that the original context is
now runnable and the engine's current context is finished,
Therefore, we can release the engine's current context and load the original
context without calling \idle or using the run queue.
This is more efficient as it uses the runqueue less and,
in turn uses the run queue's lock less.

\begin{algorithm}[tbp]
\begin{algorithmic}
    \Procedure{MR\_idle\_orig\_context}{}
        \State $join\_label \gets MR\_r1$
        \State $code\_ptr \gets$ MR\_try\_run\_local\_spark($join\_label$)
        \If{$code\_ptr \neq$ NULL}
            \Goto $code\_ptr$
        \EndIf
        \State MR\_save\_context($ctxt$);
        \State $current\_context.resume\_label \gets join\_label$
        \State $current\_context \gets$ NULL
        \Goto MR\_idle
    \EndProcedure
\end{algorithmic}
\caption{New entry point to the idle loop for dirty contexts.}
\label{alg:idle_orig_context}
\end{algorithm}

The new entry point \idleorigcontext is shown in algorithm
\ref{alg:idle_orig_context}.
This entry point expects to be called from \joinandcontinue while the engine
is holding a context that is needed to complete a parallel conjunction.
Using \tryrunlocalspark (algorithm \ref{alg:try_run_local_spark},
it tries to run a spark from the engine's local spark deque first.
When we introduced \tryrunlocalspark above,
we said that it checked whether the spark at the top of the spark deque was
compatible with the current context;
when \tryrunlocalspark is called from \idleorigcontext this test can fail
because the spark may be related to a different parallel conjunction than
the one that the engine's current context has been executing.
When this happens \tryrunlocalspark will put the spark back onto the spark
deque and return \NULL.
Below we describe why it is best to put the spark back on the deque.
It will also return \NULL when there is no spark on the deque.
If non-\NULL was returned then \idleorigcontext will jump to the code
address that \tryrunlocalspark returned,
otherwise it will save its current context,
set the context's \code{resume\_label} field
and jump to \idle which attempts to find work as described above,
which involves looking for work with \tryruncontext, \tryrunlocalspark, and
\trystealspark.
If the earlier call to \tryrunlocalspark put an incompatible spark back onto
the spark deque,
and the call to \tryruncontext did not find a runnable context then
the next call to \tryrunlocalspark
will retrieve the spark from the top of the deque and be able to execute it,
provided that the spark was not stolen in the meantime.
The spark will now be compatible because the engine released its context
before jumping to \idle.
This is one of two simple ways to handle context-spark incompatibility:
the other is to suspend the current context at the moment when we find that
a spark is incompatible.
Both solutions require a context switch,
but the solution that we have chosen will also attempt to execute a
suspended context before attempting to execute the spark using a new context
(we described above that it is best to execute a suspended context rather
than a spark).

We could add further entry points in the future if we found them useful,
and they do not have to jump to \idle when they finish.
For example,
it may be useful to create a special case for when an engine is known to
have a empty context;
this case could check for work in a different order and if it does not
find any work it could jump to \sleep the way \idle currently does.


\paul{XXX: I think I forgot to define a macro for the polling versions}

\begin{table}
\begin{center}
\begin{tabular}{r|rr|rrrr}
\multicolumn{1}{c|}{Version} &
\multicolumn{2}{c|}{Sequmential} &
\multicolumn{4}{c}{Parallel w/ $N$ Engines} \\
\Cbr{} & \C{not TS} & \Cbr{TS} & \C{1}& \C{2}& \C{3}& \C{4}\\
\hline
\hline
\multicolumn{7}{c}{Right recursive mandelbrot (max.\ 1024 contexts)} \\
\hline
Original
& 15.3 (0.99) & 15.2 (1.00)
& 15.2 (1.00) &  7.7 (1.98) &  5.2 (2.92) &  4.0 (3.79) \\
New (notify)
& 15.3 (1.00) & 15.3 (1.00) 
& 15.4 (1.00) &  8.1 (1.88) &  5.2 (2.97) &  3.9 (3.95) \\
New (poll)
& 15.5 (0.98) & 15.2 (1.00)
& 15.3 (1.00) &  7.6 (2.00) &  5.1 (2.99) &  3.8 (3.96) \\
\hline
\hline
\multicolumn{7}{c}{Left recursive mandelbrot} \\
\hline
Original
& 15.3 (0.99) & 15.2 (1.00)
& 15.2 (1.00) &  7.8 (1.95) &  5.1 (2.98) &  4.0 (3.84) \\
New (notify)
& 15.3 (1.00) & 15.3 (1.00)
& 15.4 (1.00) &  7.7 (1.99) &  5.2 (2.97) &  4.0 (3.82) \\
New (poll)
& 15.3 (0.99) & 15.1 (1.00)
& 15.1 (1.00) &  7.6 (1.99) &  5.1 (2.95) &  3.9 (3.92) \\
\hline
\hline
\multicolumn{7}{c}{Fibs program with GC (\code{Depth} = 10)} \\
\hline
Original
& 4.4 (1.06) & 4.6 (1.00)
& 4.6 (1.00) & 2.3 (2.00) & 1.6 (2.97) & 1.2 (3.97) \\
New (notify)
& 4.3 (1.08) & 4.6 (1.00)
& 4.6 (1.00) & 2.3 (2.00) & 1.6 (2.98) & 1.2 (3.97) \\
New (poll)
& 4.4 (1.06) & 4.6 (1.00)
& 4.6 (1.00) & 2.3 (2.00) & 1.6 (2.98) & 1.2 (3.94) \\
\hline
\hline
\multicolumn{7}{c}{Fibs program with GC (\code{Depth} = 20)} \\
\hline
Original
& 4.4 (1.06) & 4.6 (1.00)
& 4.7 (0.99) & 2.3 (1.98) & 1.6 (2.96) & 1.2 (3.91) \\
New (notify)
& 4.3 (1.08) & 4.6 (1.00)
& 4.7 (0.99) & 2.3 (1.98) & 1.6 (2.95) & 1.2 (3.91) \\
New (poll)
& 4.4 (1.06) & 4.6 (1.00)
& 4.7 (0.99) & 2.3 (1.97) & 1.6 (2.94) & 1.2 (3.89) \\
\hline
\hline
\multicolumn{7}{c}{Fibs program with GC (\code{Depth} = 30)} \\
\hline
Original
&  4.0 (1.07) &  4.3 (1.00)
& 23.3 (0.18) & 11.7 (0.37) &  7.8 (0.55) &  5.9 (0.73) \\
New (notify)
&  3.9 (1.09) &  4.2 (1.00)
& 25.3 (0.17) & 12.7 (0.34) &  8.4 (0.51) &  6.3 (0.67) \\
New (poll)
&  4.0 (1.07) &  4.3 (1.00)
& 25.4 (0.17) & 12.8 (0.33) &  8.5 (0.50) &  6.4 (0.67) \\
\hline
\hline
\multicolumn{7}{c}{Fibs program with GC (\code{Depth} = 40)} \\
\hline
Original
&  3.7 (1.09) &  4.1 (1.00)
& 33.5 (0.12) & 16.8 (0.24) & 11.2 (0.36) &  8.5 (0.48) \\
New (notify)
&  3.7 (1.11) &  4.1 (1.00)
& 36.5 (0.11) & 18.4 (0.22) & 12.2 (0.33) &  9.2 (0.44) \\
New (poll)
&  3.7 (1.10) &  4.1 (1.00)
& 37.5 (0.11) & 18.6 (0.22) & 12.3 (0.33) &  9.2 (0.44) \\
\hline
\hline
\multicolumn{7}{c}{Fibs program without GC} \\
\hline
Original
&  4.4 (1.06) &  4.6 (1.00)
& 32.8 (0.14) & 16.5 (0.28) & 11.0 (0.42) &  8.3 (0.56) \\
New (notify)
&  4.3 (1.08) &  4.6 (1.00)
& 35.9 (0.13) & 18.1 (0.26) & 12.0 (0.39) &  9.0 (0.51) \\
New (poll)
&  4.4 (1.06) &  4.6 (1.00)
& 35.9 (0.13) & 17.8 (0.26) & 11.9 (0.39) &  8.9 (0.52)
\end{tabular}
\end{center}
\caption{Work stealing results --- revised implementation}
\label{tab:work_stealing_revised}
\end{table}




\plan{Benchmark}
Table \ref{tab:work_stealing_revised} shows a comparison of the previous
work stealing implementation with this one.
There are several row groups with two rows each.
Each row group gives the results for a single program and configuration,
while the rows in each group give the results for the previous work stealing
system and the current one.
The first row group provides the results for the
right recursive mandelbrot program.
We compiled this with the conjunct reordering transformation from section
\ref{sec:rts_reorder} disabled,
otherwise it would have had the same results as the left recursive version
of the same program,
which is shown in the second grow group.
It is still important to test right recursion such as in this example,
as the conjunct reordering transformation cannot reorder dependent parallel
conjunctions so right recursion can still occur.

\begin{figure}
\begin{verbatim}
:- func fibs(int) = int.

fibs(N) = F :-
    ( N < 2 ->
        F = 1
    ;
        (
            F1 = fibs(N-1)
        &
            F2 = fibs(N-2)
        ),
        F = F1 + F2
    ).
\end{verbatim}
\caption{\fibs without granularity control}
\label{fig:fibs}
\end{figure}

The following five row groups provide results for a new program named fibs.
Fibs calculates the value of the 43\textsuperscript{rd} Fibonacci number.
Fibs uses the naive definition of a Fibonacci number:  The
$N$\textsuperscript{th} Fibonacci number is the sum of the previous two
Fibonacci numbers unless $N$ is 0 or 1, in which case the result is 1.
This calculation is shown as the function \fibs\footnote{
    The arity of a Mercury function is written as \code{N+1}.
    The \code{+1} refers the return value while \code{N} refers to the
    function's input arguments.}
in figure \ref{fig:fibs},
this version of \fibs is used in the ``without GC'' (seventh) row group of table
\ref{tab:work_stealing_revised}.
but this is not the one used by the fibs program as we will now explain.
This results in a program with poor performance,
however it is a suitable micro-benchmark for testing work stealing.
The recursive branch executes two recursive calls in parallel with one
another and then sums their results,
the cost of the arithmetic and if-then-else at each recursion level is very
small compared to the parallel execution overheads.
This results in \emph{embarrassing parallelism},
a case where the there is more parallel work available than the machine can
handle to such an excess that the overheads of parallel execution from
dominate the execution time
making the program slower than its sequential version;
this can be seen in the table.
We use this deliberately to allow us to measure the differences in the
overheads due to work stealing.

\begin{figure}
\begin{verbatim}
:- func fibs_gc(int, int) = int.

fibsgc(N, Depth) = F :-
    ( N < 2 ->
        F = 1
    ;
        ( Depth > 0 ->
            (
                F1 = fibs_gc(N-1, Depth-1)
            &
                F2 = fibs_gc(N-2, Depth-1)
            )
        ;
            F1 = fibs_seq(N-1),
            F2 = fibs_seq(N-2)
        ),
        F = F1 + F2
    ).
\end{verbatim}
\caption{\fibsgc, \fibs using granularity control}
\label{fig:fibsgc}
\end{figure}

To avoid this we can use \emph{granularity control} to create fewer larger
parallel tasks (more coarse-grained parallelism).
This introduces a new function \fibsgc, which takes an extra argument,
\Depth.
\fibsgc is shown in figure \ref{fig:fibsgc}.
\Depth is the number of recursion levels near the top of the call graph in
which parallel execution should be used.
\fibsgc now contains a conditional in place of the original parallel
conjunction,
depending on the value of depth we either call a parallel conjunction of two
calls to \fibsgc,
or a sequential conjunction of two calls to \fibsseq, a sequential version of
\fibs.
\fibsgc decrements \Depth before making the recursive calls to track how
close the current level is to the level at which it should switch to
sequential execution.
The third through sixth row groups in table
\ref{tab:work_stealing_revised} use \fibsgc, with varying values for
\Depth.
This is just one type of granularity control.
Granularity control is any method used to avoid
embarrassing parallelism by turning a computation with 
many small parallel tasks into one with fewer larger parallel tasks.

\plan{Evaluation}
Fibs has better parallel performance when \Depth's initial value is
smaller,
demonstrating how granularity control effects performance.
In most tests,
the mandelbrot tests and the coarse grained fibs tests,
there is very little difference between the performance of the old and new
work stealing systems.
However,
in the fine grained fibs tests we can clearly see that the new work stealing
system has much higher overheads.
This is not what we expected but it makes sense.
The new work stealing system no longer uses a timeout to wake up sleeping
engines, allowing them to check for work.
As an alternative it allows the creation of a spark to notify a sleeping
engine.
We think that this contributes to part of the slow down in performance,
but probably not all of it.
Further optimisation and tweaking could improve the situation in the future.

There are some other interesting trends in the results.
Most obviously,
as expected and explained above,
parallel fibs program with fine granularity
(high values for \Depth or disabled granularity control),
perform more slowly than those with coarse granularity.
There
We expected that sequential version versions of fibs without granularity
control to be faster than with it,
as granularity control will introduce its own overheads, this is only
somewhat true.
We also expected that the overheads of granularity control would affect the
sequential programs differently,
that smaller values of \Depth would result in faster execution times,
as less time is spent in the part of the call graph that includes the
granularity control computation.
This may be caused by some reason why the granularity controlled version of
sequential \fibs is faster.
Such a reason might be because the code generated for this procedure is
faster as it aligns labels differently or interacts with the processor's
branch prediction hardware more favourably.
This type of effect can be common with micro-benchmarks,
which is one reason why their use is often discouraged.

In the left recursive mandelbrot tests the revised work stealing
implementation is slightly slower than the original implementation.
This could be due to the same reasons that make the revised implementation
slower in the finer grained fibs tests.
Compare this to the right recursive mandelbrot tests:
in the tests with three and four Mercury engines the revised implementation
is slightly faster.
The left recursive mandelbrot program generates all its
sparks at once and the right recursive one does not.
The revised work stealing system may be slightly faster because it
notifies an engine when there is a spark to execute in parallel,
while in the original version, a sleeping engine polls the spark deques
every two milliseconds.
However, the more likely explanation is that the search through all the
deques takes longer in the original version as the number of deques is equal
to the number of contexts,
and as we saw earlier in this chapter right recursion can create a large
number of contexts.
This is especially true with larger numbers of engines;
fewer sparks are executed on the engine on which they were created where
they can re-use their parent context.

\plan{Further potential work}
\plan{Steal half}
We have noticed that in many workloads such as a left recursive parallel
loop one engine creates all the sparks.
In a situation like this,
work stealing becomes more common than local work execution.
One processor would execute work locally while $P - 1$ processors act as
thieves,
and the original engine's work queue can become a bottleneck.
To avoid this behaviour,
a thief could steal more than one spark from a victim and put all but one of
those sparks on its own deque.
Each time this happens work becomes distributed among more of the engines,
which improves the probability that a given victim has work on its deque.
The numbers of sparks on each engine's deque should stabilise,
and then work stealing will become less frequent as each engine will prefer to
take work from the hot end of its own deque.
Executing more work from one's own deque, and decreasing the amount of stealing
should both improve the efficiency of the program as a whole.

We are currently using the deque data structure and algorithms described by
\citet{Chase_2005_wsdeque} (CL Deque).
\citet{hendler:2002:stealhalf} (HS Deque) describes a similar data structure
and set of algorithms that allow half the work to be stolen in a single lock
free operation.
These data structures are similar:
they are both based on \citet{arora:1998:work-stealing}.
The HS Deque differs from 
\citet{arora:1998:work-stealing}
by stealing half the work from a victim's deque,
whilst the CL Deque differs by structuring each deque as
dynamically resizable circular buffer.
We have chosen to use the CL Deque because it is resizable,
a feature that we identified as important (section \ref{sec:rts_work_stealing}).
Furthermore,
if we ever adopt a steal half deque (or any work stealing deque that allows
us to steal more than one item in a single operation) we will have to
consider spark execution order carefully.
Therefore,
these ideas may be interesting for further work.

\plan{memory hierarchy awareness}
Another potential improvement is in the selection of a thief's victim.
A thief may wish to prefer victims that are nearby in terms of memory
topology, so that communication of the data relevant to the spark is
cheaper.
This can also be used when selecting which engine to wake up in order to
pass work to it.

