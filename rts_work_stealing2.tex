
\status{Not written}

In this section we take the opportunity to improve on our work stealing
implementation from Section \ref{sec:rts_work_stealing}.
While we made these improvments we also found it useful to change how idle
engines are managed.
Although some of these changes can be conceptually distinct,
they where made together and their implementations are interlinked.
Therefore we will present and benchmark them together as one set of changes.

\plan{Describe problems with associating stacks with contexts}
Sparks are stored on context local deques,
this has a several significant problems.

\begin{description}

    \item[There are a dyanic number of spark deques]~

    The number of contexts changes during a programs execution,
    therefore the number of spark deques also changes.
    This means that we must have code to manage this chaging number of
    deques.
    This code makes the runtime system more complicated than what is
    necessary,
    both when stealing a spark and when destroying or creating a context.

    \item[This requires locking]~

    The management of contexts must be done with thread safety so that the
    set of contexts is not currupted.
    We are using a simple implementation with an array where locking is
    necessary.
    It may be possible to replace this with a lock free datastructure,
    however removing the code is preferable.

    \item[A large number of contexts makes work stealing slower]~
    
    In previous sections
    we have shown that the number of contexts in use can often be very high,
    much higher than the number of Mercury engines.
    If there are $N$ engines and $M$ contexts,
    then there can be at most $N$ of the $M$ contexts running and
    at least $M-1$ contexts suspended (blocked and waiting to run).
    Contexts can become suspended in one of two ways:
    by blocking on a future's value in call to \wait,
    or by blocking on an incomplete conjunct in a call to \joinandcontinue.
    We attempt to minimise the former case and the later case cannot occur
    if the context has a spark on its local spark deque (it would run the
    spark rather than block).
    Therefore the large majority of the $M-N$ contexts that are suspended
    will not have sparks on their deques,
    and the probability of selecting a deque at random with a spark on its
    stack is only a little bit higher than $M \choose N$.
    Furthermore the value of $M$ can be very high in pathological cases.
    The runtime system will continue to try to steal a spark from different
    deques its complexity is $O(M)$.
    This is also true in cases where it gives up and tries again later,
    this can happen even when there is parallelism available.
    It also happens when a theif is successful,  
    as the victim counter is always incremented.

\end{description}

\noindent
\plan{We associate stacks with engines}
The solution to all of these problems is to associate spark deques with
engines rather than with contexts.
A running Mercury program has a constant number of engines,
and therefore the number of spark deques will not vary.
This means that we can remove the code used to resize the deque array when a
context is created or destroyed, simplfying the system and making these
procedures faster.
We can also remove the locking code in \trystealspark used to ensure that
the array is not changed while a theif is trying to steal a spark.
The cost of work stealing also becomes linear in the number of engines
rather than the number of contexts (which is usually larger).

\plan{Show how this is safe.}
Associating the spark deques with engines can change the order in which
sparks are executed.
A context that is suspended by a call to \wait may contain sparks,
when engines contain spark deques those sparks can be executed by the engine
that was running the context.
This does not affect the program's safety,
the program will not become deadlocked provided that
each\footnote{
    Sometimes a specific engine must execute a specific context so that the
    context's computation can access something on the C stack.
    Therefore all engines must check the context run queue for such a
    context.
    This is an implementation detail of foreign C code.
    }
of the engines checks the context run queue for a context
to execute.
It is generally a good idea to check the context run queue often as a
suspended context may be able to make more parallel work available either by
creating a spark or signaling a future.
Signaling a future as early as possible can,
in some cases,
prevent a call to \wait from blocking a context.

\begin{algorithm}
\begin{algorithmic}[1]
\Procedure{MR\_try\_steal\_spark}{$spark\_ptr$}
  \State $result \gets false$
  \For{$attempt = 0$ to $MR\_num\_engines$}
    \State $victim\_index \gets
        (MR\_engine.victim\_counter + attempt) \bmod MR\_num\_engines$
    \State $deque \gets
       MR\_spark\_deques$[$victim\_index$]
    \State $result \gets$ \steal($deque$, $spark\_ptr$)
    \If{$result$}
      \State $result \gets true$
      \State break
    \EndIf
  \EndFor
  \State $MR\_engine.victim\_counter \gets
    MR\_engine.victim\_counter + offset$
  \State \Return $result$
\EndProcedure
\end{algorithmic}
\caption{MR\_try\_steal\_spark}
\label{alg:try_steal_spark_initial}
\end{algorithm}

\plan{Show new \trystealspark}
Find out if I started tracking stealing per engine or not.

\plan{Describe the problem with the current algorithm.}
Engines wake up and periodically check for work by attempting to
steal work,
firstly, this means that there can be up to a 2ms (average 1ms) delay before
a spark is executed.
secondly, this checks for work too often, wasting resources.

\plan{Solution, wake engines for different types of work}
We modified the RTS so that waking an engine is easy, and it can be given a
message so that it knows where to look for work.

\plan{A lot of work went into preventing deadlocks due to race conditions,
a thread that is not yet sleeping if notified must wake up immediately.}

\plan{Show new \getglobalwork}
\plan{Show the algorithm for the new idle loop.}
Note that we execute contexts before sparks,
this is more-likely to produce futures and it may reduce memory consumption.

\plan{Extra benefit: when an engine is woken it can be told directly where
to find work, or be given the work directly.}

\plan{Extra benefit: we can be selective about which engine to wake,
while not implemented fully, we can wake a `nearby' engine so that
we can avoid communication between dies or sockets.}


\begin{table}
\begin{center}
\begin{tabular}{r|rr|rrrr}
\multicolumn{1}{c|}{Max no.} &
\multicolumn{2}{c|}{Sequmential} &
\multicolumn{4}{c}{Parallel w/ $N$ Engines} \\
\Cbr{of contexts} & \C{not TS} & \Cbr{TS} & \C{1}& \C{2}& \C{3}& \C{4}\\
\hline
\hline
\multicolumn{7}{c}{Prior right recursion work stealing results} \\
\hline
4        & 16.3 (0.93) & 15.2 (1.00)
         & 15.2 (1.00) & 15.2 (1.00) & 15.2 (1.00) & 15.2 (1.00) \\
64   &-&-& 15.2 (1.00) & 15.2 (1.00) & 15.2 (1.00) & 15.2 (1.00) \\
128  &-&-& 15.2 (1.00) & 15.0 (1.01) & 14.8 (1.03) & 14.9 (1.02) \\
256  &-&-& 15.2 (1.00) & 13.5 (1.12) & 12.6 (1.21) & 12.6 (1.21) \\
512  &-&-& 15.1 (1.00) &  8.8 (1.73) &  6.5 (2.34) &  5.3 (2.84) \\
1024 &-&-& 15.5 (0.98) &  7.7 (1.98) &  5.2 (2.92) &  4.1 (3.70) \\
2048 &-&-& 15.2 (1.00) &  7.9 (1.93) &  5.2 (2.92) &  4.1 (3.71) \\
\hline
\hline
\multicolumn{7}{c}{Right recursion results with revised work stealing} \\
\hline
4        & 15.3 (1.00) & 15.3 (1.00)
         & 15.3 (1.00) & 15.3 (1.00) & 15.3 (1.00) & 15.3 (1.00) \\
64   &-&-& 15.3 (1.00) & 15.3 (1.00) & 15.3 (1.00) & 15.7 (0.97) \\
128  &-&-& 15.7 (0.98) & 15.3 (1.00) & 15.7 (0.97) & 15.3 (1.00) \\
256  &-&-& 15.3 (1.00) & 14.5 (1.06) & 14.8 (1.03) & 13.5 (1.13) \\
512  &-&-& 15.3 (1.00) &  9.6 (1.59) &  8.0 (1.91) &  6.8 (2.25) \\
1024 &-&-& 15.7 (0.97) &  7.7 (1.98) &  5.3 (2.87) &  3.9 (3.90) \\
2048 &-&-& 15.3 (1.00) &  7.7 (1.97) &  5.2 (2.94) &  3.9 (3.90) \\
\hline
\hline
\multicolumn{7}{c}{Prior left recursion work stealing results} \\
\hline
N/A      & 15.3 (1.02) & 15.6 (1.00)
         & 15.2 (1.03) &  7.8 (2.00) &  5.2 (2.98) &  3.8 (4.05) \\
\hline
\hline
\multicolumn{7}{c}{Left recursion results with revised work stealing} \\
\hline
N/A      & 15.3 (1.00) & 15.3 (1.00)
         & 15.3 (1.00) &  7.7 (1.99) &  5.1 (2.97) &  4.0 (3.83) \\
\end{tabular}
\end{center}
\caption{Work stealing results --- revised implementation}
\label{tab:work_stealing_revised}
\end{table}



\plan{Benchmark}

\plan{Evaluation}

\plan{Further potential work}

