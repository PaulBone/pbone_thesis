
\plan{Introduction}
We introduced parallelism in Mercury in Sections \ref{sec:backgnd_merpar} and
\ref{sec:backgnd_deppar};
we described how the runtime system in generic terms.
In this section we will explain how sparks were originally managed
prior to 2009,
when I begun my PhD candidature.
This will provide the background for the changes we have made to the
runtime since.

\plan{Global spark queue and contention WRT global queue}
Mercury (before 2009) uses a global spark queue.
The runtime system \emph{schedules} sparks for parallel execution by placing
them onto the end of the queue.
In this chapter we use the word `schedule' to mean the act of making a spark
available for parallel or sequential work.
An idle engine runs a spark by taking it from the beginning of the queue.
The global spark queue must be protected by a lock;
this prevents concurrent access from corrupting the queue.
The global spark queue and its lock can easily become a bottleneck when many
engines contend for access to the global queue.

\plan{Local spark stack --- relieves contention on global queue}
\citet{wang:2006:hons} anticipated this problem and created context local spark
stacks to avoid contention on the global queue.
Furthermore, the local spark stacks do not require locking.
When a parallel conjunction spawns off a spark,
it places the spark either at the end of the global spark queue or at the
top of its local spark stack.
\plan{Spark scheduling decision.}
The runtime system appends the spark to the end of the global queue if
an engine is idle, and
the number of contexts in use plus the number of sparks on the global queue
does not exceed the maximum number of contexts permitted.
If either part of the condition is false,
the runtime system pushes the spark onto the top of the context's local
spark stack.
This algorithm has two aims.
The first is to reduce contention on the global queue,
especially in the common case that there is enough parallel work.
The second aim is to reduce the amount of memory allocated
as contexts' stacks by reducing the number of contexts allocated.
Globally scheduled sparks may be converted into contexts,
so they are also included in this limit.
We explain this limit on the number of context in more detail
in Section \ref{sec:original_scheduling_performance},
after covering the background information in the current section.
Note that sparks placed on the global queue are executed in a
first-in-first-out manner;
sparks placed on a context's local stack are executed in a
last-in-first-out manner.

\begin{figure}
\begin{tabular}{rl}
 1: & \code{~~MR\_SyncTerm ST;} \\
 2: & \code{~~MR\_init\_syncterm(\&ST, 2);} \\
 3: & \code{~~spawn\_off(\&ST, Spawn\_Label\_1);} \\
 4: & \code{~~}$G_1$ \\
 5: & \code{~~MR\_join\_and\_continue(\&ST, Cont\_Label);} \\
 6: & \code{Spawn\_Label:} \\
 7: & \code{~~}$G_2$ \\
 8: & \code{~~MR\_join\_and\_continue(\&ST, Cont\_Label);} \\
 9: & \code{Cont\_Label:} \\
\end{tabular}
\caption{Parallel conjunction implementation}
\label{fig:par_conj_impl_only}
\end{figure}

\begin{algorithm}
\begin{algorithmic}[1]
\Procedure{MR\_join\_and\_continue}{$ST, ContLabel$}
  \State MR\_acquire\_lock($ST.lock$)
  \State $ST.num\_outstanding \gets ST.num\_outstanding - 1$
  \If{$ST.num\_outstanding = 0$}
    \If{$ST.orig\_context = this\_context$}
      \State MR\_release\_lock($ST.lock$)
      \Goto{$ContLabel$}
    \Else
      \State $ST.orig\_context.resume\_label \gets ContLabel$
      \State MR\_schedule($ST.orig\_context$, $ContLabel$)
      \State MR\_release\_lock($ST.lock$)
      \Goto{MR\_idle}
    \EndIf
  \Else
    \State $spark \gets$ MR\_pop\_spark($this\_context.spark\_stack$)
    \If{$spark$}
      \If{$spark.ST.stack\_ptr = MR\_parent\_sp$}
%        \Comment{This spark belongs to the same parallel conjunction.
%        It can be executed immediatly.}
        \State MR\_release\_lock($ST.lock$)
        \Goto{$spark.code\_label$}
      \Else
        \State MR\_push\_spark($this\_context.spark\_stack$, $spark$)
      \EndIf
    \EndIf
    \If{$ST.orig\_context = this\_context$}
       \State MR\_save\_context($this\_context$)
       \State $this\_context \gets NULL$
    \EndIf
    \State MR\_release\_lock($ST.lock$)
    \Goto{MR\_idle}
  \EndIf
\EndProcedure
\end{algorithmic}
\caption{MR\_join\_and\_continue}
\label{alg:join_and_continue_peterw}
\end{algorithm}

\plan{barrier code, this is used to explain the right recursion problem.}
In Section \ref{sec:backgnd_merpar} we described how parallel
conjunctions are compiled
(Figure \ref{fig:par_conj} shows an example).
Consider the compiled parallel conjunction in Figure
\ref{fig:par_conj_impl_only}.
The context that executes the parallel conjunction,
lets call it $C_{Orig}$,
begins by
setting up the sync term,
spawning off $G_2$,
and executing $G_1$ (lines 1--4).
Then it executes the barrier \joinandcontinue,
whose is shown in
Algorithm \ref{alg:join_and_continue_peterw}.
Depending on how full the global run queue is,
and how parallel tasks are interleaved,
there are three important scenarios:

\begin{description}

    \item[Scenario one:]~

    In this scenario $C_{Orig}$ placed the spark for $G_2$ on the top of its
    local spark stack.
    Sparks placed on a context local spark stack cannot be executed by any
    other context.
    Therefore when $C_{Orig}$ reaches the \joinandcontinue barrier
    (line 5 in the example),
    the context $G_2$ will be outstanding and $ST.num\_outstanding$ will be
    non-zero.
    $C_{Orig}$ will execute else branch on line 14 of \joinandcontinue
    where it will pop the spark for $G_2$ off the top of the spark stack.
    It is not possible for some other spark to be on the top of the stack;
    any sparks left on the stack by $G_1$ would have been popped off by 
    the \joinandcontinue barriers of the conjunctions that spawned off the
    sparks.

    The check that the sparks stack pointer is equal to the current
    parent stack pointer\footnote{
        The code generator will ensure that $MR\_parent\_sp$ is set
        before the parallel conjunction is executed,
        and that it is restored after.}
    will succeed (line 16 of \joinandcontinue),
    and the context will execute the spark.

    After executing $G_2$,
    $C_{Orig}$ will execute the second call to \joinandcontinue,
    the one on line 8 of the example.
    This time $ST.num\_outstanding$ will be zero,
    and $C_{Orig}$ will execute the then branch on line 5 of
    \joinandcontinue.
    In the condition of the nested if-then-else,
    $C_{Orig}$ will find that the current context is the original context,
    and therefore continue execution at line 6.
    This causes $C_{Orig}$ to jump to the continuation label,
    line 9 in the example,
    completing the execution of the parallel conjunction.

    \item[Scenario two:]~

    In this scenario $C_{Orig}$ placed the spark for $G_2$ on the 
    global spark queue,
    where another context, $C_{Other}$, picked it up and executed it
    in parallel.
    Also, as distinct from scenario three,
    $C_{Other}$ reaches the barrier on line 8 of the example \emph{before}
    $C_{Orig}$ reaches the barrier on line 5.
    Even if they both seem to reach the barrier at the same time,
    their barrier operations are performed in sequence because of the
    lock protecting the barrier code.

    When $C_{Other}$ executes \joinandcontinue,
    It will find that $ST.num\_outstanding$ is non-zero,
    and will execute the else branch on line 14 of \joinandcontinue.
    It then attempts to pop a spark off its stack,
    as in another scenario a spark on the stack might represent an
    outstanding conjunction
    (it cannot tell that the outstanding conjunct is $G_1$ executing in
    parallel, and not some hypothetical $G_3$).
    $C_{Other}$ took this spark from the global queue,
    and was either empty or brand new before executing this spark,
    meaning that it had no sparks on its stack before executing $G_2$.
    Therefore $C_{Other}$ will not have any sparks of its own and
    \code{pop\_spark} will fail.
    $C_{Other}$ will continue to line 21 in \joinandcontinue
    which will also fail since it ($C_{Other}$)
    is not the original context ($C_{Orig}$).
    The lock will be released and \joinandcontinue will determine what to do
    next by jumping to \idle.

    Eventually $C_{Orig}$ will execute its call to \joinandcontinue
    (line 5 of the example),
    or be executed after waiting on the barrier's lock (line 2 of
    \joinandcontinue),.
    When this happens it will find that $ST.num\_outstanding$ is zero,
    and execute the then branch beginning at line 5 of \joinandcontinue.
    $C_{Orig}$ will test if it is the original context, 
    which it is,
    and continue on line 6 of \joinandcontinue.
    It then jumps to the continuation label on line 9 of the example,
    completing the parallel execution of the conjunction.

    \item[Scenario three:]~

    As in scenario two $C_{Orig}$ pl the spark for $G_2$ on the global spark
    queue,
    where another context, $C_{Other}$, picked it up and executed it
    in parallel.
    However,
    in this scenario
    $C_{Orig}$ reaches the barrier on line 5 in the example \emph{before}
    $C_{Other}$ reaches its barrier on line 8.

    When $C_{Orig}$ executes \joinandcontinue,
    it finds that $ST.num\_outstanding$ is non-zero,
    causing it to execute the else branch on line 14 of \joinandcontinue.
    $C_{Orig}$ will try to pop a spark of its local spark stack.
    However the the spark for $G_2$ was placed on the global
    spark queue,
    the only spark it might find is one created by an outer conjunction.
    If a spark is found, the spark's parent stack pointer will not match the
    current parent stack pointer,
    and it will put the spark back on the stack.
    $C_{Orig}$ executes the then branch (line 22 of \joinandcontinue), since this
    context is the original context.
    This branch will suspend $C_{Orig}$,
    and set the engine's context pointer to \NULL
    before jumping to \idle.

    When $C_{Other}$ reaches the barrier on line 8,
    it will find that $ST.num\_outstanding = 0$,
    and will execute the then branch of the if-then-else.
    Within this branch it will test to see if it is the original
    context,
    the test will fail and the else branch of the nested if-then-else
    will be executed.
    At this point we know that $C_{Orig}$ must be suspended because
    there where no outstanding conjuncts and the current context is not
    the original context;
    this can only happen if $C_{Orig}$ is suspended.
    The code wakes $C_{Orig}$ up by
    setting its code pointer to the continuation label,
    placing it on the global run queue,
    and then jumps to \idle.

    When $C_{Orig}$ resumes execution it executes the code on line 9,
    which is the continuation label.

\end{description}

The algorithm includes an optimisation not shown here:
if the parallel conjunction has been executed by only one context,
then a version of the algorithm without locking is used.
We have not shown the optimisation because it is equivalent and not relevant
to our discussion,
we mention it only for completeness.

\begin{algorithm}
\begin{algorithmic}
\Procedure{MR\_idle}{}
  \State MR\_acquire\_lock($MR\_runqueue\_lock$)
  \Loop
    \If{$MR\_exit\_now$}
      \State MR\_release\_lock($MR\_runqueue\_lock$)
      \State MR\_destroy\_thread()
    \EndIf
    \State $ctxt \gets$ MR\_get\_runnable\_context()
    \If{$ctxt$}
      \State MR\_release\_lock($MR\_runqueue\_lock$)
      \If{$current\_context$}
        \State MR\_release\_context($current\_context$)
      \EndIf
      \State MR\_load\_context($ctxt$)
      \Goto $ctxt.resume$
    \EndIf
    \State $spark \gets$ MR\_dequeue\_spark($MR\_global\_spark\_queue$)
    \If{$spark$}
      \State MR\_release\_lock($MR\_runqueue\_lock$)
      \If{$\neg current\_context$}
        \State $ctxt \gets$ MR\_get\_free\_context()
        \If{$\neg ctxt$}
          \State $ctxt \gets$ MR\_create\_context()
        \EndIf
        \State MR\_load\_context($ctxt$)
      \EndIf
      \State $MR\_parent\_sp \gets spark.parent\_sp$
      \State $ctxt.thread\_local\_mutables \gets
        spark.thread\_local\_mutables$
      \Goto $spark.resume$
    \EndIf
    \State MR\_wait($MR\_runqueue\_cond, MR\_runqueue\_lock$)
  \EndLoop
\EndProcedure
\end{algorithmic}
\caption{\idle}
\label{alg:MR_idle_initial}
\end{algorithm}

\plan{Explain how work begins executing, for completeness.}
When an engine cannot get any local work it must search for global work.
Newly created engines, except for the first, also search for global work.
They do this by calling \idle,
whose code is shown in Algorithm \ref{alg:MR_idle_initial}.
Only one of the idle engines can execute \idle at a time.
The \var{MR\_runqueue\_lock} lock protects the context run queue and the
global context queue from concurrent access.
After acquiring the lock,
engines execute a loop.
An engine exits the loop only when it finds some work to do or the
program is exiting.
Each iteration first checks if the runtime system is being shut down,
if so,
then this thread releases the lock,
and then destroys itself.
If the system is not being shut down,
the engine will search for a runnable context.
If it finds a context it releases the run queue lock, loads the context
and jumps to the resume point for the context.
If it already had a context then it first releases that context,
doing so is safe as
a precondition of an engine calling \idle is that if the engine has
a context, the context must not contain a spark or represent a
suspended computation.
If no context was found, the engine attempts to take a spark from the global
spark queue.
If it finds a spark then it will need a context to execute that spark.
It will try to get a context from the free list, if there is none it will
create a new context.
Once it has a context,
it copies the context's copies of registers into the engine.
It also initialises the engine's parent stack pointer
register and the spark's thread local mutables
(which are set by the context that created the spark)
into the context.
If the engine does not find any work,
it will wait using a condition variable and the run queue lock.
The pthreads wait function is able to unlock the lock and wait on the
condition atomically, preventing race conditions.
The condition variable is used to wake up the engine if either a spark is
placed on the global spark queue or a context is placed on the context run
queue.
When the engine wakes,
it will re-execute the loop.

