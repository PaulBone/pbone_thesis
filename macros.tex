
% Proper names.
\newcommand{\mapfoldl}{\code{map\_foldl}\xspace}
\newcommand{\mapfoldlpar}{\code{map\-\_\-fold\-l\-\_\-par}\xspace}
\newcommand{\LC}{\code{LC}\xspace}
\newcommand{\LCS}{\code{LCslot}\xspace}
\newcommand{\createloopgoal}{\code{create\-\_loop\_goal}\xspace}
\newcommand{\putbarriers}{\code{put\-\_bar\-riier\-s\-\_in\-\_base\_case\-s}\xspace}
\newcommand{\io}{\code{io}\xspace}
\newcommand{\di}{\code{di}\xspace}
\newcommand{\uo}{\code{uo}\xspace}
\newcommand{\n}{{\textbackslash}n\xspace}
\newcommand{\signal}{\code{future\_signal/2}\xspace}
\newcommand{\wait}{\code{future\_wait/2}\xspace}
\newcommand{\get}{\code{future\_get/2}\xspace}
\newcommand{\joinandcontinue}{\code{MR\_join\_and\_continue}\xspace}

% the name 'det' is already taken.  So I've prefixed all the detisms with 'd'
\newcommand{\ddet}{\code{det}\xspace}
\newcommand{\dsemidet}{\code{semidet}\xspace}
\newcommand{\dmulti}{\code{multi}\xspace}
\newcommand{\dnondet}{\code{nondet}\xspace}
\newcommand{\dfailure}{\code{failure}\xspace}
\newcommand{\derroneous}{\code{erroneous}\xspace}
\newcommand{\dccmulti}{\code{cc\_multi}\xspace}
\newcommand{\dccnondet}{\code{cc\_nondet}\xspace}

\newcommand{\seqfn}[0]{\texttt{seq}\xspace}
\newcommand{\parfn}[0]{\texttt{par}\xspace}

\newcommand{\PS}[0]{\code{ProcStatic}\xspace}
\newcommand{\PD}[0]{\code{ProcDynamic}\xspace}
\newcommand{\CSS}[0]{\code{CallSiteStatic}\xspace}
\newcommand{\CSD}[0]{\code{CallSiteDynamic}\xspace}
\newcommand{\Clique}[0]{\code{Clique}\xspace}

% These get used in math.
\newcommand{\prodtime}{\operatorname{prodtime}\,}
\newcommand{\prodtimep}{\operatorname{prodtime'}\,}
\newcommand{\callprodtime}{\operatorname{callprodtime}\,}
\newcommand{\timef}{\operatorname{time}\,}
\DeclareMathOperator*{\Avg}{Avg}
\newcommand{\undef}{\operatorname{undefined}\xspace}
\newcommand{\pat}[1]{\llbracket#1\rrbracket}

% Author notes.
\newcommand{\authornote}[3]{
% Comment out next line to remove author notes
    {\fbox{\sc #1}:$\triangleright$\textcolor{#2}{\textbf{#3}}$\triangleleft$}%
}

\newcommand{\paul}[1]{\authornote{Paul}{blue}{#1}}
\newcommand{\peter}[1]{\authornote{Peter}{green}{#1}}
\newcommand{\zoltan}[1]{\authornote{Zoltan}{red}{#1}}
\newcommand{\status}[1]{\authornote{Status of this section}{blue}{#1}}

\newcommand{\todoitem}[3]{\parbox{4in}{#1} & 
  \parbox{1in}{#2} & \parbox{1in}{#3} }

% Optional prose.
\newcommand{\iclp}[1]{{}}
\newcommand{\conf}[1]{{}}
\newcommand{\tr}[1]{{#1}}

% Semantic markup.
\newcommand{\code}[1]{{\tt#1}}
\newcommand{\samp}[1]{`{\tt#1}'}
\newcommand{\param}[1]{\mbox{\it{#1}}}
\newcommand{\dfn}[1]{{\em#1}}
\newcommand{\stress}[1]{{\em#1}\/}
\newcommand{\calls}[0]{\rightarrow}
\newcommand{\instr}[1]{{\red{\code{#1}}}}

% Layout, floats etc.
\newlength{\figboxwidth}
\setlength{\figboxwidth}{\textwidth}

% 2 arguments:
% 	- the figure filename
% 	- the caption
% note that the filename will be used to form the label "fig:filename"
\newcommand{\picfigure}[2]{
	\begin{figure}[t]
	\newcommand{\spf}{\footnotesize}      % Sets font size for the picture
	\input{pics/#1}                      % defines macro called \graph
	\centerline{\raise 1em\box\graph}     % Prints the picture.
	\vspace{1mm}
	\caption{#2}
	\label{fig:#1}
	\end{figure}
}

% Extra control-flow code for algorithmic
\algloopdefx[Goto]{Goto}[1]{\textbf{goto} #1}

% Don't print closing statments in algorithmic examples.
\algnotext{EndIf}
\algnotext{EndProcedure}


