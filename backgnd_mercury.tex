
\section{Mercury}
\label{sec:backgnd_mercury}
\status{This section is complete.
}

Mercury is a pure logic/functional programming language
intended for the creation of large, fast, reliable programs.
Although the syntax of Mercury is based on the syntax of Prolog,
semantically the two languages are different
due to Mercury's purity and its type, mode, determinism and module systems.

Mercury programs consist of modules,
each of which has a separate namespace.
Individual modules are compiled separately.
Each module contains predicates and functions.
Functions are syntactic sugar for predicates with an extra argument for the
their result.
In the remainder of this dissertation we will use the word predicate to
refer to either a predicate or a function.

A predicate or function $P$ is defined in terms of a goal $G$:

%\vspace{-1\baselineskip}
% \begin{figure}[htb]
$$
\begin{array}{lll}
~P
    & :~ p(X_1, \ldots, X_n)~\leftarrow~G
        & \hbox{predicates} \\
%    & |~ f(X_1, \ldots, X_n)=X_{n+1}~\leftarrow~G
%        & \hbox{functions} \\
\end{array}
$$

% \paul{Peter suggested using a more grammar-like description of the language.
% I think that his motivation was that terminals and non-terminals should look
% more distinct.
% I have capitalised the variable names since that's closer to mercury syntax,
% it is not the same as showing terminals and non-terminals differently but
% perhaps it is clearer.}

% This has to float otherwise there's too much vertical whitespace on the
% first page.
\begin{figure}
\[
\begin{array}{lll}
G
    & :~ X = Y ~|~ X = f(Y_1,~\ldots,~Y_n)
        & \hbox{unifications}\\
    & |~ p(X_1,~\ldots,~X_n)
        & \hbox{predicate calls} \\
%    & |~ X_{n+1} = f(X_1,~\ldots,~X_n)
%        & \hbox{function calls} \\
    & |~ X_0(X_1,~\ldots,~X_n)
        & \hbox{higher order calls} \\
    & |~ m(X_1,~\ldots,~X_n)
        & \hbox{method calls} \\
    & |~ \hbox{foreign}(p,
        [X_1:~Y_1,~\ldots,X_n:~Y_n],
        \hbox{\emph{foreign code}}),
        & \hbox{foreign code} \\
    & |~ (G_1,~\ldots,~G_n)
        & \hbox{sequential conjunctions}\\
    & |~ (G_1~\&~\ldots~\&~G_n)
        & \hbox{parallel conjunctions}\\
    & |~ (G_1 ; \ldots ; G_n)
        & \hbox{disjunctions}\\
    & |~ \hbox{switch}~X~(f_1:~G_1~;~\ldots~,f_n:~G_n)
        & \hbox{switches}\\
    & |~ (if~G_{cond}~then~G_{then}~else~G_{else})
        & \hbox{if-then-elses}\\
    & |~ not~G
        & \hbox{negations}\\
    & |~ some~[X_1,\ldots,X_n]~G
        & \hbox{existential quantification}\\
%    & |~ all~[X_1,\ldots,X_n]~G
%        & \hbox{universal quantification} \\
    & |~ promise\_pure~G
        & \hbox{purity promise}\\
    & |~ promise\_semipure~G
        & \hbox{purity promise}\\
\end{array}
\]
\caption{The abstract syntax of Mercury}
\label{fig:abstractsyntax}
\end{figure}

\noindent
Mercury's goal types are shown in Figure~\ref{fig:abstractsyntax}.
The atomic goals are unifications
(which the compiler breaks down until each one contains
at most one function symbol),
plain first-order calls,
higher-order calls,
typeclass method calls (similar to higher-order calls),
and calls to predicates defined by code in a foreign language (usually C).
% \peter{You only need to cover enough about Mercury for the purposes of
%   your thesis.  I do not think how inlining of foreign code works is important.}
% \paul{We use it to implement primitives on which our transformations are
% built,
% I may remove it later if the detail of how this is done is unimportant.}
The compiler may inline foreign code definitions when generating code
for a Mercury predicate.
To allow this, the representation of a foreign code construct includes
not just the name of the predicate being called,
but also the foreign code that is the predicate's definition
and the mapping from the Mercury variables that are the call's arguments
to the names of the variables that stand for them in the foreign code.
The composite goals include
sequential and parallel conjunctions,
disjunctions, if-then-elses, negations and existential quantifications.
Section~\ref{sec:backgnd_merpar} contains a detailed description of parallel
conjunctions.

The abstract syntax does not include universal quantifications:
they are allowed at the source level,
but are transformed into a combination of two negations and an existential
quantification
(Equation~\ref{eqn:expand_univ_quant}).
Universal quantifications do not appear in the compiler's internal goal
representation.
Similarly,
a negation can be simplified using the constants $true$ and $false$,
and an if-then-else (Equation~\ref{eqn:expand_neg}).
However the compiler does not make this transformation and
negations may appear in the compiler's goal representation.
Nevertheless, we will not discuss negations in this dissertation,
since our algorithms' handling of negations is equivalent
to their handling of the below if-then-else.
\paul{Bring back negation.}

\begin{align}
all~[X_1,~\ldots,~X_n]~G~&\rightarrow~not~(~some~[X_1,~\ldots,~X_n]~(~not~G~))
\label{eqn:expand_univ_quant}
\\
not~G~&\rightarrow~(if~G~then~false~else~true)
\label{eqn:expand_neg}
\end{align}

A switch is a disjunction in which
each disjunct unifies the same bound variable
with a different function symbol.
Switches in Mercury are thus analogous to switches in languages like C.
If the switch contains a case for each function symbol in the
switched-on value's type, then the switch is said to be \emph{complete}.
Otherwise the switch is \emph{incomplete}.
Note that the representation of switches in Figure~\ref{fig:abstractsyntax}
is not the same as their source code representation;
Their source code syntax is the same as a disjunction.
See page~\pageref{page:purity} for a description of the purity promises.

\label{superhomogeneous}
Note that because a unification goal contains at most one function symbol
then any unifications that would normally have more than one function symbol
are expanded into a conjunction of smaller unifications.
This also happens to any unfications that appear as arguments in calls.
A goal for a single expression such as the quadratic equation:

\begin{verbatim}
X = (-B + sqrt(B^2 - 4*A*C)) / (2*A)
\end{verbatim}

\noindent
Will be expanded into 12 conjoined goals:

\begin{verbatim}
V_1  = 2,               V_2  = B ^ V_1,         V_3 = 4,
V_4  = V_3 * A,         V_5  = V_4 * C,         V_6 = V_2 - V_5,
V_7  = sqrt(V_6),       V_8  = -1,              V_9 = V_8 * B,
V_10 = V_9 + V_7,       V_11 = V_1 * A,         X   = V_10 / V_11
\end{verbatim}

\noindent
As shown here,
constants such as integers must be introduced as unifications between a
variable and a function symbol representing the constant.
Goals flatterned in this way are said to be in
\emph{super homogeneous form}.

Mercury has a type system inspired by the type system of Hope~\citep{hope}
and is similar to Haskell's type system~\citep{haskell98};
it uses Hindley-Milner~\citep{hindley69:types,milner78:types} type
inference.
Mercury programs are statically typed; the compiler knows the type of every
argument of every predicate (from declarations or inference) and every local
variable (from inference).
Types can be parametric,
type parameters are written in uppercase and
concrete types are written in lower case.
As an example, the maybe type is defined as:

\begin{verbatim}
:- type maybe(T)
    --->    yes(T)
    ;       no.
\end{verbatim}

\noindent
A value of this type is a discriminated union;
which is either \code{yes(T)}, indicating the presence of some other value
whose type is \code{T};
or simply \code{no}.
The maybe type is used to describe a value that may or may not be present.
The type \code{maybe(int)} indicates that an integer value might be
available.
The list type uses a special syntax for nil and cons and is recursive:

\begin{verbatim}
:- type list(T)
    --->    [T | list(T)]       % cons
    ;       [].                 % nil
\end{verbatim}

% Float this so that it does not straddle a page boundary.
\begin{figure}
\begin{verbatim}
:- pred append(list(T), list(T), list(T)).

append([], Ys, Ys).
append([X | Xs], Ys, [X | Zs]) :-
    append(Xs, Ys, Zs).
\end{verbatim}
\caption{\code{append/3}'s definition and type signature}
\label{fig:append_type_and_defn}
\end{figure}

\noindent
\code{append/3}'s type signature and definition are shown in
Figure~\ref{fig:append_type_and_defn}.
Mercury also has a strong mode system.
At any given point in the program a variable has an \emph{instantiation
state},
this is one of free, ground or clobbered.
Free variables have no value,
ground variables have a fixed value,
clobbered variables once had a value, but it is no longer available.
Partial instantiation can also occur,
but the compiler does not fully support partial instantiation
and therefore it is rarely used.
Variables that are ground may also be described as unique,
meaning they are not aliased with any other value.
When variables are used, their instantiation state may change;
such a change is described by the initial and final instantiation states
separated by the \code{>>} symbol.
The instantiation states are the same if nothing changes.
Variables can only become \emph{more instantiated}:
the change `free to ground' is legal, but `ground to free' is not.
Similarly, uniqueness cannot be added to a value that is already
bound or ground.
Commonly used modes such as input (\code{in}), output (\code{out}),
destructive input (\code{di}) and unique output (\code{uo}) are
defined in the standard library:

\begin{verbatim}
:- mode in  == ground >> ground.
:- mode out == free >> ground.
:- mode di == unique >> clobbered.
:- mode uo == free >> unique.
\end{verbatim}

\noindent
A predicate may have multiple modes.
Here are two mode declarations for the \code{append/3} predicate above;
these are usually written immediately after the type declaration.

\begin{verbatim}
:- mode append(in, in, out) is det.
:- mode append(out, out, in) is multi.
\end{verbatim}

\noindent
When a predicate has a single mode,
the mode declaration may be combined with the type signature as a single
declaration such as the following declaration for \code{length/2},
the length of a list:

\begin{verbatim}
:- pred length(list(T)::in, int::out) is det.
\end{verbatim}

\noindent
The Mercury compiler generates separate code
for each mode of a predicate or function,
which we call a \emph{procedure}.
In fact, the compiler handles individual procedures as separate entities
after mode checking.
Mode correct conjunctions,
and other conjoined goals such as the cond and then parts of an if-then-else,
must produce values for variables before consuming them.
This means that
variables passed as input arguments must be ground before and after the
call and
variables passed as output arguments must be free before the call and
ground after the call.
Similarly, a variable passed in a destructive input argument will be destroyed
before the end of the call;
it must be unique before the call and clobbered afterwords.
Referencing a clobbered variable is illegal.
A variable passed in a unique output argument must initially be free and
will become unique\footnote{
    The \di and \uo modes are often used together to allow the compiler to
    destructively update data structures such as arrays.}.

Unifications can also modify the instantiation state of a variable,
Unifying a free term with a ground term will make the first term ground,
and the second term non-unique.
Unifying a ground term with a ground term is an equality test,
it does not affect instantiation.
Unifying two free terms is illegal\footnote{
Aliasing free variables is illegal,
Mercury does not support logic variables.
This is a deliberate design decision;
it makes unification in Mercury much
faster than unification in Prolog.}.
The instantiation state of each variable is known at each point
within a procedure,
therefore
the compiler knows exactly where each variable becomes ground.
The mode checking pass will minimally reorder conjuncts 
(in both sequential and parallel conjunctions)
so that multiple procedures created from the same predicate can be mode
correct.

% Mode invariants
The mode system enforces three invariants:

\begin{description}

  \item[Conjunction invariant:]
  In any set of conjoined goals,
  including not just conjuncts in conjunctions
  but also the condition and then-part of an if-then-else,
  each variable that is consumed by any of the goals
  is produced by exactly one earlier goal.
  The else part of an if-then-else is not conjoined with the cond part
  so this invariant does not apply to the cond and else parts.

  \item[Branched goal invariant:]
  In disjunctions, switches or if-then-elses,
  the goal types that contain alternative branches of execution,
  each branch of execution must produce
  the exact same set of variables
  that are consumed from outside the branched goal,
  with one exception:
  a branch of execution that cannot succeed (see determinisms below)
  may produce a subset of this set of variables.
  
  \item[Negated goal invariant:]
  A negated goal may not bind
  any variable that is visible to goals outside it,
  and the condition of an if-then-else may not bind a variable
  that is visible anywhere except in
  the then-part of that if-then-else.

\end{description}

\noindent
Each procedure and goal has a determinism,
which puts upper and lower bounds on the number of the procedure's possible
solutions
(in the absence of infinite loops and exceptions).
Mercury's determinisms are:

\begin{description}
    \item[\ddet] procedures succeed exactly once
    (upper bound is one, lower bound is one).
    \item[\dsemidet] procedures succeed at most once
    (upper bound is one, no lower bound).
    \item[\dmulti] procedures succeed at least once
    (lower bound is one, no upper bound).
    \item[\dnondet] procedures may succeed any number of times
    (no bound of either kind).
    \item[\dfailure] procedures can never succeed
    (upper bound is zero, no lower bound).
    \item[\derroneous] procedures have an upper bound of zero and a lower
    bound of one, which means they can neither succeed nor fail.
    They must either throw an exception or loop forever.
    % We use the cc detisms a little bit, since parallelism may be used in
    % cc_multi predicates.
    \item[\dccmulti] procedures may have more than one solution, like
    \dmulti,
    but they commit to the first solution.
    Operationally, they succeed exactly once.
    \item[\dccnondet] procedures may have any number of solutions, like
    \dnondet,
    but they commit to the first solution.
    Operationally, they succeed at most once.
\end{description}

\noindent
Examples of determinism declarations can be seen above in the declarations for
\code{append/3} and \code{length/2}.
In practice, most parts of most Mercury programs are deterministic (\ddet).
Each procedure's mode declaration
typically includes its determinism (see the mode declarations for
\code{append/3} above).
If this is omitted, the compiler can infer the missing information.

% Switch detection.

\begin{figure}
\parbox{0.5\textwidth}{
$$
\begin{array}{ll}
(\\
& X~=~f_1(\ldots),~G_1 \\
; \\
& X~=~f_2(\ldots),~G_2 \\
; \\
& X~=~f_n(\ldots),~G_n \\
)
\end{array}
$$}%
\parbox{0.5\textwidth}{
$$
\begin{array}{ll}
\hbox{switch}~x~( \\
f_1: \\
& X~=~f_1(\ldots),~G_1 \\
; \\
f_2: \\
& X~=~f_2(\ldots),~G_2 \\
; \\
f_n: \\
& X~=~f_n(\ldots),~G_n \\
)
\end{array}
$$}
\caption{Switch detection example}
\label{fig:switch_detect}
\end{figure}

Before the compiler attempts to check or infer
the determinism of each procedure,
it runs a switch detection algorithm that looks for disjunctions
in which each disjunct unifies the same input variable
(a variable that is already bound when the disjunction is entered)
with any number of different function symbols.
Figure~\ref{fig:switch_detect} shows an example.
When the same variable is used with multiple levels of unification in a
disjunction switch detection will rely on
a form of common subexpression elimination
to generate nested switches.

The point of this is that it allows determinism analysis
to infer much tighter bounds on the number of solutions of the goal.
For example, if each of the $G_i$ is deterministic
(i.e. it has determinism \ddet)
and the various $f_i$ comprise all the function symbols in $X$'s type,
then the switch can be inferred to be deterministic as well,
whereas a disjunction that is \emph{not} a switch and produces at least one
variable
cannot be deterministic,
since each disjunct may generate a solution.

Disjunctions that are not converted into switches can be classified into two
separate types of disjunction.
If a disjunction produces one or more variables
then it may have more than one solution
(it will be \dnondet, \dmulti, \dccnondet or \dccmulti).
If a disjunction does not produce any variables
then it has at most one solution
(it will be either \dsemidet or \ddet).
Even when there are different execution paths that may succeed,
the extra execution paths do not affect the program's semantics.
In practice disjunctions that produce no values are \dsemidet
(if they where \ddet the programmer would have not written them)
therefore we call them \emph{semidet disjunctions}.
The compiler will generate different code for them,
since they never produce more than one solution.

Mercury has a module system.
Calls may be qualified by the name of the module
that defines the predicate or function being called,
with the module qualifier and the predicate or function name
separated by a dot.
The \io (Input/Output) module of the Mercury standard library
defines an abstract type called the \io state,
which represents the entire state of the world outside the program.
The \io module also defines a large set of predicates that do I/O.
These predicates all have determinism \ddet.
and besides other arguments,
they all take a pair of \io states
whose modes are respectively \di and \uo
(as discussed above, \di being shorthand for \emph{destructive input}
and \uo for \emph{unique output}).
The \code{main/2} predicate that represents the entire program
(like \code{main()} in C)
also has two arguments, a \code{di,uo} pair of \io states.
A program is thus given
a unique reference to the initial state of the world,
every I/O operation conceptually destroys the current state of the world
and returns a unique reference to the new state of the world,
and the program must return the final state of the world
as the output of \code{main/2}.
Thus, the program (\code{main/2}) defines a relationship between the state
of the world before the program was executed,
and the state of the world when the program terminates.
These conditions guarantee that at each point in the execution
there is exactly one current state of the world.

% Float this so that it does not straddle a page boundary
\begin{figure}
\begin{verbatim}
:- pred main(io::di, io::uo) is det.

main(S0, S) :-
    io.write_string("Hello ", S0, S1),
    io.write_string("world\n", S1, S).
\end{verbatim}
\caption{Hello world and I/O example}
\label{fig:hello}
\end{figure}

As an example, a version of ``Hello world'' is shown in
Figure~\ref{fig:hello}.
\code{S0} is the initial state of the world,
\code{S1} is the state of the world after printing ``\code{Hello }'',
and \code{S} is the state of the world
after printing ``\code{world\n}'' as well.
Note that the difference e.g.\ between \code{S1} and \code{S}
represents not just the printing of ``\code{world\n}'',
but also all the changes made to the state of the world
by \emph{other} processes since the creation of \code{S1}.
This threaded state sequentialises I/O operations:
``\code{world\n}'' cannot be printed until the value for \code{S1}
is available.

%\peter{This is not relevant to your work, unless you are intent on
%  including code with state variables in your discussion.  I think
%  it is a complication better left out.}
Numbering each version of the state of the world
(or any other state that a program may pass around) is cumbersome.
Mercury has syntactic sugar to avoid the need for this,
but this sugar does not affect
the compiler's internal representation of the program.
This syntactic sugar is shown below, along with the convention of naming
the I/O state \code{!IO}.
The compiler will transform the example below into the example in
Figure~\ref{fig:hello}.

\begin{verbatim}
:- pred main(io::di, io::uo) is det.

main(!IO) :-
    io.write_string("Hello ", !IO),
    io.write_string("world\n", !IO).
\end{verbatim}

\noindent
Any term beginning with a bang (\code{!})
will be expanded into a pair of automatically named
variables.
The variable names created sequence conjunctions from left to right in
conjunctions when used with pairs of \code{di/uo} or \code{in/out}
modes.
%\peter{End of stuff I think you should leave out.}

\label{page:purity}
Mercury divides goals into three purity categories:

\begin{description}

    \item[pure goals] have no side effects
    and their outputs do not depend on side effects;

    \item[semipure goals] have no side effects
    but their outputs may depend on other side effects;

    \item[impure goals] may have side effects, and may produce outputs
      that depend on other side effects.

\end{description}

\noindent
Semipure and impure predicates and functions
have to be declared as such,
and calls to them must be prefaced with either
\code{impure} or \code{semipure} (whichever is appropriate).
The vast majority of Mercury code is pure,
with impure and semipure code confined to a few places
where it is used to implement pure interfaces.
(For example, the implementations of the all-solutions predicates
in the Mercury standard library use impure and semipure code.)
Programmers put
\code{promise\_pure} or \code{promise\_semipure} wrappers around a goal
to promise to the compiler (and to human readers) that
the goal as a whole is pure or semipure respectively,
even though some of the code inside the goal may be less pure.

The compiler keeps a lot of information associated with each goal,
whether atomic or not.
This includes:

\begin{itemize}
\item
the set of variables bound (or \emph{produced}) by the goal;
\item
the determinism of the goal.
\item
the purity of the goal
\item
the \emph{nonlocal set} of the goal,
which means the set of variables
that occur both inside the goal and outside it; and
\end{itemize}

\noindent
The language reference manual \citep{mercury_refman} contains a complete
description of Mercury.

