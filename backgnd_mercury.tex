
\status{This section is complete.
}

Mercury is a pure logic/functional programming language
intended for the creation of large, fast, reliable programs.
While the syntax of Mercury is based on the syntax of Prolog,
semantically the two languages are very different
due to Mercury's purity and its type, mode, determinism and module systems.

Mercury programs consist of modules,
each has a seperate namespace and sperate compilation is used making it easy
to create large programs.
Each module contains predicates and functions.
Functions are a syntactic sugar for predicates with an extra (result)
argument.
The intention is that
predicates define relationships between their arguments and
functions define a mapping from their arguments to their result.
in the remainder of this dissertation we will use the word predicate to refer to
either a predicate or a function.

A predicate or function $P$ is defined in terms of a goal $G$:

%\vspace{-1\baselineskip}
% \begin{figure}[htb]
$$
\begin{array}{lll}
~P
    & :~ p(X_1, \ldots, X_n)~\leftarrow~G
        & \hbox{predicates} \\
    & |~ f(X_1, \ldots, X_n)=X_{n+1}~\leftarrow~G
        & \hbox{functions} \\
\end{array}
$$

\paul{Peter suggested using a more grammer-like description of the language.
I think that his motivation was that terminals and non-terminals should look
more distinct.
I've capitalized the variable names since that's closer to mercury syntax,
it's not the same as showing terminals and non-terminals differently but
perhaps it's clearer.}

\noindent
Atomic goals do not refer to other goals;
compisite goals are defined in terms of other goals.
Mercury's goal types are:

$$
\begin{array}{lll}
G
    & :~ X = Y ~|~ X = f(Y_1,~\ldots,~Y_n)
        & \hbox{unifications}\\
    & |~ p(X_1,~\ldots,~X_n)
        & \hbox{predicate calls} \\
    & |~ X_{n+1} = f(X_1,~\ldots,~X_n)
        & \hbox{function calls} \\
    & |~ X_0(X_1,~\ldots,~X_n)
        & \hbox{higher order calls} \\
    & |~ m(X_1,~\ldots,~X_n)
        & \hbox{method calls} \\
    & |~ \hbox{foreign}(p,
        [X_1:~Y_1,~\ldots,X_n:~Y_n],
        \hbox{\emph{foreign code}}),
        & \hbox{foreign code} \\
    & |~ (G_1,~\ldots,~G_n)
        & \hbox{sequential conjunctions}\\
    & |~ (G_1~\&~\ldots~\&~G_n)
        & \hbox{parallel conjunctions}\\
    & |~ (G_1 ; \ldots ; G_n)
        & \hbox{disjunctions}\\
    & |~ \hbox{switch}~X~(f_1:~G_1~;~\ldots~,f_n:~G_n)
        & \hbox{switches}\\
    & |~ (if~G_{cond}~then~G_{then}~else~G_{else})
        & \hbox{if-then-elses}\\
    & |~ not~G
        & \hbox{negations}\\
    & |~ some~[X_1,\ldots,X_n]~G
        & \hbox{existential quantification}\\
    & |~ all~[X_1,\ldots,X_n]~G
        & \hbox{universal quantification} \\
    & |~ promise\_pure~G
        & \hbox{purity promise}\\
    & |~ promise\_semipure~G
        & \hbox{purity promise}\\
\end{array}
$$
% \caption{The abstract syntax of Mercury}
% \label{fig:abstractsyntax}
% \end{figure}
%\vspace{-1mm}

\noindent
The atomic goals are unifications
(which the compiler breaks down until they contain
at most one function symbol each),
plain first-order calls,
higher-order calls,
method calls,
and calls to predicates defined by code in a foreign language (usually C).
\peter{You only need to cover enough about Mercury for the purposes of
  your thesis.  I don't think how inlining of foreign code works is important.}
\paul{We use it to implement primatives on which our transformations are
built,
I may remove it later if the detail of how this is done is unimportant.}
The compiler can inline foreign code definitions when generating code
for a Mercury predicate.
To allow this, the representation of a foreign code construct includes
not just the name of the predicate being called
but also the foreign code that is the predicate's definition
and the mapping from the Mercury variables that are the call's arguments
to the names of the variables that stand for them in the foreign code.
The composite goals include
sequential and parallel conjunctions,
disjunctions, if-then-elses, negations and existential quantifications.
A detailed description of parallel conjunctions can be found in Section
\ref{sec:back_mer_par}.
The abstract syntax does not include universal quantifications:
they are allowed at the source level,
but are transformed into combinations of negations and existential quantification:
$all~[X_1,~\ldots,~X_n]~G~\rightarrow~not~(~some~[X_1,~\ldots,~X_n]~(~not~G~))$.
Similarly,
a negation can be simplified using the constants $true$ and $false$ and an
if-then-else:
$not~G~\rightarrow~(if~G~then~false~else~true)$.
A switch is a disjunction in which
each disjunct unifies the same bound variable
with a different function symbol.
Switches in Mercury are thus analogous to switches in languages like C.
If there is a case in the switch for each function symbol in the
switched-on value's type, then the switch is said to be complete.
Otherwise the switch is incomplete.
That is to say,
there are function symbols in the variable's type that are not covered
by a case in the switch.
The purity promise will be described on page \pageref{page:purity}.

\paul{XXX: Need to describe semidet disjunctions somewhere}

Mercury has a strong Hindley-Milner~\citep{hindley69:types,milner78:types} type
system that was inspired by that of Hope~\citep{hope_types}
and similar to Haskell's~\citep{haskell98}.
Mercury programs are statically typed; the compiler knows the type of every
argument of every predicate (from declarations or inference) and every local
variable (from inference).
Types may be parametric,
type parameters are written in uppercase while
concrete types are written in lower case.
The following are references to types.
Type declrations are not covered here,
more information can be found in \citet{mercury_refman}.

\begin{description}

    \item[\code{int}] is a concrete type describing integers.

    \item[\code{T}] is an abstract type, \code{T} may be substituted for any other
    type.

    \item[\code{list(T)}] is a type describing a list, the list is defined
    in terms of some type \code{T}.
    In this case, \code{T} is used as the type of the elements in the list.

    \item[\code{list(int)}] is a concrete type describing a list of
    integers.

\end{description}

\noindent
The definition and type signature of \code{append/3} is:

\begin{verbatim}
:- pred append(list(T), list(T), list(T)).

append([], Ys, Ys).
append([X | Xs], Ys, [X | Zs]) :-
    append(Xs, Ys, Zs).
\end{verbatim}

\noindent
Mercury also has a strong mode system.
The mode of a predicate describes the \emph{instantiation state} changes,
if any, of the predicate's arguments.
A variable may be free, ground or clobbered.
Free variables currently have no value,
ground variables have a fixed value,
clobbered variables once had a value, but it is no longer available.
Partial instantiation can also occur,
however, the compiler does not fully support partial instantiaton
and therefore it is rarely used.
Ground and bound instantiation states may also be described as unique,
meaning they are not aliased with any other value.
When variables are used, their instantiation state may change;
such a change is described by a transition between two instantiation states,
or the same instantiation state if nothing changed.
The symbol \code{>>} denotes a state change.
This is also known as a mode (of an argument),
this is not the same as the mode of a predicate.
When a variable is used as an argument or in a unification it can only
become \emph{more instantiated};
that is to say `free to ground' is legal, but `ground to free' is not.
Similarly, uniqueness cannot be added to a value that is already
bound or ground.
Commonly used modes such as input (\code{in}), output (\code{out}),
destructive input (\code{di}) and unique output (\code{uo}) are
defined in the standard library:

\begin{verbatim}
:- mode in  == ground >> ground.
:- mode out == free >> ground.
:- mode di == unique >> clobbered.
:- mode uo == free >> unique.
\end{verbatim}

\noindent
There may be multiple modes for any given predicate.
Here are two mode declarations for the append predicate above;
these are usually written immediately after the type declaration.
When a predicate has only a single mode,
the mode declration can be combined with the type signature as a single
declration.

\begin{verbatim}
:- mode append(in, in, out) is det.
:- mode append(out, out, in) is multi.
\end{verbatim}

\noindent
The compiler enforces the instantiation state constraints on variables passed
as arguments:
Variables passed as input arguments must be ground before and after the call
while variables passed as output arguments must be free before the call and
ground after the call.
Similarly, a variable passed in a destructive input argument will be destroyed
before the end of the call;
it must be unique before the call and clobbered afterwords.
It is illegal to reference a clobbered variable.
A variable passed in a unique output argument must initially be free and
will become unique.
These two modes are often used together to allow the compiler to
destructively update data structures such as arrays.
The compiler will infer the instantiation states of all variables before and
after every goal in a predicate or function's body.

The Mercury compiler generates separate code
for each mode of a predicate or function,
which we call a \emph{procedure}.
In fact, individual procedures are handled as separate entities by
the compiler after mode checking.
This allows the mode checking pass of the compiler to minimally
reorder conjuncts (in both sequential and parallel conjunctions)
so that the producer of a variable (the code that makes the variable ground)
occurs before all consumers of that variable (any code that expects the
variable to be ground).
This means that for each variable in each procedure,
the compiler knows exactly where that variable becomes ground.

% Mode invariants
The mode system both enforces three invariants
that we need to refer to later in the dissertation:

\begin{description}

  \item[Conjunction invariant:]
  In any set of conjoined goals,
  which includes not just conjuncts in conjunctions
  but also the condition and then-part of an if-then-else,
  each variable that is consumed by any one of the goals
  is produced by exactly one earlier goal.
  This does not effect the else part of an if-then-else, it is not conjoined
  with the cond part.

  \item[Branched goal invariant:]
  In disjunctions, switches or if-then-elses,
  the goal types that contain alternative branches of execution,
  each branch of execution must produce
  the exact same set of variables
  that are consumed from outside the branched goal,
  with one exception:
  a branch of execution that cannot succeed (see determinisms below)
  may produce a subset of this set of variables.
  
  \item[Negated goal invariant:]
  A negated goal may not bind
  any variable that is visible to goals outside it,
  and the condition of an if-then-else may not bind a variable
  that is visible anywhere except in
  the then-part of that if-then-else.

\end{description}

\noindent
Each procedure and goal has a determinism,
which may put upper and lower bounds on the number of its possible solutions
(in the absence of infinite loops and exceptions).
Mercury's determinisms are:

\begin{description}
    \item[\ddet] procedures succeed exactly once
    (upper bound is one, lower bound is one);
    \item[\dsemidet] procedures succeed at most once
    (upper bound is one, no lower bound);
    \item[\dmulti] procedures succeed at least once
    (lower bound is one, no upper bound);
    \item[\dnondet] procedures may succeed any number of times
    (no bound of either kind);
    \item[\dfailure] procedures can never succeed
    (upper bound is zero, no lower bound);
    \item[\derroneous] procedures have an upper bound of zero and a lower
    bound of one, which means they can neither succeed nor fail,
    they 
    they must either throw an exception or loop forever.
\end{description}

\noindent
In practice, most parts of a Mercury program are deterministic (\ddet).
Each procedure's mode declaration
typically declares its determinism (see the mode declarations for append above).
If this is omitted, the compiler can infer the missing information.


% XXX: Committed choice, if I need it.
% There are two committed choice determinisms:
% \dccmulti is analogous to \dmulti and
% \dccnondet is analogous to \dnondet;
% except that the code will commit to the first solution.
% Code generation therefore handles these determinisms as if they are
% \ddet and \dsemidet respectively.

% Switch detection.

\begin{figure}
\parbox{0.5\textwidth}{
$$
\begin{array}{ll}
(\\
& x~=~f_1(\ldots),~G_1 \\
; \\
& x~=~f_2(\ldots),~G_2 \\
; \\
& x~=~f_n(\ldots),~G_n \\
)
\end{array}
$$}%
\parbox{0.5\textwidth}{
$$
\begin{array}{ll}
\hbox{switch}~x~( \\
f_1: \\
& x~=~f_1(\ldots),~G_1 \\
; \\
f_2: \\
& x~=~f_2(\ldots),~G_2 \\
; \\
f_n: \\
& x~=~f_n(\ldots),~G_n \\
)
\end{array}
$$}
\caption{Switch detection example}
\label{fig:switch_detect}
\end{figure}

Before the compiler attempts to check or infer
the determinism of each procedure,
it runs a switch detection algorithm that looks for disjunctions
in which each disjunct unifies the same input variable
(a variable that is already bound when the disjunction is entered)
with any number of different function symbols.
Figure \ref{fig:switch_detect} shows an example.
When the same variable is used with multiple levels of unification in a
disjunction switch detection will rely on
common subexpression elimination (CSE)
to generate nested switches.

The point of this is that it allows determinism analysis
to infer much tighter bounds on the number of solutions of the goal.
For example, if each of the $G_i$ is deterministic
(i.e. it has determinism \code{det})
and the various $f_i$ comprise all the function symbols in $x$'s type,
then the switch can be inferred to be deterministic as well,
whereas a disjunction that is \emph{not} a switch and produces at least one
variable
cannot be deterministic,
since each disjunct may generate a solution.

Mercury has a module system.
Calls may be qualified by the name of the module
that defines the predicate or function being called,
with the module qualifier and the predicate or function name
separated by a dot.
The \io (Input/Output) module of the Mercury standard library
defines an abstract type called the \io state,
which represents the entire state of the world outside the program.
The \io module also defines a large set of predicates that perform I/O.
These predicates all have determinism \code{det}.
and besides other arguments,
they all take a pair of \io states
whose modes are respectively \di and \uo,
(As discussed above, \di being shorthand for \emph{destructive input}
and \uo for \emph{unique output}.)
The \code{main/2} predicate that represents the entire program
(like \code{main()} in C)
also has two arguments, a \code{di,uo} pair of \io states.
A program is thus given
a unique reference to the initial state of the world,
every I/O operation conceptually destroys the current state of the world
and returns a unique reference to the new state of the world,
and the program must return the final state of the world
as the output of \code{main/2}.
Thus, the program (\code{main/2}) defines a relationship between the state
of the world before the program was executed,
and the state of the world when the program terminates.
These conditions guarantee that at each point in the execution,
there is exactly one current state of the world.

As an example, here is one version of ``Hello world'':

\begin{verbatim}
:- pred main(io::di, io::uo) is det.

main(S0, S) :-
    io.write_string("Hello ", S0, S1),
    io.write_string("world\n", S1, S).
\end{verbatim}

\noindent
% In this version,
\code{S0} is the initial state of the world,
\code{S1} is the state of the world after printing ``\code{Hello }'',
and \code{S} is the state of the world
after printing ``\code{world\n}'' as well.
Note that the difference e.g.\ between \code{S1} and \code{S}
represents not just the printing of ``\code{world\n}'',
but also all the changes made in the state of the world
by \emph{other} processes since the creation of \code{S1}.
This threaded state sequentialises I/O operations:
``\code{world\n}'' cannot be printed until the value for \code{S1}
is available.

%\peter{This isn't relevant to your work, unless you're intent on
%  including code with state variables in your discussion.  I think
%  it's a complication better left out.}
Numbering each version of the state of the world
(or any other state that a program may pass around) is cumbersome,
Mercury has syntactic sugar to avoid the need for this,
but this sugar does not affect
the compiler's internal representation of the program.
This syntactic sugar is shown below, along with the convention of naming
the I/O state \code{!IO}.
The example below will be transformed into the example above during
compilation.

\begin{verbatim}
:- pred main(io::di, io::uo) is det.

main(!IO) :-
    io.write_string("Hello ", !IO),
    io.write_string("world\n", !IO).
\end{verbatim}

\noindent
Any term beginning with a bang (\code{!})
will be expanded into pairs of automatically named
variables.
The variable names created sequence conjunctions from left to right in
conjunctions when used with pairs of \code{di/uo} or \code{in/out}
modes.
%\peter{End of stuff I think you should leave out.}

\label{page:purity}
Mercury divides goals into three purity categories:

\begin{description}

    \item[pure goals] have no side effects
    and their outputs do not depend on side effects;

    \item[semipure goals] have no side effects
    but their outputs may depend on other side effects;

    \item[impure goals] may have side effects, and may produce outputs
      that depend on other side effects.

\end{description}

\noindent
Semipure and impure predicates and functions
have to be declared as such,
and calls to them must be prefaced with either
\code{impure} or \code{semipure} (whichever is appropriate).
The vast majority of Mercury code is pure,
with impure and semipure code confined to a few places
where it is used to implement pure interfaces.
(For example, the implementation of the all-solutions predicates
in the Mercury standard library use impure and semipure code.)
Programmers can wrap
\code{promise\_pure} or \code{promise\_semipure} goals around a goal
to promise to the compiler (and to human readers) that
the goal as a whole is pure or semipure respectively,
even though some of the code inside the goal may be less pure.

The compiler keeps a lot of information associated with each goal,
whether atomic or not.
This includes:

\begin{itemize}
\item
the set of variables bound (or \emph{produced}) by the goal;
\item
the purity of the goal
\item
the \emph{nonlocal set} of the goal,
which means the set of variables
that occur both inside the goal and outside it; and
\item
the determinism of the goal.
\end{itemize}

\noindent
A complete description of Mercury
can be found in the language reference manual \citep{mercury_refman}.

