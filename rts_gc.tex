
\status{Most of the prose is done, I want to run another benchmark
and see how the data looks before I continue}
One of the sources of poor parallel performance is the behaviour of the
garbage collector.
Like some declarative languages,
Mercury does not allow destructive update, also known as mutation.%
%\footnote{
%    Destructive update is allowed via support for mutable variables.
%    Their use does not interfere with parallelism as they are used in
%    conjunction with impurity.
%    The compiler will not parallelise impure goals.}
Therefore a calculation usually returns its value in newly allocated memory
rather than modifying the memory of one of its parameters.
This means that Mercury programs often have a high rate of allocation,
which places a lot of stress on the garbage collector.
Therefore,
allocation and garbage collection have a strong impact on a program's
performance.
This effect usually becomes more significant when parallelism is introduced into a
program.

\plan{Introduce Boehm, \& details: conservative, mark and sweep, stop the
world, parallel marking.}
Mercury uses the Boehm-Demers-Weiser Conservative Garbage Collector (Boehm GC)
\citep{boehm_gc},
which is a conservative mark and sweep collector.
Boehm GC supports parallel programming.
It will stop all the threads (\emph{stop the world}) during its marking
phase.
It also supports parallel marking,
it will use its own set of pthreads to perform parallel marking.

\plan{Introduce collector time, mutator time.}
When discussing garbage collection,
a program's execution time is divided into two alternating phases:
collection time, which is when Boehm GC performs marking,
and mutator time, which is when the Mercury program runs.
The name mutator time refers to time that mutations (changes) to memory
structures are permitted.
The collector may also perform some actions,
such as sweeping,
concurrently with the mutator.

\plan{Describe theory of GC performance.}
Amdahl's law~\citep{amdahl} describes the maximum speedup that
can theoretically  be achieved by parallelising some part of a program,
and allowing the rest of the program to run sequentially.
When sequential marking is used then can use Amdahl's
law to predict the speedup of our programs.
For example, consider a program with a runtime of 20 seconds
which can be separated into one second of collector time and 19 seconds
of mutator time.
The sequential execution time of the program is $1 + 19 = 20$.
If we parallelise mutator part of the program and do not parallelise the
collector then the minimum parallel execution time is $1 + 19{\div}P$
for $P$ processors.
Using four processors the theoretical best speedup we can achieve is:
$(1 + 19) \div (1 + 19\div4) = 3.48$.
17\% of the parallel runtime ($1 + 19\div4$) is collector time.
If we use a machine with 100 processors then this becomes:
$(1 + 19) \div (1 + 19\div100) = 16.8$;
with 84\% of the runtime spent in the collector.
Adding more processors to the system means that
As the number of processors increases a larger proportion of time is
wasted waiting for collection to complete.

\plan{Discuss predictions regarding parallel marking, locking and thread
local heaps.}
In order to reduce this problem,
\citet{boehm-gc} included parallel marking support in their collector.
Parallel garbage collection is a continuing area of research
and the Boehm GC project has only modest goals for multicore scalability.
Therefore,
parallel marking should partially remove the bottleneck described above.
Thread safety has two other significant performance costs:
The first is that
one less CPU register is available to GCC's code generator when compiling
parallel Mercury programs (Section \ref{sec:backgnd_merpar}).
The second reason is that
memory allocation routines must use locking to protect shared data
structures,
this will slow down allocation.
Boehm GC's authors recognised this problem and
added support for thread-local resources such as free lists.
Therefore,
during memory allocation a thread can use its own free lists rather than
locking a global structure.
From time to time, a thread will have to retrieve new free lists
from a global structure and will need to perform locking then,
but this cost will be amortised across a number of memory allocations.

\plan{Describe icfp2000, mandelbrot\_lowalloc and mandelbrot\_highalloc
programs.}
One of our benchmarks is a raytracer developed for the
ICFP programming contest in 2000.
For each pixel in the image the raytracer casts a ray into the scene to
determine what colour to paint that pixel.
Two nested loops build the pixels for the image:
the outer loop iterates over the rows in the image and
the inner loop iterates over the pixels in each row.
We manually parallelised the program by introducing an independent parallel
conjunction into the outer loop;
indicating that rows can be drawn in parallel.
The raytracer uses many small structures to represent vectors and a pixel's
colour.
It is therefore memory allocation intensive.
Since we suspected that garbage collection was having a negative impact on
performance we developed a mandelbrot image generator.
The mandelbrot image generator has a similar structure to the raytracer,
it draws an image using two nested loops as above.
The mandelbrot image is drawn on the complex number plane.
A complex number $C$ is in the mandelbrot set if
$|N|$ is never larger than 2 where $N_{i+1} = N_{i}^2 + C$ and $N_0 = 0$.
For each pixel in the image the program tests if the pixel's coordinates are
in the set,
The pixel's colour is chosen based on how many iterations are needed
before $|N| > 2$,
or black if the test survived 5,000 iterations.
We parallelised this program the same way as we did the raytracer:
by introducing an independent parallel conjunction into the outer loop.
We created two versions of this program,
the first version stores coordinates and complex numbers on the stack and in
registers.
Therefore, this version has a low rate of memory allocation.
We call this version `mandelbrot\_lowalloc'.
The second version of the mandelbrot program is called
`mandelbrot\_highalloc'.
This program represents its coordinates and complex numbers as structures on
the heap.
Therefore,
it has a higher rate of memory allocation.


\begin{table}
\begin{center}
\begin{tabular}{r|rr|rrrr}
\Cbr{GC Markers} &
\multicolumn{2}{c|}{Sequential} &
\multicolumn{4}{c}{Parallel w/ $N$ Mercury Engines} \\
\Cbr{} &
\C{no TS} & \Cbr{TS} &
\C{1} & \C{2} & \C{3} & \C{4} \\
\hline
\hline
\multicolumn{7}{c}{raytracer} \\
\hline
1 & 33.1 (1.13) & 37.4 (1.00) &
             37.7 (0.99) & 28.0 (1.34) & 24.8 (1.51) & 23.7 (1.58) \\
2  & - & - & 32.0 (1.17) & 21.6 (1.73) & 18.1 (2.07) & 16.7 (2.24) \\
3  & - & - & 29.6 (1.26) & 19.7 (1.90) & 16.3 (2.29) & 14.5 (2.59) \\
4  & - & - & 29.1 (1.29) & 18.9 (1.98) & 15.3 (2.45) & 13.7 (2.73) \\
\hline
\hline
\multicolumn{7}{c}{mandelbrot\_lowalloc} \\
\hline
1 & 15.4 (0.99) & 15.2 (1.00) &
             15.2 (1.00) &  5.1 (2.96) &  7.7 (1.98) &  3.9 (3.92) \\
2  & - & - & 15.3 (1.00) &  7.7 (1.98) &  5.1 (2.97) &  3.9 (3.94) \\
3  & - & - & 14.3 (1.00) &  7.7 (1.98) &  5.1 (2.97) &  3.9 (3.94) \\
4  & - & - & 15.3 (0.99) &  7.7 (1.98) &  5.1 (2.97) &  3.9 (3.92) \\ 
\hline
\hline
\multicolumn{7}{c}{mandelbrot\_highalloc} \\
\hline
1 & 41.0 (1.23) & 50.5 (1.00) &
             50.6 (1.00) & 33.4 (1.51) & 26.6 (1.90) & 23.4 (2.16) \\
2  & - & - & 46.7 (1.08) & 28.6 (1.77) & 22.1 (2.28) & 18.8 (2.69) \\
3  & - & - & 46.3 (1.09) & 27.3 (1.85) & 20.9 (2.42) & 17.3 (2.93) \\
4  & - & - & 45.6 (1.11) & 26.8 (1.89) & 20.4 (2.48) & 16.7 (3.03) \\
\end{tabular}
\end{center}
\paul{Consider drawing charts for raytracer and mandelbrot so that releative
performance between different scores can be seen easily.}
\caption{Parallelism and garbage collection}
\label{tab:gc}
\end{table}



\plan{Benchmark data.}
Table \ref{tab:gc} shows benchmark results for these programs
with different numbers of Mercury engines and garbage collector threads.
All benchmarks have an initial heap size of 16MB.
Each result is measured in seconds and represents the mean of eight samples.
The first row is the number of garbage collector threads used.
The second and third rows give sequential execution times
without and with thread safety enabled.
In these rows the programs where executed in such a way so that they did not
execute a parallel conjunction.
The remaining four rows give parallel execution times using one to four
Mercury engines.
The numbers in parentheses show the relative speedup when compared to the
sequential thread safe result for the same program.
\label{cabsav}
We performed our benchmarking on `cabsav',
a hand built system using a Intel i7-2600K CPU
with 16GB of memory.
Cabsav runs Debian/GNU Linux 6
with a 2.6.32-5-amd64 Linux kernel and GCC 4.4.5-8.
The garbage collection benchmarks were collected using a recent version of
Mercury (rotd-2012-04-29);
the other improvements discussed in this chapter and the loop control
transformation (Chapter \ref{chap:loop_control})
are enabled for these tests.
Using the improved version of Mercury allows us to observe the effects of
garbage collection without other performance problems affecting the results.

\plan{Describe performance in practice.}
The raytracer program benefits from parallelism in both Mercury and the
garbage collector.
Using four engines speeds the program up by a factor of 1.58,
compared to 1.29 for four marker threads.
When using four engines and four marker threads the raytracer
achieves a speedup of 2.73,
this likely indicates that the program's high allocation rate causes
contention on the garbage collector's locks,
although we investigate another likely cause below.
mandelbrot\_lowalloc does not see any benefit from parallel marking.
It achieves very good speedups from multiple Mercury engines.
We know that this program has a low allocation rate
but is otherwise parallelised in the same way as raytracer.
Therefore,
these results support the hypothesis that garbage collection makes it more
difficult to achieve good speedups when parallelising programs.
mandelbrot\_highalloc, which stores its data on the heap,
sees similar trends in performance as raytracer.
Also, it is universally slower than mandelbrot\_lowalloc.
mandelbrot\_highalloc achieves a speedup of 2.16 when using four Mercury
engines compared to 1.58 for the raytracer.
It also achieves a speedup of 1.11 when using four marker threads compared
to 1.29 for the raytracer.


\begin{table}
\begin{center}
\begin{tabular}{r|l| d{1} r | d{1} r | d{1} r | d{1} r }
\C{GC} & \C{Times \&} &
\multicolumn{8}{c}{Parallel w/ $N$ Mercury Engines} \\
\C{Markers} & \C{Collections}
          & \Ctwo{1}        & \Ctwo{2}        & \Ctwo{3}        & \Ctwo{4} \\
\hline
\hline
\multicolumn{10}{c}{raytracer} \\
\hline
\multirow{4}{*}{1} &
  Elapsed & 37.8 &        & 28.1 &        & 24.7 &        & 22.7 & \\
& GC      & 16.9 & 44.8\% & 16.7 & 59.4\% & 16.5 & 66.6\% & 15.8 & 69.5\% \\
& Mutator & 20.9 & 55.2\% & 11.4 & 40.6\% &  8.2 & 33.4\% &  6.9 & 30.5\% \\
& \# col. & 384. &        &346.  &        &316.  &        &288.  & \\
\hline
\multirow{4}{*}{2} &
  Elapsed & 32.6 &        & 21.8 &        & 18.3 &        & 16.0 & \\
& GC      & 11.0 & 33.8\% & 10.4 & 47.7\% & 10.3 & 56.4\% &  9.8 & 60.8\% \\
& Mutator & 21.6 & 66.2\% & 11.4 & 52.3\& &  8.0 & 43.6\% &  6.3 & 39.2\% \\
& \# col. & 384. &        &346.  &        &316.  &        &288.  & \\
\hline
\multirow{4}{*}{3} &
  Elapsed & 30.9 &        & 20.0 &        & 16.4 &        & 14.1 & \\
& GC      &  9.3 & 30.2\% &  8.5 & 42.3\% &  8.2 & 50.0\% &  7.8 & 55.4\% \\
& Mutator & 21.6 & 69.8\% & 11.5 & 57.7\% &  8.2 & 50.0\% &  6.3 & 44.6\% \\
& \# col. & 386. &        &346.  &        &316.  &        &288.  & \\
\hline
\multirow{4}{*}{4} &
  Elapsed & 30.3 &        & 19.3 &        & 15.5 &        & 13.5 & \\
& GC      &  8.7 & 28.8\% &  7.8 & 40.4\% &  7.4 & 47.8\% &  7.1 & 52.4\% \\
& Mutator & 21.6 & 71.2\% & 11.5 & 59.6\% &  8.1 & 52.2\% &  6.4 & 47.6\% \\
& \# col. & 387. &        &346.  &        &316.  &        &288.  & \\
\hline
\hline
\multicolumn{10}{c}{mandelbrot\_lowalloc} \\
\hline
\multirow{4}{*}{1} &
  Elapsed & 15.2 &        &  7.7 &        &  5.1 &        &  3.9 & \\
& GC      &  0.0 &  0.1\% &  0.0 &  0.2\% &  0.0 &  0.7\% &  0.0 &  0.4\% \\
& Mutator & 15.2 & 99.9\% &  7.6 & 99.8\% &  5.1 & 99.3\% &  3.9 & 99.6\% \\
& \# col. &  2.  &        &  2.  &        &  2.  &        &  2.  & \\
\hline
\multirow{4}{*}{2} &
  Elapsed & 15.4 &        &  7.6 &        &  5.1 &        &  3.9 & \\
& GC      &  0.0 &  0.1\% &  0.0 &  0.1\% &  0.0 &  0.2\% &  0.0 &  0.2\% \\
& Mutator & 15.4 & 99.9\% &  7.6 & 99.9\% &  5.1 & 99.8\% &  3.9 & 99.8\% \\
& \# col. &  2.  &        &  2.  &        &  2.  &        &  2.  & \\
\hline
\multirow{4}{*}{3} &
  Elapsed & 15.3 &        &  7.7 &        &  5.1 &        &  3.9 & \\
& GC      &  0.0 &  0.1\% &  0.0 &  0.1\% &  0.0 &  0.4\% &  0.0 &  0.1\% \\
& Mutator & 15.2 & 99.9\% &  7.7 & 99.9\% &  5.1 & 99.6\% &  3.9 & 99.9\% \\
& \# col  &  2.  &        &  2.  &        &  2.  &        &  2.  & \\
\hline
\multirow{4}{*}{4} &
  Elapsed & 15.2 &        &  7.6 &        &  5.1 &        &  3.9 & \\
& GC      &  0.0 &  0.1\% &  0.0 &  0.1\% &  0.0 &  0.4\% &  0.0 &  0.1\% \\
& Mutator & 15.2 & 99.9\% &  7.6 & 99.9\% &  5.1 & 99.6\% &  3.9 & 99.9\% \\
& \# col  &  2.  &        &  2.  &        &  2.  &        &  2.  & \\
\hline
\hline
\multicolumn{10}{c}{mandelbrot\_highalloc} \\
\hline
\multirow{4}{*}{1} &
  Elapsed & 51.7 &        & 34.0 &        & 27.2 &        & 24.0 & \\
& GC      & 11.0 & 21.4\% & 11.5 & 33.6\% & 11.5 & 42.2\% & 11.8 & 49.1\% \\
& Mutator & 40.7 & 78.6\% & 22.6 & 55.4\% & 15.5 & 57.8\% & 12.2 & 50.9\& \\
& \# col. &742.  &        &692.  &        &634.  &        &602. & \\
\hline
\multirow{4}{*}{2} &
  Elapsed & 48.5 &        & 29.3 &        & 22.3 &        & 19.1 & \\
& GC      &  6.4 & 13.1\% &  6.7 & 22.9\% &  6.6 & 29.7\% &  6.9 & 36.1 \\
& Mutator & 42.1 & 86.9\% & 22.6 & 77.1\% & 15.7 & 70.3\% & 12.2 & 63.9 \\
& \# col. &744.  &        &693.  &        &633.  &        &595.  & \\
\hline
\multirow{4}{*}{3} &
  Elapsed & 46.4 &        & 28.0 &        & 21.0 &        & 17.7 & \\
& GC      &  5.0 & 10.8\% &  5.3 & 18.8\% &  5.3 & 25.0\% &  5.4 & 30.4\% \\
& Mutator & 41.4 & 89.2\% & 22.8 & 81.2\% & 15.7 & 75.0\% & 12.3 & 69.6\% \\
& \# col. &737.  &        &692.  &        &629.  &        &600.  & \\
\hline
\multirow{4}{*}{4} &
  Elapsed & 46.0 &        & 27.6 &        & 20.3 &        & 16.9 & \\
& GC      &  4.5 &  9.9\% &  4.7 & 17.0\% &  4.6 & 22.6\% &  4.7 & 27.6\% \\
& Mutator & 41.4 & 90.1\% & 22.9 & 83.0\% & 15.7 & 77.4\% & 12.3 & 72.4\% \\
& \# col. &740.  &        &695.  &        &626.  &        &600.  & \\
\end{tabular}
\end{center}
\caption{Percentage of elapsed execution time used by GC/Mutator}
\label{tab:gc_amdahl}
\end{table}



\plan{Introduce the table we use for our discussion of Amdahl's law.}
We can see that mandelbrot\_highalloc benefits from parallelism in Mercury
more than raytracer does,
similarly mandelbrot\_highalloc benefits less from parallelism in the garbage
collector than raytracer does.
Using \tscope (Chapter \ref{chap:tscope}),
we analysed how much time these programs spend running the garbage collector
or the the mutator.
This results are shown in Table \ref{tab:gc_amdahl}.
The table shows the elapsed time, collector time, mutator time and the
number of collections for each program using 1--4 engines, and 1--4
collector threads.
The times are averages taken from eight samples,
using an initial heap size of 16MB.
In order to use \tscope we had to compile the runtime system differently.
Therefore,
these results differ slightly from those in Table \ref{tab:gc}.
Next to both the collector time and mutator time
we show the percentage of elapsed time taken by either collector or mutator
time respectively.

\plan{Describe trends in either the mutator or GC time.}
As expected, mandelbrot\_lowalloc spends very little time running the collector.
Typically, it ran the collector only twice during its execution,
total collector time was usually between 5 and 30 milliseconds.
The other two programs ran the collector hundreds of times.
As we increased the number of Mercury engines
we noticed that these programs made fewer collections.
As we varied the number of collector threads there was no trend in the
number of collections,
any apparent variation is most likely noise in the data.
All three programs see a speedup in the time spent in the mutator as the
number of Mercury engines is increased.
Similarly,
the raytracer and mandelbrot\_highalloc benefit from speedups
in GC time as the number of GC threads is increased.
In mandelbrot\_highalloc,
the GC time also increases slightly as Mercury engines are added.
We expect that as more Mercury engines are used the garbage collector
must use more inter-core communication which will have an overhead.
However, in the raytracer,
the GC time decreases slightly as Mercury engines are added.
We cannot guess why without understanding the collector in detail.

\plan{Compare performance with Amdahl's predictions.}
As Amdahl's law predicts,
parallelism in one part of the program has a limited effect on the program
as a whole.
When using one GC thread
and increasing the number of Mercury engines,
the raytracer's mutator time speeds up by 1.83, 2.55 and 3.03 times for
two, three and four Mercury engines.
However, the elapsed time speeds up by only: 1.35, 1.53 and 1.66 times
respectively.
While there is little change in the time spent in the collector,
there is an increase in collector time as a percentage of elapsed time.
Similarly, with one GC thread mandelbrot\_highalloc's mutator time speeds up
by 1.80, 2.63 and 3.34.
However, its elapsed time speeds up by only: 1.52, 1.90 and 2.15 times
respectively.
The elapsed time speedups are less impressive than the mutator time
speedups.
Similarly, adding parallelism in the garbage collector,
creating speedups in collector time has little benefit to elapsed time.

The results in Table \ref{tab:gc_amdahl} show that
mandelbrot\_highalloc spends less of its time (by percentage) in garbage
collection than raytracer.
This matches the results in \ref{tab:gc},
whereby mandelbrot\_highalloc has better speedups due to parallelism in
Mercury than raytracer does.
Likewise,
raytracer has better speedups due to parallelism in the collector
than mandelbrot\_highalloc does.

\begin{table}
\begin{center}
\begin{tabular}{l|rr|d{3}}
\Cbr{Program} & \C{Allocations} & \Cbr{Total alloc'd bytes} & \C{Alloc rate (M alloc/sec)} \\
\hline
raytracer   &     561,431,515 &           9,972,697,312 & 26.9 \\
mandelbrot\_lowalloc
            &       1,620,928 &              21,598,088 &  0.106 \\
mandelbrot\_highalloc
            &   3,829,971,662 &          29,275,209,824 & 94.1 \\
\end{tabular}
\end{center}
\caption{Memory allocation rates}
\label{tab:mem_alloc_rate}
\end{table}

\plan{Allocation intensity}
Because mandelbrot\_highalloc spends less of its time collecting than
raytracer,
we initially thought that mandelbrot\_highalloc is less allocation intensive
than raytracer,
but this is not true.
Using Mercury's deep profiler we measured the number and total size of memory
allocations.
This is shown in Table \ref{tab:mem_alloc_rate}.
The rightmost column in this table gives the allocation rate
measured in million allocations per second.
It is calculated by dividing the number of allocations in the second column
by the average mutator time reported in Table \ref{tab:gc_amdahl}.
Therefore, it represents the allocation rate during mutation time.
This is deliberate because, by definition there is no allocation during
collector time.
mandelbrot\_highalloc has a higher allocation rate than raytracer.
However,
it spends less time doing collection, when measured either absolutely
or by percentage of elapsed time.
Garbage collectors are tuned for particular workloads.
It is likely that Boehm GC handles mandelbrot\_highalloc's workload more
easily than raytracer's workload.

% Old idea:
%This may be due to changing allocation rates throughout the programs'
%executions.
%We noticed that while benchmarking the raytracer that if we rendered the
%image upside-down (by iterating over the rows backwards)
%we would get different performance results.
%This is because the top of the image we used contains sky,
%which has no geometry;
%it is the flat colour of the background,
%and the bottom of the image contains ground;
%which has more detail.
%Rendering the ground will naturally be more complicated and require more
%memory allocation than rendering the sky.
%Memory allocation in one part of a program can affect the performance of
%allocation and collection in another part of the program.
%This is many because the garbage collector will increase the size of the heap
%to satisfy large and/or many allocations.
%This is why rendering the image upside down had a different execution time to
%rendering it the right way up.
%Similarly,
%the mandelbrot program has sections in its image that require more iterations
%of the equation than others.


\begin{table}
\begin{center}
\begin{tabular}{r|rr|rrrr}
\Cbr{Initial} &
\multicolumn{2}{c|}{Sequential} &
\multicolumn{4}{c}{Parallel w/ $N$ Mercury Engines} \\
\Cbr{heap size} & \C{no TS}   & \Cbr{TS}    & \C{1}       & \C{2}       & \C{3}       & \C{4} \\
\hline
\hline
\multicolumn{7}{c}{raytracer} \\
\hline
1MB     & 32.8 (0.90) & 29.3 (1.01) & 29.3 (1.01) & 19.0 (1.57) & 15.4 (1.93) & 13.6 (2.19) \\
16MB    & 33.5 (0.89) & 29.7 (1.00) & 29.1 (1.02) & 18.9 (1.57) & 15.3 (1.94) & 13.7 (2.17) \\
32MB    & 34.3 (0.87) & 29.5 (1.01) & 29.5 (1.01) & 19.5 (1.52) & 15.6 (1.90) & 13.6 (2.18) \\
64MB    & 33.2 (0.89) & 30.4 (0.97) & 30.7 (0.97) & 20.3 (1.46) & 16.4 (1.81) & 14.5 (2.05) \\
128MB   & 22.7 (1.31) & 24.5 (1.21) & 24.2 (1.23) & 15.0 (1.97) & 11.7 (2.53) & 10.2 (2.91) \\
256MB   & 18.8 (1.57) & 21.4 (1.39) & 21.5 (1.38) & 12.6 (2.36) &  9.3 (3.18) &  7.7 (3.83) \\
384MB   & 17.7 (1.67) & 20.8 (1.43) & 20.7 (1.44) & 11.7 (2.53) &  8.7 (3.41) &  7.2 (4.14) \\
512MB   & 17.3 (1.71) & 20.3 (1.46) & 20.5 (1.45) & 11.5 (2.58) &  8.3 (3.56) &  6.8 (4.36) \\
\hline
\hline
\multicolumn{7}{c}{mandelbrot\_highalloc} \\
\hline
1MB     & 41.4 (1.10) & 45.5 (1.00) & 45.1 (1.00) & 26.7 (1.69) & 20.3 (2.23) & 16.7 (2.71) \\ 
16MB    & 39.7 (1.14) & 45.3 (1.00) & 45.6 (0.99) & 26.8 (1.69) & 20.4 (2.23) & 16.7 (2.72) \\
32MB    & 38.9 (1.16) & 45.2 (1.00) & 44.2 (1.02) & 26.3 (1.72) & 19.7 (2.29) & 16.5 (2.75) \\
64MB    & 36.7 (1.23) & 44.4 (1.02) & 43.7 (1.04) & 25.2 (1.80) & 18.8 (2.41) & 15.6 (2.90) \\
128MB   & 34.1 (1.33) & 42.6 (1.06) & 41.8 (1.08) & 24.1 (1.88) & 17.7 (2.56) & 14.4 (3.15) \\
256MB   & 33.2 (1.36) & 41.3 (1.10) & 41.9 (1.08) & 23.7 (1.91) & 16.9 (2.69) & 13.6 (3.34) \\ 
384MB   & 31.7 (1.43) & 41.0 (1.10) & 41.9 (1.08) & 23.3 (1.95) & 16.7 (2.71) & 13.4 (3.39) \\
512MB   & 31.1 (1.46) & 41.1 (1.10) & 41.1 (1.10) & 23.0 (1.97) & 16.7 (2.71) & 13.3 (3.40) \\ 
\end{tabular}
\end{center}
\caption{Varying the initial heapsize in parallel Mercury programs.}
\label{tab:heapsize}
\end{table}



\plan{New data, vary the heap size,}
So far,
we have shown introducing parallelism using Mercury's parallel conjunction
will have a limited impact on the overall speedup of the program.
This can be reduced by using the parallel marking feature in the garbage
collector.
We can reduce the impact of the garbage collector by varying the initial
heapsize of the program.
Table \ref{tab:gc_heapsize} shows the performance of the raytracer and
mandelbrot\_highalloc programs with various initial heap sizes.
The first column shows the heapsize uesd,
we picked a number of sizes from 1MB to 512MB.
The remaining columns show the average elapsed times in seconds.
The first of these columns shows timing for the programs compiled for
sequential execution without thread safety;
this means that the collector can not use parallel marking.
The next column is for sequential execution with thread safety;
the collector uses parallel marking with four threads.
The next four columns give the results for the programs compiled for
parallel execution, and executed with one to four Mercury engines.
These results also use four marker threads.
The numbers in parenthases describe the ratio of performance compared to the
16MB, sequential thread-safe result.
All the results are the averages of eight samples.
We ran these tests on raytracer and mandelbrot\_highalloc.
mandelbrot\_lowalloc was not used as its collection time is insagnificant.

Generally, the larger the inital heap size the better the programs
performend.
The exception is that the raytracer often ran more slowly with a heap size of 64MB
than a heapsize of 32MB.
%\paul{Does the reader know what a TLB is?  People are generally poor at
%remembering to care about normal caches much less TLBs.}
64MB is much larger than the processor's cache size (L3 cache is 8MB) and
covers more page mappings than its TLBs can cache (L2 TLB covers 2MB when
using 4KB pages).
The collector's structures and access patterns may be slower at this size
due to hardware limitiations,
and the benifits of a larger heap are not enough to overcome these
limitiations.
As the heap size gets larger performance begins to climb,
therefore in general larger heaps allow the collector to perform better.

\paul{XXX: Working here}


\plan{Criticise cache behaviour WRT parallel marking,
Thanks Simon Marlow.}
\paul{XXX: threadscope will also support our argument here.}

\plan{Discuss local heaps for threads, and their reliability problems.}
\paul{Stress this point, convince the reader that I can not test this}
Boehm GC has a few tunable parameters, one of these, \texttt{HBLKSIZE},
configures the size of a local heap among other things.
In some tests this did improve performance but we also found it to cause
reliability problems.
Therefore it cannot be used practically and we do not report on it here.

