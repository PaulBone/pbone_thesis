
%\paul{
%This should just 'set the stage' for my work.  I would
%like to leave the contrasting discussions until later in the
%thesis, perhaps at the end of each major chapter.
%}
%
%\paul{
%I don't want to introduce all my citations here (the way Liz has to
%due to the APA-formatting standard).  Only enough that the user
%understand the context for reading the research chapters.  So,
%closely related work should be introduced for the first time when
%it is compared with my work.
%}

% Parallelism and concurrency.
It is important to distinguish between parallelism and concurrency.
Concurrency is defined as multiple computations executing simultaneously,
and may communicate with one another.
On a single processor concurrency can be achieved by switching between
alternative processes quickly.
Parallelism occurs when multiple computations \textbf{actually} execute in parallel,
speeding up a computation.
Often parallelism is achieved through concurrency,
for example, multiple concurrent processes execution on a multicore processor,
this leads to a confusion between concurrency and parallelism.
Concurrency can be extremely useful when it matches the programmer's intentions.
For example,
when programming a web server that may service multiple clients at once;
it is natural to model each client's connection as a separate concurrent process.
Not all programs can be easily modeled with concurrent tasks,
in these cases it is best to introduce parallelism without concurrency.
Therefore, distinguishing these concepts is important.

\subsection{Concurrency}
\label{sec:backgnd_concurrency}

% Threading.
There are numerous imperative languages that support parallelism through
concurrency.
Those that do not support concurrency nativly can add support
with the use of a library.
For example, 
C and C++ use the POSIX Threads~\cite{butenhof1997:pthreads} or Windows
Threads~\cite{winthreads} libraries that are provided with operating systems.
These libraries tend to be low-level;
they simply provide an interface to the operating system.
These examples use the \emph{threading} model of concurrency:
Different threads of execution run independently,
sharing a heap and static data.
Threads may read and write the same data,
however,
concurrent read and write, and write and write operations are unsafe.
Code that uses shared data is known as a critical section,
Only one thread may execute a critical section for a given piece of shared data
at once,
this is known as \emph{mutual exclusion}~\cite{Dijkstra:Mutex}.
Critical sections therefore need to be protected by locks.
Upon entering a critical section a thread must acquire the lock associated with
the data,
it must release the lock when it leaves the critical section and the lock
ensures that only one thread may hold the lock at a time.
The programmer is responsible for defining critical sections and using locks
correctly.

Many other languages and libraries support the threading model of concurrency.
Many such languages support threading directly, buy building support into the
language or its standard library such as Java~\cite{java-threads}.

Mutual exclusion locking is problematic,
it is too easy to make mistakes when using locks.
Frequent mistakes include: forgetting to synchronise access to a critical
section,
making the critical section too small, or too large,
these can lead to incorrect program behaviour, inconsistent memory states and
crashes. 
When using multiple locks in a single critical section,
because that critical section uses multiple resources,
the locks must be acquired in the same order in each such critical section
otherwise deadlocks can occur.
This also prevents critical sections from being nestable.
All of these problems are difficult to debug because they are intermittent.
Often so intermittent that introducing tracing code or compiling with debugging
support can prevent the problem from occurring,
this is humorously known as a \emph{heisenbug} ---
a testament to just how frustrating it can be to find.

% Message passing.
Alternative models of concurrency exist,
another popular model is \emph{message passing}.
Notable examples of message passing libraries are MPI\cite{mpi} and
PVM~\cite{pvm}.
Unlike threads, message passing's \emph{processes} do not share static or heap data.
They communicate by sending messages to one another.
In MPI and PVM the programmer is responsible for encoding and decoding messages,
which can be tedious.
Languages such as Erlang~\cite{erlang} support message passing natively and
encoding and decoding is done for the programmer.
The major benefit of message passing is that it doesn't make assumptions about where processes
are executing:
they can be running all on one machine or spread-out across a network.
Therefore, message passing is very popular for high performance computing,
where shared memory systems are not feasible.
Message passing also avoids the problems with threading described above.
However,
it is often slower on shared memory systems since the same memory cannot easily
be referenced.
There are exceptions to this,
Go~\cite{balbaert:2012:go} for example uses message passing but also allows
access to shared memory,
and therefore has all the problems of threading.
Message passing does not avoid the problem of deadlocks,
Consider two processes,
they both attempt to send a message to their counterpart but only after
receiving such a message;
neither can proceed and therefore a deadlock occurs.
Most deadlocks are much more subtle and harder to find and fix.

% STM
A better way to avoid all these problems in concurrent programing is to use
software transational memory (STM)~%
\cite{harris_marlow_spj:haskell-stm,mika:mercury-stm}.
STM allows programmers to mark different variables as transactional
and to define critical sections.
Any changes to a transactional variable made during a critical section must
be sequentially consistent with any other operations and critical
sections executed by other threads.
If an in consistency is detected one or more critical sections must
\emph{roll back} --- undoing any changes made to transactional
variables.
It is important that a transaction has no side-effects, otherwise
consistency may not be guaranteed.


\subsection{Explicit Parallelism}
\label{sec:back_par_explicit}

% Parallelism w/o concurrency in imperative languages.
Parallelism can also be achieved without concurrency,
a popular language and library for this is OpenMP~\cite{openmp},
which allows programmers to annotate parts of their programs for parallel
execution without describing \textbf{how} this parallelism should be achieved;
making it much easier for programmers to use.
We will refer to this as \emph{explicit parallelism}
since parallelism is expressed directly by the programmer by the explicit
annotations.
In OpenMP most parallelism is achieved by annotating loops.
However,
the programmer is responsible for guaranteeing that the iterations of the loops
are independent,
that is to say, that no iteration depends on the results of affects of any of
the previous iterations.
If an iteration makes a function call, then this guarantee must also be true for
the callee all transitive callees.
Explicit parallelism otherwise avoids the problems with threading and message
passing.

% Declarative languages.
One of the easiest ways to make explicit parallelism easier is to prevent
side-effects from occurring, usually by designing them out of the language.
Pure declarative languages like Mercury~\cite{mercury_jlp},
Haskell~\cite{haskell98} and Clean~\cite{1991:concurrent-clean} do not
allow side-effects.
\paul{I should be able to find a more useful citation for Clean.}
This is done by ensuring that a function's declaration declares all of the
functions effects.
Therefore,
the programmer or the compiler can read a function's declaration to see if it
is safe to parallelize.
Mercury and Haskell both support explicit parallelism,
the programmer need only annotate their code, indicating that it should be
executed in parallel.

Glasgow Parallel Haskell (GpH) allows programmers to request parallel
evaluation of certain expressions by annotating them with a function
whose operational semantics cause parallel
evaluation~\cite{gph,loidi:2008:gph-semiexplicit-parallelism}.
Unfortunately Haskell's lazy evaluation strategy interacts poorly with
parallelism.
A parallelized computation will be evaluated to weak-head-normal-form,
that is to say that if it is a data term with parameters,
the parameters will not be evaluated.
This means that any parallelised work usually doesn't do enough work to
make parallelization worthwhile.
The GpH implementors have tried to solve this problem by allowing
the programmer to express how deep evaluation should
continue~\cite{trinder:1998:strategies}.
An alternative Haskell library called Monad.Par~\cite{marlow:monadpar}
fully evaluates the result of any spawned-off computation.
However, the programmer must still describe how any of their own
datatypes can be fully evaluated.
These solutions don't really solve the problem,
they still require the programmer to introduce strictness.
\paul{I can add support to my argument by describing problems with
space leaks, for example foldl, foldr, and foldl'.
I want to avoid a rant or a religious war, but I do want to
objectively assess the literature :-).}
It is better to avoid these and many other problems completely by using
a strict language.

\label{ref:parallel_conjunction}
Mercury allows programmers to request parallel evaluation of
conjunctions by replacing the normal conjunction operator with the
parallel conjunction operator introduced by Thomas Conway in
\cite{conway:2002:par,wang:parallel-mercury,wang:indep-and-par}.
Since Mercury is a strict language it does not have Haskell's
lazyness problems,
programmers simply annotate where parallel execution
should be used.
A more detailed description of parallelism in Mercury can be found in
section \ref{sec:backgnd_mer_par}

When a problem is not naturally a concurrent problem 
explicit parallelism such as in Mercury and Haskell
is preferable as the programmer does not need to force their program
into a concurrent model.
Especially since in all but the STM cases of concurrency programmers
must also describe how parallel computations communicate and
synchronise.

\paul{I'd like a citation here, I think I found something a while ago
about profiling}
However, explicit parallelism is a drawback because it requires the
programmer to know where their program spends most of its execution
time, it is understood that most programmers are poor at this.
Parallel execution has additional overheads such as:
spawning parallel tasks,
cleaning up completed parallel tasks,
operating system scheduling and
hardware behaviour such as cache effects.
The speedup gained when parallelising a computation will depend upon
these costs.
Programmers must therefore know whether parallelising a particular
computation is going to be an improvement in spite of the additional
costs of parallel execution.
Furthermore programmers must know whether there will be enough
processors free at runtime to execute the parallelised computation:
the additional costs of parallel execution will have an effect even
if there is not a processor available to execute the parallel work.
Parallelisation is another optimization such as inlining or efficient
register allocation.
Therefore,
it would be better for an optimizing compiler to handle parallelization
automatically;
the programmer will not need to worry about parallel evaluation any more
than they currently worry about inlining and register allocation.

\subsection{Implicit Parallelism}
\label{sec:lit_implicit-parallelism}

Some computer languages support implicit parallelism,
in these languages many parts of programs are executed in parallel.
Parallel execution is the normal mode of execution,
it is used in most places within the program.

% Implicitly parallel prologs. (and OR-parallelism)
A number of parallel Prolog-like languages that were developed during the
1980's are classified as implicitly parallel languages.
These included Concurrent
Prolog~\cite{saraswat85:probl_with_concur_prolog,saraswat86:concurrent_prolog_definition,shapiro:flat_concur_prolog},
Parlog~\cite{parlog,clark84:parlog_sys_prog} and GHC~\cite{ueda:ghc}.
Nearly all tasks in these languages were carried out in
parallel,
as a result the overheads of parallel execution are typically
greater than the benefit of running most small tasks in parallel.
Furthermore implicitly parallel programs have an \emph{embarrassingly
  parallel} workload,
this occurs when much more parallel work is available than the parallel
processing capacity of the machine;
thereby dramatically reducing the benifit of parallel execution while
the cost remained the same.
Both these effects often caused very poor performance.

% Granularity control
Granularity control was introduced in order to solve these
problems~\cite{lopez96:distance_granularity,shen99:granularity_control}.
It attempts to reduce the amount of work being executed in parallel.
There are a number of different methods, some incur a
runtime cost in order to determine if there is already ample parallel
work available while other static methods do not.
All methods help improve the performance of parallel programs and are
quite valuable, especially in recursive procedures.
\paul{I need to point out cases where GC doesn't help or isn't good
enough,
I'm going to have to find this in the literature and come back to this
paragraph and possibly the previous one.}

Some languages allow for the parallelisation of data parallel
tasks such as NESL~\cite{nesl} and Data Parallel
Haskell (DpH)~\cite{dph:2007:status_report,dph:2008:harnessing_the_multicores}.
These languages use special datatypes to denote parallelism,
sequences and parallel arrays in NESL and DpH respectivly.
Only operations on elements of these collections are parallelized.
We have none-the-less classified these systems as implicitly parallel,
since every action on these data types is executed in parallel.
Because operations are data-parallel they're independent and suitable
\paul{Define SMP in the introduction}
for execution on vector machines as well as SMP machines.
By transforming code and arranging for one thread to work on more than
one data item at a time granularity can be improved.
The drawback of these data-parallel aproches is that they can
parallelize data parallel programs
--- only a small subset of computer programs.

With the exception of DpH, implicit parallelism often performs worse
than explicit parallelism.
It is understood that carefully adding a few explicit parallelism annotations
to a program with the aid of a profiler will produce a faster-running
program than implicitly parallelising most independent computations.

\subsection{Automatic Parallelisation}
\label{sec:lit_automatic-parallelisation}

% Look at automatic parallelisation in Haskell.
Harris and Singh~\cite{harris_07_feedback_imp_par} developed a profiler
feedback directed automatic parallelisation approach for Haskell programs.
They have reported speed ups of up to 80\% compared to the sequential
execution of their test programs on a four core machine.
However they were not able to improve the performance of some
programs, they attributed this to a lack of parallelism
available in these programs.
They have shown that automatic parallelisation is a promising idea for
improving the performance of software.

We believe that more can be done to improve the effectiveness of
automatic parallelisation,
In some cases it may be possible to transform common programming
pasterns that lack parallelism into equivalent patterns with available
parallelism.
Furthermore, an advanced profiler --- such as Mercury's deep
profiler~\cite{conway:2001:mercury-deep} --- can provide information
that enables a compiler to make good parallelisation choices.
However there are a number of challenges facing automatic
parallelisation.
When using profiler feedback the profiled execution of the program may
not be a typical execution of the program, or there may be several
typical executions of the program.
We cannot control whether users profile typical executions of their
programs, but we may be able to allow users to merge execution
profiles to create a composite profile that is more representative of
their program's usage.
Another challenge can occur when a program has very little parallelism
available in it, it may be difficult to parallelise effectively.
We hope that in some cases the compiler can transform such a program
into an equivalent program with more available parallelism.

% Jerome's work.
J\'er\"ome Tannier's implicit parallelism implementation for
Mercury~\cite{tannier} also uses profiler feedback information to
automatically parallelise a program.
Tannier's approach selects the most expensive predicates of a program
and attempts to parallelise conjunctions within them.
Tannier also makes use of compile-time granularity
control to reduce the over-parallelisation that can occur in recursive
code.
Unfortunately, he estimated the costs and benefits of parallelising
dependant conjunctions based on the number of dependant variables that
they shared.
In practice most producers produce dependant variables late in their
execution and most consumers consume them early.
Therefore Tannier's calculation is na\"ive: the time that these
variables are produced by one conjunct and consumed by the other may
not correlate with the number of dependant variables.
We believe that Tannier's algorithm is, in general, too optimistic
about the parallelism available in dependant conjunctions.

% My honours thesis.
Paul Bone~\cite{bone:2008:hons} improved on this approach by using
information from a modification of Mercury's deep profiler to
calculate when the producing conjunct is most likely to produce the
dependant values and when the consuming conjunct is likely to need
them.
This information can be used to estimate the parallel speedup of
dependant conjunctions.
The effectiveness of this approach is not yet clear.

Mercury's deep profiler~\cite{conway:2001:mercury-deep} provides
detailed and accurate profiling information,
among other things the deep profiler records separate profiling
information for separate uses of the same code.
This will make it easier to implement optimisations such as
parallel specialisation --- generating sequential and parallel
versions of one procedure and using the sequential version
in situations where parallelism is not an optimisation.
Extracting information from the deep profiler to guide compiler
optimisations is supported by Bone's feedback
framework~\cite{bone:2008:hons}.
These are examples of the flexibility that the deep profiler provides,
describing other ideas is outside the context of a literature review.
No equivalent profiler exists for Haskell or Clean, making Mercury an
important choice for our implementation.

There is another challenge with automatic parallelism: a lot of
information about the execution of a program won't be recorded by the
profiler, often recording information in infeasible.
In these cases we must be careful to make safe, conservative
assumptions when calculating estimates of this information.

% Write about haskell's call centre stacks, maybe they provide enough
% information to perform similar optimisations.

% How does clean compare?

We expect that automatic parallelisation will more easily and
effectively parallelise declarative programs.
Furthermore, it will be easier to maintain such programs, as
characteristics of the program that are used to explicitly parallelise
a program won't necessarily be true in future versions or uses of that
program.
Automatic parallelisation allows the programmer to re-parallelise
their program quickly, based on a current execution profile of the
program.

