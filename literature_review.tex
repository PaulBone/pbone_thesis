
%\paul{
%This should just 'set the stage' for my work.  I would
%like to leave the contrasting discussions until later in the
%thesis, perhaps at the end of each major chapter.
%}
%
%\paul{
%I do not want to introduce all my citations here (the way Liz has to
%due to the APA-formatting standard).  Only enough that the user
%understand the context for reading the research chapters.  So,
%closely related work should be introduced for the first time when
%it is compared with my work.
%}

\status{Work in progress.}

In the rest of this chapter we will explore the literature concerning
parallel programming.
The languages in which parallelism is used have a strong influence on the
semantics and safty of the parallelism.
We group languages into the following two classifications.

\begin{description}

    \item[Imperative] programming languages are those that allow
    \emph{side-effects}.
    A side effect is an action that a computation may perform without
    needint to declare the possibility of the action.
    This can include input/output (IO) and modification of data structures.

    \item[Pure declrative] programming langauges are those that forbid
    side-effects.
    The effect of executing any computation must be declrated in the
    computation's signature.
    Because all effects are declared, there are no side-effects.
    This has numerous benifits, we are interested in the specific benifit
    that this makes it easy for both compilers and programmers to understand
    if it is safe to parallelise any particular computation.

\end{description}

\noindent
Not all researchers agree on the definitions above.
However these definitions are exactly what is required when discussing
the safty of parallelism.
Additionally a language clearly fits into one category or the other.

There are many ways to introduce parallelism in computer programming.
We find it useful to create the following classifications for these methods.

\begin{description}
    \item[Parallelism via concurrency]:
    concurrency can be used to introduce parallelism.
    While these concepts are related and often discussed together they are
    orthogonal.
    We will explain this and discuss the systems that use concurrency in
    order to achieve parallelism (\emph{parallelism via concurrency})
    (section \ref{sec:intro_concurrency}).

    \item[Explicit parallelism] is safer and easier to use compared to
    concurrency, however it is less expressive.
    We will explain how explicit parallelism is distinct from and improves
    upon \emph{concurrency for parallelism}.
    
    \item[Implicit parallelism] is a category that includes 
    systems that attempt to make parallel execution the typical form of
    execution.
    We describe a number of systems and explain why the general approach is
    flawed.
    Some systems use do not fit cleanly into either of our explicit
    or implicit parallelism definitions.
    We will discuss both appaches together in section
    \ref{sec:intro_par}i.

    \item[Automatic parallelism] is our last category.
    In these systems sequential execution is the typical form of execution
    and parallel execution is used sparingly.
    Many authors often group these systems with implicit parallelism systems.
    We have chosen to separate them because the authors of the two systems
    approach the same problem from different directions
    (parallel as default vs.\ sequential as default).
    Prior work in automatic parallelism is discussed in section
    \ref{sec:intro_auto_par}.
\end{description}    

\noindent
We explore the literature in this order as each successive category builds
on the work of the previous category.
Our own system is an automatic parallelism system;
it makes use of concepts described in all four categories.
Therefore
we cannot discuss it meaningfully until we have explored the literature
from giants upon whose shoulders we are standing.


\section{Parallelism via concurrency}
\label{sec:intro_concurrency}

% Parallelism and concurrency.
It is important to distinguish between \emph{parallelism} and
\emph{concurrency}.

\begin{description}
    \item[Concurrency] is defined as multiple computations executing simultaneously,
    which may communicate with one another \citep{hoare:1978:csp}.
    As an example consider a single-threaded web server which is required to
    handle multiple requests concurrently.

    \item[Parallelism] occurs when a program uses multiple cores or other
    components of a computer to get more work done in the same amount of
    time.
    A suitable example may be a program that uses SIMD instructions to
    multiply matrices, performing several basic operations with a single
    instruction.
\end{description}

\noindent
Often parallelism is achieved through concurrency
which leads to confusion among many programmers between the terms.
For example,
multiple concurrent processes executing on a multicore processor in parallel
is both concurrent and parallel.

Concurrency such as \citet{hoare:1978:csp}'s
communicating sequential processes (CSP)
or \citet{milner:pi}'s $\pi$-calculus
provide the programmer with a natural way to express many problems,
such as the web server example above.
In these cases concurrency is (and should be) used to make the program
easier to write.
If the concurrency uses parallelism provided by the hardware many programs
will speed up due to this parallelism,
but not all.

Often programmers will deliberatly use concurrency in order to speed up
their programs;
we call this \emph{parallelism via concurrency}.
It often works when the program is easy to express using concurrency
(but we think there are better methods).
However when the problem cannot be expressed easily using concurrency,
many programmers will attempt to \emph{force} their program into a concurrent
model in the name of parallelism.
Even if a programmer manages to profitably parallelise their program,
they will usually find that the program is no longer easy to read, understand
and debug.
In this section we will discuss systems that encourage parallelism through
concurrency.
Our critique of these systems should not be misunderstood as a critique of
concurrency which in general is outside the scope of this dissertation.

% Threading.
The most common method of introducing concurrency is with \emph{threads}.
Some languages support threads natively such as Java \citep{java-threads}
and others such as C \citep{c} support it via a library.
Libraries for using threads with C include
POSIX Threads \citep{butenhof1997:pthreads} or Windows
Threads \citep{winthreads}, depending on the operating system.
These libraries tend to be low level;
they simply provide an interface to the operating system.
Different threads of execution run independently,
sharing a heap and static data.
Because they share these areas where data may be stored they can, and often do,
use the same data (unlike CSP).
Threads communicate by writing to and reading from the shard data.
Regions of code that do this are called \emph{critical sections},
and they must be protected from concurrent access:
a thread reading shared data while another writes to it, will read
inconsistent data;
and when two threads write to the same data some of the changes to the data
may be lost.
Synchronisation such as \emph{mutual exclusion locks} (mutexes)
\citep{Dijkstra:Mutex} must be used avoid concurrent access.
Upon entering a critical section a thread must acquire the lock associated with
the data,
it must release the lock when it leaves the critical section and the lock
ensures that only one thread may hold the lock at a time.
The programmer is responsible for defining critical sections and using locks
correctly.

Mutual exclusion locking is problematic,
it is too easy to make mistakes when using locks.
Frequent mistakes include: forgetting to synchronise access to a critical
section,
making the critical section too small, or too large,
these can lead to incorrect program behaviour, inconsistent memory states and
crashes. 
When using multiple locks in a single critical section,
because that critical section uses multiple resources,
the locks must be acquired in the same order in each such critical section
otherwise deadlocks can occur.
This also prevents critical sections from being nestable,
causing significant problems for software composability.
All of these problems are difficult to debug because their symptoms are
intermittent.
Often so intermittent that introducing tracing code or compiling with debugging
support enabled can prevent the problem from occurring,
this is humorously known as a \emph{heisenbug} ---
a testament to just how frustrating it can be to find.
While it is possible to deal with these problems,
we believe that it is better to make problems impossible;
doing so guarantees that the programmer takes on fewer risks.

% Message passing.
Another popular model is \emph{message passing}.
Unlike threads, message passing's \emph{processes} do not share static or heap
data,
processes communicate by passing messages to one another.
The major benefit of message passing is that it does not make assumptions about where processes
are executing:
they can be running all on one machine or spread-out across a network.
Therefore, message passing is very popular for high performance computing,
where shared memory systems are not feasible.
Message passing also avoids the problems with threading and mutexes,
however other synchronisation techniques can introduce a subset of the
problems that threading suffers.
For example it is possible to create deadlocks with cyclic dependencies
between messages:
a process $a$ will send a message to $b$ but only after it has received a
message from $b$,
and $b$ will send a message to $a$ but only after it has received its message.
(We can create a similar situation with locks.)
Most deadlocks are more subtle and harder to find than this trivial example.

Notable examples of message passing application programming interfaces (APIs)
for C and Fortran \citep{backus:1957:fortran} are
MPI \citep{mpi} and PVM \citep{pvm}.
Every process can send and receive messages with any other process,
We can think of this in terms of \emph{mailboxes}:
each process owns (and can read from) its own mailbox and can place messages
in any other mailbox.
MPI and to a lesser extent PVM support powerful addressing modes
such as broadcasting the same message to multiple processes, or
distributing the contents of an array evenly between
multiple processes.
These features make MPI and PVM very powerful interfaces for high performance
computing.
However, the programmer is responsible for encoding and decoding messages,
which adds extra flexibility can be very tedious.
The Erlang language \citep{erlang} supports message passing natively and
encoding and decoding is done for the programmer.
Erlang allows processes to be created dynamically,
making it more flexible than MPI or PVM.
These two differences make it easier to program in Erlang than C with MPI or
PVM.
Erlang is typically used where fault tolerance is important,
where as MPI and PVM are used for high performance such as in scientific
computing.
All three systems support distributed computing.

% CSP
A different interpretation of message passing is
communicating sequential processes \citep{hoare:1978:csp}.
Even though we introduced PVM and MPI first, CSP predates them by 11 years.
In the CSP model communication is performed through \emph{channels},
channels and processes may be created dynamically.
Channels are more powerful than mailboxes as multiple processes can read from
the same channel (but not at the same time).
A process can also set up multiple channels for different types of message,
the same benefit can be gained by tagging messages in the mailbox analogy.
The first language based on CSP is occam \citep{occam1, occam3}.

\citet{milner:pi}'s $\pi$-calculus is similar to CSP.
The difference between the two process algebras is that in the $\pi$-calculus
channels are first class, meaning that a channel can be sent in a channel.
This allows programmers to compose a dynamic communication network.
%The Jomel language \citep{jomel} (an ML \citep{ml}) is based on the $\pi$-calculus
Jocaml \citep{jocaml} (an Ocaml
\citep{ml-types, ocaml-modules, ocaml-bytecode, ocaml-native}
derivative) is a language based on the related join-calculus
\citep{join-calculas}.
Join-calculus differs by using asynchronous communication and adding a number
of restrictions, which are not important to our discussion.
Jocaml differs from Occam in the same ways that the join-calculus differs
from CSP.
However Occam, in particular its first version \citep{occam1},
was intended as a research tool to experiment with CSP, Jocaml was not.
Later versions of occam \citep{occam3} were designed for more practical use.
Jocaml is more widely used and appears to be much more practical.
Jocaml, like Erlang, is a powerful language with support for distributed
computing.

There are many other languages and libraries based on these process algebra,
one that is currently enjoying a lot of popularity is Go
\citep{balbaert:2012:go}.
Go appears to follow a model similar to the $\pi$-calculus and the
join-calculus.
However Go also uses shared memory without providing any synchronisation
for communicating through shared memory.
In particular data sent in messages is sent \emph{by reference},
meaning that if one process modifies data that has been shared,
other processes can see the modification and therefore read inconsistent data
or write inconsistent data into memory.
Programmers are discouraged from using shared memory for communication,
however this makes several classes of bug possible through accidental use of
shared memory.
We suspect that the developers made this decision to make Go perform faster on
multicore systems.
It is not clear if this is part of the language specification or simply
Google's implementation.

% STM
All the systems we have introduced so far are prone to deadlocks,
and many are prone to other problems.
Many problems can be avoided using software transactional memory (STM)
\citep{stm}.
STM is similar to the threading model, it includes the concept of critical
sections.
Critical sections are not protected by mutexes,
instead optimistic locking is used.
Variables that are shared between threads are known as STM variables,
they may only be modified within a critical section and a good implementation
will enforce this.
STM works by logging all the reads and writes to STM variables within a
critical section, and not actually performing them until the end of the
critical section.
At that point the STM system acquires a mutex for each STM variable (in the
order that they were accessed)
and checks for sequential consistency by comparing its log with the current
values of the variables.
If no inconsistencies are found
it may go ahead and modify and unlock the variables.
Otherwise it must retry the critical section or fail.
Compared to the threading model,
the user does not need to use mutexes and critical sections are now
nestable (two nested sections commit when the outer one commits).
STM does not suffer from deadlocks.
However, programmers must still place their critical sections correctly in
their program.
A common criticism of STM is that it performs poorly when there is a lot of
contention as a failed critical section must be re-executed and this often
adds to the amount of contention.
Including side-effecting computations inside an STM critical section can also
be a problem.
This can be avoided in a pure declarative language such as
Mercury \citep{mercury_jlp} or Haskell \citep{haskell98} as side effects cannot
be expressed.
\citet{mika:mercury-stm} is an implementation for Mercury
and \citet{harris:2005:haskell-stm} is an implementation for Haskell.

\section{Explicit and implicit parallelism}
\label{sec:intro_par}

% Parallelism w/o concurrency in imperative languages.
Parallelism can also be achieved without concurrency.
This is done with language annotations that do not affect the (declrative)
semantics of the program.
The annotations describe \emph{what} should be parallelised,
and rarely describe \emph{how} to parallelise each computation.
We will refer to this as \emph{explicit parallelism}
since parallelism is expressed directly by the programmer by the explicit
annotations.

OpenMP \citep{openmp} is a library for C++ \citep{cplusplus} and Fortran
that allows programmers to annotate parts of their programs for parallel
execution.
In OpenMP most parallelism is achieved by annotating loops.
C++ is an imperative language and therefore the compiler cannot determine
which computations are safe to parallelise.
Therefore the programmer is responsible for guaranteeing that the iterations
of the loops are independent,
that is to say, that no iteration depends on the results of affects of any of
the previous iterations.
If an iteration makes a function call, then this guarantee must also be true for
the callee all transitive callees.
The programmer must also declare which variables are shared, local or used
for reductions.
This is necessary so that the compiled code is parallelised correctly.
The programmer is invited to specify how the iterations of the loop may be
combined into parallel tasks (known as \emph{chunking}).
This means that to some extent the programmer still describes \emph{how} to
parallelise the loop,
although the programmer has much less control than with parallelism through
concurrency systems.
OpenMP is one of the two explicit parallelism systems that we review that
does this.
Describing how something should be parallelised is rare.
Povided that the programmer can annotate their loops correctly,
explicit parallelism with OpenMP avoids the problems
with threading and message passing.  

% Declarative languages.
In pure declrative langauges
a function has no side effects,
all possible effects of calling a function are specified in the function's
declaration, (this can be considered an upper bound).
This is often done by expressing effects as part of the type system such as
with:
Monads \citep{haskell-monads}, which are used by Haskell \citep{haskell98}
or
uniquness-typing, which is used by 
Mercury\footnote{
    Mercury's uniqueness typing is part of its mode system.
    dispite this it is still considered uniqueness typing.}
\citep{mercury_jlp}
and Clean \citep{brus:1987:clean}.
There are also some very novel approaches such as effect-typing \citep{ddc}.
In any of these systems both the programmer and compiler can trivially
determine if parallel execution of a particular computation is safe.
Furthermore,
in Haskell, Mercury and Clean
the concept of a variable is different from that in imperative programming,
it is not a storage location that can be updated,
but a binding of a name to a value.
Therefore variables that are shared
(those that we would have to declare as shared in OpenMP)
are really shared immutable values\footnote{
    We find it amusing that the misnomer ``variable'' is used at all.}.
Explicit parallelism in declrative languages is much easier and safer to use
than in imperative languages.
The next two sub-sections describe parallelism in functional and logic
langauges respectivly.

%Pure declarative languages like Mercury~\citep{mercury_jlp},
%Haskell~\citep{haskell98} and Clean~\citep{1991:concurrent-clean} do not
%allow side-effects.
%\paul{I should be able to find a more useful citation for Clean.}
%This is done by ensuring that a function's declaration declares all of the
%functions effects.
%Therefore,
%the programmer or the compiler can read a function's declaration to see if it
%is safe to parallelise.
%Mercury and Haskell both support explicit parallelism,
%the programmer need only annotate their code, indicating that it should be
%executed in parallel.

\subsection{Parallelism in functional languages}
\label{sec:intro_par_func}

% \paul{Concurrent Lisp}
% Concurrent lisp clearly uses concurrency as in the previous section.

Pure functional languages like Haskell and Clean can be evaluated using a
process known as \emph{graph reduction}.
Graph reduction works by starting with an expression and a program and
applying reductions.
Reductions use an equation in the program to replace part of the expression
with a new expression.
In a lazy language (or lazy implementation of a non-strict language) 
each reduction step performs \emph{outermost reduction}.
This means that the arguments of a function are not normally evaluated
before a reduction replaces the function call with an equation for that
function.
Some reductions require the evaluation of an argument,
these include primative operations and pattern matching.
The system continues applying reductions until the espression is 
a value;
the value may have parameters which are still expressions (they have not
been evaluated).
This is called weak head normal form (WHNF);
Graph reduction itself can be parallelised:
If a subgraph (an argument in the expression) is known to be required in the
future,
then it can be evaluated in parallel using the same graph reduction process.
Several projects experimented with this technique
\citep{augustsson:1989:parallel-graph-reduction,burn:1989:parallel-reduction-machine,peyton-jones:1989:parallel-graph-reduction}.

Because the programmer does not need to annotate their program this
parallelism is implicit.
While implicit parallelism is always safe there is a serious performance
issue.
The cost of creating a parallel evaluation is sagnificant,
so is any communication between parallel tasks.
There are also distributed costs throughout the execution of these systems,
for example many parallel graph reduction systems introduce locking on every
graph node.
Such locking increases the overheads of evaluation even when no parallel
tasks are created.
Therefore parallelised computations should be large enough that the benifit
of evaluating them in parallel outweighs the costs of doing so.
Unfortunatly most parallelism in an implicit parallel system is very
fine grained, meaning that it is not large enough to make parallel
evaluation worthwhile.
There is also a lot of this parallelism,
this means that the system spends most of its time managing the parallel
tasks rather than evaluating the program.
This is called embarassing parallelism as the amount of fine-grained
parallelism can cause the evaluation to perform \emph{more slowly} than
sequential evaluation.
We will discuss other implicitly parallel systems,
most of them share the same problems.
Many systems such as \citet{peyton-jones:1989:parallel-graph-reduction} go
to lengths to exploit only course grained parallelism.
In our oppionion the problem with implicit parallelism is that it approches
the parallel programming challenge from the wrong direction:
it makes everything execute in parallel \emph{by default} and then attempts
to introduce sequential execution;
we beleive that one should make sequential execution the default and then
attempt to introduce parallel execution \emph{only where it is profitable}.

In contrast explicit parallelism allows the programmer to introduce
parallelism only where they beleive it would be profitable.
This means that the default execution stratergy is sequential.

Multilisp \citep{halstead:1984:multilisp,halstead:1985:multilisp} is an
explicitly parallel implemntation of Scheme.
Like scheme, Multilisp is not pure, it allows side effects.
This means that as programmers introduce parallelism, they must be sure that
parallel computations cannot be affected by side effects,
and do not have side effects themselves.
Therefore multilisp is not safe as side effects may cause inconsistent
results or courrupt memory.
Multilisp programmers create parallelism by creating delayed computations
called \emph{futures}.
These computations will be evaluated concurrently 
(and potentially in parallel).
When the value of a future is required,
the thread blocks until the future's value is available.
Many operations such as assignment and parameter parsing do not require the
value of a future.

Haskell is a pure functional language.
Being pure it does not have the problem with side effects that multilisp
does above.
The Haskell language standard specifies that it is non-strict\footnote{
    Non-strict evaluation is more boadly defined than lazy evaluation};
in practice however, all implementations use lazy evaluation in most cases.
Patterns and cases where a sub-expression's value is always required are
evaluated eagerly.
Haskell programmers can introduce explicit parallelism into their programs
with two functions (\code{par} and \code{pseq}).
Support for these functions first appeared in GUM
\citep{gph:gum,loidi:2008:gph-semiexplicit-parallelism}.
They later appeard in the Glasgow Haskell Compiler (GHC),
arguably the most popular implementation of Haskell
\citep{harris:2005:haskell-smp}.
The \code{par} and \code{pseq} functions are have the types:

\begin{verbatim}
par :: a -> b -> b
pseq :: a -> b -> b
\end{verbatim}
They both take two arguments and return their first;
their declrative semantics are identical to the \emph{const} function.
However their \emph{operational} semantics are different,
the \code{par} function may spawn off a parallel task that evaluates its
second argument to WHNF,
before returning its first argument.
While the \code{pseq} funtion will evaluate its second argument to WHNF
\emph{and then} return its first argument.
We can think of these functions as \code{const} with different evaulation
stratergies.

Unfortunately lazy evaluation strategy interacts poorly with
parallel evaluation.
Because parallel tasks are only evaluated to WHNF,
most tasks do not create enough work to make parallelisation worthwhile.
The developers tried to reduce this problem by allowing programmers to
specify an evaluation stratergy,
which describes how deeply to evaluate expressions \citep{trinder:98:strategies}.
his description is written in Haskell itself,
this allows it to be implemented as a library and describe higher order
evaluation stratergies.
This does not solve the problem,
it only gives the programmer a way to deal with it when they encounter it;
it is better to make it impossible for the problem to occur,
for example by restricting parallel computations to eager evaluation.
Similar problems occur with non-strict evaluation in sequential programs.
Evaluation stratergies, in particular eager vs non-eager,
are a matter of personal preference.

An alternative Haskell library called Monad.Par \citep{marlow:monadpar}
requires parallel computations to be evaluated in full.
However the programmer must provide typeclass instances for each of their
own types that describe how to completely evaluate each type.
While this still requires a non-trivial amount of effort from the
programmer,
it is harder for the programmer to make mistakes.
Their program will not compile if they forget to provide a typeclass
instance.

\subsection{Parallelism in logic languages}
\label{sec:intro_par_logic}

Prolog is the most well known logic programming language.
Although logic programming has many sub-paradigms we will discuss
those that use selective linear resolution with definite clauses
(SLD resolution) \citep{kowalski_sld}.
SLD resolution attempts to answer a query by finding a Horn clause whose
head has the same atom name as the query,
performing variable substitution and then performing a query on each
conjunct in the clause's body.
If a query of a conjunct fails then the clause fails and an alternative
clause may be tried.
As multiple clause heads may match the query there may be more than one
solution,
and as the query may fail there might be zero solutions.
As such each query may have any number of solutions.

OR-parallel Prolog implementations may attempt to resolve a query against
multiple clauses in parallel.
Like parallel graph reduction this form of implicit parallelism can create
embarassingly parallel workloads.
OR-parallelism has some additional problems.
Not all programs will provide enough OR-parallelism as many do not perform
non-determinstic searches.
Furthermore OR-parallelism is speculative:
the current goal may be executed in a context where only the first result
will be used,
therefore searching for more results in parallel is a waste of resources.
Systems that exploit OR-parallelism include
Aurora \citep{lusk:1990:aurora}
and Muse \citep{ali:1990:muse}.

AND-parallel Prolog implementations introduce parallelism by executing
conjuncts within clauses in parallel.
AND-parallelism comes in two vareities, independent and dependent.
independent AND-parallelism is much simplier, it only parallelises goals
whose executions are independent.
\&-Prolog 
\citep{Hermenegildo:1991:and-parallel,
DBLP:journals/jlp/MuthukumarBBH99}.
Independent AND-parallelism comes in two flavors, strict and non-strict.
Strict (also called restricted) independent AND-parallelism is where
conjoined goals that may be evaluated in parallel must have no variables in
common, including input variables.
It is trivial for a Prolog system to statically determine that the goals are
independent.
Non-strict (unrestricted) independent AND-parallelism is where some variables
are shared but none of the goals performs any instantiation on these variables
or anything they reference.
As non-strict parallelism is more general it is desirable for a system to
handle it.
Determining which goals are independent is much more complicated and
requires analysis 
\citep{DBLP:journals/tcs/GrasH09, Hermenegildo1995}.
Some of the analysis can be done at compile time, however it is a whole program analysis shich means that seperate compilation cannot be used.
Furthermore the analysis is often incomplete,
in some cases checks need to be done at runtime and these checks may traverse
large data structres.
In some cases they are not worthwhile.

\plan{Managing logic variables and retracting their values}
Prolog supports \emph{logic variables},
When two or more free variables are unified with one another they behave as one
variable;
any later successful unification on any of the variables implicitly modifies
the state of the other variables.
During squential execution multiple OR-branches may bind different values to
the same variable before either failing (retracting the bindings) 
or succeeding (keeping the bindings).
With OR-parallelism a variable may have to have different values in different
parts of the search tree at once.
OR-parallel systems must handle the extra overheads caused by this.

While it is straightforward to implement this correctly for OR-parallel
and independent AND-parallel systems it is much more difficult for
dependent AND-parallel systems.
AND-parallel systems must handle communcation between parallel conjuncts.
This can be implemented in a number of different ways.

1. Delayed communcation.
2. Guards and committed choice w/ suspention.
3. ??? Are there other ways?

Allowing parallel computations to see the bindings made by other
computations can help reduce the amount of wasted work,
goals that eventually fail will fail earlier and therefore waste less time
before they fail.
However, when part of a proof tree fails, any variable bindings made in that
part of the tree must be retracted.
In OR-parallel systems the Prolog implementation must handle multiple
potential bindings being produced for a variable at once.
In dependent AND-parallel systems retractions must be handled, 

Many more systems support dependent AND-parallelism as more parts of a
program are dependent,
and any system that supports dependent parallelism can handle independent
parallelism.
% An example of such a system is Andorra \citep{haridi:1990:andorra}.



Even
OR-parallel and independent AND-parallel systems must manage this correctly.

Many systems support both AND-parallelism and OR-parallelism such as:

IDIOM

PEPSys

Reform Prolog

Committed choice:

\plan{Parlog}

\plan{GHC}

\plan{ACE}

\plan{Ciao?}


To avoid the problems with implicit OR-parallelism researchers concentrate
on methods to select larger graularity tasks where possible.
Such stratergies come at a runtime cost.
For example \citet{hausman:1987:or} scans the search tree for parallelism
nearer the top, which is a linear time traversal.

Selection of which goal in a conjunction or disjunction to resolve at each
step,
usually selects the left-most goal as written by the programmer.
However Mercury's purity (unlike Prolog) makes this selection irreleant to
the programmer.


Logic programs are groups of Horn clauses,
an query may be evaluated by substitution  


\plan{Mercury}

\paul{TODO: Mercury's parallel conjunction is never speculative.}

\label{ref:parallel_conjunction}
Mercury is a pure logic/functional language,
Programmers can request parallel evaluation of a conjunction by replacing
the plain conjunction symbol (\samp{,})
with the parallel conjunction symbol (\samp{\&})
\citep{conway:2002:par,wang:2006:hons,wang:2011:dep-par}.
This is called AND-parallelism as the parallelism is introduced within
conjunctions.
We will describe how this works in section \ref{sec:bacngnd_merpar}.
Because Mercury uses a strict evaluation stratergy,
parallelism in Mercury does not have the problems of parallelism in Haskell.
No equivilient of the \code{pseq} function is required as every computation
is evaluated completely.

Other logic languages have used AND-parallelism and OR-parallelism before.


When a problem is not naturally a concurrent problem 
explicit parallelism such as in Mercury and Haskell
is preferable as the programmer does not need to force their program
into a concurrent model.
Especially since in all but the STM cases of concurrency programmers
must also describe how parallel computations communicate and
synchronise.

\paul{I would like a citation here, I think I found something a while ago
about profiling}
However, explicit parallelism is a drawback because it requires the
programmer to know where their program spends most of its execution
time, it is understood that most programmers are poor at this.
Parallel execution has additional overheads such as:
spawning parallel tasks,
cleaning up completed parallel tasks,
operating system scheduling and
hardware behaviour such as cache effects.
The speedup gained when parallelising a computation will depend upon
these costs.
Programmers must therefore know whether parallelising a particular
computation is going to be an improvement in spite of the additional
costs of parallel execution.
Furthermore programmers must know whether there will be enough
processors free at runtime to execute the parallelised computation:
the additional costs of parallel execution will have an effect even
if there is not a processor available to execute the parallel work.
Parallelisation is another optimisation such as inlining or efficient
register allocation.
Therefore,
it would be better for an optimising compiler to handle parallelisation
automatically;
the programmer will not need to worry about parallel evaluation any more
than they currently worry about inlining and register allocation.

\subsection{Implicit Parallelism}
\label{sec:intro_implicit_par}

Some computer languages support implicit parallelism,
in these languages many parts of programs are executed in parallel.
Parallel execution is the normal mode of execution,
it is used in most places within the program.

% Implicitly parallel prologs. (and OR-parallelism)
A number of parallel Prolog-like languages that were developed during the
1980's are classified as implicitly parallel languages.
These included Concurrent
Prolog~\citep{saraswat85:probl_with_concur_prolog,saraswat86:concurrent_prolog_definition,shapiro:flat_concur_prolog},
Parlog~\citep{clark:84:parlog_sys_prog,clark:86:parlog} and GHC~\citep{ueda:ghc}.
Nearly all tasks in these languages were carried out in
parallel,
as a result the overheads of parallel execution are typically
greater than the benefit of running most small tasks in parallel.
Furthermore implicitly parallel programs have an \emph{embarrassingly
  parallel} workload,
this occurs when much more parallel work is available than the parallel
processing capacity of the machine;
thereby dramatically reducing the benefit of parallel execution while
the cost remained the same.
Both these effects often caused very poor performance.

% Granularity control
Granularity control was introduced in order to solve these
problems~\citep{lopez96:distance_granularity,shen_98_granularity-control}.
It attempts to reduce the amount of work being executed in parallel.
There are a number of different methods, some incur a
runtime cost in order to determine if there is already ample parallel
work available while other static methods do not.
All methods help improve the performance of parallel programs and are
quite valuable, especially in recursive procedures.
\paul{I need to point out cases where GC does not help or is not good
enough,
I am going to have to find this in the literature and come back to this
paragraph and possibly the previous one.}

Some languages allow for the parallelisation of data parallel
tasks such as NESL~\citep{blelloch:95:nesl} and Data Parallel
Haskell (DpH)~\citep{dph:2007:status_report,dph:2008:harnessing_the_multicores}.
These languages use special data types to denote parallelism,
sequences and parallel arrays in NESL and DpH respectively.
Only operations on elements of these collections are parallelised.
We have none-the-less classified these systems as implicitly parallel,
since every action on these data types is executed in parallel.
Because operations are data-parallel they are independent and suitable
\paul{Define SMP in the introduction}
for execution on vector machines as well as SMP machines.
By transforming code and arranging for one thread to work on more than
one data item at a time granularity can be improved.
The drawback of these data-parallel approaches is that they can
parallelise data parallel programs
--- only a small subset of computer programs.

With the exception of DpH, implicit parallelism often performs worse
than explicit parallelism.
It is understood that carefully adding a few explicit parallelism annotations
to a program with the aid of a profiler will produce a faster-running
program than implicitly parallelising most independent computations.

\subsection{Other language paradigms}
\label{sec:intro_par_other}

Sisal \citep{feo:1990:sisal-report} is a strict functional language that
supports implicit dataflow parallelism.
Sisal's inbuilt types include arrays and streams.
Streams have the same form as cons-lists (Lisp style) except that as one
computation produces results on the stream, another computation may be
reading items from the stream.
There may only be one producer, which is why streams are not the same as
channels.
The other source of parallelism is the parallel execution of independent
loop iterations.
Sisal's for loops include a reduction clause,
a clause stating how the loop is transformed into a result which may be a
scalar result or an array or stream.


\section{Automatic Parallelisation}
\label{sec:intro_auto_par}

% Look at automatic parallelisation in Haskell.
\citet{harris_07_feedback_imp_par} developed a profiler
feedback directed automatic parallelisation approach for Haskell programs.
They have reported speed ups of up to 80\% compared to the sequential
execution of their test programs on a four core machine.
However they were not able to improve the performance of some
programs, they attributed this to a lack of parallelism
available in these programs.
They have shown that automatic parallelisation is a promising idea for
improving the performance of software.

We believe that more can be done to improve the effectiveness of
automatic parallelisation,
In some cases it may be possible to transform common programming
pasterns that lack parallelism into equivalent patterns with available
parallelism.
Furthermore, an advanced profiler --- such as Mercury's deep
profiler~\citep{conway:2001:mercury-deep} --- can provide information
that enables a compiler to make good parallelisation choices.
However there are a number of challenges facing automatic
parallelisation.
When using profiler feedback the profiled execution of the program may
not be a typical execution of the program, or there may be several
typical executions of the program.
We cannot control whether users profile typical executions of their
programs, but we may be able to allow users to merge execution
profiles to create a composite profile that is more representative of
their program's usage.
Another challenge can occur when a program has very little parallelism
available in it, it may be difficult to parallelise effectively.
We hope that in some cases the compiler can transform such a program
into an equivalent program with more available parallelism.

% Jerome's work.
\citet{tannier:2007:parallel_mercury} previously attempted to automatically
parallelise Mercury programs using profiler feedback
information to automatically parallelise a program.
\citet{tannier:2007:parallel_mercury} approach selects the most expensive predicates
of a program and attempts to parallelise conjunctions within them.
Tannier also makes use of compile-time granularity
control to reduce the over-parallelisation that can occur in recursive
code.
Unfortunately, he estimated the costs and benefits of parallelising
dependant conjunctions based on the number of dependant variables that
they shared.
In practice most producers produce dependant variables late in their
execution and most consumers consume them early.
Therefore Tannier's calculation is na\"ive: the time that these
variables are produced by one conjunct and consumed by the other may
not correlate with the number of dependant variables.
We believe that Tannier's algorithm is, in general, too optimistic
about the parallelism available in dependant conjunctions.

% My honours thesis.
\citet{bone:2008:hons} improved on this approach by using
information from a modification of Mercury's deep profiler to
calculate when the producing conjunct is most likely to produce the
dependant values and when the consuming conjunct is likely to need
them.
This information can be used to estimate the parallel speedup of
dependant conjunctions.
The effectiveness of this approach is not yet clear.

Mercury's deep profiler~\citep{conway:2001:mercury-deep} provides
detailed and accurate profiling information,
among other things the deep profiler records separate profiling
information for separate uses of the same code.
This will make it easier to implement optimisations such as
parallel specialisation --- generating sequential and parallel
versions of one procedure and using the sequential version
in situations where parallelism is not an optimisation.
Extracting information from the deep profiler to guide compiler
optimisations is supported by the feedback framework developed by
\citet{bone:2008:hons}.
These are examples of the flexibility that the deep profiler provides,
describing other ideas is outside the context of a literature review.
No equivalent profiler exists for Haskell or Clean, making Mercury an
important choice for our implementation.

There is another challenge with automatic parallelism: a lot of
information about the execution of a program will not be recorded by the
profiler, often recording information in infeasible.
In these cases we must be careful to make safe, conservative
assumptions when calculating estimates of this information.

% Write about haskell's call centre stacks, maybe they provide enough
% information to perform similar optimisations.

% How does clean compare?

We expect that automatic parallelisation will more easily and
effectively parallelise declarative programs.
Furthermore, it will be easier to maintain such programs, as
characteristics of the program that are used to explicitly parallelise
a program will not necessarily be true in future versions or uses of that
program.
Automatic parallelisation allows the programmer to re-parallelise
their program quickly, based on a current execution profile of the
program.

