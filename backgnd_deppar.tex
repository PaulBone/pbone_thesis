
\status{I'm currently working on this section.}

% This is a figure since there are a number of references to it.
\begin{figure}
\begin{verbatim}
map_foldl(M, F, L, Acc0, Acc) :-
    (
        L = [],
        Acc = Acc0
    ;
        L = [H | T],
        (
            M(H, MappedH),
            F(MappedH, Acc0, Acc1)
        &
            map_foldl(M, F, T, Acc1, Acc)
        )
    ).
\end{verbatim}
%\vspace{2mm}
\caption{Parallel \mapfoldl{}}
\label{fig:mapfoldl}
%\vspace{-1\baselineskip}
\end{figure}

Mercury's mode system allows a parallel conjunct to consume variables
that are produced by conjuncts to its left, but not to its right.
This guarantees the absence of circular dependencies
and hence the absence of deadlocks between the conjuncts,
but it does allow a conjunct to depend on a variable that is yet to be
computed by a conjunct running in parallel.
These variables can be identified easily at compile time:
% I mentioned compile time because anyone reading this with a Prolog
% background may make a false assumption, even though we gave them the
% background in the Mercury section above.
they are bound by the parallel conjunction,
and they appear in multiple conjuncts of the parallel conjunction.
In Figure \ref{fig:mapfoldl} \code{Acc1} is an example of such a
\emph{shared variable}.
We handle these variables through a source-to-source transformation
\citep{wang_dep_par_conj,wang_hons_thesis}.

%For each of these shared variables,
%it creates a data structure called a \emph{future} \citep{multilisp}.
%When the producer has finished computing the value of the variable,
%it puts the value in to the future which signals its availability.
%When a consumer attempts to read a value from a future
%it waits for this signal (if it has not yet happened),
%and then retrieves the value.
%Both operations are protected by mutexes.

\begin{figure}
\begin{verbatim}
XXX: Maybe make the int an enum?
struct MR_Future {
    /* lock preventing concurrent accesses */
    MercuryLock     MR_fut_lock;
    /* whether this future has been signalled yet */
    int             MR_fut_signalled;

    /* linked list of all the contexts blocked on this future */
    MR_Context      *MR_fut_suspended;
    MR_Word         MR_fut_value;
};
\end{verbatim}
\caption{Future datastructure}
\label{fig:future}
\end{figure}

For each of these shared variables,
the compiler creates a data structure called a \emph{future} \citep{multilisp},
which contains room for the value of the variable,
a flag indicating whether the variable has been produced yet,
a queue of consumer contexts waiting for the value, and a mutex.
The initial value of the future has the flag set to `not yet produced'.
The signal operation on the future sets the value of the variable,
sets the flag to `produced',
and wakes up all the waiting consumers,
all under the protection of the mutex.
The wait operation on the future is also protected by the mutex:
it checks the value of the flag,
and if it says `not yet produced',
the engine will put its context on the queue and suspend it before
looking for global work.
When it wakes up,
or if the flag said that the value was already `produced',
the wait operation simply gets the value of the variable.
The wait operation includes an optimization,
it is able to check the flag before it takes the mutex provided that it
checks the flag once more within the critical section,
This way it does not need to use the mutex if the value has already been
produced.
Futhermore, if the compiler sees mone than one read of the same future
on the same execution path,
it can optimize the later ones so they read the value without checking
the flag or mutex.

To minimize waiting,
the compiler pushes signal operations on each future
as far to the left into the producer conjunct as possible,
and it pushes wait operations
as far to the right into each of its consumer conjuncts as possible.
This means not only pushing them
into the body of the predicate called by the conjunct,
but also into the bodies of the predicates they call,
with the intention being that
each signal is put immediately after
the primitive goal that produces the value of the variable,
and each wait is put immediately before
the leftmost primitive goal that consumes the value of the variable.
Since the compiler has complete information
about which goals produce and consume which variables,
the only things that can stop the pushing process from placing the
wait immediately before the value is to be used and the signal
immediately after it is produced are
higher order calls and module boundaries:
the compiler cannot push a wait or signal operation
into the body of a predicate
if it does not have access to the identity or to the body of the predicate.


\begin{figure}
\begin{verbatim}
map_foldl(M, F, L, Acc0, Acc) :-
    (
        L = [],
        Acc = Acc0
    ;
        L = [H | T],
        new_future(FutureAcc1),
        (
            M(H, MappedH),
            F(MappedH, Acc0, Acc1),
            signal_future(FutureAcc1, Acc1)
        &
            map_foldl_par(M, F, T, FutureAcc1, Acc)
        )
    ).

map_foldl_par(M, F, L, FutureAcc0, Acc) :-
    (
        L = [],
        wait_future(FutureAcc0, Acc0),
        Acc = Acc0
    ;
        L = [H | T],
        new_future(FutureAcc1),
        (
            M(H, MappedH),
            wait_future(FutureAcc0, Acc0),
            F(MappedH, Acc0, Acc1),
            signal_future(FutureAcc1, Acc1)
        &
            map_foldl_par(M, F, T, FutureAcc1, Acc)
        )
    ).
\end{verbatim}
%\vspace{2mm}
\caption{\mapfoldl{} with synchronization}
\label{fig:map_foldl_sync}
%\vspace{-1\baselineskip}
\end{figure}

Given the \mapfoldl predicate in Figure~\ref{fig:mapfoldl},
this synchronization transformation
generates the code in Figure~\ref{fig:map_foldl_sync}.


