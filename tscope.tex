
\begin{abstract}
The behavior of parallel programs is even harder to understand
than the behavior of sequential programs.
Parallel programs may suffer from
any of the performance problems afflicting sequential programs,
as well as from several problems unique to parallel systems.
Many of these problems are quite hard
(or even practically impossible)
to diagnose without help from specialized tools.
We present a proposal for a tool
for profiling the parallel execution of Mercury programs,
a proposal whose implementation we have already started.
This tool is an adaptation and extension of the ThreadScope profiler
that was first built to help programmers visualize
the execution of parallel Haskell programs.
\end{abstract}

\section{Introduction}

When programmers need to improve the performance of their program,
they must first understand it.
The standard way to do this is to use a profiler.
Profilers record performance data from executions of the program,
and give this data to programmers
to help them understand how their program behaves.
This enhanced understanding then makes it easier
for programmers to speed up the program.

Profilers are needed because the actual behavior of a program
often differs from the behavior imagined by the programmer.
This is true even for sequential programs.
For parallel programs,
the gulf between human estimates and machine realities
is usually even wider.
This is because every cause
that can cause such differences for sequential programs
is present in parallel programs as well,
while parallel programs also have several causes unique to them.
% When writing software that uses parallelism,
% understanding how the program runs is more difficult.
% Parallelism may be used in different parts of the program,
% causing some parts to speed up while other parts remain sequential.

Consider a program containing a loop with many iterations
in which the iterations do not depend on one another
and can thus be done in parallel.
% is an independent parallel task.
Actually executing every iteration in parallel
may generate an overwhelming number of parallel tasks.
It may be that the average amount of computation done by one of these tasks
is less than the amount of work it takes to spawn one of them off.
In that case, executing all the iterations in parallel
will increase overheads to the point of actually slowing the program down,
perhaps quite substantially.
The usual way to correct this is to use granularity control
to make each spawned-off task execute
not one but several iterations of the loop,
thus making fewer but larger parallel tasks.
However, if this is done too aggressively,
performance may once again suffer.
For example,
there may be fewer tasks created than the computer has processors,
causing some processors to be idle when they could be used.
More commonly, the iterations of the loop and thus the tasks
may each need a different amount of CPU time.
If there are eight processors and eight unevenly sized tasks,
then some processors will finish their work early,
and be idle while waiting for the others.

All these problems can arise in programs with independent parallelism:
programs in which parallel tasks do not need to communicate.
Communication between tasks makes this situation even more complicated,
especially when a task may block waiting for information from another task.
A task that blocks may in turn further delay
other tasks that depend on data \emph{it} produces.
Chains of tasks that produce and consume values from one another are common,
Programmers need tools to help them
identify and understand performance problems,
whether they result from such dependencies or other causes.

Profiling tools that can help application programmers
can also be very useful for the implementors of runtime systems.
While optimizing Mercury's parallel runtime system,
we have needed to measure the costs of certain operations
and the frequency of certain behaviors.
Some examples are:
when a piece of work is made available,
how quickly can a sleeping worker-thread respond to this request?
When a task is made runnable after being blocked,
how often will it be executed on the same CPU that was previously executing it,
so that its cache has a chance to be warm?
Such information from profiles of typical parallel programs
can be used to improve the runtime system,
which can help improve the performance of \emph{all} parallel programs.

A third category of people who can use profiling tools for parallel programs
is researchers working on automatic parallelization tools.
Doing a good job of automatically parallelizing a program
requires a cost-benefit analysis for each parallelism opportunity,
which requires estimates of both the cost and the benefit of each opportunity.
Profiling tools can help researchers calibrate
the algorithms they use to generate estimates of both costs and benefits.
% For example, if it is more expensive to spawn off a computation
% than it is to execute it locally, we prefer to execute it locally.
% To answer this we need to know how expensive
% it is to spawn off the computation as a new task.
% We want to use profiling data from the parallel runtime to anwser this.

As researchers working on the parallel implementation of Mercury~\cite{jlp},
a pure logic programming language designed for
the creation of large, reliable and efficient application programs,
we fall into all three of the above categories.
We have long needed a tool
to help us understand the behavior of parallel Mercury programs,
of the parallel Mercury runtime system
and of our autoparallelization tool \cite{overlap},
but the cost of building such a tool seemed daunting.
We therefore looked around for alternative approaches.
The one we selected was to adapt to Mercury
an existing profiler for another parallel system.

The profiler we chose was ThreadScope~\cite{threadscope},
a profiler built by the team behind the Glasgow Haskell Compiler (GHC)
for their parallel implementation of Haskell.
We chose ThreadScope because
the runtime system it was designed to record information from
has substantial similarities to the Mercury runtime system,
because it was \emph{designed} to be extensible,
and, like many visualization tools, it is extremely useful
for finding subtle issues that cause significant problems.

% This paper presents the adaption of ThreadScope~\cite{threadscope}
% --- a tool for profiling the parallel execution of Haskell programs
% --- to Mercury~\cite{jlp} --- a pure logic programming language.
% We will use the above three types of users to determine the usefulness of our
% work.

The structure of the paper is as follows.
Section~\ref{sec:background} gives the background
needed for the rest of the paper:
it introduces Mercury,
the machinery involved in the parallel execution of Mercury programs,
and the ThreadScope profiler.
Section~\ref{sec:newevents}
describes how we extended the ThreadScope system
to collect the kinds of data needed to describe
the parallel execution of Mercury programs,
while section~\ref{sec:analysis} describes how
one can analyze that data to yield insights
that can be useful to our three audiences:
application programmers,
runtime system implementors,
and the implementors of autoparallelization tools.
Section~\ref{sec:conc}
concludes with a discussion of related work.

\section{Background}
\label{sec:background}

\subsection{Mercury}
\label{sec:backmer}

% (The handling of each of the omitted constructs, e.g.\ method calls,
% is analogous to one of the constructs that \emph{is} listed in the figure.)
% \paul{And here, the reader doesn't even need to know this.}

% \vspace{-2mm}
% % \begin{figure}[htb]
% $$
% \begin{array}{lll}
% \mbox{pred}~P
%     & :~ p(x_1, \ldots, x_n)~\leftarrow~G
%         & \hbox{predicates} \\
% %   & |~ f(x_1, \ldots, x_n)=r~\leftarrow~G
% %       & \hbox{functions} \\
% \mbox{goal}~G
%     & :~ x = y ~|~ x = f(y_1,~\ldots,~y_n)
%         & \hbox{unifications}\\
%     & |~ p(x_1,~\ldots,~x_n) ~|~ x_0(x_1,~\ldots,~x_n)
%         & \hbox{first and higher order calls} \\
% %   & |~ \hbox{foreign}(p, \hbox{\emph{foreign code}},
% %       & \\
% %   & ~~~ [x_1:~f_1,~\ldots,x_n:~f_n])
% %       & \hbox{foreign code} \\
%     & |~ (G_1,~\ldots,~G_n) ~|~ (G_1~\&~\ldots~\&~G_n)
%         & \hbox{seq and par conjunctions}\\
%     & |~ (G_1 ; \ldots ; G_n) ~|~ \hbox{switch}~x~(\ldots;~f_i: G_i ; \ldots)
%         & \hbox{disjunctions and switches}\\
%     & |~ (if~G_c~then~G_t~else~G_e) ~|~ not~G
%         & \hbox{if-then-elses and negations}\\
%     & |~ some~[x_1,\ldots,x_n]~G
%         & \hbox{quantifications}\\
% \end{array}
% $$
% % \caption{The abstract syntax of Mercury}
% \label{fig:abstractsyntax}
% % \end{figure}
% \vspace{-1mm}

Mercury is a pure logic programming language.
The atomic constructs of Mercury are unifications,
% (which the compiler breaks down until they contain
% at most one function symbol each),
plain first-order calls,
and higher-order calls.
% and calls to predicates defined by code in a foreign language (usually C).
% \paul{foreign code construct is not last}
% To allow inlining, the representation of the last construct includes
% not just the name of the predicate being called
% but also the foreign code that is predicate's definition
% and the mapping from the Mercury variables that are the call's arguments
% to the names of the variables that stand for them in the foreign code.
Its composite constructs are
sequential and parallel conjunctions, disjunctions, switches,
if-then-elses, negations and existential quantifications.
A switch is a disjunction in which
each disjunct unifies the same bound variable
with a different function symbol;
the others should all be self-explanatory.
% The abstract syntax does not include universal quantification:
% they are allowed at the source level,
% but the compiler turns them into
% combinations of existential quantifications and negations.
% PURITY We will discuss and purity assertions below.

Mercury has a strong mode system.
The mode system classifies each argument of each predicate
as either input or output;
there are exceptions, but they are not relevant to this paper.
If input, the caller must pass a ground term as the argument.
If output, the caller must pass a distinct free variable,
which the predicate will instantiate to a ground term.
% Modes may also require an input argument to be a unique reference,
% and may promise an output argument to be unique.
% \peter{Can we leave uniqueness out of this?}
It is possible for a predicate to have more than one mode;
% the usual example is \code{append}, which has two principal modes:
% \code{append(in,in,out)} and \code{append(out,out,in)}.
we call each mode of a predicate a \emph{procedure}.
The compiler generates separate code
for each procedure of a predicate.
%% (In fact, different procedures are handled as separate entities by
%% most parts of the Mercury debugger and by all parts of the compiler
%% after mode checking.
The mode checking pass of the compiler is responsible for
reordering conjuncts (in both sequential and parallel conjunctions)
as necessary to ensure that for each variable shared between conjuncts,
the goal that generates the value of the variable (the \emph{producer})
comes before all goals that use this value (the \emph{consumers}).
This means that for each variable in each procedure,
the compiler knows exactly where that variable is bound to a ground term.

% INVARIANTS \noindent
Each procedure and goal has a determinism,
which may put upper and lower bounds on the number of its possible solutions
(in the absence of infinite loops and exceptions).
A determinism may impose an upper bound of one solution,
and it may impose a lower bound of one solution.
\emph{det} procedures succeed exactly once;
\emph{semidet} procedures succeed at most once;
\emph{multi} procedures succeed at least once;
\emph{nondet} procedures may succeed any number of times.

% \noindent
% A complete description of Mercury
% can be found in the language reference manual \cite{mercuryrefman}.

\subsection{Parallelism in Mercury}
\label{sec:backpar}

The Mercury runtime system has a construct called a Mercury \emph{engine}
that represents a virtual CPU.
Each engine is independently schedulable by the OS, usually as a POSIX thread.
The number of engines that a parallel Mercury program will allocate on startup
is configurable by the user,
but it defaults to the actual number of CPUs.
Another construct in the Mercury runtime system is a \emph{context},
which represents a computation in progress.
An engine may be idle, or it may be executing a context;
a context can be running on an engine, or it may be suspended.
% Each context has its own stack
% (actually more than one, since ...)
When a context finishes execution,
its storage is put back into a pool of free contexts for later reuse.
Following \cite{multicorerts},
we use \emph{sparks} to represent goals that have been spawned off
but whose execution has not yet been started.

The only parallel construct in Mercury is parallel conjunction:
$(G_1~\&~\ldots~\&~G_n)$.
All the conjuncts must be deterministic,
that is, they must all have exactly one solution.
This restriction greatly simplifies the implementation,
since it guarantees that there can never be any need
to execute $(G_2~\&~\ldots~\&~G_n)$ multiple times,
just because $G_1$ has succeeded multiple times.
(Any local backtracking inside $G_1$ will not be visible to the other conjuncts;
bindings made by det code are never retracted.)
% The current Mercury implementation supports parallelism only for det code,
% since supporting it for code that may have no solution
% would represent speculative execution,
% while supporting for code that may have more than one solution
% would require significant new infrastructure for managing bindings.
In practice, we have not found this to be a significant limitation.
% Since the design of Mercury strongly encourages deterministic code,
% in our experience, about 75 to 85\% of all Mercury procedures are det,
% and most programs spend an even greater fraction of their time in det code.
% % \peter{Do we have any statistics regarding what proportion of execution time
% %   is spent in det code?  That would be a more relevant statistic.}
% Existing algorithms for executing nondeterministic code in parallel
% have very significant overheads, generating slowdowns by integer factors.
% Thus we have given priority to parallelizing deterministic code,
% which we can do with \emph{much} lower overhead.
% We think that avoiding such slowdowns is a good idea,
% even if it does mean foregoing the parallelization of 10 to 25\% of a program.

The Mercury compiler implements $(G_1~\&~G_2~\&~\ldots~\&~G_n)$
by creating a data structure representing a barrier,
and then spawning off $(G_2~\&~\ldots~\&~G_n)$ as a spark.
% \paul{In the future, these queues need to be owed by engines. this
% simplifies stealing.
% }
Since $(G_2~\&~\ldots~\&~G_n)$ is itself a parallel conjunction,
it is handled the same way:
the context executing it
first spawns off $(G_3~\&~\ldots~\&~G_n)$ as a spark,
and then executes $G_2$ itself.
Eventually, the spawned-off remainder of the conjunction
consists only of the final conjunct, $G_n$,
and the context just executes it.
The code of each conjunct synchronizes on the barrier once it has
completed its job.
When all conjuncts have done so,
the original context will continue execution after the parallel conjunction.

Mercury's mode system allows a parallel conjunct to consume variables
that are produced by conjuncts to its left, but not to its right.
This guarantees the absence of circular dependencies
and hence the absence of deadlocks between the conjuncts,
but it does allow a conjunct to depend on data that is yet to be computed
by a conjunct running in parallel.
We handle these dependencies through a source-to-source transform
\cite{wangdepparconj}.
The compiler knows which variables
are produced by one parallel conjunct and consumed by another.
For each of these shared variables,
it creates a data structure called a \emph{future} \cite{multilisp}.
When the producer has finished computing the value of the variable,
it puts the value in the future and signals its availability.
When a consumer needs the value of the variable,
it waits for this signal (if it has not yet happened),
and then retrieves the value from the future.
% Both operations are protected by mutexes.

To minimize waiting,
the compiler pushes signal operations on each future
as far to the left into its producer conjunct as possible,
and it pushes wait operations
as far to the right into each of its consumer conjuncts as possible.
This means not only pushing them
into the body of the predicate called by the conjunct,
but also into the bodies of the predicates they call,
with the intention being that
each signal is put immediately after
the primitive goal that produces the value of the variable,
and each wait is put immediately before
the leftmost primitive goal that consumes the value of the variable.
% Since the compiler has complete information
% about which goals produce and consume which variables,
% the only things that can stop the pushing process are
% higher order calls and module boundaries:
% the compiler cannot push a wait or signal operation
% into code it cannot identify or cannot access.
% % into the body of a predicate
% % if it does not have access to the identity or to the body of the predicate.

\subsection{ThreadScope}

ThreadScope was originally built
to help programmers visualize the parallel execution of Haskell programs
compiled with the dominant implementation of Haskell,
the Glasgow Haskell Compiler or GHC.
The idea is that during the execution of a parallel Haskell program,
the Haskell runtime system writes time-stamped reports
about significant events to a log file.
The ThreadScope tool later reads this log file,
and shows the user graphically what each CPU was doing over time.
The diagrams it displays reveal to the programmer
the places where the program is getting the expected amount of parallelism,
as well as the places where it isn't.

We could adapt the ThreadScope system to Mercury for two main reasons.
The first is that the parallel implementations of Haskell and Mercury
in GHC and mmc (the Melbourne Mercury compiler)
are quite similar in several important respects,
even though they use different terms for
(slightly different implementations of) the same concepts.
For example, what Mercury calls an engine GHC calls a \emph{capability},
and what Mercury calls a context GHC calls a \emph{thread}.
(We will use Mercury terminology in the rest of the paper.)
The second reason is that
the ThreadScope log file format was designed to be extensible.
Each log file starts with a description
of each kind of event that may occur in it.
This description lists the arguments of the event and their total size,
which may be variable.
%\zoltan{Is this true, or can only one argument be variable size?}
%\paul{Only one argument can have a variable size, unless extra
%  argument are added to the event to describe $N-1$ of the
%  variable sizes, but this is not important enough to include in the
%  paper}
This description is precise enough that tools 
can skip events, and even of events, that they do not understand,
and process the rest.

Mercury is able to make use of a number of event types already supported by
ThreadScope and GHC,
in other cases we are able to add support for Mercury-specific event types to
ThreadScope.
Here we introduce the events in the first category,
The next section describes those in the second category.
Events of all types have a timestamp
that records the time of their occurrence, measured in nanoseconds.
This unit illustrates the level of precision ThreadScope aims for,
although of course there is no guarantee
that the system clock is capable of achieving it.

Most events in the log file are associated with an engine.
To avoid having to include an engine id with every event in the log file,
the log file format
groups sequences of events that are all from the same engine into a block,
and writes out the engine id just once, in a block header pseudo-event.
Since tools reading the log file can trivially remember
the engine id in the last block header they have read,
this makes the current engine id
(the id of the engine in whose block an event appears)
an implicit parameter of most event types.

The event types supported by the original version of ThreadScope 
that are relevant to our work are the following.

\emph{STARTUP:
Marks the beginning of the execution of the program,
and records the number of engines that the program will use.}

\emph{SHUTDOWN:
Marks the end of the execution of the program.}

\emph{CREATE\_THREAD:
Records the act of the current engine creating a context,
and gives the id of the context being created.}
(What Mercury calls a context Haskell calls a thread;
the name of the event uses Haskell terminology.)
Context ids are allocated sequentially,
but the parameter cannot be omitted,
since the runtime system can and will reuse the storage of a context
after the termination of the computation that has previously used it.

\emph{RUN\_THREAD:
Records the scheduling event
of the current engine switching to execute a context,
and gives the id of the context being switched to.}

\emph{STOP\_THREAD:
Records the scheduling event
of the current engine switching away from executing a context.}
Gives the id of the context being switched from,
as well as the reason for the switch.
The possible reasons include:
(a) the heap is full, and so the engine must invoke the garbage collector;
(b) the context has blocked, and
(c) the context has finished.

\emph{THREAD\_RUNNABLE:
Records that the current engine has made a blocked context runnable,
and gives the id of the newly-runnable context.}

\emph{RUN\_SPARK:
Records that the current engine is starting to run a spark
that it retrieved from its own local spark queue.}
Gives the id of the context that will execute the spark,
although this can be inferred by context.
% \zoltan{It can be inferred from the context id of the next
% CREATE\_SPARK\_THREAD event in Mercury,
% is this true in Haskell as well?}

\emph{STEAL\_SPARK:
Records that the current engine will run a spark
that it stole from the spark queue of another engine.}
Gives the id of the context that will execute the spark,
although again, this can be inferred by context.
% \zoltan{It can be inferred from the context id of the next
% CREATE\_SPARK\_THREAD event in Mercury,
% is this true in Haskell as well?}
Also gives the id of the engine that the spark was stolen from.
% This should really be called STOLE_SPARK.

\emph{CREATE\_SPARK\_THREAD:
Records the creation of a new context or the reuse of an old context
in order to execute a spark.}
Gives the id of the context.
Does not say whether the context is new or reused.
In most cases, that information is simply not needed,
but if it is, it can be inferred from context.
% In both Mercury and Haskell,
% the context will probably be used to execute other sparks
% once it finishes the current one.

\emph{GC\_START:
The current engine has initiated garbage collection;
control of the engine has been taken away from the mutator.}

\emph{GC\_END:
Garbage collection has finished on the current engine;
control of the engine has been returned to the mutator.}
% \paul{I don't know the intended semantics of the GC events that Haskell uses}
% More information about how Mercury uses these events is in Section
% \ref{sec:gc_stats} below.

% The concepts of \emph{engine} and \emph{capability} are the same in
% Mercury and Haskell respectively.
% so are the concepts of Contexts and Threads.
% Mercury and Haskell both use sparks, although in Haskell sparks do not need to
% be evaluated, and may be garbage collected.
% Because Mercury is strict sparks do not represent thunks and therefore
% sparks must be executed in order to write their
% results into their parent's stack frame.

% Mercury also adds the concept of futures (Section \ref{sec:futures}),
% Haskell does not need futures since any unevaluated value is a thunk
% and can be evaluated directly by the code trying to read it's value.
% If the thunk is already being evaluated, the thread will block;
% otherwise, the thread will evaluate the thunk itself.

\noindent
The longer a parallel program being profiled runs,
the more events it will generate.
Long running programs can generate enormous log files.
To keep the sizes of log files down as much as possible,
events include only the information they have to.
If some information about an event
can be inferred from information recorded for other events,
then the design principles of ThreadScope say
that information should be inferred at analysis time
rather than recorded at profiling runtime.
(The presence of context ids in RUN\_SPARK and STEAL\_SPARK events
violates this principle,
since they are guaranteed to be the same as the id of the context already
running on that engine or
the context id in the following CREATE\_SPARK\_THREAD event.
but their removal is prevented by backwards compatibility concerns.)
In most cases, the required inference algorithm is quite simple.
For example, answering the question of whether the context
mentioned by a CREATE\_SPARK\_THREAD event is new or reused
merely requires searching the part of the log up to that event
looking for mentions of the same context id.
And we have already seen how the engine id parameter
missing from most of the above events
can be deduced from block headers.

\section{New events}
\label{sec:newevents}

In order to support the profiling of parallel Mercury programs,
we had to add new arguments to two of the existing ThreadScope events.
Originally, the RUN\_SPARK and STEAL\_SPARK events
did \emph{not} specify the identity of the spark being run or stolen.
This is not a problem for the Haskell version of ThreadScope,
since it does not care about sparks' identities.
However, we do, since sparks correspond to conjuncts in parallel conjunctions,
and we want to give to the user not just general information
about the behavior of the program as a whole,
but also about the behavior of individual parallel conjunctions,
and of the conjuncts in them.
We have therefore added an id that uniquely identifies each spark
as an additional argument of the RUN\_SPARK and STEAL\_SPARK events.
Note that CREATE\_SPARK\_THREAD does not need a spark id,
since the only two ways it can get the spark it converts into a context
is by getting it from its own queue or from the queue of another engine.
A CREATE\_SPARK\_THREAD event will therefore always be preceded
by either a RUN\_SPARK event or a STEAL\_SPARK event,
and the id of the spark in that event
will be the id of the spark being converted.

% In Haskell, their identities can always be inferred
% by modelling the spark queues of all the capabilities (engines) in the system.

We have extended the set of ThreadScope event types
to include several new types of events.
Most of these record information about constructs that do not exist in Haskell:
parallel conjunctions, conjuncts in those conjunctions, and futures.
Some provide information about the behavior of Mercury engines
that ThreadScope for Haskell does not need,
either because that information is of no interest,
or because the information is of interest
but it can be deduced from other events.
Even though the Haskell and Mercury runtime systems
generate many of the same events,
the stream of these common events they generate
do not necessarily obey the same invariants.

However, most of the work we have done towards adapting ThreadScope to Mercury
has been in the modification of the parallel Mercury runtime system
to generate all of the existing, modified and new events when called for.

In the rest of this section,
we report the event types we have added to ThreadScope.
In the next section, section \ref{sec:analysis},
we will show how the old and new events can be used together
to infer interesting and useful information
about the behavior of parallel Mercury programs.

% \item[STRING] Record an association between a String and an integer so
% that a static string occurring many times within a log file can be
% replaced with an integer.
% Making the log files shorter and therefore not flushing them to disk
% as often.

\emph{START\_PAR\_CONJUNCTION: records the fact that
the current engine is about to start executing a parallel conjunction.}
It identifies the parallel conjunction in two different ways.
The static id identifies
the location of the conjunction in the source code of the program.
The dynamic id is the address of the barrier structure
that all the conjuncts in the conjunction will synchronize on when they finish.
Note that barrier structures may be garbage collected,
and their storage reused for other barriers used by later code,
so these dynamic ids are not unique across time,
but they \emph{do} uniquely identify a parallel conjunction
at any given moment in time.
See the end of this section for a discussion of why we chose this design.

\emph{END\_PAR\_CONJUNCTION:
records the end of a parallel conjunction}.
It gives the dynamic id of the finishing conjunction.
Its static id can be looked up in the matching START\_PAR\_CONJUNCTION event.

\emph{CREATE\_SPARK: records the creation of a spark
for a parallel conjunct by the current engine.}
% \paul{XXX We call this SPARKING in the source code but shouldn't}
Gives the id of the spark itself,
and the dynamic id of the parallel conjunction the conjunct is from.
To keep the log file small,
it does \emph{not} give the position of the conjunct in the conjunction.
However, since in every parallel conjunction
the spark for conjunct $n+1$
will be created by the context executing conjunct $n$,
and unless $n=1$, that context will itself have been created
from the spark for conjunct $n$.
(The first conjunct of a parallel conjunction is always executed directly,
without ever being represented by a spark.)
This means that the sparks for the non-first conjuncts
will always be created in order,
which makes it quite easy to determine which spark represents which conjunct.

The id of a spark is an integer consisting of two parts.
% spark IDs are 32 bits wide, the 8 most significant bits are the engine ID,
The first part is the number of the engine that created the spark,
and the second is an engine-specific sequence number,
so that in successive sparks created by the same engine,
the second part will be 1, 2, 3 etc.
This design allows the runtime system
to allocate globally unique spark ids without synchronization.

\emph{END\_PAR\_CONJUNCT:
records that the current engine
has finished the execution of a conjunct in a parallel conjunction.}
Gives the dynamic id of the parallel conjunction.
Note that there is no START\_PAR\_CONJUNCT event
to mark the start of execution of any conjunct.
For the first conjunct, its execution will start
as soon as the engine has finished recording the START\_PAR\_CONJUNCTION event.
The first thing the conjunct will do is create a spark
representing the second and later conjuncts
(which above we informally referred to as the spark for the second conjunct).
The CREATE\_SPARK event records the id of the spark,
then, either the RUN\_SPARK or STEAL\_SPARK event records which engine and context
executes the spark,
If the engine doesn't yet have a context a CREATE\_SPARK\_THREAD event is posted
that identifies the newly created context.
RUN\_THREAD and STOP\_THREAD events
also tell us when that context is executing, and on which engine.
The first thing the second conjunct does is spawn off a spark
representing the third and later conjuncts.
By following these links,
a single forward traversal of the log file can find out,
for each engine, which conjunct of which dynamic parallel conjunction
it is running at any given time
(if in fact it is running any;
an engine can be idle,
or it may be running code that is outside of any parallel conjunction).
When an engine records an END\_PAR\_CONJUNCT event,
it can only be for the conjunct it is currently executing.

\emph{FUTURE\_CREATE: records the creation of a future by the current engine.}
Gives the name of the variable the future is for,
as well as the dynamic id of the future.
It does not give the dynamic id of the parallel conjunction
whose conjuncts the future is intended to synchronize,
but that can be found out quite easily:
just scan forward for the next START\_PAR\_CONJUNCTION event.
The reason why this inference works is that
the code that creates futures
is only ever put into programs by the Mercury compiler,
and mmc never puts that code anywhere
except just before the start of a parallel conjunction.
If a parallel conjunction uses $n$ futures,
then every one of its START\_PAR\_CONJUNCTION events
will be preceded by $n$ FUTURE\_CREATE events,
one for each future.

The dynamic id of the future is its address in memory.
This has the same issues with reuse
as the dynamic ids of parallel conjunctions.

\emph{FUTURE\_SIGNAL: records that code running on the current engine
has signalled the availability
of the value of the variable protected by a given future.}
Gives the id of the future.
If another conjunct is already waiting for the value stored in this future,
this event will unblock its context.

\emph{FUTURE\_WAIT\_NO\_SUSPEND: records that the current engine
retrieved the value of the future without blocking.}
Gives the id of the future.
The future's value was already available
when the engine tried to wait for the value of the future,
so the context running on the current engine was not blocked.

\emph{FUTURE\_WAIT\_SUSPEND: records that the current engine
tried to retrieve the value of a future, but was blocked.}
Gives the id of the future.
Since the future's value was not already available,
the current context has been be suspended until it is.

We have added a group of four event types
that record the actions of engines
that cannot continue to work on what they were working before,
either because the conjunct they were executing finished,
or because it suspended.

\emph{TRY\_GET\_RUNNABLE\_CONTEXT: This engine is checking the global
run queue for a context to execute.}

\emph{TRY\_GET\_LOCAL\_SPARK: This engine is attempting to get a spark
from its own stack.}

\emph{TRY\_STEAL\_SPARK: This engine is attempting to steal a spark
from another engine.}

\emph{ENGINE\_WILL\_SLEEP: The engine is about to sleep.}
The next event from this engine will be from when it is next awake.

The idea is that an idle engine can look for work in three different places:
(a) the global runnable context queue,
(b) its own local spark queue, and
(c) another engine's spark queue.
An idle engine will try these in some order,
the actual order depending on the scheduling algorithm.
The engine will post the try event for a queue
before it actually looks for work in that queue.
If one of the tests is successful, the engine will post
a START\_THREAD event, a RUN\_SPARK event or a STEAL\_SPARK event respectively.
If one of the tests fails, the engine will go on the next.
If it does not find work in any of the queues,
it will go to sleep after posing the ENGINE\_WILL\_SLEEP event.

This design uses two events to record
the successful search for work in a queue:
TRY\_GET\_RUNNABLE\_CONTEXT and START\_THREAD,
TRY\_GET\_LOCAL\_SPARK and RUN\_SPARK,
or TRY\_STEAL\_SPARK and STEAL\_SPARK.
This may seem wasteful compared to using one event,
but this design enables us to measure several interesting things:
\vspace{-2mm}
\begin{itemize}
\item
how quickly a sleeping engine can wake up and try to find work
when work is made available by a CREATE\_SPARK and or CONTEXT\_RUNNABLE event;
% This can remove the space between items, as well as above and below
% the environment, I like the look of -0.5
% \vspace{-1\baselineskip}
\item
how often an engine is successful at finding work;
% \vspace{-1\baselineskip}
\item
when it is successful,
how long it takes an engine to find work and begin its execution;
% \vspace{-1\baselineskip}
\item
when it is unsuccessful,
how long it takes to try to find work from other sources,
fail, and then to go to sleep.
\end{itemize}
\vspace{-2mm}

% I've removed the idle and working events.
% I spoke with Simon Marlow about this, I can't construct a sound
% argument why I need them, and in all cases the state of an engine
% can be inferred by other events.  Although this inference feels
% dirty/clumsy it is the correct option since it keeps the log file
% shorter.
% The problem now is that Haskell and Mercury both behave differently,
% and therefore may have different rules of inference.

\noindent
As we mentioned above,
using the address of the barrier structure
as the dynamic id of the parallel conjunction whose end the barrier represents
and using the address of the future structure
as the dynamic id of the future
are both design choices.
The obvious alternative would be to give them both sequence numbers
using either global or engine-specific counters.
However, this would require
adding a field to both structures to hold the id, which costs memory, and
filling in the field, which costs time.
Both these costs would be incurred by the program execution
whose performance we are trying to measure,
interfering with the measurement itself.
While such interference cannot be avoided
(writing events out to the log file also takes time),
we nevertheless want to minimize it if at all possible.
In this case, it is possible:
the addresses of the structures are trivially available,
and require no extra time or space during the profiled run.
The tradeoff is that we now need a pre-pass over the log file
that consistently renames apart
the different incarnations of the same id for a parallel conjunction or future.

\tr{
The tradeoff is that if an analysis requires globally unique dynamic ids
for parallel conjunctions or for futures,
and most analyses do,
then it needs to ensure that a pre-pass has been run over the log file.
This pre-pass would maintain a map from future ids
to a pair containing an active/inactive flag, and a replacement id,
and another map from dynamic conjunction ids
to a tuple containing an active/inactive flag, a replacement id,
and a list of future ids.
When the pre-pass sees a dynamic conjunction id or future id in an event,
it looks it up in the relevant table.
If the flag says the id is active,
it replaces the id with the replacement.
If the flag says the id is inactive,
it gets a new replacement id (e.g.\ by incrementing a global counter),
and sets the flag to active.
If the event is a FUTURE\_CREATE event,
the pre-pass adds the future's original id to a list.
If the event is a START\_PAR\_CONJUNCTION event,
it copies this list to the conjunction's entry, and then clears the list.
If the event is an END\_PAR\_CONJUNCTION event,
which ends the lifetime not only of the parallel conjunction
but also of all futures created for that conjunction,
the pre-pass sets to inactive
that flags of both the conjunction itself
and the futures listed in its entry.

This algorithm consistently renames apart
the different incarnations of the same id,
replacing them with globally unique values.
Its complexity can be close to linear, provided
the maps are implemented using a suitable data structure, such as a hash table.
This transformation does not even need an extra traversal of the log file data.
The ThreadScope tool must traverse the log file anyway
when it gets ready to display its contents,
and this transformation can be done
as part of that traversal.
The extra cost is therefore quite small.
}

\section{Deriving metrics from events}
\label{sec:analysis}

ThreadScope is primarily used to visualize the execution of parallel programs.
However, we can also use it, along with our new events,
to calculate and to present to the user
a number of different metrics about the execution of parallel Mercury programs.
% \zoltan{needs discussion about how metrics are presented to users}
% \zoltan{needs discussion about audiences}
Some of the metrics are of interest
only to application programmers,
or only to runtime system implementors,
or only to autoparallelization tool implementors,
but several are useful to two or even all three of those audiences.
Also, some of these metrics say something about the whole program;
some say something about individual parallel conjunctions,
and some say something about individual conjuncts inside those conjunctions.
We discuss our proposed metrics in that order;
we then discuss how we intend to present them to users.

% Such as how much overlap we're \emph{really} getting,
% and how efficient the RTS is and how this can drive settings in our
% automatic parallelization analysis.
% 
% Discuss granularity and parallel slackness, show how we can measure these.

\subsection{Whole program metrics}

% \emph{PARCONJ\_CONTEXT\_SELF:
% The number of contexts that are actually used
% by a given parallel conjunction.}
% This measurement is computed similarly to PARCONJ\_RUNNABLE\_SELF,
% but it doesn't count the part of a conjunct's lifetime
% that it spends as a spark;
% it counts only the part that it spends as a context.
% The difference is important,
% because contexts occupy substantial amounts of memory,
% while sparks do not.

% For optimum memory efficiency,
% the ratio between the time-weighted averages of
% PARCONJ\_CONTEXT\_SELF and PARCONJ\_RUNNABLE\_SELF
% should be as low as possible for all parallel conjuncts.

\emph{CPUS\_OVER\_TIME:
The number of CPUs being used by the program at any given time.}
It is trivial to scan all the events in the trace,
keeping track of the number of engines currently being used,
each of which corresponds to a CPU.
The resulting curve tells programmers
which parts of their program's execution is already sufficiently parallelized,
which parts are parallelized but not yet enough to use all available CPUs,
and which parts still have no parallelism at all.
They can then focus their efforts on the latter.
The visualization of this curve has been implemented,
and is described by \cite{threadscope}.

% number of sparks available for execution at any given time
% number of runnable but not running contexts at any given time
% number of sparks plus runnable but not running contexts at any given time

% \label{sec:gc_stats}
\emph{GC\_STATS: The number of garbage collections,
and the average, minimum, maximum and variance
of the elapsed time taken by each collection.}
Mercury uses the Boehm-Demers-Weiser garbage collector~\cite{boehm}.
This collector supports parallel marking using GC-specific helper threads
rather than the OS threads that run Mercury engines.
Therefore, even when parallel marking is used,
we can only calculate the elapsed time used by garbage collection,
and not the \emph{cpu time} used by garbage collection.
The elapsed time of each garbage collection
is the interval between pairs of GC\_START and GC\_END events.

\emph{MUTATOR\_VS\_GC\_TIME:
The fraction of the program's runtime used by the garbage collector.}
We calculate this by summing the times
between pairs of GC\_START and GC\_END events,
and dividing the result by the program's total runtime.
(Both sides of the division refer to elapsed time, not CPU time.)
Due to Amdahl's law~\cite{amdahlslaw},
this fraction limits the best possible speedup we can get
for the program as a whole by parallelizing the mutator.
For example, if the program spends one third of its time doing GC
(which unfortunately actually happens for some parallel programs),
then no parallelization of the program can yield a speedup of more than three,
\emph{regardless} of the number of CPUs available.

\emph{NANOSECS\_PER\_CALL:
The average number of nanoseconds between successive procedure calls.}
The Mercury profiler for sequential programs~\cite{conway:2001:mercury-deep}
(which has nothing to do with ThreadScope)
measures time in call sequence counts (CSCs);
the call sequence counter is incremented at every procedure call.
It does this because there is no portable, or even semi-portable way
to access any realtime clocks that may exist on the machine,
and even the non-portable method on x86 machines (the RDTSC instruction)
is too expensive for it.
(The Mercury sequential profiler needs to look up the time at every call,
which in typical programs will happen every couple of dozen instructions or so.
The Mercury parallel profiler needs to look up the time only at each event,
which occur \emph{much} more rarely.)
% RDTSC still has some overhead:
% a) It's a sequencing operation for the CPU.
% b) It requires some arithmetic and maybe a function call, I don't
%    know if we can afford that.

The final value of the call sequence counter
at the end of a sequential execution of a program 
gives its length in CSCs; say $n$ CSCs.
When a parallelized version of the same program is executed on the same data
under ThreadScope,
we can compute the total amount of \emph{user time}
taken by the program on all CPUs; say $m$ nanoseconds.
The ratio $n/m$ gives the average number of nanoseconds
in the parallel execution of the program
per CSC in its sequential execution.
% \paul{I am having trouble parsing the above sentence, I've tried to
% read it 3 times now, I don't know what you're trying to say}
% \zoltan{is the new phrasing clear?}
% Clearer, but not great, I get stuck at ``in its sequential execution''
% but now I know what you mean and this phrase is somewhat implicit.
This is useful information,
because our automatic parallelization system
uses CSCs, as measured by the Mercury sequential profiler,
as its unit of measurement of the execution time 
of both program components and system overheads.
Using this scale, we can convert predictions about time made by the tool
from being expressed in CSCs to being expressed in nanoseconds,
which is an essential first step in comparing them to reality.

\subsection{Conjunction specific metrics}

\emph{PARCONJ\_TIME: The time taken by a given parallel conjunction.}
For each dynamic parallel conjunction id that occurs in the trace,
it is easy to compute the difference between
the times at which that parallel conjunction starts and ends,
and it is just as trivial to associate these time intervals
with the conjunction's static id.
From this, we can compute,
for each parallel conjunction in the program that was actually executed,
both the average time its execution took,
and the variance in that time.
This information can then be compared,
either by programmers or by automatic tools,
with the sequential execution time of that conjunction recorded by
Mercury's sequential profiler,
to see whether executing the conjunction was a good idea or not.
\emph{provided} that the two measurements are done in the same units.
At the moment, they are not, but as we discussed above,
CSCs can be converted to nanoseconds using the NANOSECS\_PER\_CALL metric.

\emph{PARCONJ\_RUNNABLE\_SELF:
The number of CPUs that can be used by a given parallel conjunction.}
For each dynamic parallel conjunction id that occurs in the trace,
we can divide the time between its start and end events into blocks,
with the blocks bounded by
the CREATE\_SPARK\_THREAD and STOP\_THREAD events for its conjuncts,
and the FUTURE\_WAIT\_SUSPEND and FUTURE\_SIGNAL events
of the futures used by those conjuncts.
We can then compute, for each of these blocks,
the number of runnable tasks that these conjuncts represent.
From this we can compute the history of the number
of runnable tasks made available by this dynamic conjunction.
We can derive the maximum and the time-weighted average of this number,
and we can summarize those
across all dynamic instances of a given static conjunction.

Consider a conjunction with $n$ conjuncts.
If it has a maximum number of runnable tasks
that is significantly less than $n$,
this is a sign that the parallelism the programmer aimed for
could not be achieved.
If instead the average but not the maximum number of runnable tasks
is significantly less than $n$,
this suggests that the parallelism the programmer aimed for was achieved,
but only briefly.
The number of runnable tasks can drop due to
either a wait operation that suspends or the completion of a conjunct.
Both dependencies among conjuncts
and differences in the execution times of the conjuncts
limit the amount of parallelism available in the conjunction.
The impact of the individual drops on the time-weighted average
shows which effects do the most limiting in any given parallel conjunction.

\emph{PARCONJ\_RUNNABLE\_SELF\_AND\_DESC:
The number of CPUs that can be used by a given parallel conjunction
and its descendants.}
This metric is almost the same as PARCONJ\_RUNNABLE\_SELF,
but it operates
not just on a given dynamic parallel conjunction,
but also on the parallel conjunctions
spawned by the call-trees of its conjuncts as well.
It takes their events into account when it divides time into blocks,
and it counts the number of runnable tasks they represent.

The two metrics,
PARCONJ\_RUNNABLE\_SELF and PARCONJ\_RUNNABLE\_SELF\_AND\_DESC,
can be used together to see whether
the amount of parallelism that can be exploited
by the two parallel conjunctions together is substantially greater than
the amount of parallelism that can be exploited
by just the outer parallel conjunction alone.
If it is, then executing the inner conjunction in parallel is a good idea;
if it is not, then it is a bad idea.

If the outer and inner conjunction in the dynamic execution
come from different places in the source code,
then acting on such conclusions is relatively straightforward.
If they represent different invocations of the same conjunction in the program,
which can happen if one of the conjuncts contains a recursive call,
then acting on such conclusions will typically require
the application of some form of runtime granularity control.

% \paul{The number of parallel conjunctions in this case is as deep as
% the recursion, not just two.
% Can you explain how considering the first two levels of the recursion
% bares on all the levels of recursion?
% Or is this a case where a parallel conjunction is nested within
% another and no code within either conjunction is recursive?  Thanks}

\emph{PARCONJ\_RUNNING\_SELF:
The number of CPUs that are actually used by a given parallel conjunction.}
This metric is computed similarly to PARCONJ\_RUNNABLE\_SELF,
but it doesn't count a runnable conjunct until gets to use a CPU,
and stops counting a conjunct when it stops using the CPU
(when it blocks on a future, and when it finishes).
% \zoltan{Does the CPU ever get taken away from a conjunct?}
% \paul{Only when it blocks on a future, finishes a conjunct or calls
% \code{sched.yield}}

Obviously, PARCONJ\_RUNNING\_SELF can never exceed PARCONJ\_RUNNABLE\_SELF
for any parallel conjunction at any given point of time.
However, one important difference between the two metrics
is that PARCONJ\_RUNNING\_SELF can never exceed
the number of CPUs on the system either.
If the maximum value of PARCONJ\_RUNNABLE\_SELF for a parallel conjunction
does not exceed the number of CPUS,
then its PARCONJ\_RUNNING\_SELF metric
can have the same value as its PARCONJ\_RUNNABLE\_SELF metric,
barring competition for CPUs by other parallel conjunctions (see later)
or by the garbage collector.

On the other hand, some conjunctions
do generate more runnable tasks than there are CPUs.
In such cases, we want the extra parallelism
that the system hardware cannot accommodate
in the period of peak demand for the CPU
to ``fill in'' later valleys,
periods of time when the conjunction demands
less than the available number of CPUs.
This will happen only to the extent that
the delayed execution of tasks that lost the competition for the CPU
does not lead to further delays in later conjuncts
through variable dependencies.

The best way to measure this effect
is to visually compare the PARCONJ\_RUNNING\_SELF curves for the conjunction
taken from two different systems,
e.g.\ one with four CPUs and one with eight.
However, given measurements taken from e.g.\ a four CPU system,
it should also be possible to predict with \emph{some} certainty
what the curve would look like on an eight CPU system,
by using the times and dependencies recorded in the four-CPU trace
to simulate how the scheduler would handle the conjunction
on an eight CPU machine.
Unfortunately, the simulation cannot be exact
unless it correctly accounts for \emph{everything},
including cache effects and the effects on the GC system.

The most obvious use of the PARCONJ\_RUNNING\_SELF curve of a conjunction
is to tell programmers whether and to what extent
that parallel conjunction can exploit the available CPUs.

\emph{PARCONJ\_RUNNING\_SELF\_AND\_DESC:
The number of CPUs that are actually used by a given parallel conjunction
and its descendants.}
This metric has the same relationship to PARCONJ\_RUNNING\_SELF
as PARCONJ\_RUNNABLE\_SELF\_AND\_DESC has to PARCONJ\_RUNNABLE\_SELF.
Its main use is similar to the main use of PARCONJ\_RUNNING\_SELF:
to tell programmers whether and to what extent
that parallel conjunction and its descendants can exploit the available CPUs.
This is important because a conjunction and its descendants
may have enough parallelism
even if the top-level conjunction by itself does not,
and in such cases the programmer can stop looking for more parallelism,
at least in that part of the program's execution timeline.

\emph{PARCONJ\_AVAIL\_CPUS:
The number of CPUs available to a given parallel conjunction.}
Scanning through the entire trace,
we can compute and record the number of CPUs being used
at any given time during the execution of the program.
For any given parallel conjunction,
we can also compute PARCONJ\_RUNNING\_SELF\_AND\_DESC,
the number of CPUs used by that conjunction and its descendants.
By taking the difference between the curves of those two numbers,
we can compute the curve of the number of CPUs
that execute some task \emph{outside} that conjunction,
Subtracting that difference curve
from the constant number of the available CPUs
gives the number of CPUs available for use by this conjunction.

This number's maximum, time-weighted average
and the shape of the curve of its value over time,
averaged over the different dynamic occurrences
of a given static parallel conjunction,
tell programmers the level of parallelism they should aim for.
% \paul{Should there be a new sentence here.  Reading the first half of
% the sentence it sounds like you're talking about increasing the
% amount of parallelism,
% but the second half of the sentence talks about decreasing it --- at
% least that's how I understood it on first impression.}
Generating parallelism in a conjunction
that consistently exceeds the number of CPUs available for that conjunction
is more likely to lead to slowdowns from overheads
than to speedups from parallelism.

\emph{PARCONJ\_CONTINUE\_ON\_BEFORE:
the probability, for any given parallel conjunction,
that the code after the conjunction
will continue executing on the same CPU
as the code before the conjunction.}
If the parallel conjunction as a whole took only a small amount of time,
then CPUs other than the original CPU will still have relatively cold caches,
even if they ran some part of the parallel conjunction.
We want to keep using the warm cache of the original CPU.
The better the scheduling strategy is at ensuring this,
the more effectively the system as a whole will exploit the cache system,
and the better overall system performance will be.

\emph{PARCONJ\_CONTINUE\_ON\_LAST:
the probability, for any given parallel conjunction,
that the code after the conjunction
will continue executing on the same CPU
as the last conjunct to finish.}
If the parallel conjunction as a whole took a large amount of time,
then how warm the cache of a CPU will be
for the code after the conjunction
depends on how recently that CPU executed
either a conjunct of this conjunction or the code before the conjunction.
Obviously, all the conjuncts execute after the code before the conjunction,
so if the conjunction takes long enough for most of the data accessed
before the conjunction to 
% \paul{Cached data doesn't become cold, a cache becomes cold after
% useful data has been removed}.
% become cold and therefore
be evicted from the cache,
then only the CPUs executing the conjuncts will have useful data in
their caches.
% \paul{what race?}
% are in the race.
In the absence of specific information
about the code after the conjunct preferentially accessing data
that was also accessed (read or written) by specific conjuncts,
the best guess is that the CPU with the warmest cache
will be the one that last executed a conjunct of this conjunction.
The more often the scheduling strategy executes
the code after the conjunction on that CPU,
the more effectively the system as a whole will exploit the cache system,
and the better overall system performance will be.

\subsection{Conjunct specific metrics}

There are several simple times
whose maximums, minimums, averages and variances
can be computed for each static parallel conjunct.

\emph{CONJUNCT\_TIME\_AS\_SPARK:
the time between the conjunct's creation
(which will be as a spark) and the start of its execution
(when the spark will be converted into a context).}

\emph{CONJUNCT\_TIME\_AS\_CONTEXT:
the time between the start of the conjunct's execution and its end.}

\emph{CONJUNCT\_TIME\_BLOCKED:
the total amount of time
between the start of the conjunct's execution and its end
that the conjunct spends blocked waiting for a future.}

% We also want a new metric,
% \item[CONJUNCT_TIME_BLOCKED_BARRIER] but discussing this means
% discussing right vs left recursion with Mercury's parallel runtime,
% which like to avoid for this paper.

\emph{CONJUNCT\_TIME\_RUNNABLE:
the total amount of time
between the start of the conjunct's execution and its end
that the conjunct spends runnable but not running.}

\emph{CONJUNCT\_TIME\_RUNNING:
the total amount of time
between the start of the conjunct's execution and its end
that the conjunct spends actually running on a CPU.}

Since every context is always either running, runnable or blocked,
the last three numbers must sum up to the second
(in absolute terms and on average, not in e.g.\ maximum or variance).

Programmers may wish to look at conjuncts
that spend a large percentage of their time blocked
to see whether the dependencies that cause those blocks can be eliminated.

\tr{
If such a dependency cannot be eliminated,
it may still be possible to improve
at least the memory impact of the conjunct
by converting some of the blocked time into spark time.
The scheduler should definitely prefer executing an existing runnable context
over taking a conjunct that is still a spark,
converting the spark into a context and running that context.
When there are no existing runnable contexts
and it must convert a spark into a context,
the scheduler should try to choose a spark whose consumed variables
(the variables it will wait for) are all currently available.
Of course, such a scheduling algorithm is not possible
without information about which variables conjuncts consume,
information that schedulers do not typically have access to,
but which it is easy to give them.

If all the sparks consume at least one variable
that is not currently available,
the scheduler should prefer to execute the one
whose consumed variables are the \emph{closest} to being available.
This requires knowledge of the expected behavior of the program,
to wit, the expected running times of the conjuncts generating those variables.
In some cases, that information may nevertheless be available,
derived from measured previous runs of the program,
although of course it can only ever be a guess.
% \paul{Should there be a comma after although?
% and maybe another after of course?}
% \zoltan{no}

Note also this is only one of several considerations
that an ideal scheduler should take into account.
For example, schedulers should also prefer to execute the conjunct
(whether it is a context or a spark)
that is currently next on the critical path to the end of the conjunction.
Knowledge of the critical path is also knowledge about the future,
and similarly must also be a guess.
Ideally, the scheduler should take into account
all these different considerations before coming to a decision
based on balancing their relevance in the current situation.
}

% When it chooses between several existing runnable contexts,
% ideally it should choose one XXX

% If such a dependency cannot be eliminated,
% the programmer may still be able to improve
% at least the memory impact of the conjunct
% by converting some of the blocked time into spark time.
% The idea is that if a conjunct starts but then blocks waiting for some data,
% and does not generate any of its own data in the meantime,
% then if possible its start should be delayed
% until a time closer to when the data it needs will be available.
% This reduces the contribution made by the conjunct
% to the integral of the program's aggregate memory demand over time.
% Such as delay is often possible
% by merging two parallel conjuncts in a larger parallel conjunction into one.

% As an example, consider the three-way parallel conjunction
% p(In1, X, W) & q(In2, X, Y) & r(In3, Y, Z),
% where In1, In2 and In3 are available at the start of the conjunction,
% p generates X and W, q generates Y and r generates Z.
% If r needs Y very soon after the start of its execution,
% then it maybe preferable to change that conjunction into the two-way
% p(In1, X, W) & ( q(In2, X, Y), r(In3, Y, Z) ).
% By executing q and r in sequence instead of in parallel, ...
% XXX this example illustrates the wrong point

\emph{CONJUNCT\_TIME\_AFTER:
the time, for each parallel conjunct,
between the end of its execution and the end of the conjunction as a whole.}
It is easy to compute this for every dynamic conjunct,
and to summarize it as a minimum, maximum, average and variance
for any given static conjunct.
If the average times after
for different conjuncts in a given parallel conjunction
are relatively stable (have low variance)
but differ substantially compared to the runtime of the conjunction as a whole,
then the speedup from the parallel execution of the conjunction
will be significantly limited by Amdahl's law.
In such cases, the programmer may wish to take
two conjuncts that are now executed in parallel
and execute them in sequence.
Provided the combined conjunct is not the last-ending conjunct,
and provided that delaying the execution of one of the original conjuncts
does not unduly delay any other conjuncts that consume the data it generates,
the resulting small loss of parallelism
may be more than compensated for
by the reduction in parallelism overhead.
It is of course much harder and maybe impossible
to select the two conjuncts to execute in sequence
if the times after for at least some of the conjuncts
are \emph{not} stable (i.e. they have high variance).

% \emph{FUTURE_BLOCK_PROBABILITY:
% the probability that a consuming conjunct
% will ever block on the future for a given variable.}
% The higher the value of this metric is for a variable,
% the more likely that variable is to be a bottleneck.

% \emph{FUTURE_BLOCK_TIME:
% the average, minimum, maximum and variance
% in the time that any consuming conjunct
% that blocks on a future for a given variable
% actually spends blocked on that future.}
% The more time that blocked conjuncts spend blocked waiting for a variable,
% the bigger the bottleneck created by read accesses to that variable.

% probability of blocking,
% time between signal and first wait in nonblocking case,
% time between first wait and signal (blocking time) in blocking case,
% total time between all waits and signal (blocking time) in blocking case:
% average, variance

\emph{FUTURE\_SUSPEND\_TIME\_FIRST\_WAITS:
for each shared variable in each parallel conjunction:
the minimum, average, maximum and variance of the time
of the first wait event on a future for this variable
by each consuming conjunct
minus the time of the corresponding signal event.}
If this value is consistently positive,
then contexts never or rarely suspend waiting for this future;
if this value is consistently negative,
then contexts often suspend on this future, and do so for a long time.
% but if the variance is low they are suspended for long.
Programmers should look at shared variables
that fit into the second category
and try to eliminate the dependencies they represent.
If that is not possible, they may nevertheless try to reduce them
by computing the variable earlier
or pushing the point of first consumption later.
% After optimizing these cases,
% programmer may wish to look at shared variables
% that fit into neither category.

\emph{FUTURE\_SUSPEND\_TIME\_ALL\_WAITS:
for each future in each dynamic parallel conjunction:
the minimum, average, maximum and variance of the time
of all wait events minus the time of the corresponding signal event,
and the count of all such wait events.}
This metric helps programmers understand how hard
delaying the point of first consumption of a shared variable in a conjunct
is likely to be.
Delaying the first consumption is easiest
when the variable is consumed in only a few places,
and the first consumption occurs a long time before later consumptions.
If the variable is consumed in many few places,
many of these points of consumption are just slightly after
the original point of first consumption,
then significantly delaying the point of first consumption
requires eliminating the consumption of the shared variable
in \emph{many} places in the code,
or at least significantly delaying the execution
of those many pieces of code.

% \emph{FUTURE_SUSPEND_TIME_ALL_WAITS:
% for each future in each dynamic parallel conjunction:
% the minimum, average, maximum and variance of
% the time of all wait events minus the time of the corresponding signal event.}
% When this value is consistently positive,
% then contexts never or rarely suspend waiting for this future;
% when this value is consistently negative,
% contexts often suspended on this future and for a long time;
% if the average value is close to zero,
% then contexts sometimes suspend on this future.
% % but if the variance is low they are suspended for long.
% Programmers should look at futures that fit into the second category
% and try to remove or reduce
% the dependencies they represent between parallel conjuncts;
% after optimizing these cases,
% they should look futures that fit into the third category.

% \emph{FUTURE_SUSPEND_TIME_FIRST_WAIT:
% for each future in each dynamic parallel conjunction:
% the minimum, average, maximum and variance of the time
% of the first wait event minus the time of the corresponding signal event.}
% Like FUTURE_SUSPEND_TIME_ALL_WAITS,
% this metric helps programmers understand
% how dependencies affect their program's performance.
% However, when compared to FUTURE_SUSPEND_TIME_ALL_WAITS,
% it can inform the user about cases where a dependency is a problem in
% only some of the parallel conjuncts of the parallel conjunction.
% \zoltan{what does this mean}

\emph{FUTURE\_SUSPEND\_PROBABILITY\_FIRST\_WAIT:
for the first wait event of each future:
the number of times the signal event occurs before the wait event versus
the number of times the wait event occurs before the signal event.}

\emph{FUTURE\_SUSPEND\_PROBABILITY\_ALL\_WAITS:
for each wait event of each future,
the number of times the signal event occurs before the wait event versus
the number of times the wait event occurs before the signal event.}
Like FUTURE\_SUSPEND\_TIME\_ALL\_WAITS and FUTURE\_SUSPEND\_TIME\_FIRST\_WAIT,
these two metrics can help programmers find out
how often contexts are suspended waiting for futures.

\emph{WAITED\_FUTURE\_SIGNAL\_TO\_CONJUNCT\_END\_TIME:
the average, maximum and variance of the time
between the signaling of a future on which another context is blocked
and the end of the parallel conjunct that signalled the future.}
This should be computed for every parallel conjunct
that signals at least one future.
If the average is below the time that it normally takes
for a context made runnable to begin executing on an engine,
and the variance is low, then it suggests that
it is better to execute the context made runnable by the signal
on the same engine immediately after the current conjunct finishes.

\emph{FUTURE\_SIGNAL\_TO\_CONJUNCT\_END\_TIME:
the average, maximum and variance of the time
between the signaling of a future
and the end of the parallel conjunct that signaled that future.}
As above, this should be computed for
every parallel conjunct that signals at least one future.
This value can be used to determine
how often the optimization described above will not be useful.

\emph{OUT\_OF\_ORDER\_PROBABILITY:
given two parallel conjuncts $A$ and $B$,
with $A$ coming logically before $B$
either by being to the left of $B$ in some conjunction,
or by some ancestor of $A$ being to the left of an ancestor of $B$
in some conjunction,
how much time does the system spend with $B$ running
while $A$ is runnable but not running,
compared to the time it spends with either $A$ or $B$ running?}
Ideally, when $A$ and $B$ are both runnable
but the system has only one available CPU,
it should choose to run $A$.
Since it comes logically earlier,
the consumers that depend on its outputs
will also tend to come logically earlier.
Delaying the running of $A$ has a substantial risk of delaying them,
thus delaying the tasks depending on \emph{their} outputs, and so on.
Any efficient scheduling algorithm
must take this effect into consideration.

\emph{OUT\_OF\_ORDER\_SCHEDULING\_BLOCKS\_ANOTHER\_CONTEXT:
when tasks are executed out of order, as described above,
how much of the time is another context $C$
blocked on a future produced by $A$?}
This metric describes how often
out of order execution has a direct impact on another task.

\emph{OUT\_OF\_ORDER\_SCHEDULING\_BLOCKS\_ANOTHER\_CONTEXT\_TC:
When tasks are executed out of order, as described above,
how much of the time do other contexts $D, E, F\ldots$ block
waiting on a future signaled by $C, D, E, \ldots$
which, eventually, depend on $A$?}
This metric considers the transitive closure of the dependency chain
measured by the previous metric.

\subsection{Presenting metrics to users}

The ThreadScope tool is a graphical program.
Its main display screen shows
a continuous timeline of the program's execution on the horizontal axis,
while along the vertical axis,
the display is divided into a discrete number of rows,
with one row per CPU (per engine in Mercury, per capability in Haskell).
The color of a display point for CPU $n$ at time $t$
shows what CPU $n$ was doing at time $t$:
whether it was idle, doing GC, or executing the program.
The time axis can be scaled to see an overview of the execution as a whole
or to focus on a specific period during the program run.
If the current scale allows for it,
the display also shows the individual events
that the on-screen picture is derived from.
Above the per-CPU rows ThreadScope shows a plot that is virtually identical
to the curve of the CPUS\_OVER\_TIME metric we defined above,
and now it can do so for Mercury programs as well as for Haskell programs.

The main ThreadScope GUI also has a mechanism that allows the user
to ask for simple scalar data about the profiled program run.
Originally, it was used only to display the overall time taken by the run.
It is trivial to modify this to let the user input
the number of call sequence counts (CSCs)
executed by a sequence version of the same program on the same data,
which would allow the tool to compute and display
the value of the NANOSECS\_PER\_CALL metric.
It is not much harder to modify it to compute and then output
the values of the GC\_STATS and MUTATOR\_VS\_GC\_TIME metrics.

We plan to modify the ThreadScope GUI so that
if the user hovers the pointer over the representation
of an event connected with a parallel conjunction as a whole
(such as a START\_PAR\_CONJUNCTION or END\_PAR\_CONJUNCTION event),
they get a menu allowing them to select
one of the conjunction-specific metrics listed above.
The system should then compute and print the selected metric.
Similarly if the user hovers the pointer over the representation
of an event connected with a specific parallel conjunct,
they should be able to ask for and get
the value of one of the conjunct-specific metrics.

We plan to add to ThreadScope a screen
that lists all the parallel conjunctions executed by the program.
(The information needed for that list is available in the strings
representing the static ids of the executed parallel conjunctions.)
By selecting items from this list,
the tool should be able to print summaries
from all the dynamic occurrences of the conjunction
for any of the conjunction-specific metrics.
It should also be able to take users
to all those occurrences in the main display.

We also plan to add to ThreadScope
a mechanism that allows the user to request
the generation of a plain text file in a machine-readable format
containing all the metrics that may be of interest
to our automatic parallelization tool.
We want our tool to be able to compare
its predictions of autoparallelized programs
with metrics derived from actual measurements of such programs,
so that we can help tune its performance to reduce the resulting discrepancies.

% \section{Time recording and synchronization}
%
% A critical part of profiling knowing how long certain things take,
% and at what time certain events occur.
% One of the most reliable and portable ways of getting the current time is the
% \code{gettimeofday} system call.
% However,
% system calls can be expensive due to the switch between privilege levels.
% Some operating systems have efficient solutions to these problems,
% such as Linux's VSO support~\cite{linux:vso};
% however, such features are not standard.
% Therefore,
% on i386 and amd64 systems that support it we use the \RDTSCP and
% \RDTSC instructions~\cite{intel:rdtsc}.
% These instructions read a time stamp counter (TSC) from the CPU's internal
% counter.
% In some processors there is no linear relationship between the time stamp
% counter and real time,
% id est, they do not have a \emph{constant TSC}.
% For example, the processor may increment it's time stamp counter less when
% it a lower power state.
% This means that a different meaning of time is used,
% one in which time corresponds to the amount of work a processor can do
% rather than time as it pases in the world outside the processor.
% We do not know if this meaning of time is useful,
% or if programmers would find it confusing;
% therefore,
% if the implementation does not detect the \emph{constant TSC} feature then it
% falls back to \code{gettimeofday}.
% Support for \RDTSCP and \RDTSC is detected dynamically,
% if unavailable our implementation falls back to \code{gettimeofday}
%
% % XXX: Describe the alrogithm and cite it.
% When there are multiple cores each core may have it's own time stamp counter.
% Assuming that time passes at the same rate for each core we can
% calculate the amount that the time on each core is different to the first core.
% This offset is subtracted from the times recorded by each of these cores.
% In practice we have not needed this algorithm,
% Whenever it was enabled it simply conformed that the TSCs in the different
% cores where already synchronized.

\section{Related work and conclusion}
\label{sec:conc}

% It is well established that a visualization tool is extremely useful
% for programmers debugging and tuning complex systems.
% ThreadScope definitely supports this when used either with
% Haskell~\cite{threadscope},
% or, as we have proposed, when used with Mercury.
% Furthermore, extending ThreadScope to work with Mercury has been
% relatively straight forward,
% largely due to ThreadScope's extensible file format.
% 
% Other tools have existed for profiling the parallel execution of other
% languages such as Eden's trace viewer~\cite{edentraceviewer}.
% This shows a similar timeline to that of threadscope,
% but since Eden is a distributed programming language it also shows
% communication.
% 
% GranSim~\cite{loidl98:gransim} is a simulator that determines how much
% parallelism is in a Haskell program.
% Ben Lippmeier wrote a similar tool/paper, it's on his website.
% 
% \cite{runciman93:profilingparfp} introduces another visualization tool.
% This also shows the amount of parallelism in a Haskell program, but it does it
% over time and also shows the number of blocked tasks.
% 
% Apparently GUM has some good profiling tools, but I didn't read this
% paper~\cite{trinder:1996:gum}

% Use this as a source of feedback.
% Discuss what this work implies for Haskell.
% Discuss how this can possibly be tied to perf(1).

Many parallel implementations of programming languages
come with visualization tools
since (as we argued in the introduction)
the people writing parallel programs
need such tools to understand the actual behavior of their programs,
and the people implementing those languages find them useful too.
Such tools have been created for all kinds of programming languages:
imperative (even Visual Studio has one),
functional
(\cite{edentraceviewer,loidl98:gransim,runciman93:profilingparfp} among others)
and logic (\cite{Foster96,vace}
and the systems cited by \cite{Gupta95parallelexecution}).

Many of these visualizers share common ideas and features.
For example, many systems (including ThreadScope)
use a horizontal bar with different parts colored differently
to display the behavior of one CPU,
while many other systems use trees
that display the structure of the computation.
(The latter approach is more common
among visualizers for sequential languages.)
Nevertheless, each visualizer is necessarily oriented
towards giving its users what they want,
and the users of different systems want different information,
since they are using different languages based on different concepts.
For example, functional languages have no concept of parallel conjunctions,
and their concept of data flow between computations
is quite different from Mercury's,
and users of concurrent logic programming languages
such as Concurrent Prolog, Parlog and Guarded Horn Clauses
have quite different concerns from users of Mercury
(where to disable parallelism, not where to enable it).

We know of only one visualizer
for a not-inherently-concurrent logic programming language
that does nevertheless support dependent AND-parallelism: VACE \cite{vace}.
However, the ACE system supports
OR-parallelism and independent AND-parallelism
as well as dependent AND-parallelism,
and its features don't seem designed
to help programmers exploit dependent AND-parallelism better.
As far as we know, noone has built a profiler
for a dependent AND-parallel logic programming language
with the capabilities that we propose.
We haven't yet built one either :-),
but we are working on it.
We expect to have at least a third of our proposed metrics implemented
by the time of the workshop,
and many more by the end of the year.

% I think our basic argument should be:
% - visualization tools have long proven to be useful
%   in improving the performance of parallel programs
%   and in helping the developers of parallel systems
% - such tools have been built for many different systems:
%   logic, functional, and imperative,
% - some features are common to many of these systems:
%   many show trees, while many others show CPU utilization
% - yet each visualizer is nevertheless specific to its own language
% - the only visualizer (VACE) we know of for a language
%   that supports dependent AND parallelism
%   does not focus on dependent AND parallelism,
%   and it does not gather the metrics we think we need

% XXX Acknowledgements.
We would like to thank everyone
who has contributed to the development of ThreadScope,
in particular Simon Marlow and Duncan Coutts,
who have answered many of our questions,
and helped us add Mercury specific features to ThreadScope.

% You can find these papers in ~pbone/papers/
% berthold_eden.pdf
% runciman_Profiling-par-fp.pdf
% I did not download the gransim or gum papers.

% zs: the relevant papers I found are in the files
% vace.ps
% hotpar10.pdf
% par_prolog_survey.pdf
% CRPC-TR93446.pdf
% in my home directory.

