
\status{This chapter is currently WIP}

Early in the project
we tested two manually parallelised programs:
a raytracer and a mandelbrot image generator.
Both of them have a single significant loop
whose iterations are independent of one-another.
We parallelised this loop in each program as it was the best place to
introduce parallelism,
but we did not get the speedups that we expected.
For these programs, automatic parallelisation cannot achieve
a greater speedup than we have achieved with manual parallelism.
Therefore,
we chose to address the performance problems with manual parallelism
before we worked on automatic parallelism.

The chapter is structured as follows:
Section \ref{sec:gc} describes the garbage collector's effect on parallel
execution performance and how tuning some parameters of the garbage
collector can improve performance.
Section \ref{sec:old_scheduling} describes how the existing runtime
system schedules sparks.
It provides background material for Section
\ref{sec:old_scheduling_performance}
which benchmarks the runtime system and describes a significant problem with
spark scheduling.
We address those problems by introducing work stealing in Section
\ref{sec:work_stealing}.
This section is separated into two sub-sections
which describe the initial and revised versions of the work stealing
implementation.
Section \ref{sec:work_stealing} also includes benchmarks that show
how work stealing fixes the spark scheduling problem.
We made a number of improvements to the way that Mercury engines are created,
this includes thread pinning and support for SMT systems,
we describe these improvements in Section \ref{sec:thread_pinning}.
Section \ref{sec:idle_loop} describes our changes to how engines idle,
and how and when they wake up to execute parallel work.
Finally, in Section \ref{sec:kernel_scheduling_help}
we describe an area of research that would improve resource usage when
multiple instances of a parallel Mercury program are executing in parallel.

\section{Garbage collection tweaks}
\label{sec:gc}


\section{Garbage Collector Tweaks}
\label{sec:gc}

One of the sources of poor parallel performance is the behaviour of the
garbage collector.
Like other pure declarative languages,
Mercury does not allow destructive update.
%\footnote{
%    Destructive update is allowed via support for mutable variables.
%    Their use does not interfere with parallelism as they are used in
%    conjunction with impurity.
%    The compiler will not parallelise impure goals.}
Therefore a call usually returns its value in newly allocated memory
rather than modifying the memory of its parameters.
Likewise, a call cannot modify global memory.
This means that Mercury programs often have a high rate of allocation,
which places significant stress on the garbage collector.
Therefore,
allocation and garbage collection can reduce a program's
performance.
This usually becomes more significant when parallelism is introduced into a
program.

\plan{Introduce Boehm, \& details: conservative, mark and sweep, stop the
world, parallel marking.}
Mercury uses the Boehm-Demers-Weiser conservative garbage collector (Boehm GC)
\citep{boehm:1988:gc},
which is a conservative mark and sweep collector.
Boehm GC supports parallel programming:
it will stop all the program's threads (\emph{stop the world}) during its
marking phase.
It also supports parallel marking:
it will use its own set of pthreads to do parallel marking.

\plan{Introduce collector time, mutator time.}
For the purposes of this section
we separate the program's execution time into two alternating phases:
Collection time, which is when Boehm GC performs marking,
and mutator time, which is when the Mercury program runs.
The name `mutator time' refers to time that mutations (changes) to memory
structures are permitted.
The collector may also perform some actions,
such as sweeping,
concurrently with the mutator.

\plan{Describe theory of GC performance.}
Amdahl's law~\citep{amdahl:1967:law} describes the maximum speedup that
can theoretically be achieved by parallelising a part of a program.
We use Amdahl's law to predict the speedup of a program whose
mutator is parallelised but whose collector runs sequentially.
Consider a program with a runtime of 20 seconds
which can be separated into one second of collector time and 19 seconds
of mutator time.
The sequential execution time of the program is $1 + 19 = 20$.
If we parallelise the mutator and do not parallelise the
collector then the minimum parallel execution time is $1 + 19/P$
for $P$ processors.
Using four processors the theoretical best speedup is:
%\paul{I prefer how \\frac displays maths but in text the fonts become tiny.}
$(1 + 19) / (1 + 19/4) = 3.48$.
17\% of the parallel runtime ($1 + 19/4$) is collector time.
If we use a machine with 100 processors then this becomes:
$(1 + 19) / (1 + 19/100) = 16.8$;
with 84\% of the runtime spent in the collector.
As the number of processors increases,
the mutator threads spend a larger proportion of their time waiting for
collection to complete.

\plan{Discuss predictions regarding parallel marking, locking and thread
local heaps.}
To reduce this problem,
\citet{boehm:1988:gc} included parallel marking support in their collector.
Ideally this would remove the bottleneck described above.
However,
parallel garbage collection is a continuing area of research and
the Boehm GC project has only modest multicore scalability goals.
Therefore,
we expect parallel marking to only partially reduce the bottleneck,
rather than remove it completely.

Furthermore, thread safety has two significant performance costs;
These types of costs prevent us from achieving the theoretical maximum
speedups that Amdahl's law predicts.
The first of these is that
one less CPU register is available to GCC's code generator when compiling
parallel Mercury programs (section \ref{sec:backgnd_merpar}).
The second is that
memory allocation routines must use locking to protect shared data
structures,
which will slow down allocation.
Boehm GC's authors recognised this problem and
added support for thread-local resources such as free lists.
Therefore,
during memory allocation,
a thread uses its own free lists rather than locking a global structure.
From time to time, a thread will have to retrieve new free lists
from a global structure and will need to lock the structure.
The costs of locking will be amortised across several memory allocations.

\plan{Describe icfp2000, mandelbrot\_lowalloc and mandelbrot\_highalloc
programs.}
To how well Mercury and Boehm GC scale to multiple cores we used several
benchmark programs with different memory allocation requirements.
Our first benchmark is a raytracer developed for the
ICFP programming contest in 2000.
For each pixel in the image,
the raytracer casts a ray into the scene to determine what colour to paint that pixel.
Two nested loops build the pixels for the image:
the outer loop iterates over the rows in the image and
the inner loop iterates over the pixels in each row.
We manually parallelised the program by introducing an parallel
conjunction into the outer loop;
indicating that rows should be drawn in parallel.
This parallelisation is independent.
The raytracer uses many small structures to represent vectors and a pixel's
colour.
It is therefore memory allocation intensive.
Since we suspected that garbage collection was having a negative affect on
performance,
we developed a mandelbrot image generator.
The mandelbrot image generator has a similar structure to the raytracer:
it draws an image using two nested loops as above.
The mandelbrot image is drawn on the complex number plane.
A complex number $C$ is in the mandelbrot set if
$\forall i \cdot |N_i| < 2$ where $N_0 = 0$ and $N_{i+1} = N_{i}^2 + C$.
For each pixel in the image the program tests if the pixel's coordinates are
in the set.
The pixel's colour is chosen based on how many iterations $i$ are needed
before $|N_i| \ge 2$,
or black if the test survived 5,000 iterations.
We parallelised this program the same way as we did the raytracer:
by introducing an independent parallel conjunction into the outer loop.
We created two versions of this program.
The first version represents coordinates and complex numbers as structures
on the heap.
Therefore it has a high rate of memory allocation.
We call this version `mandelbrot\_highalloc'.
The second version of the mandelbrot program,
called `mandelbrot\_lowalloc',
stores its coordinates and complex numbers on the stack and in registers.
Therefore it has a lower rate of memory allocation.


\begin{table}
\begin{center}
\begin{tabular}{r|rr|rrrr}
\Cbr{GC Markers} &
\multicolumn{2}{c|}{Sequential} &
\multicolumn{4}{c}{Parallel w/ $N$ Mercury Engines} \\
\Cbr{} &
\C{no TS} & \Cbr{TS} &
\C{1} & \C{2} & \C{3} & \C{4} \\
\hline
\hline
\multicolumn{7}{c}{raytracer} \\
\hline
1 & 33.1 (1.13) & 37.4 (1.00) &
             37.7 (0.99) & 28.0 (1.34) & 24.8 (1.51) & 23.7 (1.58) \\
2  & - & - & 32.0 (1.17) & 21.6 (1.73) & 18.1 (2.07) & 16.7 (2.24) \\
3  & - & - & 29.6 (1.26) & 19.7 (1.90) & 16.3 (2.29) & 14.5 (2.59) \\
4  & - & - & 29.1 (1.29) & 18.9 (1.98) & 15.3 (2.45) & 13.7 (2.73) \\
\hline
\hline
\multicolumn{7}{c}{mandelbrot\_highalloc} \\
\hline
1 & 41.0 (1.23) & 50.5 (1.00) &
             50.6 (1.00) & 33.4 (1.51) & 26.6 (1.90) & 23.4 (2.16) \\
2  & - & - & 46.7 (1.08) & 28.6 (1.77) & 22.1 (2.28) & 18.8 (2.69) \\
3  & - & - & 46.3 (1.09) & 27.3 (1.85) & 20.9 (2.42) & 17.3 (2.93) \\
4  & - & - & 45.6 (1.11) & 26.8 (1.89) & 20.4 (2.48) & 16.7 (3.03) \\
\hline
\hline
\multicolumn{7}{c}{mandelbrot\_lowalloc} \\
\hline
1 & 15.4 (0.99) & 15.2 (1.00) &
             15.2 (1.00) &  5.1 (2.96) &  7.7 (1.98) &  3.9 (3.92) \\
2  & - & - & 15.3 (1.00) &  7.7 (1.98) &  5.1 (2.97) &  3.9 (3.94) \\
3  & - & - & 14.3 (1.00) &  7.7 (1.98) &  5.1 (2.97) &  3.9 (3.94) \\
4  & - & - & 15.3 (0.99) &  7.7 (1.98) &  5.1 (2.97) &  3.9 (3.92) \\ 
\end{tabular}
\end{center}
\paul{Consider drawing charts for raytracer and mandelbrot so that releative
performance between different scores can be seen easily.}
\caption{Parallelism and garbage collection}
\label{tab:gc}
\end{table}



\plan{Benchmark data.}
We benchmarked these three programs with different numbers of Mercury
engines and garbage collector threads.
We show the results in table \ref{tab:gc}.
All benchmarks have an initial heap size of 16MB.
Each result is measured in seconds and represents the mean of eight samples.
The first column is the number of garbage collector threads used.
The second and third columns give sequential execution times
without and with thread safety enabled.
In these columns the programs where executed in such a way so that they did
not execute a parallel conjunction.
The remaining four columns give parallel execution times using one to four
Mercury engines.
The numbers in parentheses show the relative speedup when compared with the
sequential thread safe result for the same program.
\label{cabsav}
We ran our benchmarks on
\paul{Find a brand name i7 I can borrow for some weeks}
a Intel i7-2600K system
with 16GB of memory,
running Debian/GNU Linux 6
with a 2.6.32-5-amd64 Linux kernel and GCC 4.4.5-8.
The garbage collection benchmarks were gathered using a recent version of
Mercury (rotd-2012-04-29).
This version does not have the performance problems described in the
rest of this chapter,
and it does have the loop control transformation described
in chapter \ref{chap:loop_control}.
Therefore we can observe the effects of garbage collection without
interference from any other performance problems.

\plan{Describe performance in practice.}
The raytracer program benefits from parallelism in both Mercury and the
garbage collector.
Using Mercury's parallelism only (four Mercury engines, and one GC thread) 
speeds the program up by a factor of 1.58,
compared to 1.29 when using the GC's parallelism only (one Mercury engine,
and four GC threads).
When using both Mercury and the GC's parallelism (four engines and four
marker threads)
the raytracer achieves a speedup of 2.73.
These speedups are much lower than we might expect from such a program:
either the mutator, the collector or both are not being parallelised well.
mandelbrot\_lowalloc does not see any benefit from parallel marking.
It achieves very good speedups from multiple Mercury engines.
We know that this program has a low allocation rate
but is otherwise parallelised in the same way as raytracer.
Therefore,
these results support the hypothesis that heavy use of garbage collection
makes it difficult to achieve good speedups when parallelising programs.
The more time spend in garbage collection,
the worse the speedup due to parallelism.
mandelbrot\_highalloc, which stores its data on the heap,
sees similar trends in performance as raytracer.
It is also universally slower than mandelbrot\_lowalloc.
mandelbrot\_highalloc achieves a speedup of 2.16 when using parallelism in
Mercury (four Mercury engines, and one GC thread).
The corresponding figure for the raytracer is 1.58.
When using the GC's parallelism
(one Mercury engine, and four GC threads)
mandelbrot\_highalloc achieves a speedup of 1.11,
compared with 1.29 for the raytracer.


\begin{table}
\begin{center}
\begin{tabular}{r|l| d{1} r | d{1} r | d{1} r | d{1} r }
\C{GC} & \C{Times \&} &
\multicolumn{8}{c}{Parallel w/ $N$ Mercury Engines} \\
\C{Markers} & \C{Collections}
          & \Ctwo{1}        & \Ctwo{2}        & \Ctwo{3}        & \Ctwo{4} \\
\hline
\hline
\multicolumn{10}{c}{raytracer} \\
\hline
\multirow{4}{*}{1} &
  Elapsed & 37.8 &        & 28.1 &        & 24.7 &        & 22.7 & \\
& GC      & 16.9 & 44.8\% & 16.7 & 59.4\% & 16.5 & 66.6\% & 15.8 & 69.5\% \\
& Mutator & 20.9 & 55.2\% & 11.4 & 40.6\% &  8.2 & 33.4\% &  6.9 & 30.5\% \\
& \# col. & 384. &        &346.  &        &316.  &        &288.  & \\
\hline
\multirow{4}{*}{2} &
  Elapsed & 32.6 &        & 21.8 &        & 18.3 &        & 16.0 & \\
& GC      & 11.0 & 33.8\% & 10.4 & 47.7\% & 10.3 & 56.4\% &  9.8 & 60.8\% \\
& Mutator & 21.6 & 66.2\% & 11.4 & 52.3\& &  8.0 & 43.6\% &  6.3 & 39.2\% \\
& \# col. & 384. &        &346.  &        &316.  &        &288.  & \\
\hline
\multirow{4}{*}{3} &
  Elapsed & 30.9 &        & 20.0 &        & 16.4 &        & 14.1 & \\
& GC      &  9.3 & 30.2\% &  8.5 & 42.3\% &  8.2 & 50.0\% &  7.8 & 55.4\% \\
& Mutator & 21.6 & 69.8\% & 11.5 & 57.7\% &  8.2 & 50.0\% &  6.3 & 44.6\% \\
& \# col. & 386. &        &346.  &        &316.  &        &288.  & \\
\hline
\multirow{4}{*}{4} &
  Elapsed & 30.3 &        & 19.3 &        & 15.5 &        & 13.5 & \\
& GC      &  8.7 & 28.8\% &  7.8 & 40.4\% &  7.4 & 47.8\% &  7.1 & 52.4\% \\
& Mutator & 21.6 & 71.2\% & 11.5 & 59.6\% &  8.1 & 52.2\% &  6.4 & 47.6\% \\
& \# col. & 387. &        &346.  &        &316.  &        &288.  & \\
\hline
\hline
\multicolumn{10}{c}{mandelbrot\_highalloc} \\
\hline
\multirow{4}{*}{1} &
  Elapsed & 51.7 &        & 34.0 &        & 27.2 &        & 24.0 & \\
& GC      & 11.0 & 21.4\% & 11.5 & 33.6\% & 11.5 & 42.2\% & 11.8 & 49.1\% \\
& Mutator & 40.7 & 78.6\% & 22.6 & 55.4\% & 15.5 & 57.8\% & 12.2 & 50.9\% \\
& \# col. &742.  &        &692.  &        &634.  &        &602. & \\
\hline
\multirow{4}{*}{2} &
  Elapsed & 48.5 &        & 29.3 &        & 22.3 &        & 19.1 & \\
& GC      &  6.4 & 13.1\% &  6.7 & 22.9\% &  6.6 & 29.7\% &  6.9 & 36.1\% \\
& Mutator & 42.1 & 86.9\% & 22.6 & 77.1\% & 15.7 & 70.3\% & 12.2 & 63.9\% \\
& \# col. &744.  &        &693.  &        &633.  &        &595.  & \\
\hline
\multirow{4}{*}{3} &
  Elapsed & 46.4 &        & 28.0 &        & 21.0 &        & 17.7 & \\
& GC      &  5.0 & 10.8\% &  5.3 & 18.8\% &  5.3 & 25.0\% &  5.4 & 30.4\% \\
& Mutator & 41.4 & 89.2\% & 22.8 & 81.2\% & 15.7 & 75.0\% & 12.3 & 69.6\% \\
& \# col. &737.  &        &692.  &        &629.  &        &600.  & \\
\hline
\multirow{4}{*}{4} &
  Elapsed & 46.0 &        & 27.6 &        & 20.3 &        & 16.9 & \\
& GC      &  4.5 &  9.9\% &  4.7 & 17.0\% &  4.6 & 22.6\% &  4.7 & 27.6\% \\
& Mutator & 41.4 & 90.1\% & 22.9 & 83.0\% & 15.7 & 77.4\% & 12.3 & 72.4\% \\
& \# col. &740.  &        &695.  &        &626.  &        &600.  & \\
\hline
\hline
\multicolumn{10}{c}{mandelbrot\_lowalloc} \\
\hline
\multirow{4}{*}{1} &
  Elapsed & 15.2 &        &  7.7 &        &  5.1 &        &  3.9 & \\
& GC      &  0.0 &  0.1\% &  0.0 &  0.2\% &  0.0 &  0.7\% &  0.0 &  0.4\% \\
& Mutator & 15.2 & 99.9\% &  7.6 & 99.8\% &  5.1 & 99.3\% &  3.9 & 99.6\% \\
& \# col. &  2.  &        &  2.  &        &  2.  &        &  2.  & \\
\hline
\multirow{4}{*}{2} &
  Elapsed & 15.4 &        &  7.6 &        &  5.1 &        &  3.9 & \\
& GC      &  0.0 &  0.1\% &  0.0 &  0.1\% &  0.0 &  0.2\% &  0.0 &  0.2\% \\
& Mutator & 15.4 & 99.9\% &  7.6 & 99.9\% &  5.1 & 99.8\% &  3.9 & 99.8\% \\
& \# col. &  2.  &        &  2.  &        &  2.  &        &  2.  & \\
\hline
\multirow{4}{*}{3} &
  Elapsed & 15.3 &        &  7.7 &        &  5.1 &        &  3.9 & \\
& GC      &  0.0 &  0.1\% &  0.0 &  0.1\% &  0.0 &  0.4\% &  0.0 &  0.1\% \\
& Mutator & 15.2 & 99.9\% &  7.7 & 99.9\% &  5.1 & 99.6\% &  3.9 & 99.9\% \\
& \# col  &  2.  &        &  2.  &        &  2.  &        &  2.  & \\
\hline
\multirow{4}{*}{4} &
  Elapsed & 15.2 &        &  7.6 &        &  5.1 &        &  3.9 & \\
& GC      &  0.0 &  0.1\% &  0.0 &  0.1\% &  0.0 &  0.4\% &  0.0 &  0.1\% \\
& Mutator & 15.2 & 99.9\% &  7.6 & 99.9\% &  5.1 & 99.6\% &  3.9 & 99.9\% \\
& \# col  &  2.  &        &  2.  &        &  2.  &        &  2.  & \\
\end{tabular}
\end{center}
\caption{Percentage of elapsed execution time used by GC/Mutator}
\label{tab:gc_amdahl}
\end{table}



\plan{Introduce the table we use for our discussion of Amdahl's law.}
We can see that mandelbrot\_highalloc benefits from parallelism in Mercury
more than raytracer does,
conversely mandelbrot\_highalloc benefits from parallelism in the garbage
collector less than raytracer does.
Using \tscope (chapter \ref{chap:tscope}),
we analysed how much time these programs spend running the garbage collector
or the mutator.
These results are shown in table \ref{tab:gc_amdahl}.
The table shows the elapsed time, collector time, mutator time and the
number of collections for each program using one to four engines, and one
to four collector threads.
The times are averages taken from eight samples,
using an initial heap size of 16MB.
To use \tscope we had to compile the runtime system differently.
Therefore,
these results differ slightly from those in table \ref{tab:gc}.
Next to both the collector time and mutator time
we show the percentage of elapsed time taken by the collector or mutator
respectively.

\plan{Describe trends in either the mutator or GC time.}
As we expected, mandelbrot\_lowalloc spends very little time running the collector.
Typically, it ran the collector only twice during its execution.
Also, total collector time was usually between 5 and 30 milliseconds.
The other two programs ran the collector hundreds of times.
As we increased the number of Mercury engines
we noticed that these programs made fewer collections.
As we varied the number of collector threads,
we saw no trend in the number of collections.
Any apparent variation is most likely noise in the data.
All three programs see a speedup in the time spent in the mutator as the
number of Mercury engines is increased.
Similarly,
the raytracer and mandelbrot\_highalloc benefit from speedups
in GC time as the number of GC threads is increased.
In mandelbrot\_highalloc,
the GC time also increases slightly as Mercury engines are added.
We expect that as more Mercury engines are used the garbage collector
must use more inter-core communication which has additional costs.
However, in the raytracer,
the GC time decreases slightly as Mercury engines are added.
To understand why this happens we would need to understand the collector in
detail,
however Boehm GC's sources are notorious for being difficult to read and
understand.

\plan{Compare performance with Amdahl's predictions.}
As Amdahl's law predicts,
parallelism in one part of the program has a limited effect on the program
as a whole.
While using one GC thread
we tested with one to four Mercury engines;
the mutator time speedups for two, three and four engines were
1.83, 2.55 and 3.03 respectively.
However, the elapsed time speedups for the same test were only
1.35, 1.53 and 1.66 respectively.
Although there is little change in absolute time spent in the collector;
there is an increase in collector time as a percentage of elapsed time.
The corresponding mutator time speedups for mandelbrot\_highalloc were
similar,
at 1.80, 2.63 and 3.34 for two, three and four Mercury engines.
The elapsed time speedups on the same test for mandelbrot\_highalloc were
1.52, 1.90 and 2.15.
Like raytracer above, the elapsed time speedups are lower than the mutator
time speedups.
Similarly,
when we increase the number of threads used by the collector,
it improves collection time more than it does elapsed time.

When we increased both the number of threads used by the collector and the
number of Mercury engines
(a diagonal path through the table)
both the collector and mutator time decrease.
This shows that parallelism in Mercury and in Boehm GC both contribute to
the elapsed time speedup we saw in the diagonal path in table \ref{tab:gc}.
As table \ref{tab:gc_amdahl} give us more detail,
we can also see that collector time as a percentage of elapsed time
increases as we add threads and engines to the collector and Mercury.
This occurs for both the raytracer and mandelbrot\_highalloc.
It suggests that the mutator makes better use of additional Mercury
engines
than the collector makes use of additional threads.
We can confirm this by comparing the speedup for the mutator with the speedup
in the collector.
In the case of raytracer using four Mercury engines and one GC thread
the mutator's speedup is $20.9 / 6.9 = 3.03$ over the case for one Mercury
engine and one GC thread.
The collectors speedup with four GC threads and one Mercury engine over
the case for one GC thread and one Mercury engine is $16.9 / 8.7 = 1.94$.
The equivalent speedups for mandelbrot\_highalloc are:
$40.7 / 12.2 = 3.34$ and $11.0 / 4.5 = 2.44$.
Similar comparisons can be made along the diagonal of the table.

\begin{table}
\begin{center}
\begin{tabular}{l|rr|d{3}}
\Cbr{Program} & \C{Allocations} & \Cbr{Total alloc'd bytes} & \C{Alloc rate (M alloc/sec)} \\
\hline
raytracer   &     561,431,515 &           9,972,697,312 & 26.9 \\
mandelbrot\_highalloc
            &   3,829,971,662 &          29,275,209,824 & 94.1 \\
mandelbrot\_lowalloc
            &       1,620,928 &              21,598,088 &  0.106 \\
\end{tabular}
\end{center}
\caption{Memory allocation rates}
\label{tab:mem_alloc_rate}
\end{table}

\plan{Allocation intensity}
table \ref{tab:gc_amdahl} also shows that raytracer spends more of its
elapsed time doing garbage collection than mandelbrot\_highalloc does.
This matches the results in table \ref{tab:gc},
where mandelbrot\_highalloc has better speedups because of parallelism in
Mercury than raytracer does.
Likewise,
raytracer has better speedups because of parallelism in the collector
than mandelbrot\_highalloc does.
This suggests that mandelbrot\_highalloc is less allocation intensive than
raytracer,
but this is not true.
Using Mercury's deep profiler we measured the number and total size of memory
allocations.
This is shown in table \ref{tab:mem_alloc_rate}.
The rightmost column in this table gives the allocation rate
measured in million allocations per second.
It is calculated by dividing the number of allocations in the second column
by the average mutator time reported in table \ref{tab:gc_amdahl}.
Therefore, it represents the allocation rate during mutation time.
This is deliberate because,
by definition,
allocation cannot occur during collector time.
mandelbrot\_highalloc has a higher allocation rate than raytracer.
However,
it spends less time doing collection, when measured either absolutely
or by percentage of elapsed time.
Garbage collectors are tuned for particular workloads.
It is likely that Boehm GC handles mandelbrot\_highalloc's workload more
easily than raytracer's workload.

% Old idea:
%This may be due to changing allocation rates throughout the programs'
%executions.
%We noticed that while benchmarking the raytracer that if we rendered the
%image upside-down (by iterating over the rows backwards)
%we would get different performance results.
%This is because the top of the image we used contains sky,
%which has no geometry;
%it is the flat colour of the background,
%and the bottom of the image contains ground;
%which has more detail.
%Rendering the ground will naturally be more complicated and require more
%memory allocation than rendering the sky.
%Memory allocation in one part of a program can affect the performance of
%allocation and collection in another part of the program.
%This is many because the garbage collector will increase the size of the heap
%to satisfy large and/or many allocations.
%This is why rendering the image upside down had a different execution time to
%rendering it the right way up.
%Similarly,
%the mandelbrot program has sections in its image that require more iterations
%of the equation than others.


\begin{table}
\begin{center}
\begin{tabular}{r|rr|rrrr}
\Cbr{Initial} &
\multicolumn{2}{c|}{Sequential} &
\multicolumn{4}{c}{Parallel w/ $N$ Mercury Engines} \\
\Cbr{heap size} & \C{no TS}   & \Cbr{TS}    & \C{1}       & \C{2}       & \C{3}       & \C{4} \\
\hline
\hline
\multicolumn{7}{c}{raytracer} \\
\hline
1MB     & 32.8 (0.89) & 29.3 (1.00) & 29.3 (1.00) & 19.0 (1.54) & 15.4 (1.90) & 13.6 (2.16) \\
16MB    & 33.5 (0.89) & 29.7 (1.00) & 29.1 (1.02) & 18.9 (1.57) & 15.3 (1.94) & 13.7 (2.17) \\
32MB    & 34.3 (0.86) & 29.5 (1.00) & 29.5 (1.00) & 19.5 (1.51) & 15.6 (1.89) & 13.6 (2.16) \\
64MB    & 33.2 (0.92) & 30.4 (1.00) & 30.7 (0.99) & 20.3 (1.50) & 16.4 (1.86) & 14.5 (2.10) \\
128MB   & 22.7 (1.08) & 24.5 (1.00) & 24.2 (1.01) & 15.0 (1.63) & 11.7 (2.09) & 10.2 (2.40) \\
256MB   & 18.8 (1.13) & 21.4 (1.00) & 21.5 (0.99) & 12.6 (1.70) &  9.3 (2.29) &  7.7 (2.76) \\
384MB   & 17.7 (1.17) & 20.8 (1.00) & 20.7 (1.01) & 11.7 (1.77) &  8.7 (2.39) &  7.2 (2.90) \\
512MB   & 17.3 (1.17) & 20.3 (1.00) & 20.5 (0.99) & 11.5 (1.77) &  8.3 (2.44) &  6.8 (2.99) \\
\hline
\hline
\multicolumn{7}{c}{mandelbrot\_highalloc} \\
\hline
1MB     & 41.4 (1.10) & 45.5 (1.00) & 45.1 (1.01) & 26.7 (1.70) & 20.3 (2.24) & 16.7 (2.72) \\ 
16MB    & 39.7 (1.14) & 45.3 (1.00) & 45.6 (0.99) & 26.8 (1.69) & 20.4 (2.23) & 16.7 (2.72) \\
32MB    & 38.9 (1.16) & 45.2 (1.00) & 44.2 (1.02) & 26.3 (1.72) & 19.7 (2.29) & 16.5 (2.74) \\
64MB    & 36.7 (1.21) & 44.4 (1.00) & 43.7 (1.02) & 25.2 (1.76) & 18.8 (2.36) & 15.6 (2.84) \\
128MB   & 34.1 (1.25) & 42.6 (1.00) & 41.8 (1.02) & 24.1 (1.77) & 17.7 (2.41) & 14.4 (2.96) \\
256MB   & 33.2 (1.24) & 41.3 (1.00) & 41.9 (0.99) & 23.7 (1.74) & 16.9 (2.45) & 13.6 (3.04) \\ 
384MB   & 31.7 (1.29) & 41.0 (1.00) & 41.9 (0.98) & 23.3 (1.76) & 16.7 (2.46) & 13.4 (3.07) \\
512MB   & 31.1 (1.32) & 41.1 (1.00) & 41.1 (1.00) & 23.0 (1.79) & 16.7 (2.46) & 13.3 (3.08) \\ 
\end{tabular}
\end{center}
\caption{Varying the initial heapsize in parallel Mercury programs.}
\label{tab:heapsize}
\end{table}



\plan{New data, vary the heap size.}
So far,
we have shown that speedups due to Mercury's parallel conjunction
are limited by the garbage collector.
Better speedups can be achieved by using the parallel marking feature in the
garbage collector.
We also attempted to improve performance further by modifying the initial
heap size of the program.
table \ref{tab:gc_heapsize} shows the performance of the raytracer and
mandelbrot\_highalloc programs with various initial heap sizes.
The first column shows the heap size used;
we picked a number of sizes from 1MB to 512MB.
The remaining columns show the average elapsed times in seconds.
The first of these columns shows timing for the programs compiled for
sequential execution without thread safety;
this means that the collector cannot use parallel marking.
The next column is for sequential execution with thread safety;
the collector uses parallel marking with four threads.
The next four columns give the results for the programs compiled for
parallel execution, and executed with one to four Mercury engines.
These results also use four marker threads.
The numbers in parentheses are the ratio of elapsed time compared with the
sequential thread-safe result for the same initial heap size.
All the results are the averages of eight test runs.
We ran these tests on raytracer and mandelbrot\_highalloc.
We did not use mandelbrot\_lowalloc as its collection time is insignificant
and would not have provided useful data.

\plan{Observations: Programs get faster with a larger heap size.}
Generally, the larger the initial heap size the better the programs
performed.
The Boehm GC will begin collecting if it cannot satisfy a memory allocation
request.
If, after a collection, it still cannot satisfy the memory request then it
will increase the size of the heap.
In each collection,
the collector must read all the stacks, global data, thread local data and
all the in-use memory in the heap 
(memory that is reachable from the stacks, global data and thread local
data).
This causes a lot of cache misses, especially when the heap is large.
The larger the heap size,
the less frequently memory is exhausted and needs to be collected.
Therefore,
programs with larger heap sizes garbage collect less often and
have fewer cache misses due to garbage collection.
This explains the trend of increased performance as we increase the initial
heap size of the program.

Although performance improved with larger heap sizes,
there is an exception.
The raytracer often ran more slowly with a heap size of 64MB
than with a heap size of 32MB.
%\paul{Does the reader know what a TLB is?  People are generally poor at
%remembering to care about normal caches much less TLBs.}
64MB is much larger than the processor's cache size
(this processor's L3 cache is 8MB) and
covers more page mappings than its TLBs can hold (L2 TLB covers 2MB when
using 4KB pages).
The collector's structures and access patterns may be slower at this size
because of these hardware limitations,
and the benefits of a 64MB heap are not enough to overcome the effects of
these limitations,
however the benefits of a 128MB or larger heap are enough.

\plan{Speedup observation with larger heap sizes.}
Above, we said that programs perform better with larger heap sizes.
This measurement compares their elapsed time with one heap
size with the elapsed time using a different heap size.
We also found that programs \emph{exploited parallelism more easily} with
larger heap sizes:
This measurement compares programs' speedups (which are themselves
comparisons of elapsed time).
When the raytracer uses a 16MB initial heap its speedup using four
cores is 2.17 times.
However, when it uses a 512MB initial heap the corresponding speedup is
2.99.
Similarly,
mandelbrot\_highalloc has a 4-engine speedup of 2.72 using a 16MB initial
heap and a speedup of 3.08 using a 512MB initial heap.
Generally,
larger heap sizes allow programs to exploit more parallelism.
There are two reasons for this

\begin{description}

    \item[Reason 1]
    In a parallel program with more than one Mercury engine,
    each collection must \emph{stop-the-world}:
    All Mercury engines are stopped so that they do not modify the heap during
    the marking phase.
    This requires synchronisation which reduces the performance of parallel
    programs.
    Therefore, the less often collection occurs, the less stop-the-world's
    synchronisation affects performance.

    \item[Reason 2]
    Another reason was suggested by Simon Marlow:
    Because Boehm GC uses its own threads for marking and not Mercury's,
    it cannot mark objects with the same thread that allocated the objects.
    Unless Mercury and Boehm GC both mapped and pinned their threads to
    processors,
    and agreed on the mapping then making will cause a large number of cache
    misses.
    For example, during collection, processor one (P1) marks one of processor two's
    (P2) objects,
    causing a cache miss in P1's cache and invalidating the corresponding cache
    line in P2's cache.
    Later, when collection is finished,
    P2's process resumes execution and incurs a cache miss for the object that
    it had been using.
    % He never states this in either of his parallel GC papers.
    Simon Marlow made a similar observation when working with GHC's garbage
    collector.

\end{description}

\plan{Discuss local heaps for threads, and their reliability problems.}
We also investigated another area for increased parallel performance.
Boehm GC maintains thread local free lists that allow memory to be allocated
without contending for locks on global free lists.
When a local free list cannot satisfy a memory request, the global free
lists must be used.
The local free lists amortise the costs of locking the global free lists.
We anticipate that increasing the size of the local free lists will cause even
less contention for global locks,
allowing allocation intensive programs to have better parallel
performance.
The size of the local free list can be adjusted by increasing the
\texttt{HBLKSIZE} tunable.
Unfortunately this feature is experimental,
and adjusting \texttt{HBLKSIZE} caused our programs to crash.
Therefore we cannot evaluate how \texttt{HBLKSIZE} affects our
programs.
Once this feature is no longer experimental,
adjusting \texttt{HBLKSIZE} to improve parallel allocation should be
investigated.




\section{Original spark scheduling algorithm}
\label{sec:old_scheduling}

\status{Checking plan for this section}

\plan{Global spark queue and Contention wrt global queue}
Mercury has a global spark queue.
The runtime can easily schedule a spark by placing it on the end of the
global queue.
An idle engine can run a spark by takeing it spark from the beginning of the queue.
The global spark queue must be protected by a lock;
this prevents concurrent access from corrupting the queue.
The global spark queue and its lock can easily become a bottleneck when many
engines content for access to the global queue.

\plan{Local spark stack --- relieves contention on global queue}
\citet{wang-hons} expected this problem and created context local spark stacks
to avoid contention on the global queue.
Furthermore, the local spark stacks do not require locking.
When a parallel conjunction spawns off a spark it can place the spark either
at the end of the global spark queue or at the top of its local spark stack.
\plan{Spark scheduling decision}
The runtime system appends the spark to the end of the global queue if:
an engine is idle, and
the number of contexts in use plus the number of sparks on the global queue
does not exceed the maximum number of contexts permitted.
Otherwise,
the runtime system pushes the spark onto the top of the context's local
spark stack.
% Note, if I ever change citations I need to fix 'Wang' here.
% XXX The next line can be searched for to find citations.
% \citet
Wang's scheduling decision has two aims:
Firstly, to reduce the contention on the global queue,
especially in the common case that there is enough parallel work.
This also reduces the amount of locking.
Secondly, to reduce the amount of memory allocated
in contexts' stacks by reducing the number of contexts allocated.
Globally scheduled sparks may be converted into contexts so they are also
included in this limit.
Note that sparks placed on the global queue are executed in a
first-in-first-out manner, while
sparks placed on a context's local stack are executed in a
last-in-first-out manner.

\begin{algorithm}
\begin{algorithmic}
\Procedure{MR\_join\_and\_continue}{$ST, ContLabel$}
  \State aquire\_lock($ST.lock$)
  \State $ST.num\_outstanding \gets ST.num\_outstanding - 1$
  \If{$ST.num\_outstanding = 0$}
    \If{$ST.orig\_context = this\_context$}
      \State release\_lock($ST.lock$)
      \Goto{$ContLabel$}
    \Else
      \State schedule($ST.parent$)
      \State release\_lock($ST.lock$)
      \Goto{MR\_get\_global\_work}
    \EndIf
  \Else
    \State $spark \gets$ pop\_spark
    \If{$spark$}
      \If{$spark{\rightarrow}ST.stack\_ptr = MR\_parent\_sp$}
%        \Comment{This spark belongs to the same parallel conjunction.
%        It can be executed immediatly.}
        \State release\_lock($ST.lock$)
        \Goto{$spark.code\_label$}
      \EndIf
      \State push\_spark($spark$)
    \EndIf
    \If{$ST.orig\_context = this\_context$}
       \State suspend($this\_context$)
       \State $this\_context \gets$ NULL
    \EndIf
    \State release\_lock($ST.lock$)
    \Goto{MR\_get\_global\_work}
  \EndIf
\EndProcedure
\end{algorithmic}
\caption{MR\_join\_and\_continue}
\label{alg:join_and_continue_peterw}
\end{algorithm}

\plan{barrier code, this is used to explain the right recursion problem.}
As an engine finishes executing a parallel conjunct,
it will execute the the barrier at the end of the conjunct.
The barrier is named \joinandcontinue and shown in
Algorithm \ref{alg:join_and_continue_peterw}.
If the parallel conjunction has only been executed by
one context,
then it is safe to use a version of the barrier code that does not use
locking,
this optimisation is not shown as it is equivalent and not relevant to
our discussion,
we mention it only for completeness.

The algorithm begins by checking if there are any outstanding conjuncts in
the parallel conjunction.
If there are no outstanding conjuncts and the current context is the parent
context,
then execution jumps to the label after the parallel conjunction.
If the current context is not the parent context then
we can infer that the parent context is suspended.
Therefore,
the engine will schedule the parent context.
The current context's local work queue is guaranteed to be empty,
as this conjunction and any nested conjunctions are complete.
Therefore,
the engine will continue by jumping to the \getglobalwork routine
which is described below.
Alternatively,
if \joinandcontinue found that there are outstanding conjuncts then
the local spark stack is checked for a spark.
A spark is used only if it was spawned off by the same parallel conjunction.
The only other spark that might exist, is one for a caller's parallel
conjunction provided.~\footnote{
    It is impossible to find a callee's spark on our local spark stack as
    all the parallel conjuncts in a callee must be complete before context
    would be allowed to continue from the barrier at the callee's parallel
    conjunction.}
This can only haoppen if this context is the original context for the
current parallel conjunction.
We cannot execute a spark for a caller's parallel conjunction because it may
block, causing our context to block the current conjunction.
Therefore, only sparks for the current parallel conjunction are executed.
If there was no suitable spark,
the algorithm checks if this context is the parent context.
If so, the algorithm suspends the context and
jumps to \getglobalwork.


\plan{Explain how work begins executing, for completeness}
\paul{Talk about get\_global\_work().}
%An engine looks for global work first by checking the global context run queue.
%If it finds a runnable context and is still holding a context from a
%previous execution, it saves the old context onto the free context list.
%If there are no runnable contexts,
%it will take a spark from the global spark queue,
%and either use its current context to execute the spark,
%or allocate a new context (from the free context list if possible).
%If it is unsuccessful at finding work,
%it will go to sleep using a pthread condition variable and the global run
%queue's lock.
%This condition is used to wake engines when either contexts are added to the
%run queue,
%or sparks are added to the spark run queue.

\section{Prior spark scheduling performance}
\label{sec:old_scheduling_performance}

\status{Checking plan for this section.}

\begin{figure}
\begin{center}
\subfigure[Right recursive]{%
\label{fig:map_right_recursive}
\begin{tabular}{l}
\code{map(\_, [], []).} \\
\code{map(P, [X $|$ Xs], [Y $|$ Ys]) :-} \\
\code{~~~~P(X, Y) \&} \\
\code{~~~~map(P, Xs, Ys).} \\
\end{tabular}}
\subfigure[Left recursive]{%
\label{fig:map_left_recursive}
\begin{tabular}{l}
\code{map(\_, [], []).} \\
\code{map(P, [X $|$ Xs], [Y $|$ Ys]) :-} \\
\code{~~~~map(P, Xs, Ys) \&} \\
\code{~~~~P(X, Y).} \\
\end{tabular}}%
\end{center}
\caption{Right and left recursive map/3}
\label{fig:map_right_and_left_recursive}
\end{figure}

\plan{Describe the problem where we noticed that the context limit could limit
the amount of parallelism that can be exploited.}

\plan{Introduce right recursion.}

\plan{Show performance figures where limit is increased, confirming this problem.}

\plan{Suggest that left recursion might fix this,
(This is what we thought at the time).}
Explain why we think that left recursion will not have this problem.

\plan{Show performance figures for left recursion.}
The left recursive figures are underwhelming, they are worse than right recursion.

\plan{Explain the premature scheduling problem that affects left-recursive programs.}

\plan{A second table of results shows that adjusting the context limit allows left-recursion to run faster.}
\paul{I would like to try yet-another experiment on left recursion where we remove this limit completely}

\plan{Reinforce that these results support the idea that scheduling decisions are made prematurely}



%Figure \ref{fig:map_right_and_left_recursive} shows two alternative, parallel
%implementations of \code{map/3}.
%While their declarative semantics are identical,
%their operational semantics are very different.  In Section
%\ref{sec:backgnd_merpar} we explained that parallel conjunctions are
%implemented by spawning off the tail of the conjunction and executing the
%head directly.
%This means that in the right recursive case (Figure
%\ref{fig:map_right_recursive}), the recursive call is spawned off as a
%spark,
%and that in the left recursive case (Figure \ref{fig:map_left_recursive}),
%the recursive call is executed directly, and the loops \emph{body} is
%spawned off.

\begin{table}
\paul{Make this easier to analyse by making the second column the actual
number of contexts, not contexts per engine.}
\begin{center}
\begin{tabular}{lr|rrrrrr}
\multicolumn{1}{c|}{Recursion type} &
\multicolumn{1}{c|}{Max no.\ of contexts} &
\multicolumn{2}{|c|}{Sequmential} &
\multicolumn{4}{|c}{Parallel w/ $N$ Engines} \\
\Cbr{} & & \C{not TS} & \Cbr{TS} & \C{1}& \C{2}& \C{3}& \C{4}\\
\hline
\multirow{5}{*}{Right} &
 2      & 23.2       & 21.5     & 21.5 & 21.6 & 21.6 & 21.6 \\
&32     & -          & -        & 21.5 & 21.6 & 21.5 & 21.2 \\
&64     & -          & -        & 21.5 & 19.8 & 18.5 & 16.5 \\
&128    & -          & -        & 21.5 & 13.2 &  8.2 &  6.1 \\
&256    & -          & -        & 21.5 & 12.2 &  8.1 &  6.1 \\
\hline
\multirow{5}{*}{Left} &
 2      & 23.2       & 21.5     & 21.5 & 21.5 & 21.5 & 21.5 \\
&32     & -          & -        & 21.5 & 21.5 & 21.5 & 21.5 \\
&64     & -          & -        & 21.5 & 21.5 & 20.5 & 19.0 \\
&128    & -          & -        & 21.5 & 18.5 & 15.8 & 12.9 \\
&256    & -          & -        & 21.5 & 17.8 & 15.6 & 14.1 \\
\end{tabular}
\end{center}
\caption{Right vs.\ left recursion.}
\label{tab:right_v_left}
\end{table}

%Table \ref{tab:right_v_left} shows average elapsed time in seconds for the
%mandelbrot image generating program over 20 samples.
%The program iterates over the rows in the mandelbrot image using
%a left or right recursive parallel
%\code{map/3} predicate such as those in Figure
%\ref{fig:map_right_and_left_recursive},
%the leftmost column in the table indicates which predicate was used.
%The next column describes how many contexts may exist at once per Mercury
%engine.
%Values are omitted in the range 3--31 as they are not interesting:
%they are the same as the case for 2 contexts per engine.
%The next two columns give the runtime for a sequential version of the
%program,
%in this version of the program no parallel conjunctions where used.
%The first of these, labelled ``not TS'',
%is compiled without thread safety;
%the second, labelled ``TS'',
%is compiled with thread safety, meaning that it allows multiple engines to be
%used: requiring a register to point to a pthread's engine structure
%(Section \ref{sec:backgnd_merpar}),
%and compiling the garbage collector for thread safety.
%The following four columns give the runtimes for the parallel mandelbrot
%program with 1--4 Mercury engines.
%
%We can see that the right recursive program performs better than the left
%recursive one.
%The right recursive program, when using 4 cores and at least 128 contexts,
%achieves a speedup of 3.52 compared to the right recursive sequential thread
%safe program.
%At best the left recursive program is sped up by a factor of 1.67.
%The next observation is that as we allow more contexts to reside in memory
%at once,
%we achieve greater speedups.
%The exception to this is the case for left recursion with four engines and
%256 contexts, it is slower than the case for 256 contexts.
%These two results have standard deviations of 6.40 and 6.15 seconds
%respectively meaning that any apparent difference is most likely
%noise in the data.
%All the other results have a standard deviation of less than two seconds,
%most of these are less than half a second.
%We saw the same high variance in these two results when repeating the test.
%We will explain the probable cause for this later in this section.
%
%Parallel conjunctions are evaluated as described in Section
%\ref{sec:backgnd_merpar},
%the second and later conjuncts are spawned off while the first conjunct is
%executed by the current context.
%The barrier at the end of the first conjunct will block the context if
%the other conjuncts have not yet completed.
%It must be blocked because it would normally continue by executing the code
%after the parallel conjunction
%which may depend on values produced by the other conjuncts ---
%it is not safe for it to continue executing until all the conjuncts have
%been completely executed.
%
%\paul{consider diagram with stack}
%In the right recursive example,
%the recursive call is spawned off,
%the spark for the recursive call will be placed on the global spark queue.
%Another engine will wake up and take the spark from the global queue and
%convert it to a context.
%When it makes the recursive call it will execute the parallel conjunction
%inside and perform the same process.
%There is at most one spark on the global spark queue at any time.
%Each of these contexts,
%after creating the spark for their recursive call,
%will execute \code{P(X, Y)} and
%\joinandcontinue which blocks the context on the completion of
%its recursive call.
%This process continues:
%the runtime system will convert each spark into a context,
%and block each one at the barrier for the conjunction within its
%recursive call.
%This will quickly consume a lot of memory,
%most of which is used for the stacks within each context.
%
%Eventually the number of contexts in memory plus
%the number of sparks on the global queue is equal to the maximum number of
%contexts.
%At this point sparks are not added to the global queue but to their parent
%context's local stack.
%Contexts will not be created to execute these sparks.
%Therefore,
%this restricts the amount of memory allocated in contexts.
%Without this limit,
%a simple loop, such as in Figure \ref{fig:map_right_recursive},
%can easily consume all the physical memory in a system.
%This limit has the adverse effect of restricting how much parallelism
%in the program is exploited.
%This is why performance improves as we allow more contexts per engine.
%We will solve the memory usage problem and context limit in
%Chapter \ref{chap:loop_control}.
%
%In the left recursive program scheduling is quite different.
%The parallel conjunction creates a spark for \code{P(X, Y)} and executes the
%recursive call directly.
%The spark is converted into a context,
%that context does not execute another parallel conjunction since it does not
%execute the recursive call.
%Therefore, it will not become blocked on the \joinandcontinue barrier in any
%nested parallel conjunction.
%It will execute the barrier after \code{P(X, Y)},
%this however does not block this context.
%The context is not the conjunction's original context and therefore once it
%reaches this barrier it is free,
%if it has any sparks on its local queue it may execute them,
%otherwise the engine executing it will look for global work,
%either another context or a spark from the global queue.
%If there is a spark on the global queue the engine will use this context to
%execute it since the context is otherwise unused.
%
%This led us to believe that the left recursion would be more efficient than
%right recursion,
%namely that since contexts are reused, the number of contexts wouldn't climb
%and prevent parallelism from being exploited.
%As Table \ref{tab:right_v_left} shows, we were wrong:
%the context limit is affecting performance.
%As discussed, a left-recursive loop spawns of calls to \code{P} as sparks
%and executes its recursive call directly.
%It will, very quickly,
%make many recursive calls, spawn off many sparks.
%The context limit includes sparks on the global queue since
%executing them can require the creation of new contexts,
%Furthermore, if they were not included and the runtime system refused to
%convert a spark on the global queue into a context the system could become
%deadlocked.
%In the left recursive case,
%the context limit will be reached very quickly,
%often before engines have begun taking sparks from the queue and executing
%them.
%Once the limit is reached sparks are placed on the contexts local queues
%where they cannot be executed in parallel.
%The smaller the context limit,
%the more quickly the limit is reached and the fewer contexts are placed on
%the global queue.
%Additionally,
%the loop placing sparks on its context's local stack will execute very
%quickly.
%
%We concluded that
%in the left recursive case
%the scheduling decision for each spark is made much earlier than the spark's
%execution.
%Specifically,
%when the decision to place the spark on the global queue or local stack is
%made,
%often the context limit has already been reached:
%\paul{Need to decide how I communicate who the actor is for scheduling
%decisions.}
%the context will place the spark on its local stack.
%Later, when a different engine becomes idle,
%it cannot access the spark since it is on another engine's context's spark
%stack.
%At this point it is apparent that the scheduling decision made when the
%spark was placed on the local stack was incorrect,
%as there is an idle engine ready to execute the spark,
%and because contexts are re-used (in left recursion) there is either a free
%context or we can easily create one.

\begin{table}
\begin{center}
\begin{tabular}{lr|rrrrrr}
\multicolumn{1}{c|}{} &
\multicolumn{1}{c|}{Max no.\ of contexts} &
\multicolumn{2}{|c|}{Sequmential} &
\multicolumn{4}{|c}{Parallel w/ $N$ Engines} \\
\Cbr{} & & \C{not TS} & \Cbr{TS}  & \C{1}& \C{2}& \C{3}& \C{4}\\
\hline
\multirow{5}{*}{Include} &
 2       & 23.2       & 21.5      & 21.5 & 21.5 & 21.5 & 21.5 \\
&32      & -          & -         & 21.5 & 21.5 & 21.5 & 21.5 \\
&64      & -          & -         & 21.5 & 21.5 & 20.5 & 19.0 \\
&128     & -          & -         & 21.5 & 18.5 & 15.8 & 12.9 \\
&256     & -          & -         & 21.5 & 17.8 & 15.6 & 14.1 \\
\hline
Exclude &
-        & 23.3       & 21.5      & 21.7 & 17.9 & 15.6 & 14.2 \\
\end{tabular}
\end{center}
\caption{Left recursion with and without global sparks included in the context
limit}
\label{tab:2009_nolimit}
\end{table}

\plan{Avoid the right-recursion problem by reordering independent conjunctions.}


\section{Work stealing implementation}
\label{sec:work_stealing}

The stack data structure used is described by \citet{workstealing_queue},
this was chosen as it could support work stealing (Section
\ref{sec:work_stealing}).
%The top of the stack is sometimes referred to as the \emph{hot end},
%since it is used frequently.
%Likewise, the bottom of the stack is sometimes called the
%\emph{cold end}, since it is only used to implement work stealing.
%The top of the stack, or hot end, can be used without locking or atomic
%operations.
%This is desirable, since it makes common operations inexpensive.
%The local context uses the top of the stack only,
%therefore sparks placed on this stack are scheduled in a
%last-in-first-out manner.

\status{Not written}

\plan{Acknowledge work stealing used elsewhere in other runtime systems.}

\subsection{Initial work stealing implementation}

\plan{I believe that Peter discussed work stealing briefly in his thesis,
I should check this and see what he said.}

\plan{Say that stacks queues are associated with contexts.}

\plan{Describe the data structure used to implement these stacks and its properties.}
Local context can use the hot end without synchronisation.
Other contexts can use the cold end with a CAS,
Memory barriers are used to ensure writes appear in the correct order.

\plan{Describe how work stealing policies work}
What happens to the global queue,
in what cases is work stolen,
Show algorithms for work stealing attempts.
Describe work stealing timeout.

\plan{Benchmark}

This subsection was joint work with Peter Wang.

\subsection{Final work stealing implementation}

\plan{Describe problems with associating stacks with contexts}
The number of stacks varies,
Stealing uses a global lock to determine which stack to steal from.

\plan{Prove that even though there are N engines and M contexts and M may be
larger than N, that there will be at most N of the M contexts with work on
their queues}
Therefore:
Stealing is unnecessary complicated.
In pathological cases many attempts can be made without success,
a thief may give up even though there is parallelism.

\plan{We associate stacks with engines}
This removes the above problems how.

\plan{Show stealing algorithm there are any,}
Find out if I started tracking stealing per engine or not.

\plan{Show how this is safe.}
When a parallel conjunction's barrier is executed and a conjunct is
outstanding, if its spark is on this engine's stack it must be at the top of
the stack.
This invariant should be kept because the context can be re-used if
'compatible' work is found at the top of the spark stack.
This invariant is soft.

\plan{Benchmark}

\section{Thread pinning}
\label{sec:thread_pinning}
\status{Not written}

\plan{Explain why we want $P$ engines when there are $P$ processors}

\plan{Explain briefly how we detect how many processors there are,}
Explain that this method is good because it is (mostly) cross platform.

\plan{Explain why we want thread pinning.}
I wonder if there's any relevant literature.

\plan{Explain how we get thread pinning.}
This method is also (mostly) cross platform.

\plan{What about SMT, not all processors are equal.}
This does not matter when we are creating $P$ engines.
But it does matter when we create less than $P$ engines,
explain how bad CPU assignments are sub-optimal.

\plan{How do we handle SMT}
This uses a support library, which is cross platform, provided that it is installed.
We fall back to setcpuafinity() when it is not.

\paul{I am not going to talk about busy waiting since I have not written the
runtime system in a way that I can test or change this easily.}

\section{Idle loop}
\label{sec:idle_loop}
\status{Not written}

\plan{Describe the problem with the current algorithm.}
Engines wake up and periodically check for work by attempting to
steal work,
firstly, this means that there can be up to a 2ms (average 1ms) delay before
a spark is executed.
secondly, this checks for work too often, wasting resources.

\plan{Solution, wake engines for different types of work}
We modified the RTS so that waking an engine is easy, and it can be given a
message so that it knows where to look for work.

\plan{A lot of work went into preventing deadlocks due to race conditions,
a thread that is not yet sleeping if notified must wake up immediately.}

\plan{Extra benefit: when an engine is woken it can be told directly where
to find work, or be given the work directly.}

\plan{Extra benefit: we can be selective about which engine to wake,
while not implemented fully, we can wake a `nearby' engine so that
we can avoid communication between dies or sockets.}

\plan{Show the algorithm for the new idle loop.}
Note that we execute contexts before sparks,
this is more-likely to produce futures and it may reduce memory consumption.

\section{Proposed scheduling tweaks}
\label{sec:proposed_tweaks}
\status{Not written, May move to TS chapter}

I really think that this section will move to the threadscope chapter,
it will have more in common with that chapter and more data will be
available.
Secondly, threadscope can be used with micro-benchmarks to measure the
average costs of certain operations in the RTS.
I will not write it until at least the rest of this chapter is finished.

%\section{Proposed kernel support to manage processor resources}
%\label{sec:kernel_scheduling_help}
%
%\status{This may not be worth discussing until someone actually does it}
%
%Should I describe our proposal for OS kernel's to help
%applications with how many threads to use.
%GCD is related but does not fit into a language runtime system so
%easily~\cite{apple_gcd}.
%See also N:M threading.

%\section{Spare text}
%
%\status{This text will be moved up into one of the work stealing sections
%once those sections are ready}

%The spark is added to a global run queue of sparks, or if that queue is too
%full, because there's already enough parallelism,
%then the spark is added to a queue owned by the current context.
%\citet{wang_hons_thesis} intended to use the local queues for work stealing
%but had not completed his implementation,
%The work-stealing dequeue structure
%is described in \citet{Chase_2005_wsdeque}.
%see Chapter \ref{chap:rts} for details.
%
%and finds that spark is still at the head of its queue,
%it will pick it up and run it itself.
%This is a useful optimisation,
%% it is also really well-known.
%since it avoids using a separate context in the relatively common case
%that all the other CPUs are busy with their own work.
%This optimisation is also useful since it can avoid the creation of superfluous
%contexts and their stacks.
%

%When an engine becomes idle, it will first try
%to resume a suspended but runnable context if there is one.
%If not, it will attempt to run a spark from the global spark queue.
%If it successfully finds a spark, it will allocate a context,
%and start running the spark in that context.

% XXX: Mention global spark queue and spark sheduling above.
% XXX:

%Barrier code is placed at the end of each conjunct,
%this is named \code{join\_and\_continue} (Figure \ref{fig:par_conj}).
%This code starts by atomically decrementing the number of outstanding
%conjuncts in the conjunction's syncterm and checking the result for zero
%(the whole operation is thread-safe, not just the decrement).
%Algorithm \ref{alg:join_and_continue} shows the pseudo code for
%join\_and\_continue.


%XXX
%Parallel conjunctions are evaluated as described in Section
%\ref{sec:backgnd_merpar}.
%Sparks are added to the global spark queue if the global queue has room,
%otherwise they are pushed onto a context-local spark stack.
%Creating a lot of parallel work and using a single global work queue can be
%pesimistic,
%the queue itself can become a bottleneck:
%when many processors try to access it the same location in memory there will
%be many cache misses and delays.
%This is why \citet{wang_hons_thesis} choose to introduce local queues,
%and place work on them when there is a surplus of work on the global queue.
%
%When a engine finishes executing a context and reaches the barrier at the
%end of a parallel conjunct,
%it will check the context's local spark stack for any other work and attempt
%to execute it.
%Otherwise, it will either save the context to resume later or release the
%context before checking the global spark queue and global context queue.
%
%XXX Algorithm.
%
%Deciding too early, results.
%
