
\chapter{Runtime System Improvements}
\label{chap:rts}

\status{This chapter is finished!}

Early in the project
we tested two manually parallelised programs:
a raytracer and a mandelbrot image generator.
Both programs have a single significant loop
whose iterations are independent of one another.
We expect that a good automatic parallelisation system will parallelise this
loop as it is the best place to introduce parallelism.
When we parallelised this loop manually,
we did not get the speedups that we expected.
Therefore,
we chose to address the performance problems
before we worked on automatic parallelism.
Throughout this chapter
we continue to use these two benchmarks, along with a naive Fibonacci
program (Page~\pageref{page:fibs}).
These benchmarks are not diverse and they all create a lot of
AND-parallelism,
most of which is independent.
We use these benchmarks deliberately to test that our runtime system can
handle large amounts of parallelism efficiently.

In this chapter we investigate and correct these performance problems.
We start with the garbage collector in Section~\ref{sec:rts_gc};
we analyse the collector's effects on performance and tune its parameters
to improve performance.
In Section~\ref{sec:rts_original_scheduling} we describe how the existing runtime
system schedules sparks,
and provide background material for
Section~\ref{sec:rts_original_scheduling_performance},
which benchmarks the runtime system and describes two significant problems with
spark scheduling.
We address one of these problems by introducing work stealing in
Section~\ref{sec:rts_work_stealing}.
Then in Section~\ref{sec:rts_reorder} we reorder conjuncts in independent
parallel conjunctions to work around the second spark scheduling problem.
Finally, in Section~\ref{sec:rts_work_stealing2} we make further
improvements to
work stealing and change the data structures and algorithms used to manage
idle engines,
including how idle engines look for work, sleep and are woken up.


\section{Garbage collector tweaks}
\label{sec:rts_gc}

One of the sources of poor parallel performance is the behaviour of the
garbage collector.
Like other pure declarative languages,
Mercury does not allow destructive update.
%\footnote{
%    Destructive update is allowed via support for mutable variables.
%    Their use does not interfere with parallelism as they are used in
%    conjunction with impurity.
%    The compiler will not parallelise impure goals.}
Therefore a call usually returns its value in newly allocated memory
rather than modifying the memory of its parameters.
Likewise, a call cannot modify data that may be aliased.
This means that Mercury programs often have a high rate of allocation,
which places significant stress on the garbage collector.
Therefore,
allocation and garbage collection can reduce a program's
performance as we will show in this section.

\plan{Introduce Boehm, \& details: conservative, mark and sweep, stop the
world, parallel marking.}
Mercury uses the Boehm-Demers-Weiser conservative garbage collector (Boehm GC)
\citep{boehm:1988:gc},
which is a conservative mark and sweep collector.
Boehm GC supports parallel programming:
it will stop all the program's threads (\emph{stop the world}) during its
marking phase.
It also supports parallel marking:
it will use its own set of pthreads to do parallel marking.

\plan{Introduce collector time, mutator time.}
For the purposes of this section
we separate the program's execution time into two alternating phases:
Collector time, which is when Boehm GC performs marking,
and mutator time, which is when the Mercury program runs.
The name `mutator time' refers to time that mutations (changes) to memory
structures are permitted.
The collector may also perform some actions,
such as sweeping,
concurrently with the mutator.

\plan{Describe theory of GC performance.}
Amdahl's law~\citep{amdahl:1967:law} describes the maximum speedup that
can theoretically be achieved by parallelising a part of a program.
We use Amdahl's law to predict the speedup of a program whose
mutator is parallelised but whose collector runs sequentially.
Consider a program with a runtime of 20 seconds
which can be separated into one second of collector time and 19 seconds
of mutator time.
The sequential execution time of the program is $1 + 19 = 20$.
If we parallelise the mutator and do not parallelise the
collector then the minimum parallel execution time is $1 + 19/P$
for $P$ processors.
Using four processors the theoretical best speedup is:
%\paul{I prefer how \\frac displays maths but in text the fonts become tiny.}
$(1 + 19) / (1 + 19/4) = 3.48$.
17\% of the parallel runtime ($1 + 19/4$) is collector time.
If we use a machine with 100 processors then this becomes:
$(1 + 19) / (1 + 19/100) = 16.8$;
with 84\% of the runtime spent in the collector.
As the number of processors increases,
the mutator threads spend a larger proportion of their time waiting for
collection to complete.

\plan{Discuss predictions regarding parallel marking, locking and thread
local heaps.}
To reduce this problem,
\citet{boehm:1988:gc} included parallel marking support in their collector.
Ideally this would remove the bottleneck described above.
However,
parallel garbage collection is a continuing area of research and
the Boehm GC project has only modest multicore scalability goals.
Therefore,
we expect parallel marking to only partially reduce the bottleneck,
rather than remove it completely.

Furthermore, thread safety has two significant performance costs;
These types of costs prevent us from achieving the theoretical maximum
speedups that Amdahl's law predicts.
The first of these is that
one less CPU register is available to GCC's code generator when compiling
parallel Mercury programs (Section~\ref{sec:backgnd_merpar}).
The second is that
memory allocation routines must use locking to protect shared data
structures,
which will slow down allocation.
Boehm GC's authors recognised this problem and
added support for thread-local resources such as free lists.
Therefore,
during memory allocation,
a thread uses its own free lists rather than locking a global structure.
From time to time, a thread will have to retrieve new free lists
from a global structure and will need to lock the structure,
but the costs of locking will be amortised across several memory allocations.

\plan{Describe icfp2000, mandelbrot\_lowalloc and mandelbrot\_highalloc
programs.}
To test how well Mercury and Boehm GC scale to multiple cores we used several
benchmark programs with different memory allocation requirements.
We wanted to determine how memory allocation rates affect
performance of parallel programs.
Our first benchmark is a raytracer developed for the
ICFP programming contest in 2000.
For each pixel in the image,
the raytracer casts a ray into the scene to determine what colour to paint that pixel.
Two nested loops build the pixels for the image:
the outer loop iterates over the rows in the image and
the inner loop iterates over the pixels in each row.
We manually parallelised the program by introducing a parallel
conjunction into the outer loop,
indicating that rows should be drawn in parallel.
This parallelisation is independent.
The raytracer uses many small structures to represent vectors and a pixel's
colour.
It is therefore memory allocation intensive.
Since we suspected that garbage collection in memory allocation intensive
programs was having a negative affect on performance,
we developed a mandelbrot image generator.
The mandelbrot image generator has a similar structure to the raytracer,
but is designed so that we can test programs with different allocation rates.
It draws an image using two nested loops as above.
The mandelbrot image is drawn on the complex number plane.
A complex number $C$ is in the mandelbrot set if
$\forall i \cdot |N_i| < 2$ where $N_0 = 0$ and $N_{i+1} = N_{i}^2 + C$.
For each pixel in the image the program tests if the pixel's coordinates are
in the set.
The pixel's colour is chosen based on how many iterations $i$ are needed
before $|N_i| \ge 2$,
or black if the test survived 5,000 iterations.
We parallelised this program the same way as we did the raytracer:
by introducing an independent parallel conjunction into the outer loop.
We created two versions of this program.
The first version represents coordinates and complex numbers as structures
on the heap.
Therefore it has a high rate of memory allocation.
We call this version `mandelbrot\_highalloc'.
The second version of the mandelbrot program,
called `mandelbrot\_lowalloc',
stores its coordinates and complex numbers on the stack and in registers.
Therefore it has a lower rate of memory allocation.


\begin{table}
\begin{center}
\begin{tabular}{r|rr|rrrr}
\Cbr{GC Markers} &
\multicolumn{2}{c|}{Sequential} &
\multicolumn{4}{c}{Parallel w/ $N$ Mercury Engines} \\
\Cbr{} &
\C{no TS} & \Cbr{TS} &
\C{1} & \C{2} & \C{3} & \C{4} \\
\hline
\hline
\multicolumn{7}{c}{raytracer} \\
\hline
1 & 33.1 (1.13) & 37.4 (1.00) &
             37.7 (0.99) & 28.0 (1.34) & 24.8 (1.51) & 23.7 (1.58) \\
2  & - & - & 32.0 (1.17) & 21.6 (1.73) & 18.1 (2.07) & 16.7 (2.24) \\
3  & - & - & 29.6 (1.26) & 19.7 (1.90) & 16.3 (2.29) & 14.5 (2.59) \\
4  & - & - & 29.1 (1.29) & 18.9 (1.98) & 15.3 (2.45) & 13.7 (2.73) \\
\hline
\hline
\multicolumn{7}{c}{mandelbrot\_lowalloc} \\
\hline
1 & 15.4 (0.99) & 15.2 (1.00) &
             15.2 (1.00) &  5.1 (2.96) &  7.7 (1.98) &  3.9 (3.92) \\
2  & - & - & 15.3 (1.00) &  7.7 (1.98) &  5.1 (2.97) &  3.9 (3.94) \\
3  & - & - & 14.3 (1.00) &  7.7 (1.98) &  5.1 (2.97) &  3.9 (3.94) \\
4  & - & - & 15.3 (0.99) &  7.7 (1.98) &  5.1 (2.97) &  3.9 (3.92) \\ 
\hline
\hline
\multicolumn{7}{c}{mandelbrot\_highalloc} \\
\hline
1 & 41.0 (1.23) & 50.5 (1.00) &
             50.6 (1.00) & 33.4 (1.51) & 26.6 (1.90) & 23.4 (2.16) \\
2  & - & - & 46.7 (1.08) & 28.6 (1.77) & 22.1 (2.28) & 18.8 (2.69) \\
3  & - & - & 46.3 (1.09) & 27.3 (1.85) & 20.9 (2.42) & 17.3 (2.93) \\
4  & - & - & 45.6 (1.11) & 26.8 (1.89) & 20.4 (2.48) & 16.7 (3.03) \\
\end{tabular}
\end{center}
\paul{Consider drawing charts for raytracer and mandelbrot so that releative
performance between different scores can be seen easily.}
\caption{Parallelism and garbage collection}
\label{tab:gc}
\end{table}



\plan{Benchmark data.}
We benchmarked these three programs with different numbers of Mercury
engines and garbage collector threads.
We show the results in Table~\ref{tab:gc}.
All benchmarks have an initial heap size of 16MB.
Each result is measured in seconds and represents the mean of eight test runs.
The first column is the number of garbage collector threads used.
The second and third columns give sequential execution times
without and with thread safety\footnote{
    A thread safe build of a Mercury program enables thread safety in the
    garbage collector and runtime system.
    It also requires that one less CPU register is available to Mercury
    programs (see Section~\ref{sec:backgnd_merpar})}
enabled.
In these columns the programs were compiled in such a way so that they did
not execute a parallel conjunction.
The remaining four columns give parallel execution times using one to four
Mercury engines.
The numbers in parentheses show the relative speedup when compared with the
sequential thread safe result for the same program.
\label{cabsav}
We ran our benchmarks on
%\paul{Find a brand name i7 I can borrow for some weeks}
a four-core Intel i7-2600K system
with 16GB of memory,
running Debian/GNU Linux 6
with a 2.6.32-5-amd64 Linux kernel and GCC 4.4.5-8.
We kept frequency scaling (Speedstep and TurboBoost) disabled.
The garbage collection benchmarks were gathered using a recent version of
Mercury (rotd-2012-04-29).
This version does not have the performance problems described in the
rest of this chapter,
and it does have the loop control transformation described
in Chapter~\ref{chap:loop_control}.
Therefore we can observe the effects of garbage collection without
interference from any other performance problems.

\plan{Describe performance in practice.}
The raytracer program benefits from parallelism in both Mercury and the
garbage collector.
Using Mercury's parallelism only (four Mercury engines, and one GC thread) 
speeds the program up by a factor of 1.58,
compared to 1.29 when using the GC's parallelism only (one Mercury engine,
and four GC threads).
When using both Mercury and the GC's parallelism (four engines and four
marker threads)
the raytracer achieves a speedup of 2.73.
These speedups are much lower than we might expect from such a program:
either the mutator, the collector or both are not being parallelised well.
mandelbrot\_lowalloc does not see any benefit from parallel marking.
It achieves very good speedups from multiple Mercury engines.
We know that this program has a low allocation rate
but is otherwise parallelised in the same way as raytracer.
Therefore,
these results support the hypothesis that heavy use of garbage collection
makes it difficult to achieve good speedups when parallelising programs.
The more time spend in garbage collection,
the worse the speedup due to parallelism.
mandelbrot\_highalloc, which stores its data on the heap,
sees similar trends in performance as raytracer.
It is also universally slower than mandelbrot\_lowalloc.
mandelbrot\_highalloc achieves a speedup of 2.16 when using parallelism in
Mercury (four Mercury engines, and one GC thread).
The corresponding figure for the raytracer is 1.58.
When using the GC's parallelism
(one Mercury engine, and four GC threads)
mandelbrot\_highalloc achieves a speedup of 1.11,
compared with 1.29 for the raytracer.


\begin{table}
\begin{center}
\begin{tabular}{r|l| d{1} r | d{1} r | d{1} r | d{1} r }
\C{GC} & \C{Times \&} &
\multicolumn{8}{c}{Parallel w/ $N$ Mercury Engines} \\
\C{Markers} & \C{Collections}
          & \Ctwo{1}        & \Ctwo{2}        & \Ctwo{3}        & \Ctwo{4} \\
\hline
\hline
\multicolumn{10}{c}{raytracer} \\
\hline
\multirow{4}{*}{1} &
  Elapsed & 37.8 &        & 28.1 &        & 24.7 &        & 22.7 & \\
& GC      & 16.9 & 44.8\% & 16.7 & 59.4\% & 16.5 & 66.6\% & 15.8 & 69.5\% \\
& Mutator & 20.9 & 55.2\% & 11.4 & 40.6\% &  8.2 & 33.4\% &  6.9 & 30.5\% \\
& \# col. & 384. &        &346.  &        &316.  &        &288.  & \\
\hline
\multirow{4}{*}{2} &
  Elapsed & 32.6 &        & 21.8 &        & 18.3 &        & 16.0 & \\
& GC      & 11.0 & 33.8\% & 10.4 & 47.7\% & 10.3 & 56.4\% &  9.8 & 60.8\% \\
& Mutator & 21.6 & 66.2\% & 11.4 & 52.3\& &  8.0 & 43.6\% &  6.3 & 39.2\% \\
& \# col. & 384. &        &346.  &        &316.  &        &288.  & \\
\hline
\multirow{4}{*}{3} &
  Elapsed & 30.9 &        & 20.0 &        & 16.4 &        & 14.1 & \\
& GC      &  9.3 & 30.2\% &  8.5 & 42.3\% &  8.2 & 50.0\% &  7.8 & 55.4\% \\
& Mutator & 21.6 & 69.8\% & 11.5 & 57.7\% &  8.2 & 50.0\% &  6.3 & 44.6\% \\
& \# col. & 386. &        &346.  &        &316.  &        &288.  & \\
\hline
\multirow{4}{*}{4} &
  Elapsed & 30.3 &        & 19.3 &        & 15.5 &        & 13.5 & \\
& GC      &  8.7 & 28.8\% &  7.8 & 40.4\% &  7.4 & 47.8\% &  7.1 & 52.4\% \\
& Mutator & 21.6 & 71.2\% & 11.5 & 59.6\% &  8.1 & 52.2\% &  6.4 & 47.6\% \\
& \# col. & 387. &        &346.  &        &316.  &        &288.  & \\
\hline
\hline
\multicolumn{10}{c}{mandelbrot\_lowalloc} \\
\hline
\multirow{4}{*}{1} &
  Elapsed & 15.2 &        &  7.7 &        &  5.1 &        &  3.9 & \\
& GC      &  0.0 &  0.1\% &  0.0 &  0.2\% &  0.0 &  0.7\% &  0.0 &  0.4\% \\
& Mutator & 15.2 & 99.9\% &  7.6 & 99.8\% &  5.1 & 99.3\% &  3.9 & 99.6\% \\
& \# col. &  2.  &        &  2.  &        &  2.  &        &  2.  & \\
\hline
\multirow{4}{*}{2} &
  Elapsed & 15.4 &        &  7.6 &        &  5.1 &        &  3.9 & \\
& GC      &  0.0 &  0.1\% &  0.0 &  0.1\% &  0.0 &  0.2\% &  0.0 &  0.2\% \\
& Mutator & 15.4 & 99.9\% &  7.6 & 99.9\% &  5.1 & 99.8\% &  3.9 & 99.8\% \\
& \# col. &  2.  &        &  2.  &        &  2.  &        &  2.  & \\
\hline
\multirow{4}{*}{3} &
  Elapsed & 15.3 &        &  7.7 &        &  5.1 &        &  3.9 & \\
& GC      &  0.0 &  0.1\% &  0.0 &  0.1\% &  0.0 &  0.4\% &  0.0 &  0.1\% \\
& Mutator & 15.2 & 99.9\% &  7.7 & 99.9\% &  5.1 & 99.6\% &  3.9 & 99.9\% \\
& \# col  &  2.  &        &  2.  &        &  2.  &        &  2.  & \\
\hline
\multirow{4}{*}{4} &
  Elapsed & 15.2 &        &  7.6 &        &  5.1 &        &  3.9 & \\
& GC      &  0.0 &  0.1\% &  0.0 &  0.1\% &  0.0 &  0.4\% &  0.0 &  0.1\% \\
& Mutator & 15.2 & 99.9\% &  7.6 & 99.9\% &  5.1 & 99.6\% &  3.9 & 99.9\% \\
& \# col  &  2.  &        &  2.  &        &  2.  &        &  2.  & \\
\hline
\hline
\multicolumn{10}{c}{mandelbrot\_highalloc} \\
\hline
\multirow{4}{*}{1} &
  Elapsed & 51.7 &        & 34.0 &        & 27.2 &        & 24.0 & \\
& GC      & 11.0 & 21.4\% & 11.5 & 33.6\% & 11.5 & 42.2\% & 11.8 & 49.1\% \\
& Mutator & 40.7 & 78.6\% & 22.6 & 55.4\% & 15.5 & 57.8\% & 12.2 & 50.9\& \\
& \# col. &742.  &        &692.  &        &634.  &        &602. & \\
\hline
\multirow{4}{*}{2} &
  Elapsed & 48.5 &        & 29.3 &        & 22.3 &        & 19.1 & \\
& GC      &  6.4 & 13.1\% &  6.7 & 22.9\% &  6.6 & 29.7\% &  6.9 & 36.1 \\
& Mutator & 42.1 & 86.9\% & 22.6 & 77.1\% & 15.7 & 70.3\% & 12.2 & 63.9 \\
& \# col. &744.  &        &693.  &        &633.  &        &595.  & \\
\hline
\multirow{4}{*}{3} &
  Elapsed & 46.4 &        & 28.0 &        & 21.0 &        & 17.7 & \\
& GC      &  5.0 & 10.8\% &  5.3 & 18.8\% &  5.3 & 25.0\% &  5.4 & 30.4\% \\
& Mutator & 41.4 & 89.2\% & 22.8 & 81.2\% & 15.7 & 75.0\% & 12.3 & 69.6\% \\
& \# col. &737.  &        &692.  &        &629.  &        &600.  & \\
\hline
\multirow{4}{*}{4} &
  Elapsed & 46.0 &        & 27.6 &        & 20.3 &        & 16.9 & \\
& GC      &  4.5 &  9.9\% &  4.7 & 17.0\% &  4.6 & 22.6\% &  4.7 & 27.6\% \\
& Mutator & 41.4 & 90.1\% & 22.9 & 83.0\% & 15.7 & 77.4\% & 12.3 & 72.4\% \\
& \# col. &740.  &        &695.  &        &626.  &        &600.  & \\
\end{tabular}
\end{center}
\caption{Percentage of elapsed execution time used by GC/Mutator}
\label{tab:gc_amdahl}
\end{table}



\plan{Introduce the table we use for our discussion of Amdahl's law.}
We can see that mandelbrot\_highalloc benefits from parallelism in Mercury
more than raytracer does,
conversely mandelbrot\_highalloc benefits from parallelism in the garbage
collector less than raytracer does.
Using \tscope (Chapter~\ref{chap:tscope}),
we analysed how much time these programs spend running the garbage collector
or the mutator.
These results are shown in Table~\ref{tab:gc_amdahl}.
The table shows the elapsed time, collector time, mutator time and the
number of collections for each program using one to four engines, and one
to four collector threads.
The times are averages taken from eight samples,
using an initial heap size of 16MB.
To use \tscope we had to compile the runtime system differently.
Therefore,
these results differ slightly from those in Table~\ref{tab:gc}.
Next to both the collector time and mutator time
we show the percentage of elapsed time taken by the collector or mutator
respectively.

\plan{Describe trends in either the mutator or GC time.}
As we expected, mandelbrot\_lowalloc spends very little time running the collector.
Typically, it ran the collector only twice during its execution.
Also, total collector time was usually between 5 and 30 milliseconds.
The other two programs ran the collector hundreds of times.
As we increased the number of Mercury engines
we noticed that these programs made fewer collections.
As we varied the number of collector threads,
we saw no trend in the number of collections.
Any apparent variation is most likely noise in the data.
All three programs see a speedup in the time spent in the mutator as the
number of Mercury engines is increased.
Similarly,
the raytracer and mandelbrot\_highalloc benefit from speedups
in GC time as the number of GC threads is increased.
In mandelbrot\_highalloc,
the GC time also increases slightly as Mercury engines are added.
We expect that as more Mercury engines are used the garbage collector
must use more inter-core communication which has additional costs.
However, in the raytracer,
the GC time decreases slightly as Mercury engines are added.
To understand why this happens we would need to understand the collector in
detail,
however Boehm GC's sources are notoriously difficult to read and
understand.

\plan{Compare performance with Amdahl's predictions.}
As Amdahl's law predicts,
parallelism in one part of the program has a limited effect on the program
as a whole.
While using one GC thread
we tested with one to four Mercury engines;
the mutator time speedups for two, three and four engines were
1.83, 2.55 and 3.03 respectively.
However, the elapsed time speedups for the same test were only
1.35, 1.53 and 1.66 respectively.
Although there is little change in absolute time spent in the collector;
there is an increase in collector time as a percentage of elapsed time.
The corresponding mutator time speedups for mandelbrot\_highalloc were
similar,
at 1.80, 2.63 and 3.34 for two, three and four Mercury engines.
The elapsed time speedups on the same test for mandelbrot\_highalloc were
1.52, 1.90 and 2.15.
Like raytracer above, the elapsed time speedups are lower than the mutator
time speedups.
Similarly,
when we increase the number of threads used by the collector,
it improves collector time more than it does elapsed time.

When we increased both the number of threads used by the collector and the
number of Mercury engines
(a diagonal path through the table)
both the collector and mutator time decrease.
This shows that parallelism in Mercury and in Boehm GC both contribute to
the elapsed time speedup we saw in the diagonal path in Table~\ref{tab:gc}.
As Table~\ref{tab:gc_amdahl} give us more detail,
we can also see that collector time as a percentage of elapsed time
increases as we add threads and engines to the collector and Mercury.
This occurs for both the raytracer and mandelbrot\_highalloc.
It suggests that the mutator makes better use of additional Mercury
engines
than the collector makes use of additional threads.
We can confirm this by comparing the speedup for the mutator with the speedup
in the collector.
In the case of raytracer using four Mercury engines and one GC thread
the mutator's speedup is $20.9 / 6.9 = 3.03$ over the case for one Mercury
engine and one GC thread.
The collectors speedup with four GC threads and one Mercury engine over
the case for one GC thread and one Mercury engine is $16.9 / 8.7 = 1.94$.
The equivalent speedups for mandelbrot\_highalloc are:
$40.7 / 12.2 = 3.34$ and $11.0 / 4.5 = 2.44$.
Similar comparisons can be made along the diagonal of the table.

\begin{table}
\begin{center}
\begin{tabular}{l|rr|d{3}}
\Cbr{Program} & \C{Allocations} & \Cbr{Total alloc'd bytes} & \C{Alloc rate (M alloc/sec)} \\
\hline
raytracer   &     561,431,515 &           9,972,697,312 & 26.9 \\
mandelbrot\_highalloc
            &   3,829,971,662 &          29,275,209,824 & 94.1 \\
mandelbrot\_lowalloc
            &       1,620,928 &              21,598,088 &  0.106 \\
\end{tabular}
\end{center}
\caption{Memory allocation rates}
\label{tab:mem_alloc_rate}
\end{table}

\plan{Allocation intensity}
Table~\ref{tab:gc_amdahl} also shows that raytracer spends more of its
elapsed time doing garbage collection than mandelbrot\_highalloc does.
This matches the results in Table~\ref{tab:gc},
where mandelbrot\_highalloc has better speedups because of parallelism in
Mercury than raytracer does.
Likewise,
raytracer has better speedups because of parallelism in the collector
than mandelbrot\_highalloc does.
This suggests that mandelbrot\_highalloc is less allocation intensive than
raytracer,
but this is not true.
Using Mercury's deep profiler we measured the number and total size of memory
allocations.
This is shown in Table~\ref{tab:mem_alloc_rate}.
The rightmost column in this table gives the allocation rate
measured in millions of allocations per second.
It is calculated by dividing the number of allocations in the second column
by the average mutator time reported in Table~\ref{tab:gc_amdahl}.
Therefore, it represents the allocation rate during mutator time.
This is deliberate because,
by definition,
allocation cannot occur during collector time.
mandelbrot\_highalloc has a higher allocation rate than raytracer.
However,
it spends less time doing collection, when measured either absolutely
or by percentage of elapsed time.
Garbage collectors are tuned for particular workloads.
It is likely that Boehm GC handles mandelbrot\_highalloc's workload more
easily than raytracer's workload.

% Old idea:
%This may be due to changing allocation rates throughout the programs'
%executions.
%We noticed that while benchmarking the raytracer that if we rendered the
%image upside-down (by iterating over the rows backwards)
%we would get different performance results.
%This is because the top of the image we used contains sky,
%which has no geometry;
%it is the flat colour of the background,
%and the bottom of the image contains ground;
%which has more detail.
%Rendering the ground will naturally be more complicated and require more
%memory allocation than rendering the sky.
%Memory allocation in one part of a program can affect the performance of
%allocation and collection in another part of the program.
%This is many because the garbage collector will increase the size of the heap
%to satisfy large and/or many allocations.
%This is why rendering the image upside down had a different execution time to
%rendering it the right way up.
%Similarly,
%the mandelbrot program has sections in its image that require more iterations
%of the equation than others.


\begin{table}
\begin{center}
\begin{tabular}{r|rr|rrrr}
\Cbr{Initial} &
\multicolumn{2}{c|}{Sequential} &
\multicolumn{4}{c}{Parallel w/ $N$ Mercury Engines} \\
\Cbr{heap size} & \C{no TS}   & \Cbr{TS}    & \C{1}       & \C{2}       & \C{3}       & \C{4} \\
\hline
\hline
\multicolumn{7}{c}{raytracer} \\
\hline
1MB     & 32.8 (0.90) & 29.3 (1.01) & 29.3 (1.01) & 19.0 (1.57) & 15.4 (1.93) & 13.6 (2.19) \\
16MB    & 33.5 (0.89) & 29.7 (1.00) & 29.1 (1.02) & 18.9 (1.57) & 15.3 (1.94) & 13.7 (2.17) \\
32MB    & 34.3 (0.87) & 29.5 (1.01) & 29.5 (1.01) & 19.5 (1.52) & 15.6 (1.90) & 13.6 (2.18) \\
64MB    & 33.2 (0.89) & 30.4 (0.97) & 30.7 (0.97) & 20.3 (1.46) & 16.4 (1.81) & 14.5 (2.05) \\
128MB   & 22.7 (1.31) & 24.5 (1.21) & 24.2 (1.23) & 15.0 (1.97) & 11.7 (2.53) & 10.2 (2.91) \\
256MB   & 18.8 (1.57) & 21.4 (1.39) & 21.5 (1.38) & 12.6 (2.36) &  9.3 (3.18) &  7.7 (3.83) \\
384MB   & 17.7 (1.67) & 20.8 (1.43) & 20.7 (1.44) & 11.7 (2.53) &  8.7 (3.41) &  7.2 (4.14) \\
512MB   & 17.3 (1.71) & 20.3 (1.46) & 20.5 (1.45) & 11.5 (2.58) &  8.3 (3.56) &  6.8 (4.36) \\
\hline
\hline
\multicolumn{7}{c}{mandelbrot\_highalloc} \\
\hline
1MB     & 41.4 (1.10) & 45.5 (1.00) & 45.1 (1.00) & 26.7 (1.69) & 20.3 (2.23) & 16.7 (2.71) \\ 
16MB    & 39.7 (1.14) & 45.3 (1.00) & 45.6 (0.99) & 26.8 (1.69) & 20.4 (2.23) & 16.7 (2.72) \\
32MB    & 38.9 (1.16) & 45.2 (1.00) & 44.2 (1.02) & 26.3 (1.72) & 19.7 (2.29) & 16.5 (2.75) \\
64MB    & 36.7 (1.23) & 44.4 (1.02) & 43.7 (1.04) & 25.2 (1.80) & 18.8 (2.41) & 15.6 (2.90) \\
128MB   & 34.1 (1.33) & 42.6 (1.06) & 41.8 (1.08) & 24.1 (1.88) & 17.7 (2.56) & 14.4 (3.15) \\
256MB   & 33.2 (1.36) & 41.3 (1.10) & 41.9 (1.08) & 23.7 (1.91) & 16.9 (2.69) & 13.6 (3.34) \\ 
384MB   & 31.7 (1.43) & 41.0 (1.10) & 41.9 (1.08) & 23.3 (1.95) & 16.7 (2.71) & 13.4 (3.39) \\
512MB   & 31.1 (1.46) & 41.1 (1.10) & 41.1 (1.10) & 23.0 (1.97) & 16.7 (2.71) & 13.3 (3.40) \\ 
\end{tabular}
\end{center}
\caption{Varying the initial heapsize in parallel Mercury programs.}
\label{tab:heapsize}
\end{table}



\plan{New data, vary the heap size.}
So far,
we have shown that speedups due to Mercury's parallel conjunction
are limited by the garbage collector.
Better speedups can be achieved by using the parallel marking feature in the
garbage collector.
We also attempted to improve performance further by modifying the initial
heap size of the program.
Table~\ref{tab:gc_heapsize} shows the performance of the raytracer and
mandelbrot\_highalloc programs with various initial heap sizes.
The first column shows the heap size used;
we picked a number of sizes from 1MB to 512MB.
The remaining columns show the average elapsed times in seconds.
The first of these columns shows timing for the programs compiled for
sequential execution without thread safety;
this means that the collector cannot use parallel marking.
The next column is for sequential execution with thread safety;
the collector uses parallel marking with four threads.
The next four columns give the results for the programs compiled for
parallel execution, and executed with one to four Mercury engines.
These results also use four marker threads.
The numbers in parentheses are the ratio of elapsed time compared with the
sequential thread-safe result for the same initial heap size.
All the results are the averages of eight test runs.
We ran these tests on raytracer and mandelbrot\_highalloc.
We did not use mandelbrot\_lowalloc as its collection time is insignificant
and would not have provided useful data.

\plan{Observations: Programs get faster with a larger heap size.}
Generally, the larger the initial heap size the better the programs
performed.
The Boehm GC will begin collecting if it cannot satisfy a memory allocation
request.
If, after a collection, it still cannot satisfy the memory request then it
will increase the size of the heap.
In each collection,
the collector must read all the stacks, global data, thread local data and
all the in-use memory in the heap 
(memory that is reachable from the stacks, global data and thread local
data).
This causes a lot of cache misses, especially when the heap is large.
The larger the heap size,
the less frequently memory is exhausted and needs to be collected.
Therefore,
programs with larger heap sizes garbage collect less often and
have fewer cache misses due to garbage collection.
This explains the trend of increased performance as we increase the initial
heap size of the program.

Although performance improved with larger heap sizes,
there is an exception.
The raytracer often ran more slowly with a heap size of 64MB
than with a heap size of 32MB.
%\paul{Does the reader know what a TLB is?  People are generally poor at
%remembering to care about normal caches much less TLBs.}
64MB is much larger than the processor's cache size
(this processor's L3 cache is 8MB) and
covers more page mappings than its TLBs can hold (L2 TLB covers 2MB when
using 4KB pages).
The collector's structures and access patterns may be slower at this size
because of these hardware limitations,
and the benefits of a 64MB heap are not enough to overcome the effects of
these limitations,
however the benefits of a 128MB or larger heap are enough.

\plan{Speedup observation with larger heap sizes.}
Above, we said that programs perform better with larger heap sizes.
This measurement compares their elapsed time with one heap
size with the elapsed time using a different heap size.
We also found that programs \emph{exploited parallelism more easily} with
larger heap sizes:
This measurement compares programs' speedups (which are themselves
comparisons of elapsed time).
When the raytracer uses a 16MB initial heap its speedup using four
cores is 2.17 times.
However, when it uses a 512MB initial heap the corresponding speedup is
2.99.
Similarly,
mandelbrot\_highalloc has a 4-engine speedup of 2.72 using a 16MB initial
heap and a speedup of 3.08 using a 512MB initial heap.
Generally,
larger heap sizes allow programs to exploit more parallelism.
There are two reasons for this

\begin{description}

    \item[Reason 1]
    In a parallel program with more than one Mercury engine,
    each collection must \emph{stop-the-world}:
    All Mercury engines are stopped so that they do not modify the heap during
    the marking phase.
    This requires synchronisation which reduces the performance of parallel
    programs.
    Therefore, the less often collection occurs, the rarer stop-the-world
    events are, and the less their synchronisation affects performance.

    \item[Reason 2]
    Another reason was suggested by Simon Marlow:
    Because Boehm GC uses its own threads for marking and not Mercury's,
    it cannot mark objects with the same thread that allocated the objects.
    Unless Mercury and Boehm GC both map and pin their threads to
    processors,
    and agree on the mapping,
    then marking will cause a large number of cache misses.
    For example, during collection, processor one (P1) marks one of processor
    two's
    (P2) objects,
    causing a cache miss in P1's cache and invalidating the corresponding cache
    line in P2's cache.
    Later, when collection is finished,
    P2's process resumes execution and incurs a cache miss for the object that
    it had been using.
    % He never states this in either of his parallel GC papers.
    Simon Marlow made a similar observation when working with GHC's garbage
    collector.

\end{description}

\plan{Discuss local heaps for threads, and their reliability problems.}
We also investigated another area for increased parallel performance.
Boehm GC maintains thread local free lists that allow memory to be allocated
without contending for locks on global free lists.
When a local free list cannot satisfy a memory request, the global free
lists must be used.
The local free lists amortise the costs of locking the global free lists.
We anticipate that increasing the size of the local free lists will cause even
less contention for global locks,
allowing allocation intensive programs to have better parallel
performance.
The size of the local free list can be adjusted by increasing the
\texttt{HBLKSIZE} tunable.
Unfortunately this feature is experimental,
and adjusting \texttt{HBLKSIZE} caused our programs to crash.
Therefore we cannot evaluate how \texttt{HBLKSIZE} affects our
programs.
Once this feature is no longer experimental,
adjusting \texttt{HBLKSIZE} to improve parallel allocation should be
investigated.



\plan{Introduction}
We introduced parallelism in Mercury in Sections \ref{sec:backgnd_merpar} and
\ref{sec:backgnd_deppar};
we described how the runtime system in generic terms.
In this section we will explain how sparks were originally managed
prior to 2009,
when I begun my PhD candidature.
This will provide the background for the changes we have made to the
runtime since.

\plan{Global spark queue and contention WRT global queue}
Mercury (before 2009) uses a global spark queue.
The runtime system \emph{schedules} sparks for parallel execution by placing
them onto the end of the queue.
In this chapter we use the word `schedule' to mean the act of making a spark
available for parallel or sequential work.
An idle engine runs a spark by taking it from the beginning of the queue.
The global spark queue must be protected by a lock;
this prevents concurrent access from corrupting the queue.
The global spark queue and its lock can easily become a bottleneck when many
engines contend for access to the global queue.

\plan{Local spark stack --- relieves contention on global queue}
\citet{wang:2006:hons} anticipated this problem and created context local spark
stacks to avoid contention on the global queue.
Furthermore, the local spark stacks do not require locking.
When a parallel conjunction spawns off a spark,
it places the spark either at the end of the global spark queue or at the
top of its local spark stack.
\plan{Spark scheduling decision.}
The runtime system appends the spark to the end of the global queue if
an engine is idle, and
the number of contexts in use plus the number of sparks on the global queue
does not exceed the maximum number of contexts permitted.
If either part of the condition is false,
the runtime system pushes the spark onto the top of the context's local
spark stack.
This algorithm has two aims.
The first is to reduce contention on the global queue,
especially in the common case that there is enough parallel work.
The second aim is to reduce the amount of memory allocated
as contexts' stacks by reducing the number of contexts allocated.
Globally scheduled sparks may be converted into contexts,
so they are also included in this limit.
We explain this limit on the number of context in more detail
in Section \ref{sec:original_scheduling_performance},
after covering the background information in the current section.
Note that sparks placed on the global queue are executed in a
first-in-first-out manner;
sparks placed on a context's local stack are executed in a
last-in-first-out manner.

\begin{figure}
\begin{tabular}{rl}
 1: & \code{~~MR\_SyncTerm ST;} \\
 2: & \code{~~MR\_init\_syncterm(\&ST, 2);} \\
 3: & \code{~~spawn\_off(\&ST, Spawn\_Label\_1);} \\
 4: & \code{~~}$G_1$ \\
 5: & \code{~~MR\_join\_and\_continue(\&ST, Cont\_Label);} \\
 6: & \code{Spawn\_Label:} \\
 7: & \code{~~}$G_2$ \\
 8: & \code{~~MR\_join\_and\_continue(\&ST, Cont\_Label);} \\
 9: & \code{Cont\_Label:} \\
\end{tabular}
\caption{Parallel conjunction implementation}
\label{fig:par_conj_impl_only}
\end{figure}

\begin{algorithm}
\begin{algorithmic}[1]
\Procedure{MR\_join\_and\_continue}{$ST, ContLabel$}
  \State MR\_acquire\_lock($ST.lock$)
  \State $ST.num\_outstanding \gets ST.num\_outstanding - 1$
  \If{$ST.num\_outstanding = 0$}
    \If{$ST.orig\_context = this\_context$}
      \State MR\_release\_lock($ST.lock$)
      \Goto{$ContLabel$}
    \Else
      \State $ST.orig\_context.resume\_label \gets ContLabel$
      \State MR\_schedule($ST.orig\_context$, $ContLabel$)
      \State MR\_release\_lock($ST.lock$)
      \Goto{MR\_idle}
    \EndIf
  \Else
    \State $spark \gets$ MR\_pop\_spark($this\_context.spark\_stack$)
    \If{$spark$}
      \If{$spark.ST.stack\_ptr = MR\_parent\_sp$}
%        \Comment{This spark belongs to the same parallel conjunction.
%        It can be executed immediatly.}
        \State MR\_release\_lock($ST.lock$)
        \Goto{$spark.code\_label$}
      \Else
        \State MR\_push\_spark($this\_context.spark\_stack$, $spark$)
      \EndIf
    \EndIf
    \If{$ST.orig\_context = this\_context$}
       \State MR\_save\_context($this\_context$)
       \State $this\_context \gets NULL$
    \EndIf
    \State MR\_release\_lock($ST.lock$)
    \Goto{MR\_idle}
  \EndIf
\EndProcedure
\end{algorithmic}
\caption{MR\_join\_and\_continue}
\label{alg:join_and_continue_peterw}
\end{algorithm}

\plan{barrier code, this is used to explain the right recursion problem.}
In Section \ref{sec:backgnd_merpar} we described how parallel
conjunctions are compiled
(Figure \ref{fig:par_conj} shows an example).
Consider the compiled parallel conjunction in Figure
\ref{fig:par_conj_impl_only}.
The context that executes the parallel conjunction,
lets call it $C_{Orig}$,
begins by
setting up the sync term,
spawning off $G_2$,
and executing $G_1$ (lines 1--4).
Then it executes the barrier \joinandcontinue,
whose is shown in
Algorithm \ref{alg:join_and_continue_peterw}.
Depending on how full the global run queue is,
and how parallel tasks are interleaved,
there are three important scenarios:

\begin{description}

    \item[Scenario one:]~

    In this scenario $C_{Orig}$ placed the spark for $G_2$ on the top of its
    local spark stack.
    Sparks placed on a context local spark stack cannot be executed by any
    other context.
    Therefore when $C_{Orig}$ reaches the \joinandcontinue barrier
    (line 5 in the example),
    the context $G_2$ will be outstanding and $ST.num\_outstanding$ will be
    non-zero.
    $C_{Orig}$ will execute else branch on line 14 of \joinandcontinue
    where it will pop the spark for $G_2$ off the top of the spark stack.
    It is not possible for some other spark to be on the top of the stack;
    any sparks left on the stack by $G_1$ would have been popped off by 
    the \joinandcontinue barriers of the conjunctions that spawned off the
    sparks.

    The check that the sparks stack pointer is equal to the current
    parent stack pointer\footnote{
        The code generator will ensure that $MR\_parent\_sp$ is set
        before the parallel conjunction is executed,
        and that it is restored after.}
    will succeed (line 16 of \joinandcontinue),
    and the context will execute the spark.

    After executing $G_2$,
    $C_{Orig}$ will execute the second call to \joinandcontinue,
    the one on line 8 of the example.
    This time $ST.num\_outstanding$ will be zero,
    and $C_{Orig}$ will execute the then branch on line 5 of
    \joinandcontinue.
    In the condition of the nested if-then-else,
    $C_{Orig}$ will find that the current context is the original context,
    and therefore continue execution at line 6.
    This causes $C_{Orig}$ to jump to the continuation label,
    line 9 in the example,
    completing the execution of the parallel conjunction.

    \item[Scenario two:]~

    In this scenario $C_{Orig}$ placed the spark for $G_2$ on the 
    global spark queue,
    where another context, $C_{Other}$, picked it up and executed it
    in parallel.
    Also, as distinct from scenario three,
    $C_{Other}$ reaches the barrier on line 8 of the example \emph{before}
    $C_{Orig}$ reaches the barrier on line 5.
    Even if they both seem to reach the barrier at the same time,
    their barrier operations are performed in sequence because of the
    lock protecting the barrier code.

    When $C_{Other}$ executes \joinandcontinue,
    It will find that $ST.num\_outstanding$ is non-zero,
    and will execute the else branch on line 14 of \joinandcontinue.
    It then attempts to pop a spark off its stack,
    as in another scenario a spark on the stack might represent an
    outstanding conjunction
    (it cannot tell that the outstanding conjunct is $G_1$ executing in
    parallel, and not some hypothetical $G_3$).
    $C_{Other}$ took this spark from the global queue,
    and was either empty or brand new before executing this spark,
    meaning that it had no sparks on its stack before executing $G_2$.
    Therefore $C_{Other}$ will not have any sparks of its own and
    \code{pop\_spark} will fail.
    $C_{Other}$ will continue to line 21 in \joinandcontinue
    which will also fail since it ($C_{Other}$)
    is not the original context ($C_{Orig}$).
    The lock will be released and \joinandcontinue will determine what to do
    next by jumping to \idle.

    Eventually $C_{Orig}$ will execute its call to \joinandcontinue
    (line 5 of the example),
    or be executed after waiting on the barrier's lock (line 2 of
    \joinandcontinue),.
    When this happens it will find that $ST.num\_outstanding$ is zero,
    and execute the then branch beginning at line 5 of \joinandcontinue.
    $C_{Orig}$ will test if it is the original context, 
    which it is,
    and continue on line 6 of \joinandcontinue.
    It then jumps to the continuation label on line 9 of the example,
    completing the parallel execution of the conjunction.

    \item[Scenario three:]~

    As in scenario two $C_{Orig}$ pl the spark for $G_2$ on the global spark
    queue,
    where another context, $C_{Other}$, picked it up and executed it
    in parallel.
    However,
    in this scenario
    $C_{Orig}$ reaches the barrier on line 5 in the example \emph{before}
    $C_{Other}$ reaches its barrier on line 8.

    When $C_{Orig}$ executes \joinandcontinue,
    it finds that $ST.num\_outstanding$ is non-zero,
    causing it to execute the else branch on line 14 of \joinandcontinue.
    $C_{Orig}$ will try to pop a spark of its local spark stack.
    However the the spark for $G_2$ was placed on the global
    spark queue,
    the only spark it might find is one created by an outer conjunction.
    If a spark is found, the spark's parent stack pointer will not match the
    current parent stack pointer,
    and it will put the spark back on the stack.
    $C_{Orig}$ executes the then branch (line 22 of \joinandcontinue), since this
    context is the original context.
    This branch will suspend $C_{Orig}$,
    and set the engine's context pointer to \NULL
    before jumping to \idle.

    When $C_{Other}$ reaches the barrier on line 8,
    it will find that $ST.num\_outstanding = 0$,
    and will execute the then branch of the if-then-else.
    Within this branch it will test to see if it is the original
    context,
    the test will fail and the else branch of the nested if-then-else
    will be executed.
    At this point we know that $C_{Orig}$ must be suspended because
    there where no outstanding conjuncts and the current context is not
    the original context;
    this can only happen if $C_{Orig}$ is suspended.
    The code wakes $C_{Orig}$ up by
    setting its code pointer to the continuation label,
    placing it on the global run queue,
    and then jumps to \idle.

    When $C_{Orig}$ resumes execution it executes the code on line 9,
    which is the continuation label.

\end{description}

The algorithm includes an optimisation not shown here:
if the parallel conjunction has been executed by only one context,
then a version of the algorithm without locking is used.
We have not shown the optimisation because it is equivalent and not relevant
to our discussion,
we mention it only for completeness.

\begin{algorithm}
\begin{algorithmic}
\Procedure{MR\_idle}{}
  \State MR\_acquire\_lock($MR\_runqueue\_lock$)
  \Loop
    \If{$MR\_exit\_now$}
      \State MR\_release\_lock($MR\_runqueue\_lock$)
      \State MR\_destroy\_thread()
    \EndIf
    \State $ctxt \gets$ MR\_get\_runnable\_context()
    \If{$ctxt$}
      \State MR\_release\_lock($MR\_runqueue\_lock$)
      \If{$current\_context$}
        \State MR\_release\_context($current\_context$)
      \EndIf
      \State MR\_load\_context($ctxt$)
      \Goto $ctxt.resume$
    \EndIf
    \State $spark \gets$ MR\_dequeue\_spark($MR\_global\_spark\_queue$)
    \If{$spark$}
      \State MR\_release\_lock($MR\_runqueue\_lock$)
      \If{$\neg current\_context$}
        \State $ctxt \gets$ MR\_get\_free\_context()
        \If{$\neg ctxt$}
          \State $ctxt \gets$ MR\_create\_context()
        \EndIf
        \State MR\_load\_context($ctxt$)
      \EndIf
      \State $MR\_parent\_sp \gets spark.parent\_sp$
      \State $ctxt.thread\_local\_mutables \gets
        spark.thread\_local\_mutables$
      \Goto $spark.resume$
    \EndIf
    \State MR\_wait($MR\_runqueue\_cond, MR\_runqueue\_lock$)
  \EndLoop
\EndProcedure
\end{algorithmic}
\caption{\idle}
\label{alg:MR_idle_initial}
\end{algorithm}

\plan{Explain how work begins executing, for completeness.}
When an engine cannot get any local work it must search for global work.
Newly created engines, except for the first, also search for global work.
They do this by calling \idle,
whose code is shown in Algorithm \ref{alg:MR_idle_initial}.
Only one of the idle engines can execute \idle at a time.
The \var{MR\_runqueue\_lock} lock protects the context run queue and the
global context queue from concurrent access.
After acquiring the lock,
engines execute a loop.
An engine exits the loop only when it finds some work to do or the
program is exiting.
Each iteration first checks if the runtime system is being shut down,
if so,
then this thread releases the lock,
and then destroys itself.
If the system is not being shut down,
the engine will search for a runnable context.
If it finds a context it releases the run queue lock, loads the context
and jumps to the resume point for the context.
If it already had a context then it first releases that context,
doing so is safe as
a precondition of an engine calling \idle is that if the engine has
a context, the context must not contain a spark or represent a
suspended computation.
If no context was found, the engine attempts to take a spark from the global
spark queue.
If it finds a spark then it will need a context to execute that spark.
It will try to get a context from the free list, if there is none it will
create a new context.
Once it has a context,
it copies the context's copies of registers into the engine.
It also initialises the engine's parent stack pointer
register and the spark's thread local mutables
(which are set by the context that created the spark)
into the context.
If the engine does not find any work,
it will wait using a condition variable and the run queue lock.
The pthreads wait function is able to unlock the lock and wait on the
condition atomically, preventing race conditions.
The condition variable is used to wake up the engine if either a spark is
placed on the global spark queue or a context is placed on the context run
queue.
When the engine wakes,
it will re-execute the loop.



\begin{figure}
\begin{center}
\subfigure[Right recursive]{%
\label{fig:map_right_recursive}
\begin{tabular}{l}
\code{map(\_, [], []).} \\
\code{map(P, [X $|$ Xs], [Y $|$ Ys]) :-} \\
\code{~~~~P(X, Y) \&} \\
\code{~~~~map(P, Xs, Ys).} \\
\end{tabular}
}
\subfigure[Left recursive]{%
\label{fig:map_left_recursive}
\begin{tabular}{l}
\code{map(\_, [], []).} \\
\code{map(P, [X $|$ Xs], [Y $|$ Ys]) :-} \\
\code{~~~~map(P, Xs, Ys) \&} \\
\code{~~~~P(X, Y).} \\
\end{tabular}
}%
\end{center}
\caption{Right and left recursive map/3}
\label{fig:map_right_and_left_recursive}
\end{figure}

In section \ref{sec:gc} we ran our benchmarks with a recent version of the
runtime system.
In the rest of this chapter we describe many of the improvements to the
runtime system that improved parallel performance.

\plan{Introduce right recursion.}
Figure \ref{fig:map_right_and_left_recursive} shows two alternative, parallel
implementations of \code{map/3}.
While their declarative semantics are identical,
their operational semantics are very different.
In section \ref{sec:backgnd_merpar} we explained that parallel conjunctions
are implemented by spawning off the second and later conjuncts and executing
the first conjunct directly.
In the right recursive case (figure \ref{fig:map_right_recursive}),
the recursive call is spawned off as a spark,
and in the left recursive case (figure \ref{fig:map_left_recursive}),
the recursive call is executed directly, and the loop \emph{body} is
spawned off.
Declarative programmers are taught to prefer tail recursion,
and therefore tend to write right recursive code.

\begin{table}
\begin{center}
\begin{tabular}{r|rr|rrrr}
\multicolumn{1}{c|}{Max no.} &
\multicolumn{2}{c|}{Sequmential} &
\multicolumn{4}{c}{Parallel w/ $N$ Engines} \\
\Cbr{of contexts} & \C{not TS} & \Cbr{TS} & \C{1}& \C{2}& \C{3}& \C{4}\\
\hline
4        & 23.2 (0.93) & 21.5 (1.00)
         & 21.5 (1.00) & 21.5 (1.00) & 21.5 (1.00) & 21.5 (1.00) \\
64   &-&-& 21.5 (1.00) & 21.5 (1.00) & 21.5 (1.00) & 21.5 (1.00) \\
128  &-&-& 21.5 (1.00) & 19.8 (1.09) & 20.9 (1.03) & 21.2 (1.01) \\
256  &-&-& 21.5 (1.00) & 13.2 (1.63) & 15.5 (1.38) & 16.5 (1.30) \\
512  &-&-& 21.5 (1.00) & 11.9 (1.81) &  8.1 (2.66) &  6.1 (3.55) \\
1024 &-&-& 21.5 (1.00) & 11.8 (1.81) &  8.0 (2.67) &  6.1 (3.55) \\
2048 &-&-& 21.5 (1.00) & 11.9 (1.81) &  8.0 (2.67) &  6.0 (3.55) \\
\end{tabular}
\end{center}
\caption{Right recursion performance.}
\label{tab:right}
\end{table}

\plan{Show performance figures.}
Table \ref{tab:right} shows average elapsed time in seconds for the
mandelbrot\_lowalloc program from 20 test runs.
We use the mandelbrot\_lowalloc program from section \ref{sec:gc}.
Using this program we can easily observe the
speedup due to parallelism in Mercury without the effects of the garbage
collector.
The loop that iterates over the rows in the image uses right recursion.
It is similar to \code{map/3}
in figure \ref{fig:map_right_recursive}.
The left most column shows the maximum number of contexts permitted at
any time.
This is the limit that was mentioned in the previous section.
The next two columns give the elapsed execution time for a sequential
version of the program,
in this version of the program no parallel conjunctions where used.
These two columns give results without and with thread safety.
The following four columns give the elapsed execution times
using one to four Mercury engines.
The numbers in parentheses are the ratio between the time and the
sequential thread safe time.

\plan{Observations}
In general we achieve more parallelism when we use more contexts,
until a threshold of 601 contexts,
as there are 600 rows in the image and a base case each one consumes a
context.
This is why the program does not benefit greatly from a high limit such as
1024 or 2048 contexts.
This program may use fewer than 600 contexts as any sequentially executed
sparks use their parent context.
This is possibly why a limit of 512 contexts also results in a good parallel
speedup.
When only 256 contexts are used the four core version achieves a speedup
of 1.30,
compared to 3.55 for 512 or more contexts.
Given that mandelbrot uses independent parallelism there should never be any
need to suspend a context.
Therefore the program should parallelise well enough when restricted to
a small number of contexts (four to eight).
Too many contexts are needed to execute this program at the level of
parallelism that we want.
We noticed the same pattern in other programs with right recursive
parallelism.

\label{context_limit}
\plan{Describe the context limit problem.}
%Right recursion uses a context for each iteration of the loop,
%and then suspends that context.
To understand this problem we must consider how parallel conjunctions are
executed (see section \ref{sec:backgnd_merpar}).
The original context creates a spark for the second and later conjuncts and
puts it on the global spark queue.
It then executes the first conjunct \code{P/2}.
This takes much less time to execute than the spawned off call to
\code{map/3}.
After executing \code{P/2} the original context will call the
\joinandcontinue barrier.
It then attempts to execute a spark from its local spark stack,
this will fail because the only spark was placed on the global spark queue.
The original context will be needed after the other conjunct finishes, 
to execute the code after the parallel conjunction.
The original context cannot proceed but will be needed later,
therefore it is suspended until all the other 599 iterations of the loop
have finished.
Meanwhile the spark that was placed on the global run queue is converted
into a new context.
This new context enters the recursive call and
becomes blocked within the recursive instance of the same parallel
conjunction, it must wait for the remaining 598 iterations of the loop.
This process continues to repeat itself,
allocating more contexts which can consume large amounts of memory.
Each context consumes a significant amount of memory,
much more than one stack frame,
Therefore
this problem makes programs that look tail recursive
\emph{consume much more memory than}
sequential programs that are not tail recursive.
We implement the limit on the number of contexts to prevent pathological
cases such as this one from crashing the program or the whole computer.
Once the limit is reached sparks cannot be placed on the global run queue
and sequential execution is used.
Therefore the context limit is a trade off between memory usage and
parallelism.

\picfigure{linear_context_usage}{Linear context usage in right recursion}

\plan{Diagram from the loop control talk}
An example of context usage over time using four engines is shown in
figure \ref{fig:linear_context_usage}.
Time is shown on the Y axis and contexts are drawn so that each additional
context is further along the X axis.
At the top left,
four contexts are created and they execute four iterations of a loop;
this execution is indicated by boxes.
Once each of these iterations finishes,
its context stays in memory but is suspended,
indicated by the long vertical lines.
Another four iterations of the loop create another four contexts, and so on.
Later when all iterations of the loop have been executed,
each of the blocked contexts resumes execution and immediately exits,
indicated by the horizontal line at the bottom of each of the vertical
lines.
Later in the project we developed a solution to this problem
(see chapter \ref{chap:loop_control}).
Before developing the solution, we developed a work around (see section
\ref{sec:rts_reorder}.

\plan{Extra engines require more contexts}
When using either 128 or 256 contexts,
we noticed that when we used three or four Mercury engines
the program ran more slowly than with two Mercury engines.
A possible reason is that:
the more Mercury engines there are the more often at least one engine is
idle.
When creating a spark the runtime system checks four an idle engine,
if there is one then the spark may be placed on the global queue, subject to
    the context limit (see section \ref{sec:original_scheduling}).
If there is no idle engine,
a spark is placed on the context's local queue regardless of the current
number of contexts.
When there are fewer Mercury engines being used then it is less likely that
one of them is idle.
Therefore more sparks are placed on the local queue and executed
sequentially and the program does not hit the context limit as quickly as
one using more Mercury engines.
The program can exploit more parallelism as it hits the context limit
later,
and achieve better speedups.

\plan{Suggest that left recursion might fix this,
(This is what we thought at the time).}
Only right recursion use one context for each iteration of a loop. 
A left recursive predicate (figure \ref{fig:map_left_recursive} has its
recursive call on the left of the parallel conjunction.
The context executing the conjunction sparks off the body of the loop,
in the figure this is \code{P},
and executes the recursive call directly.
By the time the recursive call finishes,
the conjunct containing \code{P} should have also finished.
The context that created the parallel conjunction is less likely to be
blocked at the barrier,
and the context executing the spark is never blocked since it is not the
original context that executed the conjunction.

\begin{table}
\begin{center}
\begin{tabular}{r|rrrrrr}
\multicolumn{1}{c|}{Max no.} &
\multicolumn{2}{c|}{Sequmential} &
\multicolumn{4}{c}{Parallel w/ $N$ Engines} \\
\Cbr{of contexts} & \C{not TS} & \Cbr{TS}  & \C{1}& \C{2}& \C{3}& \C{4}\\
\hline
\hline
\multicolumn{7}{c}{Right recursion} \\
\hline
4        & 23.2 (0.00) & 21.5 (0.04)
         & 21.5 (0.02) & 21.5 (0.00) & 21.5 (0.01) & 21.5 (0.02) \\
64   &-&-& 21.5 (0.03) & 21.5 (0.01) & 21.5 (0.01) & 21.5 (0.02) \\
128  &-&-& 21.5 (0.02) & 19.8 (0.01) & 20.9 (0.03) & 21.2 (0.03) \\
256  &-&-& 21.5 (0.02) & 13.2 (0.05) & 15.5 (0.06) & 16.5 (0.07) \\
512  &-&-& 21.5 (0.02) & 11.9 (0.11) &  8.1 (0.09) &  6.1 (0.08) \\
1024 &-&-& 21.5 (0.03) & 11.8 (0.11) &  8.0 (0.06) &  6.1 (0.08) \\
2048 &-&-& 21.5 (0.03) & 11.9 (0.10) &  8.0 (0.08) &  6.0 (0.06) \\
\hline
\hline
\multicolumn{7}{c}{Left recursion (Sparks included in context limit)} \\
\hline
4        & 23.2 (0.01) & 21.5 (0.03)
         & 21.5 (0.02) & 21.5 (0.04) & 21.5 (0.04) & 21.5 (0.02) \\
64   &-&-& 21.5 (0.02) & 21.5 (0.02) & 21.4 (0.03) & 21.5 (0.03) \\
128  &-&-& 21.5 (0.04) & 21.5 (0.03) & 21.5 (0.03) & 21.5 (0.03) \\
256  &-&-& 21.5 (0.02) & 18.3 (0.75) & 18.2 (0.03) & 19.6 (1.37) \\
512  &-&-& 21.5 (0.02) & 17.9 (0.83) & 15.5 (1.29) & 16.4 (7.09) \\
1024 &-&-& 21.5 (0.03) & 18.0 (0.85) & 14.7 (2.18) & 16.1 (5.23) \\
2048 &-&-& 21.5 (0.02) & 18.0 (0.85) & 15.4 (2.15) & 17.8 (5.25) \\
\hline
\hline
\multicolumn{7}{c}{Left recursion (Sparks excluded from context limit)} \\
\hline
N/A      & 23.3 (0.01) & 21.5 (0.02)
         & 21.7 (0.78) & 17.9 (0.60) & 15.6 (2.49) & 14.2 (6.94) \\
\end{tabular}
\end{center}
\caption{
Right and Left recursion shown with standard deviation}
\label{tab:2009_left_nolimit}
\end{table}

\plan{Show performance figures for left recursion.}
Table \ref{tab:2009_left_nolimit} shows benchmark results using left
recursion.
The table is broken into three sections:
a copy of the right recursion data from table \ref{tab:right},
which is presented again for comparison;
the left recursive data,
which we will discuss now;
and left recursive data with a modified context limit,
which we will discuss in a moment.
The figures in parenthesis are the standard deviation of the samples.

The left recursive figures are underwhelming;
they are worse than right recursion.
Furthermore,
the standard deviations for left recursion results are much
higher.
In particular,
the more contexts and Mercury engines are used,
the higher the standard deviation.
Despite this,
we can see that the left recursion results are much slower than the right
recursive results.

\plan{Explain the premature scheduling problem that affects left-recursive
programs.}
Both left and right recursion are affected by the context limit,
however the cause for left recursion is different.
Consider a case of two Mercury engines and a context limit of eight.
The first Mercury engine is executing the original context,
which enters the left recursive parallel conjunction and spawns off a spark,
adding it to the global spark queue.
The original context then executes the recursive call and
continues to spawn off sparks.
As we described in section \ref{sec:original_scheduling},
each spark on the global spark queue may be converted into a new
context,
and therefore sparks on the global queue contribute towards the context
limit.
With left recursion,
the first engine executes its tight loop which spawns off sparks.
Meanwhile the second engine takes sparks from the queue and executes them,
the execution of the work that a spark represents takes more time than each
recursive call executed by the first engine.
Therefore the first engine will put sparks on the global queue more quickly
than the second engine can remove and execute them.
After eight to twelve recursions,
it is very likely that the sparks on the global queue will exceed the
context limit,
and that new sparks are now placed on the original context's local spark
stack.
This happens sooner if the context limit is low.

\plan{Second problem, discuss the high variance}
The high variance in all the left recursive results is
indicative of nondeterminism.
The cause is going to be something that either varies between execution
runs or does not always effect execution runs.
A potential explanation is that in different runs different numbers of
sparks are executed in parallel before the context limit is reached.
However this cannot be true, or at least is only partly true,
as the variance is higher when the context limit is higher including cases
where 2048 contexts are permitted and we know the program needs at most 601
(usually less).
Fortunately there is a better explanation.
The context limit is one of the two things used to make this scheduling
decision;
the other is, if there is no idle engine then the spark is always
placed on the context local spark stack
(section \ref{sec:original_scheduling}).
At the time when the parallel conjunction in \code{map/3} is executed,
the other engines will initially be idle but will quickly become busy,
once they are busy the original context will not place sparks on the
global spark queue.
If the other engines are slow to respond to the first sparks placed
on the global queue,
then more sparks are placed on this queue and more
parallel work is available.
If they are quick to respond,
then more sparks will be placed on the original contexts local queue,
where they will be executed sequentially.

\plan{Left recursion w/out limit results}
Mandelbrot uses only independent parallelism, which means no context can
ever be blocked by a future.
In a left recursive case a computation that is spawned off on a spark is
never blocked by a \joinandcontinue barrier,
only the original context executes a \joinandcontinue that may block.
Therefore the program never uses more than one context per Mercury engine
plus the original context.
For this program we can safely modify the runtime system so that globally
scheduled sparks do not count towards the context limit.
We did to test 
that a low context limit prevents the majority of sparks from being placed on
the global queue and being eligible for parallel execution.
The final row group in table
\ref{tab:2009_left_nolimit} shows results for the left recursive test
where sparks in the global queue do not count towards the context limit.
As expected, we got similar results for different values of the context
limit.
We have therefore shown only one row of results in the table.
This result is similar to those in the left recursive group with an
unmodified runtime system and a sufficiently high context limit.
Therefore we can say that the context limit is affecting left recursion in
this way.

\plan{Reinforce that these results support the idea that scheduling
decisions are made prematurely}
Both of the left recursive problems have the same cause:
the decision to execute a spark sequentially or in parallel is
made too early.
This decision is made when the spark is scheduled,
and placed either on the global spark queue or the context local spark
stack.
The data used to make the decision includes the number of contexts in
use,
the number of sparks on the global queue,
and if there is an idle engine.
These conditions will be different when the spark is scheduled compared
to when it may be executed,
and the conditions when it is scheduled are not a good indication of
the conditions when it may be executed.
We will refer to this problem as the \emph{premature spark scheduling problem}.
In the next section,
we solve this problem by delaying the decision to execute a spark in
parallel or in sequence.

%In the left recursive program scheduling is quite different.
%The parallel conjunction creates a spark for \code{P(X, Y)} and executes the
%recursive call directly.
%The spark is converted into a context,
%that context does not execute another parallel conjunction since it does not
%execute the recursive call.
%Therefore, it will not become blocked on the \joinandcontinue barrier in any
%nested parallel conjunction.
%It will execute the barrier after \code{P(X, Y)},
%this however does not block this context.
%The context is not the conjunction's original context and therefore once it
%reaches this barrier it is free,
%if it has any sparks on its local queue it may execute them,
%otherwise the engine executing it will look for global work,
%either another context or a spark from the global queue.
%If there is a spark on the global queue the engine will use this context to
%execute it since the context is otherwise unused.
%
%This led us to believe that the left recursion would be more efficient than
%right recursion,
%namely that since contexts are reused, the number of contexts wouldn't climb
%and prevent parallelism from being exploited.
%As table \ref{tab:right_v_left} shows, we were wrong:
%the context limit is affecting performance.
%As discussed, a left-recursive loop spawns of calls to \code{P} as sparks
%and executes its recursive call directly.
%It will, very quickly,
%make many recursive calls, spawn off many sparks.
%The context limit includes sparks on the global queue since
%executing them can require the creation of new contexts,
%Furthermore, if they were not included and the runtime system refused to
%convert a spark on the global queue into a context the system could become
%deadlocked.
%In the left recursive case,
%the context limit will be reached very quickly,
%often before engines have begun taking sparks from the queue and executing
%them.
%Once the limit is reached sparks are placed on the contexts local queues
%where they cannot be executed in parallel.
%The smaller the context limit,
%the more quickly the limit is reached and the fewer contexts are placed on
%the global queue.
%Additionally,
%the loop placing sparks on its context's local stack will execute very
%quickly.
%
%We concluded that
%in the left recursive case
%the scheduling decision for each spark is made much earlier than the spark's
%execution.
%Specifically,
%when the decision to place the spark on the global queue or local stack is
%made,
%often the context limit has already been reached:
%\paul{Need to decide how I communicate who the actor is for scheduling
%decisions.}
%the context will place the spark on its local stack.
%Later, when a different engine becomes idle,
%it cannot access the spark since it is on another engine's context's spark
%stack.
%At this point it is apparent that the scheduling decision made when the
%spark was placed on the local stack was incorrect,
%as there is an idle engine ready to execute the spark,
%and because contexts are re-used (in left recursion) there is either a free
%context or we can easily create one.



\section{Initial work stealing implementation}
\label{sec:rts_work_stealing}

\plan{Refer to Peter Wang's conclusion of the early scheduling problem}
Work stealing addresses the premature spark scheduling problem that we
described in the previous section.
\citet{wang:2006:hons} recognised this problem and proposed work stealing as
its solution.
In a work stealing system,
sparks placed on a context's local spark stack
are not committed to running in that context;
they may be executed in a different context if they are stolen.
This delays the decision of where to execute a spark until the moment
before it is executed.

\plan{Introduce work stealing}
Work stealing is a popular method for managing parallel work in a shared
memory multiprocessor system.
% XXX: There may be earlier papers by keller 1984 as cited by halstead.
Multilisp \citep{halstead:1985:multilisp} was one of the first systems to
use work stealing,
which Halstead calls ``an unfair scheduling policy''.
The term ``unfair'' is not an expression of the morals of stealing (work),
instead it refers to the unfairness of \emph{cooperative multitasking} when
compared to something like \emph{round-robin scheduling}.
Each processor in Multilisp has a currently running task,
and a stack of suspended tasks.
If the current task is suspended on a future or finishes,
the processor will execute the task at the top of its task stack.
If there is no such task,
it attempts to steal a task from the bottom of another processor's stack.

Halstead's ``unfair scheduling'' is motivated by his need to reduce the
problems of embarrassing parallelism.
Most significantly,
he says that the system can deadlock due to running out of resources but
needing resources to proceed.
We would like to point out that
this is not the same as our right recursion problem,
even though the two have similar symptoms.
Furthermore,
work stealing does not fix the right recursion problem.

Work stealing's other benefits can be summarised as follows.

\begin{itemize}

    \item
    It is quicker to schedule sparks because
    a context can place sparks on its local spark stack more quickly
    than it can place sparks on the global spark queue.
    This is because there is less contention on a context's
    local spark stack.
    The only source of contention is work stealing,
    which is balanced among all of the contexts.
    Assume each of the $C$ contexts' spark stacks contains sparks.
    Contention on any one of these stacks is $1/C$ times less than it would be
    on a global queue.

    \item
    An idle worker can take work from its own queue quickly.
    Again, there is less contention on a local queue.

    \item
    In ideal circumstances,
    an idle worker rarely needs to steal work from another's queue.
    Stealing work is more costly as it requires communication between
    processors.

\end{itemize}

The initial work stealing implementation was built jointly by
Peter Wang and myself,
Wang contributed about 80\% of the work
and I contributed the remaining 20\%.
Wang's honours thesis \citep{wang:2006:hons} describes his proposal on which
our implementation is heavily based.

\begin{figure}
\begin{center}

\begin{minipage}[b][3in]{0.29\textwidth}
\subfigure[Nested parallel conjunctions]{%
\label{fig:nested_par_conjunctions}
\begin{tabular}{l}
\code{p(...) :-} \\
\code{~~~~(} \\
\code{~~~~~~~~q(Y::out, ...)} \\
\code{~~~~\&} \\
\code{~~~~~~~~r(Y::in, ...)} \\
\code{~~~~\&} \\
\code{~~~~~~~~s(...)} \\
\code{~~~~).} \\
\code{} \\
\code{q(Y::out, ...) :-} \\
\code{~~~~(} \\
\code{~~~~~~~~t(X::out, ...)} \\
\code{~~~~\&} \\
\code{~~~~~~~~u(X::in, Y::out)} \\
\code{~~~~).} \\
\end{tabular}
}
\hfill
\end{minipage}
\begin{minipage}[b][3in]{0.69\textwidth}
\begin{minipage}[b]{0.32\textwidth}
\subfigure[Step 1]{%
\label{fig:spark_stack_step1}
\picfigurenofloat{spark_stack_step1} }
\hfill
\end{minipage}
\begin{minipage}[b]{0.32\textwidth}
\subfigure[Step 2]{%
\label{fig:spark_stack_step2}
\picfigurenofloat{spark_stack_step2} }
\hfill
\end{minipage}
\begin{minipage}[b]{0.32\textwidth}
\subfigure[Step 3]{%
\label{fig:spark_stack_step3}
\picfigurenofloat{spark_stack_step3} }
\hfill
\end{minipage}

\vspace{0.5in}
\begin{minipage}[b]{0.32\textwidth}
\subfigure[Step 4]{%
\label{fig:spark_stack_step4}
\picfigurenofloat{spark_stack_step2} }
\hfill
\end{minipage}
\begin{minipage}[b]{0.32\textwidth}
\subfigure[Step 5]{%
\label{fig:spark_stack_step5}
\picfigurenofloat{spark_stack_step5} }
\hfill
\end{minipage}
\begin{minipage}[b]{0.32\textwidth}
\subfigure[Step 6]{%
\label{fig:spark_stack_step6}
\picfigurenofloat{spark_stack_step1} }
\hfill
\end{minipage}
\hfill
\end{minipage}

\end{center}
\caption{Spark execution order}
\label{fig:spark_execution_order}
\end{figure}

\plan{Mercury's needs for a deque}
Until now, each context has used a stack to manage sparks.
The last item pushed onto the stack is the first to be popped off the
stack.
This \emph{last-in-first-out} order is important for Mercury's parallel
conjunctions.
Parallel conjunctions may be nested, either in the same procedure or through
procedure calls such as in
Figure~\ref{fig:nested_par_conjunctions}.
Consider a context whose spark stack's contents are initially undefined.
The spark stack is represented by
Figure~\ref{fig:spark_stack_step1}.
When the context calls \code{p} it creates a spark
for the second and third conjuncts \code{r(\ldots)~\&~q(\ldots)}
and places the spark on its local stack
(Figure~\ref{fig:spark_stack_step2}).
It then calls \code{q}, where it creates another spark.
This spark represents the call to \code{u},
and now the spark stack looks like
Figure~\ref{fig:spark_stack_step3}.
Assuming that no work stealing has occurred,
when the context finishes executing \code{t} it pops the spark from the top
of its stack;
the popped spark represents the call to \code{u} because
the stack returns items in a last-in-first-out order.
This is desirable because it follows the same execution order as sequential
execution would.
We call this left-to-right execution since by default the left operand of a
conjunction is executed first.
The conjunction in \code{q} contains a shared variable named \code{X};
this execution order ensures that the \signal operation for \code{X}'s future
occurs before the \wait operation;
the context is not blocked by the call to \wait.
(Note that the mode annotations are illustrative,
they are not part of Mercury's syntax.)
Once the spark for \code{u} is popped off the spark stack then the spark
stack looks like Figure~\ref{fig:spark_stack_step4}.
After the execution of \code{u}, \code{q} returns control to \code{p},
which then continues the execution of its own parallel conjunction.
\code{p} pops a spark off the spark stack;
the spark popped off is the parallel conjunction
\code{r(\ldots) \& s(\ldots)}.
This immediately creates a spark for \code{s} and pushes it onto the spark
stack (Figure~\ref{fig:spark_stack_step5}) and
then executes \code{r}.
The execution of \code{r} at this point is also the same as what would occur
if sequential conjunctions were used.
This is straightforward because
the spark represented the \emph{rest} of the parallel
conjunction.
Finally \code{r} returns and \code{p} pops the spark for \code{s} off of the
spark stack and executes it.
At this point the context's spark stack looks like
Figure~\ref{fig:spark_stack_step6}.

If at step 3,
the context executed the spark for \code{r(\ldots) \& s(\ldots)} then
\code{r} would have blocked on the future for \code{Y}.
This would have used more contexts than necessary and more overheads than
necessary.
It may have also caused a deadlock:
since stealing a spark may create a context which may not be permitted due
to the context limit.
Furthermore
the original context would not be able to continue the execution in \code{p}
without special stack handling routines;
this would make the implementation more complicated than necessary.
%The system must be able to make progress by executing contexts from the
%global context run queue or by executing sparks from a context's own local
%spark stack.
Therefore from a context's perspective its local spark storage must behave
like a stack.
Keeping this sequential execution order for parallel conjunctions has
an additional benefit:
it ensures that a popped spark's data has a good chance of being hot in the
processor's cache.

Prior to work stealing,
context local spark stacks returned sparks in last-in-first-out
order,
the global spark queue returns sparks in \emph{first-in-first-out} order.
This does not encourage a left-to-right execution order,
however \citet{wang:2006:hons} proposes that this order may be better:

\begin{quote}
The global spark queue, however, is a queue because we assume that a
spark generated earlier will have greater amounts of work underneath it
than the latest generated spark, and hence is a better candidate for
parallel execution.
\end{quote}

\noindent
If we translate this execution order into the work stealing system,
it means that work is stolen from the bottom end of a context's local spark
stack.
Consider step 3 of the example above.
If a thief removes the spark for \code{r \& s}
then only the spark for \code{u} is on the stack, and
the original context can continue its execution by taking the spark to
\code{u}.
By comparison if the spark for \code{u} was stolen,
it would force the original context to try to execute \code{r \& s}
which, for the reasons above, is less optimal.
Therefore a thief steals work from the bottom of a context local stack
so that a first-in-first-out order is used for parallel tasks.

\citet{halstead:1985:multilisp} also uses processor-local task stacks,
so that when a processor executes a task from its own stack,
a last-in-first-out order is used.
He chose to have thieves steal work from the top of a task stack so
that a thief also executes work in the last-in-first-out order.
However, Halstead remarks that it may be better to steal work from the
bottom of a stack.
Halstead chose a last-in-first-out execution order for the same reasons
that we did:
so that when a processor runs its own task,
the task is executed in the same order as it would be if the program were
running sequentially.
We agree with his claim that this reduces resource usage overheads.
We also agree with his speculation that it results in more efficient cache
usage.
However,
Halstead also claims that this can improve performance for
embarrassingly parallel workloads.
This claim is only true insofar as work stealing reduces runtime costs
\emph{in general}.
No runtime method can completely eliminate the overheads of parallel
execution.
%therefore only during development or compile time can
%embarrassing parallelism be addressed effectively.
Furthermore, in our experience,
using sparks to represent computations whose execution has
not begun has had a more significant improvement in reducing runtime costs.

\plan{Describe the data structure used to implement these stacks and its
properties.}
To preserve the last-in-first-out behaviour of context-local spark
stacks,
and first-in-first-out behaviour of the global queue,
we chose to use double ended queues (deques) for context local spark
storage.
Since the deque will be used by multiple Mercury engines we must either
choose a data structure that is thread safe or use a mutex to protect a non
thread safe data structure.
The pop and push operations are always made by the same thread,%
\footnote{
    Contexts move between engines by being suspended and resumed,
    but this always involves the context run queue's lock or other
    protection.
    Therefore the context's data structures including the
    spark deque are protected from concurrent access.}
therefore the stealing of work by another context is what creates the need
for synchronisation.
The deque described by \citet{Chase_2005_wsdeque} supports lock free,
nonblocking
operation, has low overheads (especially for pop and push operations),
and is dynamically resizable;
all of these qualities are very important to our runtime system's
implementation.
The deque provides the following operations.

\begin{description}

    \item[\code{void MR\_push\_spark(deque *d, spark *s)}]
    The deque's owner can call this to push an item onto the
    top%
\footnote{
        Note when we refer to the top of the deque
        \citet{Chase_2005_wsdeque} refers to the bottom and vice-versa.
        Most people prefer to imagine that stacks grow upwards,
        and since the deque is most often used as a stack we consider this
        end of the deque to be the top.
        We also call this the \emph{hot end} as it is the busiest end of the
        deque.}
    or \emph{hot end} of the deque.
    This can be done with only a \emph{memory barrier} for synchronisation
    (see below).
    If necessary,
    \push will grow the array that is used to implement the
    deque.

    \item[\code{MR\_bool MR\_pop\_spark(deque *d, spark *s)}]
    The deque's owner can call this to pop an item from the top of the
    queue.
    In the case that the deque contains only one item,
    a compare and swap
    operation is used to determine if the thread lost a race to another
    thread attempting to steal the item from the deque (a \emph{thief}).
    When this happens, the single item was stolen by the thief and the
    owner's call to \pop returns false,
    indicating that the deque was empty.
    In the common case the compare and swap is not used.
    In all cases a memory barrier is also used (see below).
    Internally the deque is stored as an array of sparks, not spark
    pointers.
    This is why the second argument, in which the result is returned,
    is not a double pointer as one might expect.
    This implementation detail avoids memory allocation for sparks inside
    the deque implementation.
    A callers of any of these functions temporarily stores the spark on
    its program stack.

    \item[\code{MR\_ws\_result MR\_steal\_spark(deque *d, spark *s)}]
    A thread other than the deque's owner can steal
    items from the bottom of the deque.
    This always uses an atomic compare and swap operation as multiple
    thieves may call \steal on the same deque at the same time.
    \steal can return one of three different values:
    ``success'', ``failure'' and ``abort''.
    ``abort'' indicates that the thief lost a race with either the owner
    (\emph{victim}) or another thief.

\end{description}

%All the atomic compare and swap operations here are wait free,
%rather than looping \pop will return false and \steal will abort.
%\push and \pop are very fast, this is good as a context uses its own
%spark deque more often than calling \steal on another's.
%\steal is slightly slower than either \push or \pop,
%but is still quite fast.
%This is good because a context will
%\push and \pop sparks onto and off of its local deque
%more often than it will \steal a spark from another's deque.
%This means that the top of the deque is used more often than the bottom.
%Therefore, we refer to the top as the \emph{hot} end,
%and to the bottom as the \emph{cold} end.
%These terms are less confusing than the disagreement between the stack
%like behaviour of a deque from its context's perspective,
%and the terminology used by \citet{Chase_2005_wsdeque}.

\begin{algorithm}[tbp]
\begin{verbatim}
void MR_push_spark(MR_SparkDeque *dq, const MR_Spark *spark)
{
    int                     bot;
    int                     top;
    volatile MR_SparkArray  *arr;
    int                     size;

    bot = dq->MR_sd_bottom;
    top = dq->MR_sd_top;
    arr = dq->MR_sd_active_array;
    size = top - bot;

    if (size >= arr->MR_sa_max) {
        /* grow array omitted */
    }

    MR_sa_element(arr, top) = *spark;
    /*
    ** Make sure the spark data is stored before we store the value of
    ** bottom.
    */
    __asm__ __volatile__ ("sfence");
    dq->MR_sd_top = top + 1;
}
\end{verbatim}
\caption{\push with memory barrier}
\label{alg:MR_push_spark}
\end{algorithm}

\begin{algorithm}[tbp]
\begin{verbatim}
volatile MR_Spark* MR_pop_spark(MR_SparkDeque *dq)
{
    int                     bot;
    int                     top;
    int                     size;
    volatile MR_SparkArray  *arr;
    bool                    success;
    volatile MR_Spark       *spark;
   
    top = dq->MR_sd_top;
    arr = dq->MR_sd_active_array;

    top--;
    dq->MR_sd_top = top;

    /* top must be written before we read bottom. */
    __asm__ __volatile__ ("mfence");

    bot = dq->MR_sd_bottom;
    size = top - bot;

    if (size < 0) {
        dq->MR_sd_top = bot;
        return NULL;
    }

    spark = &MR_sa_element(arr, top);
    if (size > 0) {
        return spark;
    }

    /* size = 0 */
    success = MR_compare_and_swap_int(&dq->MR_sd_bottom, bot, bot + 1);
    dq->MR_sd_top = bot + 1;
    return success ? spark : NULL;
}
\end{verbatim}
\caption{\pop with memory barrier}
\label{alg:MR_pop_spark}
\end{algorithm}

Mercury was already using \citet{Chase_2005_wsdeque}'s deques for spark
storage,
most likely because Wang had always planned to implement work stealing.
We did not need to replace the data structure used for a context local
spark storage,
but we will now refer to it as a deque rather than a stack.
We have removed the global spark queue,
as work stealing does not use one.
Consequently,
when a context creates a spark,
that spark is always placed on the context's local spark deque.

\citet{Chase_2005_wsdeque}'s paper uses Java to describe the deque structure
and algorithms.
Java has a strong memory model compared to C (the implementation language of
the Mercury runtime system).
C's \code{volatile} variable storage keyword prevents the compiler from
reordering operations on that variable with respect to other code and
prevents generated code from caching the value of the variable.
CPUs will generally reorder memory operations including executing
several operations concurrently with one another and other instructions.
Therefore, C's \code{volatile} keyword is insufficient when the order of
memory operations is important.
In contrast,
Java's \code{volatile} keyword has the constraints of C's keyword
plus it guarantees that memory operations become visible to other
processors in the order that they appear in the program.
% Java has a strong memory model;
% memory operations on \code{volatile} variables become visible to other
% processors in the order that they appear in the program.
% This is much stronger than C's \code{volatile} keyword which only controls
% how the compiler may order instructions and cache values.
\citet{Chase_2005_wsdeque} uses Java's \code{volatile} qualifier in the
declaration for the \var{top} and \var{bottom} fields of the deque
structure.
When adding the deque algorithms to Mercury's runtime system,
we translated them to C.
To maintain correctness we had to introduce two \emph{memory barriers},\footnote{
    Memory barriers are also known as memory fences.
    In particular the x86/x86\_64 instructions contain the word fence.}
CPU instructions which place ordering constraints on memory operations.
First, we placed a store barrier in the \push procedure
(the inline assembly\footnote{
    In our implementation a C macro is used to abstract away the assembly
    code for different architectures;
    we have shown our figures without the macro to aid the reader.}
in Algorithm~\ref{alg:MR_push_spark}).
This barrier ensures that the spark is written into the array before the pointer to
the top of the array is updated,
indicating that the spark is available.
Second, we placed a full barrier in the \pop procedure
(the inline assembly in Algorithm~\ref{alg:MR_pop_spark}).
This barrier requires that the procedure updates the pointer to the top of the
deque before it reads the pointer to the bottom (which it uses to
calculate the size).
This barrier prevents a race between an owner popping a spark and two
thieves stealing sparks.
It ensures that each of the threads sees the correct number of items on the
deque and therefore executes correctly.
Both these barriers have a significant cost,
in particular the full barrier in \pop.
This means that the \push and \pop functions are not as efficient as they
appear to be.
The deque algorithms are still efficient enough;
we have not needed to improve them further.
For more information about C's weak memory ordering see
\citet{Boehm:2005:threads-as-a-library}.
See also \citet{Adve:2010:memory-models} which discusses the Java memory
model and the new C++0x\footnote{C++0x is also known as C++11} memory model.

\begin{algorithm}[tbp]
\paul{XXX: Place these environments once we know what pagination will be
used}
\begin{verbatim}
 1  void MR_join_and_continue(MR_SyncTerm *st, MR_Code *cont_label) {
 2      MR_bool     finished, got_spark;
 3      MR_Context  *current_context = MR_ENGINE(MR_eng_current_context); 
 4      MR_Context  *orig_context = st->MR_st_orig_context;
 5      MR_Spark    spark;
 6
 7      finished = MR_atomic_dec_and_is_zero(&(st->MR_st_num_outstanding));
 8      if (finished) {
 9          if (orig_context == current_context) {
10              MR_GOTO(cont_label)
11          } else {
12              while (orig_context->MR_ctxt_resume != cont_label) {
13                  CPU_spin_loop_hint();
14              }
15              MR_schedule_context(orig_context);
16              MR_GOTO{MR_idle};
17          }
18      } else {
19          got_spark = MR_pop_spark(current_context->MR_ctxt_spark_deque, &spark);
20          if (got_spark) {
21              MR_GOTO(spark.MR_spark_code);
22          } else {
23              if (orig_context == current_context) {
24                   MR_save_context(current_context);
25                   current_context->MR_ctxt_resume = cont_label;
26                   MR_ENGINE(MR_eng_current_context) = NULL;
27              }
28              MR_GOTO(MR_idle);
29          }
30      }
31  }    
\end{verbatim}
\caption{\joinandcontinue --- initial work stealing version}
\label{alg:join_and_continue_ws1}
\end{algorithm}

\plan{Barrier code}
A context accesses its own local spark queue in the \joinandcontinue barrier
introduced in Section~\ref{sec:rts_original_scheduling}.
The introduction of work stealing has allowed us to optimise
\joinandcontinue.
The new version of \joinandcontinue is shown in
Algorithm~\ref{alg:join_and_continue_ws1}, and
the previous version is shown in
Algorithm~\ref{alg:join_and_continue_peterw}
on page~\pageref{alg:join_and_continue_peterw}.
The first change to this algorithm
is that this version is lock free.
All the synchronisation is performed by atomic CPU instructions, memory
write ordering and one spin loop.
The number of outstanding contexts in the synchronisation term is
decremented and the result is checked for zero atomically.\footnote{
    On x86/x86\_64 this is a \instruction{lock dec} instruction.
    We read the zero flag to determine if the decrement caused the value to
    become zero.}
This optimisation could have been made without introducing work stealing,
but it was convenient to make both changes at the same time.
Note that in the previous version, a single lock was shared between all of
the synchronisation terms in the system.

The next change prevents a race condition that would otherwise be possible
without locking, which could occur as follows.
A conjunction of two conjuncts is executed in parallel.
The original context, $C_{Orig}$,
enters the barrier, decrements the counter from two to one,
and because there is another outstanding conjunct,
it executes the else branch on lines 19--29.
At almost the same time another context, $C_{Other}$,
enters the barrier, decrements the counter and finds that there is no more
outstanding work.
$C_{Other}$ attempts to schedule $C_{Orig}$ on line 15.
However attempting to schedule $C_{Orig}$ before it has finished suspending
would cause an inconsistent state and memory corruption.
Therefore lines 12--14 wait until $C_{Orig}$ has been suspended.
The engine that is executing $C_{Orig}$ first suspends $C_{Orig}$ and then
indicates that it has been suspended by setting $C_{Orig}$'s resume label
(lines 24--25).
The spin loop on lines 12--14 includes a use of a macro named
\code{CPU\_spin\_loop\_hint},
This resolves to the \instruction{pause} instruction on x86 and x86\_64,
which instructs the CPU to pipeline the loop differently in order to reduce
memory traffic and allow the loop to exit without a pipeline stall
\citep{intel:pause}.
Lines 24--25 also include memory write ordering (not shown).

The other change is around line 20:
when the context pops a spark off its stack, it does not check if the spark
was created by a callee's parallel conjunction.
This check can be avoided
because all sparks are placed on the context local spark deques and
thieves never steal work from the hot end of the deque.
Hence if there is an outstanding conjunct then either
its spark will be at the hot end of the deque
or the deque will be empty and the spark will have been stolen.
Therefore any spark at the hot end of the deque cannot possibly be created
by a callee's parallel conjunction.

\begin{algorithm}[tbp]
\begin{verbatim}
 1  void MR_idle() {
 2      MR_Context  *ctxt;
 3      MR_Context  *current_context = MR_ENGINE(MR_eng_current_context);
 4      MR_Code     *resume;
 5      MR_Spark    spark;
 6
 7      MR_acquire_lock(&MR_runqueue_lock);
 8      while(MR_True) {
 9          if (MR_exit_now) {
10              MR_release_lock(MR_runqueue_lock);
11              MR_destroy_thread();                        // does not return.
12          }
13
14          // Try to run a context
15          ctxt = MR_get_runnable_context();
16          if (ctxt != NULL) { 
17              MR_release_lock(&MR_runqueue_lock);
18              if (current_context != NULL) {
19                  MR_release_context(current_context);
20              }
21              MR_load_context(ctxt);
22              resume = ctxt->MR_ctxt_resume;
23              ctxt->MR_ctxt_resume = NULL;
24              MR_GOTO(resume);
25          }
26      
27          // Try to run a spark.
28          if ((current_context != NULL) ||
29                  (MR_num_outstanding_contexts < MR_max_contexts))
30          {
31              if (MR_try_steal_spark(&spark)) {
32                  MR_release_lock(&MR_runqueue_lock);
33                  if (current_context == NULL) {
34                      ctxt = MR_get_free_context();
35                      if (ctxt == NULL) {
36                          ctxt = MR_create_context();
37                      }
38                      MR_load_context(ctxt);
39                      current_context = ctxt;
40                  }
41                  MR_parent_sp = spark.MR_spark_parent_sp;
42                  current_context->MR_ctxt_thread_local_mutables =
43                      spark.MR_spark_thread_local_mutables;
44                  MR_GOTO(spark.MR_spark_code);
45              }
46          }
47
48          MR_timed_wait(&MR_runqueue_cond, &MR_runqueue_lock);
49      }
50  }
\end{verbatim}
\caption{\idle --- initial work stealing version}
\label{alg:MR_idle_wsinitial}
\end{algorithm}

\plan{Other changes to the idle loop?}
We have also modified \idle to use spark stealing;
its new code is shown in Algorithm~\ref{alg:MR_idle_wsinitial}.
We have replaced the old code which dequeued a spark from the global spark
queue
with a call to \trystealspark (see below).
Before attempting to steal a spark,
\idle must ensure that the context limit will not be exceeded
(this was previously done when choosing where to place a new spark).
This check is on lines 28--29; it is true if either
the engine already has a context (which is guaranteed to be available for a
new spark),
or the current number of contexts is lower than the limit.
When sparks are placed on a context's local deque,
the run queue's condition variable is not signalled,
as doing so would be wasteful.
Therefore engines must wake up periodically to check if there is any
parallel work available.
This is done on line 48 using a call to \code{MR\_timed\_wait} with a
configurable timeout (it defaults to 2ms).
We discuss this timeout later in Section~\ref{sec:rts_work_stealing2}.

\plan{Accessing the array of deques}
Engines that attempt to steal a spark must have access to all the spark
deques.
We do this with a global array of pointers to contexts' deques.
When a context is created,
the runtime system attempts to add the context's deque to this array by
finding an empty slot,
one containing a null pointer,
and writing the pointer to the context's deque to that index in the array.
If there is no unused slot in this array, the runtime system will resize the
array.
When the runtime system destroys a context,
it writes \NULL to that context's index in the deque array.
To prevent concurrent access from corrupting the array,
these operations are protected by \code{MR\_spark\_deques\_lock}.

\begin{algorithm}[tbp]
\begin{verbatim}
 1  MR_bool MR_try_steal_spark(MR_Spark *spark) {
 2      int         max_attempts;
 3      MR_Deque    *deque;
 4
 5      MR_acquire_lock(&MR_spark_deques_lock);
 6      max_attempts = MR_MIN(MR_max_spark_deques, MR_worksteal_max_attempts);
 7      for (int attempt = 0; attempts < max_attempts; attempt++) {
 8          MR_victim_counter++;
 9          deque = MR_spark_deques[MR_victim_counter % MR_max_spark_deques];
10          if (deque != NULL) {
11              if (MR_steal_spark(deque, spark));
12                  MR_release_lock(&MR_spark_deques_lock);
13                  return MR_true;
14              }
15          }
16      }
17      MR_release_lock(&MR_spark_deques_lock);
18      return MR_false;
19  }
\end{verbatim}
\caption{\trystealspark --- initial work stealing version} 
\label{alg:try_steal_spark_initial}
\end{algorithm}

\plan{Thief behaviour}
The algorithm for \trystealspark is shown in
Algorithm~\ref{alg:try_steal_spark_initial}.
It must also acquire the lock before using the array.
A thief may need to make several attempts before it successfully finds a
deque with work it can steal.
Line 6 sets the number of attempts to make, which is either the user
configurable \var{MR\_worksteal\_max\_attempts} or the size of the array,
whichever is smaller.
A loop (beginning on line 7) attempts to steal work until it succeeds or it
has made \var{max\_attempts}.
We use a global variable, \var{MR\_victim\_counter},
to implement round-robin selection of the victim.
On line 11 we attempt to steal work from a deque,
provided that its deque pointer in the array is non-null.
If the call to \steal succeeded,
it will have written spark data into the memory pointed to by
\var{spark},
then \trystealspark releases the lock and returns true.
Eventually \trystealspark may give up (lines 17--18).
If it does so it will release the lock and return false.
Whether or not a thief steals work or gives up,
the next thief will resume the round-robin selection where the previous
thief left off.
This is guaranteed because \var{MR\_victim\_counter} is protected by the
lock acquired on line 5.

\plan{Work stealing fixes the premature scheduling decision problem}
All sparks created by a context are placed on its local spark deque.
Sparks are only removed to be executed in parallel when an idle engine
executes \trystealspark.
Therefore
the decision to execute a spark in parallel is only made once an engine is
idle and able to run the spark.
We expect this to correct the premature scheduling decision problem we
described in Section~\ref{sec:rts_original_scheduling_performance}.


\begin{table}
\begin{center}
\begin{tabular}{r|rr|rrrr}
\multicolumn{1}{c|}{Max no.} &
\multicolumn{2}{c|}{Sequmential} &
\multicolumn{4}{c}{Parallel w/ $N$ Engines} \\
\Cbr{of contexts} & \C{not TS} & \Cbr{TS} & \C{1}& \C{2}& \C{3}& \C{4}\\
\hline
\hline
\multicolumn{7}{c}{Prior right recursion results} \\
\hline
4        & 23.2 (0.93) & 21.5 (1.00)
         & 21.5 (1.00) & 21.5 (1.00) & 21.5 (1.00) & 21.5 (1.00) \\
64   &-&-& 21.5 (1.00) & 21.5 (1.00) & 21.5 (1.00) & 21.5 (1.00) \\
128  &-&-& 21.5 (1.00) & 19.8 (1.09) & 20.9 (1.03) & 21.2 (1.01) \\
256  &-&-& 21.5 (1.00) & 13.2 (1.63) & 15.5 (1.38) & 16.5 (1.30) \\
512  &-&-& 21.5 (1.00) & 11.9 (1.81) &  8.1 (2.66) &  6.1 (3.55) \\
1024 &-&-& 21.5 (1.00) & 11.8 (1.81) &  8.0 (2.67) &  6.1 (3.55) \\
2048 &-&-& 21.5 (1.00) & 11.9 (1.81) &  8.0 (2.67) &  6.0 (3.55) \\
\hline
\hline
\multicolumn{7}{c}{New right recursion results with work stealing} \\
\hline
4        & 23.2 (0.93) & 21.6 (1.00)
         & 21.5 (1.01) & 21.7 (1.00) & 21.7 (1.00) & 21.5 (1.01) \\
64   &-&-& 21.7 (1.00) & 21.5 (1.01) & 21.5 (1.01) & 21.5 (1.01) \\
128  &-&-& 21.7 (1.00) & 21.6 (1.00) & 21.1 (1.03) & 21.2 (1.02) \\
256  &-&-& 21.7 (1.00) & 19.5 (1.11) & 18.1 (1.20) & 18.0 (1.20) \\
512  &-&-& 21.5 (1.01) & 12.9 (1.67) &  9.0 (2.41) &  7.9 (2.73) \\
1024 &-&-& 21.5 (1.01) & 10.8 (2.00) &  7.3 (2.96) &  5.6 (3.87) \\
2048 &-&-& 21.5 (1.01) & 10.8 (2.00) &  7.3 (2.95) &  5.7 (3.81) \\
\hline
\hline
\multicolumn{7}{c}{Prior left recursion results} \\
\hline
4        & 23.2 (0.93) & 21.5 (1.00)
         & 21.5 (1.00) & 21.5 (1.00) & 21.5 (1.00) & 21.5 (1.00) \\
64   &-&-& 21.5 (1.00) & 21.5 (1.00) & 21.4 (1.00) & 21.5 (1.00) \\
128  &-&-& 21.5 (1.00) & 21.5 (1.00) & 21.5 (1.00) & 21.5 (1.00) \\
256  &-&-& 21.5 (1.00) & 18.3 (1.17) & 18.2 (1.18) & 19.6 (1.09) \\
512  &-&-& 21.5 (1.00) & 17.9 (1.20) & 15.5 (1.39) & 16.4 (1.31) \\
1024 &-&-& 21.5 (1.00) & 18.0 (1.19) & 14.7 (1.46) & 16.1 (1.33) \\
2048 &-&-& 21.5 (1.00) & 18.0 (1.19) & 15.4 (1.40) & 17.8 (1.21) \\
\hline
\hline
\multicolumn{7}{c}{New left recursion results with work stealing} \\
\hline
4        & 23.2 (0.93) & 21.5 (1.00)
         & 21.5 (1.00) & 10.8 (1.99) &  7.3 (2.96) &  5.4 (3.95) \\
8    &-&-& 21.7 (0.99) & 10.8 (1.99) &  7.3 (2.94) &  5.5 (3.92) \\
16   &-&-& 21.6 (0.99) & 10.8 (1.99) &  7.2 (2.98) &  5.5 (3.92) \\
32   &-&-& 21.5 (1.00) & 10.8 (1.99) &  7.2 (2.98) &  5.5 (3.92) \\
\end{tabular}
\end{center}
\caption{Work stealing results --- initial implementation}
\label{tab:work_stealing_initial}
\end{table}



\plan{Benchmark}
We benchmarked our initial work stealing implementation with the mandelbrot
program from previous sections.
Table~\ref{tab:work_stealing_initial} shows the results of our benchmarks.
The first and third row groups are the same results as the first and second
row groups from Table~\ref{tab:2009_left_nolimit} respectively.
They are included to allow for easy comparison.
The second and fourth row groups were generated with a newer version of
Mercury that implements work stealing as described above.\footnote{
    The version of Mercury used is slightly modified,
    due to an oversight we had forgotten to include the context limit
    condition in \idle.
    The version we benchmarked includes this limit and matches the
    algorithms shown in this section.}
The figures in parenthesis are speedup figures.

The first conclusion that we can draw from the results is that
left recursion with work stealing is much faster than left recursion without
work stealing.
We can see that work stealing has solved the premature spark scheduling problem
for mandelbrot.
Given that left recursive mandelbrot represented a pathological case of this
problem,
we are confident that the problem has generally been solved.

We can also see that the context limit no longer affects performance
with left recursion.
There is no significant difference between the results for the various
settings of the context limit.

\paul{XXX: Consider always putting left and right in italics.}
When we consider the case for right recursion we can see that
work stealing has improved performance when there is a
high context limit.
This is somewhat expected since work stealing is known as a very
efficient way of managing parallel work on a shared memory system.
However we did not expect such a significant improvement in
performance.
In the non work stealing system,
a spark is placed on the global spark queue if both the context limit has
not been reached,
and there is an idle engine.
When a spark is being created there may not be an idle engine,
however an engine may become idle soon and wish to execute a spark.
With work stealing,
all sparks are placed on local deques and an idle
engine will attempt to steal work when it is ready ---
delaying the decision to run the spark in parallel.
This suggests that the premature scheduling problem was also affecting right
recursion.
But the effect was minor compared to the pathological case we saw with left
recursion.

There is a second observation we can make about right recursion.
In the cases for 256 or 512 contexts work stealing performs worse than
without work stealing.
We believe that the original results were faster because more work was done
in parallel.
When a context spawned off a spark,
it would often find that there was no free engine,
and place the spark on its local queue.
After completing the first conjunct of the parallel conjunction it would
execute \joinandcontinue.
\joinandcontinue would find that the context had a spark on its local
stack and would execute this spark directly.
When a spark is executed directly the existing context is re-used and so the
context limit does not increase as quickly.
The work stealing version does not do this,
it always places the spark on its local stack and almost always
another engine will steal that spark,
and the creation of a new context will bring the system closer to the
context limit.
The work stealing program is more likely to reach the context limit earlier,
where it will be forced into sequential execution.

%\paul{XXX: Consider verifying this point with instrumentation.}



\status{Writing}

\plan{Acknoledge results of previous chapter}
Right recursion has a context limit problem and is generally slower.
Left recursion is faster.

\plan{Describe the intuation for the transformation.}

\plan{Describe the transformation}

\plan{Discuss reasoning for not showing results}.


%
\status{Not written}

\plan{Explain why we want $P$ engines when there are $P$ processors}
As explained in Chapter \ref{chap:background} Mercury manages its parallel
work with userspace scheduling.
Parallel tasks are represented by contexts or sparks,
depending on whether the task has started execution or not.
Engines run contexts and can switch between them without using the operating
system.
We use a set of operating system threads,
each thread running a Mercury engine.
The operating system manages these threads including their mapping onto
processors.



\plan{Explain briefly how we detect how many processors there are,}
Explain that this method is good because it is (mostly) cross platform.

\plan{Explain why we want thread pinning.}
I wonder if there's any relevant literature.

\plan{Explain how we get thread pinning.}
This method is also (mostly) cross platform.

\plan{What about SMT, not all processors are equal.}
This does not matter when we are creating $P$ engines.
But it does matter when we create less than $P$ engines,
explain how bad CPU assignments are sub-optimal.

\plan{How do we handle SMT}
This uses a support library, which is cross platform, provided that it is installed.
We fall back to setcpuafinity() when it is not.

\paul{I am not going to talk about busy waiting since I have not written the
runtime system in a way that I can test or change this easily.}



\status{Not written}

In this section we take the opportunity to improve on our work stealing
implementation from Section \ref{sec:rts_work_stealing}.
While we made these improvements we also found it useful to change how idle
engines behave.
Although these too changes are conceptually distinct,
they where made together and their implementations are interlinked.
Therefore we will present and benchmark them together as one set of changes.

\plan{Describe problems with associating stacks with contexts}
Sparks are stored on context local deques,
this has a several significant problems.

\begin{description}

    \item[There is a dynamic number of spark deques]~

    The number of contexts changes during a programs execution,
    therefore the number of spark deques also changes.
    This means that we must have code to manage this changing number of
    deques.
    This code makes the runtime system more complicated necessary,
    both when stealing a spark and when destroying or creating a context.

    \item[This requires locking]~

    The management of contexts must be thread safe so that the set of
    spark deques is not corrupted.
    We store spark deques in a global array protected by a lock.
    It may be possible to replace the array with a lock free data structure.
    However it is better to remove the need of thread safety by using a
    constant set of deques.

    \item[A large number of contexts makes work stealing slower]~
    
    In prior sections
    we have shown that the number of contexts in use can often be very high,
    much higher than the number of Mercury engines.
    If there are $N$ engines and $M$ contexts,
    then there can be at most $N$ contexts running and
    at least $M-N$ contexts suspended (blocked and waiting to run),
    In some cases there can be at most $M-1$ suspended contexts.
    A context can become suspended in one of two ways:
    by blocking on a future's value in call to \wait,
    or by blocking on an incomplete conjunct in a call to \joinandcontinue.
    We attempt to minimise the former case and the later case cannot occur
    if the context has a spark on its local spark deque (it would run the
    spark rather than block).
    Therefore the large majority of the suspended contexts will not have
    sparks on their deques,
    and the probability of selecting a deque at random with a spark on its
    stack is only a little bit higher than $M \choose N$ at best and
    and $M \choose 1$ at worst.
    Furthermore the value of $M$ can be very high in pathological cases.
    If an engine does not successfully steal a spark from a deque,
    it will continue by trying to steal a spark from a different deque.
    An engine can exhaust all its attempts, even when there is work
    available.
    Upon exhausting all its attempts or all the deques,
    the engine will sleep before making another round of attempts.
    Each round of attempts (regardless of success or failure) has a
    complexity of $O(M)$.

\end{description}

\noindent
\plan{We associate stacks with engines}
The solution to all of these problems is to associate spark deques with
engines rather than with contexts.
A running Mercury program has a constant number of engines,
and therefore the number of spark deques will not vary.
This has allowed us to remove the code used to resize the deque array,
the array is filled in when the engines are created during startup,
after which it is never modified.
This code is much simpler as it never needs to resize the array.
Context creation and destruction is now simpler and faster,
it does not need to use the spark array at all, and certainly never contends
for the spark array lock.
We can also remove the locking code in \trystealspark used to ensure that
the array is not changed while a thief is trying to steal a spark.
The cost of work stealing also becomes linear in the number of engines
rather than the number of contexts (there are usually fewer engines than
contexts).

\begin{algorithm}[tbp]
\begin{algorithmic}[1]
\Procedure{try\_steal\_spark}{$spark\_ptr$}
  \State $result \gets false$
  \For{$attempt = 0$ to $MR\_num\_engines$}
    \State $victim\_index \gets
        (MR\_engine.victim\_counter + attempt) \bmod MR\_num\_engines$
    \State $deque \gets
       MR\_spark\_deques$[$victim\_index$]
    \State $result \gets$ steal\_spark($deque$, $spark\_ptr$)
    \If{$result$}
      \State break
    \EndIf
  \EndFor
  \State $MR\_engine.victim\_counter \gets
    MR\_engine.victim\_counter + offset$
  \State \Return $result$
\EndProcedure
\end{algorithmic}
\caption{try\_steal\_spark}
\label{alg:try_steal_spark_revised}
\end{algorithm}

\plan{Show new \trystealspark}
We have made some other changes to \trystealspark,
its new code is shown in Algorithm \ref{alg:try_steal_spark_revised}.
The lock was also used to protect the victim counter,
ensuring that round robin selection of the victim was maintained.
Now each engine independently performs its own round robin selection of the
victim using a new field \code{victim\_counter} in the engine structure.
We have not evaluated if this policy works better or worse,
even if it is worse,
it avoids the cost of a lock which is more significant.
The two other changes are minor,
we have removed the configurable limit of the number of work stealing
attempts per round,
and the test for null array slots has been removed as it is now unnecessary.

\plan{Show how this is safe.}
Associating the spark deques with engines can change the order in which
sparks are executed,
we have to ensure that the system is still deadlock free.
In the previous version
a context that is blocked on a call to \wait may contain sparks
which can only be accessed by a thief,
In the current version,
those sparks are now associated with the engine that the context was running
on,
and therefore that engine has the opportunity to run those sparks by
retrieving them from the hot end of its own deque.
If the engine attempts to execute one of these sparks it will need a
context,
and the creation of a new context may exceed the context limit even though
the spark is being executed locally.
This is just one instance where an engine \emph{without a context}
may try to run a spark from its own deque,
in the previous version a context would always be available because sparks
where associated with contexts.

\begin{algorithm}[tbp]
\begin{multicols}{2}

\parbox{\textwidth}{
\begin{algorithmic}
    \Procedure{MR\_idle}{}
        \State \tramp{try\_run\_context()}
        \State \tramp{try\_run\_local\_spark($NULL$)}
        \State \tramp{try\_steal\_spark()}
        \Goto MR\_sleep
    \EndProcedure
\end{algorithmic}
}

\begin{minipage}{\textwidth}
\begin{verbatim}
#define TRAMPOLINE(call)        \\
    do {                        \\
        MR_Code *code;          \\
        code = (call);          \\
        if (code) {             \\
            goto code;          \\
        }                       \\
    } while(0)
\end{verbatim}
\end{minipage}

\end{multicols}
\caption{New \idle code}
\label{alg:idle_entry_point}
\end{algorithm}

\paul{Rename \getglobalwork throughout previous sections to \idle}.

An idle engine without a context calls \idle to acquire new work,
\idle's code is shown in Algorithm \ref{alg:idle_entry_point}.
\idle has changed significantly,
many of its details shown in previous sections have been moved into C
functions,
these include:
\tryruncontext which tries to execute a suspended but runnable context;
\tryrunlocalspark which tries to run a spark from the top of the engine's
spark deque, possibly creating a new context for the spark;
and 
\trystealspark which was shown above in Algorithm
\ref{alg:try_steal_spark_revised}.
Because running a local spark may allocate a context and exceed the context
limit,
the engine must execute any suspended but runnable context instead.
Doing so is always permitted despite the context limit.
Furthermore,
a context whose execution has already begin may make more parallel work
available creating sparks or signalling futures.
If the engine is able to run the context until the completion of its work
(until the contexts executes the \joinandcontinue barrier at the end of a
parallel conjunct)
then this may make the context available to execute a spark from the
engine's local spark queue.
Therefore we always try to run a context (using \tryruncontext)
before attempting to run a local spark (using \tryrunlocalspark).
Of course there will be cases when \tryruncontext fails and an engine
without a context then attempts to run a local spark.
To facilitate this, the context limit is not checked in \tryrunlocalspark
(it is still used in \trystealspark).

\idle is a hard coded Mercury procedure and it does not return control to
its caller,
therefore it continues execution by jumping to the code address of the next
thing to execute.
This can either be the resume address of a context,
the entry point of a spark,
or the Mercury procedure \sleep.
However
the C functions are used to find the next context or spark to execute,
if successful these C functions prepare the engine to execute the spark or
context and then return the address that the engine should execute.
they \emph{must} return rather than execute the spark or context directly so
that the C stack pointer has the same value that it did upon entering \idle.
\idle uses a \emph{trampoline}\footnote{
    Trampoline is an overloaded term in computer science.
    The reader may or may not agree with our use of this term.
    }
macro to jump to the returned code
address or fall through to the next instruction if the function returned
false.
If the idle engine cannot find a spark or context to execute then it jumps
to \sleep.

\begin{algorithm}[tbp]
\begin{minipage}{\textwidth}
\begin{verbatim}
struct engine_sleep_sync {
    sem_t                               sleep_sem;
    lock                                lock;
    volatile unsigned                   state;
    volatile unsigned                   action;
    union MR_engine_wake_action_data    action_data;
};

union MR_engine_wake_action_data {      
    MR_EngineId     worksteal_engine;
    MR_Context      *context;
};
\end{verbatim}

\begin{algorithmic}
    \Procedure{MR\_sleep}{}
        \Loop
            \State $eng\_data \gets MR\_engine\_sleep\_data$[$engine\_id$]
            \State $eng\_data.state \gets$ SLEEPING
            \State $sem\_wait(eng\_data.sleep_sem)$
            \Switch{$eng\_data.action$}
              \Case{ACTION\_SHUTDOWN}
                \State $\cdots$
              \EndCase
              \Case{ACTION\_RUN\_CONTEXT}
                \State $ctxt \gets eng\_data.action\_data.context$
                \State MR\_load\_context($ctxt$)
                \Goto $ctxt.resume\_label$
              \EndCase
              \Case{ACTION\_STEAL\_SPARK}
                \State $MR\_engine.victim\_counter \gets
                    eng\_data.action\_data.worksteal\_engine$ 
                \State \tramp{try\_steal\_spark()}
                \State \tramp{try\_run\_context()}
                \State break
              \EndCase
              \Case{ACTION\_NONE}
                \State \tramp{try\_run\_context()}
                \State \tramp{try\_steal\_spark()}
                \State break
              \EndCase
            \EndSwitch
        \EndLoop
    \EndProcedure
\end{algorithmic}

\end{minipage}
\caption{The \sleep code}
\end{algorithm}

\sleep is another hand written Mercury procedure.
It uses an
\enginesleepsync structure to manage the state of each engine.
These structures are organised into a global array indexed by engine ids,
making them accessible to other engines.
The structure contains a semaphore and a lock;
the engine owning the structure waits on \code{sleep\_sem} in order to
sleep,
and \code{lock} is used by other engines to control their mutual
access to the \enginesleepsync structure.
There are three other fields in the structure:
\code{state} is used by the engine that owns the structure to communicate
its state to other engines,
\code{action} and \code{action\_data} are used by other threads when
communicating to this engine.

Upon jumping into \idle an engine sets its state to \code{SLEEPING} and
waits on \code{sleep\_sem}.
If another engine posts to \code{sleep\_sem} then the engine is woken up,
and retrieves the value of the \code{action} field.
This field can be used to tell the engine what to do upon wakening.
It may be instructed to shutdown, to execute a context or to steal a spark.
If it is instructed to run a context it can be passed the context directly
using the \code{action\_data} field,
similarly, if it instructed to steal a spark \code{action\_data} will tell
it whose spark stack it should begin its round robin search from.
If thread stealing fails the engine will check the global contest queue for
a runnable context,
if that fails then \idle will loop and putting the engine back to sleep.
If no action is specified
(the value of the \code{action} field is \code{ACTION\_NONE})
them \sleep will try to run a context from the global context queue before
attempting to steal a spark.

\begin{algorithm}[tbp]
\begin{algorithmic}
\Procedure{wake\_engine}{$engine\_id, action, action\_data, states$}
    \State $eng\_data \gets MR\_engine\_sleep\_data$[$engine\_id$]
    \State acquire\_lock($eng\_data.lock$)
    \If{$eng\_data.state \in states$}
        \State $eng\_data.action \gets action$
        \State $eng\_data.action\_data \gets action\_data$
        \State $eng\_data.state \gets$ WOKEN
        \State sem\_post($eng\_data.sleep\_sem$)
        \State $result \gets$ true
    \Else
        \State $result \gets$ false
    \EndIf
    \State release\_lock($eng\_data.lock$)
    \Return $result$
\EndProcedure
\end{algorithmic}
\caption{\wakeengine}
\label{alg:wake_engine}
\end{algorithm}

Engines may be woken with a call to \wakeengine whose code is shown in
Algorithm \ref{alg:wake_engine}.
\wakeengine takes id of the engine to awaken,
the action and its data that the engine should perform upon awakening,
and the set of states that we may awaken the engine from (usually
\code{SLEEPING}).
The implementation of \wakeengine is mostly straightforward,
the important details are that we must only wake the engine if it has not
been woken already, this is done by checking the value of
\code{eng\_data.state} after acquiring the lock,
and setting the new state while holding the lock.
The action and action data must also be written to the
\enginesleepsync structure before posting to the engine's sleep
semaphore.
The parameter \code{states} is a bit field, with each state represented by
its own bit.
This is flexible, it allows us to message an engine regardless of its
current state.
For example,
we shutdown an engine by calling \wakeengine with the shutdown action a and
all bits set in the \code{states} bit field.
Compared to the previous system where a lock and condition variable were
used,
this system allows us to wake engines selectively.
This is useful in the case when a context must be executed on a particular
engine because that engine's C stack contains a frame that made a call to
some Mercury code and now must be returned into.
In the future this feature could be used to schedule computations on
\emph{nearby} processors,
processors that may share a second or third level cache with the current
processor.

\begin{algorithm}[tbp]
\begin{algorithmic}[1]
\Procedure{MR\_join\_and\_continue}{$ST, ContLabel$}
  \State $finished \gets$ atomic\_dec\_and\_is\_zero($ST.num\_outstanding$)
  \If{$finished$}
    \If{$ST.orig\_context = this\_context$}
      \Goto{$ContLabel$}
    \Else
      \While{$ST.orig\_context.resume\_label \neq ContLabel$}
        \State CPU\_relax
      \EndWhile
      \State release\_context($this\_context$)
      \State $this\_context \gets$ NULL
      \State load\_context($ST.parent$)
      \Goto{$ContLabel$}
    \EndIf
  \Else
    \If{$ST.orig\_context = this\_context$}
      \State $MR\_r1 \gets ContLabel$
      \Goto{MR\_idle\_orig\_context}
    \Else
      \Goto{MR\_idle}
    \EndIf
  \EndIf
\EndProcedure
\end{algorithmic}
\caption{\joinandcontinue}
\label{alg:join_and_continue_ws2}
\end{algorithm}

By moving spark deques from contexts to engines we had to change \idle.
We have also had to change the \joinandcontinue barrier.
The new version of \joinandcontinue is shown in Algorithm
\ref{alg:join_and_continue_ws2}.
The most significant change is on line 15,
\joinandcontinue no longer attempts to run a spark from the local deque,
it relies on \idle to do this.
The second change is that \joinandcontinue no longer suspends the original
context, it executes \idle while holding the original context.
This would invalidate \idle's precondition that if an engine has a context,
that that context is otherwise free to execute \emph{any} spark,
but an original context may only execute sparks that were created by the
parallel conjunction that it is currently executing.
The new \idle code allows us to create any number of entry points,
we have created the \idleorigcontext entry point which expects to be called
with such a context.
However, since we do not set the contest's \code{resume\_label} field until
the engine has given up the context,
we must pass the value of the continuation label into \idleorigcontext.
This is done using \code{MR\_r1}, the first Mercury abstract machine
register;
this is the normal calling convention used by compiled Mercury procedures.

The other change is an optimisation on line 10.
if the parallel conjunction has been finished and this engine is not
executing the original context,
then instead of placing the original context on the context run queue as the
previous version did,
the new version unloads its current context and resumes execution of the
original context,
This has three benefits,
firstly, this does not use the run queue's lock,
making it more efficient.
Secondly, in the old algorithm, the engine would schedule the context and
call \idle,
where it would check the context run queue and was likely to run the same
context anyway.
Thirdly, by deliberately running the context on the CPU that most
recently completed the parallel conjunction,
we may be able to take advantage of relevant data being hot in the CPU's
cache.

\begin{algorithm}[tbp]
\begin{algorithmic}
    \Procedure{MR\_idle\_orig\_context}{}
        \State $join\_label \gets MR\_r1$
        \State \tramp{try\_run\_local\_spark($join\_label$)}
        \State suspend($join\_label$);
        \State $this\_context.resume\_label \gets join\_label$
        \State $this\_context \gets$ NULL
        \State \tramp{try\_run\_context()}
        \State \tramp{try\_run\_local\_spark(NULL)}
        \State \tramp{try\_steal\_spark()}
        \Goto MR\_sleep
    \EndProcedure
\end{algorithmic}
\caption{New entry point to the idle loop for dirty contexts.}
\label{alg:idle_orig_context}
\end{algorithm}

The new entry point \idleorigcontext is shown in Algorithm
\ref{alg:idle_orig_context}.
This entry point expects to be called from \joinandcontinue while the engine
is holding a context that is needed to complete a parallel conjunction.
It tries to run a spark from the engine's local spark deque first,
the \tryrunlocalspark will check if the spark at the top of the spark deque
is compatible with the current context,
if it is not it will put the spark back onto the spark deque and return
\NULL, it also returns \NULL if there is no spark on the deque.
\idleorigcontext uses the trampoline macro from Algorithm
\ref{alg:idle_entry_point} to jump to the spark's entry point
\tryrunlocalspark found a valid spark.
Next \idleorigcontext saves the context that the engine was using and then
attempts to resume a suspended context,
failing that it retries running a local spark,
this is done because there may have been a spark on the engine's spark deque
that was not compatible, but now that the engine has no context, any spark
is compatible.
If this also fails then
\idleorigcontext will attempt to steal a spark from another engine and run
it,
failing that the engine will jump to \sleep.
We could add further entry points in the future if we found them useful,
for example,
it may be useful to distinguish between cases where an engine already has a
context or does not have a context.

\paul{XXX: Consider removing MR\_ from most if not all symbols.}


\begin{table}
\begin{center}
\begin{tabular}{r|rr|rrrr}
\multicolumn{1}{c|}{Max no.} &
\multicolumn{2}{c|}{Sequmential} &
\multicolumn{4}{c}{Parallel w/ $N$ Engines} \\
\Cbr{of contexts} & \C{not TS} & \Cbr{TS} & \C{1}& \C{2}& \C{3}& \C{4}\\
\hline
\hline
\multicolumn{7}{c}{Prior right recursion work stealing results} \\
\hline
4        & 16.3 (0.93) & 15.2 (1.00)
         & 15.2 (1.00) & 15.2 (1.00) & 15.2 (1.00) & 15.2 (1.00) \\
64   &-&-& 15.2 (1.00) & 15.2 (1.00) & 15.2 (1.00) & 15.2 (1.00) \\
128  &-&-& 15.2 (1.00) & 15.0 (1.01) & 14.8 (1.03) & 14.9 (1.02) \\
256  &-&-& 15.2 (1.00) & 13.5 (1.12) & 12.6 (1.21) & 12.6 (1.21) \\
512  &-&-& 15.1 (1.00) &  8.8 (1.73) &  6.5 (2.34) &  5.3 (2.84) \\
1024 &-&-& 15.5 (0.98) &  7.7 (1.98) &  5.2 (2.92) &  4.1 (3.70) \\
2048 &-&-& 15.2 (1.00) &  7.9 (1.93) &  5.2 (2.92) &  4.1 (3.71) \\
\hline
\hline
\multicolumn{7}{c}{Right recursion results with revised work stealing} \\
\hline
4        & 15.3 (1.00) & 15.3 (1.00)
         & 15.3 (1.00) & 15.3 (1.00) & 15.3 (1.00) & 15.3 (1.00) \\
64   &-&-& 15.3 (1.00) & 15.3 (1.00) & 15.3 (1.00) & 15.7 (0.97) \\
128  &-&-& 15.7 (0.98) & 15.3 (1.00) & 15.7 (0.97) & 15.3 (1.00) \\
256  &-&-& 15.3 (1.00) & 14.5 (1.06) & 14.8 (1.03) & 13.5 (1.13) \\
512  &-&-& 15.3 (1.00) &  9.6 (1.59) &  8.0 (1.91) &  6.8 (2.25) \\
1024 &-&-& 15.7 (0.97) &  7.7 (1.98) &  5.3 (2.87) &  3.9 (3.90) \\
2048 &-&-& 15.3 (1.00) &  7.7 (1.97) &  5.2 (2.94) &  3.9 (3.90) \\
\hline
\hline
\multicolumn{7}{c}{Prior left recursion work stealing results} \\
\hline
N/A      & 15.3 (1.02) & 15.6 (1.00)
         & 15.2 (1.03) &  7.8 (2.00) &  5.2 (2.98) &  3.8 (4.05) \\
\hline
\hline
\multicolumn{7}{c}{Left recursion results with revised work stealing} \\
\hline
N/A      & 15.3 (1.00) & 15.3 (1.00)
         & 15.3 (1.00) &  7.7 (1.99) &  5.1 (2.97) &  4.0 (3.83) \\
\end{tabular}
\end{center}
\caption{Work stealing results --- revised implementation}
\label{tab:work_stealing_revised}
\end{table}



\plan{Benchmark}

\plan{Evaluation}

\plan{Further potential work}
\plan{Steal half}
Our work stealing implementation could be improved and tuned further.
We have noticed that in many workloads such as a left recursive parallel
loop one engine creates all the sparks, all the parallelism comes from one
place.
In a situation like this work stealing becomes more common than local work
execution,
one processor would execute work locally while $P - 1$ processors act as
thieves, and a single work queue can again become a bottleneck.
To avoid this behaviour
a steal-half implementation (such as \citet{hendler:2002:stealhalf})
should quickly distributed work evenly between the processors,
reducing the overall number of work stealing operations.
\plan{memory hierarchy awareness}
Another potential improvement is in the selection of a thief's victim.
A thief may wish to prefer victims that are nearby in terms of memory
topology, so that communication of the data relevant to the spark is
cheaper.
This can also be used when selecting which engine to wake up in order to
pass work to it.
We discuss memory hierarchy awareness in more detail in the next section.



%\section{Proposed scheduling tweaks}
%\label{sec:proposed_tweaks}
%\status{Not written, May move to TS chapter}
%
%I really think that this section will move to the \tscope chapter,
%it will have more in common with that chapter and more data will be
%available.
%Secondly, \tscope can be used with micro-benchmarks to measure the
%average costs of certain operations in the RTS.
%I will not write it until at least the rest of this chapter is finished.

