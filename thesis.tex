\documentclass[a4paper,twoside]{report}

\title{Automatic Parallelisation for Mercury}
\author{Paul Bone}

% According to gradresearch.unimelb.edu.au:
%
% The disteration should be roughly 80,000 words and strictly fewer than
% 100,000 words.
%
% Printed on a4 paper.
%
% 1.5 spaced.
%
% Margins no smaller than 30mm.
%
% Page numbers must be consecutive and printed within the margin.
%

% WARNING: citesort, hyperref and algorithm don't play nice together.
% This can be fixed by using natbib (which is apparently better) instead of
% citesort.
% And by using hyperref before algorithm.

\usepackage{hyperref}

\usepackage{algorithm}
\usepackage{amsmath}
\usepackage[margin=3cm]{geometry}
\usepackage[sort&compress]{natbib}
\usepackage{pstricks}
\usepackage{setspace}
\usepackage{breakurl}
\usepackage{xspace}

\onehalfspacing


% Proper names.
\newcommand{\mapfoldl}{\code{map\_foldl}\xspace}
\newcommand{\mapfoldlpar}{\code{map\-\_\-fold\-l\-\_\-par}\xspace}
\newcommand{\LC}{\code{LC}\xspace}
\newcommand{\LCS}{\code{LCslot}\xspace}
\newcommand{\createloopgoal}{\code{create\-\_loop\_goal}\xspace}
\newcommand{\putbarriers}{\code{put\-\_bar\-riier\-s\-\_in\-\_base\_case\-s}\xspace}
\newcommand{\io}{\code{io}\xspace}
\newcommand{\di}{\code{di}\xspace}
\newcommand{\uo}{\code{uo}\xspace}
\newcommand{\NULL}{\code{NULL}\xspace}
\newcommand{\n}{{\textbackslash}n\xspace}
\newcommand{\signal}{\code{future\_signal/2}\xspace}
\newcommand{\wait}{\code{future\_wait/2}\xspace}
\newcommand{\get}{\code{future\_get/2}\xspace}
\newcommand{\joinandcontinue}{\code{MR\_join\_and\_continue}\xspace}
\newcommand{\getglobalwork}{\code{MR\_get\_global\_work}\xspace}
\newcommand{\trystealspark}{\code{MR\_try\_steal\_spark}\xspace}
\newcommand{\tscope}{ThreadScope\xspace}

% the name 'det' is already taken.  So I've prefixed all the detisms with 'd'
\newcommand{\ddet}{\code{det}\xspace}
\newcommand{\dsemidet}{\code{semidet}\xspace}
\newcommand{\dmulti}{\code{multi}\xspace}
\newcommand{\dnondet}{\code{nondet}\xspace}
\newcommand{\dfailure}{\code{failure}\xspace}
\newcommand{\derroneous}{\code{erroneous}\xspace}
\newcommand{\dccmulti}{\code{cc\_multi}\xspace}
\newcommand{\dccnondet}{\code{cc\_nondet}\xspace}

\newcommand{\seqfn}[0]{\texttt{seq}\xspace}
\newcommand{\parfn}[0]{\texttt{par}\xspace}

\newcommand{\PS}[0]{\code{ProcStatic}\xspace}
\newcommand{\PD}[0]{\code{ProcDynamic}\xspace}
\newcommand{\CSS}[0]{\code{CallSiteStatic}\xspace}
\newcommand{\CSD}[0]{\code{CallSiteDynamic}\xspace}
\newcommand{\Clique}[0]{\code{Clique}\xspace}
\newcommand{\push}[0]{\code{push\_spark()}\xspace}
\newcommand{\pop}[0]{\code{pop\_spark()}\xspace}
\newcommand{\steal}[0]{\code{steal\_spark()}\xspace}

% These get used in math.
\newcommand{\prodtime}{\operatorname{prodtime}\,}
\newcommand{\prodtimep}{\operatorname{prodtime'}\,}
\newcommand{\callprodtime}{\operatorname{call\_prodtime}\,}
\newcommand{\constime}{\operatorname{constime}\,}
\newcommand{\constimep}{\operatorname{constime'}\,}
\newcommand{\callconstime}{\operatorname{call\_constime}\,}
\newcommand{\iteconstime}{\operatorname{iteconstime}\,}
\newcommand{\timef}{\operatorname{time}\,}
\DeclareMathOperator*{\Avg}{Avg}
\newcommand{\undef}{\operatorname{undefined}\xspace}
\newcommand{\canthappen}{\operatorname{cannot\,happen}\xspace}
\newcommand{\pat}[1]{\llbracket#1\rrbracket}
\newcommand{\timeofcall}{\operatorname{time\_of\_call}\xspace}
\newcommand{\reorder}{\operatorname{reorder}\xspace}
\newcommand{\trypushconjlater}{\operatorname{try\_push\_conj\_later}\xspace}
\newcommand{\canswap}{\operatorname{can\_swap}\xspace}

% I use this in tables to create a cell whose contents are centered.
\newcommand{\C}[1]{\multicolumn{1}{c}{#1}}
\newcommand{\Ctwo}[1]{\multicolumn{2}{c}{#1}}
% Centre with a boarder on the right.
\newcommand{\Cbr}[1]{\multicolumn{1}{c|}{#1}}
% Decimal alignment type (dcolumn package)
\newcolumntype{d}[1]{D{.}{.}{#1}}

% Author notes.
\newcommand{\authornote}[3]{
% Comment out next line to remove author notes
    {\fbox{\sc #1}:$\triangleright$\textcolor{#2}{\textbf{#3}}$\triangleleft$}%
}

\newcommand{\paul}[1]{\authornote{Paul}{blue}{#1}}
\newcommand{\peter}[1]{\authornote{Peter}{green}{#1}}
\newcommand{\zoltan}[1]{\authornote{Zoltan}{red}{#1}}
\newcommand{\status}[1]{\authornote{Status of this section}{red}{#1}}
\newcommand{\plan}[1]{\authornote{Plan}{red}{#1}}

\newcommand{\todoitem}[3]{\parbox{4in}{#1} & 
  \parbox{1in}{#2} & \parbox{1in}{#3} }

% Optional prose.
\newcommand{\iclp}[1]{{}}
\newcommand{\conf}[1]{{}}
\newcommand{\tr}[1]{{#1}}

% Semantic markup.
\newcommand{\code}[1]{{\tt#1}}
\newcommand{\samp}[1]{`{\tt#1}'}
\newcommand{\param}[1]{\mbox{\it{#1}}}
\newcommand{\dfn}[1]{{\em#1}}
\newcommand{\stress}[1]{{\em#1}\/}
\newcommand{\calls}[0]{\rightarrow}
\newcommand{\instr}[1]{{\red{\code{#1}}}}

% Layout, floats etc.
\newlength{\figboxwidth}
\setlength{\figboxwidth}{\textwidth}
\addtolength{\subfigcapskip}{1em}

% 2 arguments:
% 	- the figure filename
% 	- the caption
% note that the filename will be used to form the label "fig:filename"
\newcommand{\picfigure}[2]{
	\begin{figure}[t]
	\newcommand{\spf}{\footnotesize}      % Sets font size for the picture
	\input{pics/#1}                      % defines macro called \graph
	\centerline{\raise 1em\box\graph}     % Prints the picture.
	\vspace{1mm}
	\caption{#2}
	\label{fig:#1}
	\end{figure}
}

% Extra control-flow code for algorithmic
\algloopdefx[Goto]{Goto}[1]{\textbf{goto} #1}

% Don't print closing statments in algorithmic examples.
\algnotext{EndIf}
\algnotext{EndProcedure}




\begin{document}

\begin{titlepage}
\begin{center}
\vspace*{3cm}
{\Huge \textbf{Automatic Parallelisation for Mercury}} \par
\vspace*{2cm}
{\huge \textbf{Paul Bone}} \par
\vspace*{4cm}
{\Large Submitted in total fulfilment of the requirements of the degree of
        Doctor of Philosophy} \par
\vspace*{1cm}
% XXX: Revise this date.
{\Large April 2012} \par
\vspace*{1cm}
{\Large Department of Computing and Information Systems} \par
{\Large The University of Melbourne} \par
\end{center}
\end{titlepage}

\chapter*{Abstract}

\paul{300-500 words}

\chapter*{Declaration}

% This is the prose that gradresearch.unimelb.edu.au said I should use.

This is to certify that:

\begin{itemize}

    \item the thesis comprises only my original work towards the PhD except
          where indicated in the Preface,

    \item due acknowledgement has been made in the text to all other material
          used,

    \item the thesis is fewer than 100,000 words in length, exclusive of
          tables, maps, bibliographies and appendices.

\end{itemize}

\vspace{1em}

\noindent Signed:

\vspace{1em}

\noindent Date:

\chapter*{Preface}

I begun working on automatic parallelism in Mercury as part of my Honours project
for the degree of Bachelor of Computer Science at the University of Melbourne.
This work occured before my PhD candidature commenced.
It included the profiler feedback framework that is discussed in Section
\ref{sec:background}.
It also includes an initial, rudimentary, version of the automatic
parallelisation tool
(Section \ref{sec:background}
including an algorithm for determining when a sub-computation produces or
consumes a variable (Section \ref{sec:var_use_time}).

Peter Wang and I worked together to implement work stealing for the Mercury
runtime system.
We estimate that Peter contributed 80\% of the work stealing implementation
and that I contributed 20\%.
\paul{I haven't written up this work, I must decide which chapter it belongs
in, maybe it should be part of the background}.

Chapter \ref{chap:overlap} is derived from a journal paper:

% XXX: not really a quote, I want this text to be typeset as a paragraph whose
% indent (not just first line) is larger than normal.

\begin{quote}
Paul Bone, Zoltan Somogyi and Peter Schachte.
Estimating the overlap between dependent computations for automatic
parallelization.
{\em Theory and Practice of Logic Programming}, 11(4--5):575--591, 2011.
\end{quote}

% Zoltan, Peter and I worked together to develop the overlap algorithm.
% \paul{The basis for this algorithm first appears in Zoltan's Louven slides}
% The implementation and testing is all my own work.
% Zoltan and Peter assisted me in writing and editing the paper.

Chapter \ref{chap:loop_control} is derived from a conference paper:

\begin{quote}
Paul Bone, Zoltan Somogyi and Peter Schachte.
Controlling Loops in Parallel Mercury Code.
{\em Declarative Aspects and Applications of Multicore Programming},
Philadelphia PA USA, Janurary 2012.
\end{quote}

% Zoltan and I developed the datastructures and algorithm.
% 80\% of the implementation is my own work.
% \paul{Zoltan worked on the code generator changes, but are they actually part
% of the research?}
% Zoltan and Peter assisted me in writing and editing the paper.

Chapter \ref{chap:tscope} is derived from a workshop paper:

\begin{quote}
Paul Bone and Zoltan Somogyi
Profiling parallel Mercury programs with ThreadScope.
{\em 21st Workshop on Logic-based methods in Programming Environments},
Lexington KY USA, July 2011.
\end{quote}

% Zoltan provided feedback about my ideas and contributed a number of ideas of
% his own.
% I am the sole author of the implementation.
% Zoltan assisted me in writing and editing the paper.

\chapter*{Acknowledgements}

\tableofcontents

\listoffigures

\listoftables

\chapter{Introduction}
\label{chap:intro}
%
% vim: ft=tex ts=4 sw=4 
%
\chapter{Introduction}
% XXX
%\pagenumbering{arabic}
%\setcounter{page}{1}
\label{chap:intro}

\status{This chapter is ready for its \textbf{second} review by Zoltan.}

\plan{Why multicore}
In 1965 Moore \citep{moore} predicted that the transistor 
density on processors would grow exponentially with time,
and the manufacturing cost would fall.
The smaller transistors are, the faster they can switch
(given adequate power),
and therefore manufacturers can ship faster processors.
The industry celebrated this trend,
calling it Moore's Law.
However as faster processors require more power,
they create more heat which must be dissipated.
Without novel power saving techniques
(such as \citet{intel-high-k}),
this limits increases of processors' clock speeds.

Around 2005 it became clear that significant improvements in performance
would not come from increased clock speeds but from multicore parallelism
\citep{free_lunch}.
Manufacturers now build processors with multiple processing cores,
which can be placed in the same package,
and usually on the same die.
Individual cores work separately, communicating through the memory
hierarchy.
% Multicore computing is only the latest form of multiprocessor computing.

Other methods of improving performance without increasing clock speed have
also been tried.

\begin{itemize}

\item
Modern processors perform super-scalar execution:
processor instructions form a pipeline,
with several instructions at different stages of execution at once,
and by adding extra circuitry, several instructions may be at the same stage
of execution.
However, we have just about reached the limits of what super-scalar
execution can offer.

\item
Manufacturers have also added Single Instruction Multiple Data (SIMD)
instructions to their processors;
this allows programmers to perform the same operation on multiple pieces of
data.
In practice however, SIMD is useful only in some specific circumstances.

\item
Multicore computing has the potential to be useful in many circumstances,
and does not appear to have limitations that cannot be overcome.
Cache coherency could be a limitation for processors with many cores.
However there are solutions to this such as directory based memory
coherency;
there are also research opportunities such as making compilers responsible
for cache management.

\end{itemize}

%There are a number of ways to achieve this including:
%Single Instruction Multiple Data (SIMD) instruction sets,
%and super-scalar execution such as pipelining and reordering.
%One of the most notable improvements is multicore execution.

\plan{SMP \& programming}
Multicore computing is a special case of multiprocessing.
Most multiprocessing systems are symmetric multiprocessing (SMP) systems.
An SMP system consists of several homogeneous processors and some memory
connected together.
Usually all processors are equally-distant from all memory location.
Most multicore systems are SMP systems;
they may have more than one CPU each with any number of cores.
Some multiprocessing systems use a non-uniform memory architecture (NUMA).
Usually this means that each processor has fast access to some local memory
and slower access to the other processors' memories.
%there may be multiple memory areas, some of which a processor cannot access
%or can access only with high latency and or low bandwidth.
%NUMA systems do not always support cache coherency.
%The benefit of NUMA is that it is easier to build large NUMA systems than
%large SMP systems.
%The drawback is that it is harder to program.
SMP systems are currently vastly more common, so programmers are usually
more interested in programming for them.
Therefore, in this dissertation we are only concerned with SMP systems.
Our approach will work with NUMA systems, but not optimally.

%\plan{We need parallelism}
%For a long time the software industry enjoyed the exponential growth of
%clock speeds.
%Programmers did not need to spend any effort in order to use the extra
%clock-speed of a newer model processor.
%Now that manufacturers add performance by adding parallel processing
%features rather than increasing clock speeds,
%there is extra pressure on the software industry.
%Multicore computing is one of the most powerful features,
%however it is arguably the most difficult to use feature;
%which is why we choose to concentrate our effort on making it easier to use.

To use a multicore system, or multiprocessing in general,
programmers must parallelise their software.
This is normally done by dividing the software into multiple threads of
execution
which execute in parallel with one another.
This is very difficult in imperative languages as the programmer is
responsible for coordinating the threads \citep{sutter:2005:concurrency}.
Few programmers have the skills necessary to accomplish this,
and those that do, still make expensive mistakes as
threaded programming is inherently error prone. 
Bugs such as data corruption, deadlocks and race conditions
can be extremely tedious to find and fix.
These bugs increase the costs of software development.
Software companies who want their software to out-perform their competitors
will usually take on the costs of multicore programming.
We will explain the problems with parallelism in imperative languages in
Section~\ref{sec:intro_concurrency}.

In contrast to imperative languages,
it is trivial to express parallelism in pure declarative languages.
Expressing this parallelism creates two strongly-related problems.
First,
one must overcome the costs of parallel execution.
For example,
it may take hundreds of instructions to make a task available for execution on
another processor.
However, if that task only takes a few instructions to execute,
then there is no benefit to executing it in parallel.
Even if the task creates hundreds of instructions to execute,
parallel execution is probably not worthwhile.
Most easy-to-exploit parallelism is \emph{fine grained} such as this.
Second,
an abundance of coarse grained parallelism can also be a problem.
Whilst the amount of parallelism the machine can exploit
cannot increase beyond the number of processors,
the more parallel tasks a program creates,
the more the overheads of parallel execution will have an effect on
performance.
In these situations,
there is no benefit in parallelising many of the tasks,
and yet the overheads of their parallel executions will still have an
effect.
This often cripples the performance of such programs.
%For example a ray-tracer that creates an image
%1,000$\times$1,000 pixels in size has 1,000,000 independent computations
%available for parallelisation.
%Parallelising all of these on a four processor machine creates too
%many overheads, more than would be necessary to parallelise this program
%optimally.
This is known as an \emph{embarrassingly parallel} workload.
Programs with either of these problems almost always perform more
\emph{slowly} than their sequential equivalents.
Programmers must therefore find the parts of their program where parallel execution
is profitable and parallelise those parts of their program \emph{only},
whilst being careful to avoid embarrassing parallelism.
This means that a programmer must have a strong understanding of their
program's  computations' costs,
how much parallelism they are making available,
and how many processors may be available at any point in the program's
execution.
Programmers are not good at identifying the hotspots in their programs
or in many cases understanding the costs of computations,
consequently programmers are not good at manually parallelising programs.
Programmers are encouraged to use profilers to help them identify the
hotspots in their programs and speed them up;
this also applies to parallelisation.

Automatic parallelism aims to make it easy to introduce parallelism.
Software companies will not need to spend as much effort on parallelising
their software.
% Programmers will not have to painstakingly find and fix race conditions.
%When programmers are using parallelism with pure declarative languages they
%will not need to tediously introduce parallelism.
Better yet,
it will be easier for programmers to take advantage of the extra cores on
newer processors.
Furthermore, 
as a parallel program is modified its performance characteristics will
change,
and some changes may affect the benefit of the parallelism that has already
been exploited.
In these cases automatic parallelism will make it easy to
\emph{re-parallelise} the program,
saving programmers a lot of time.


\section{Parallelism in Programming Languages}
\label{sec:literature_review}

\status{Mostly notes}

%\paul{
%This should just 'set the stage' for my work.  I would
%like to leave the contrasting discussions until later in the
%thesis, perhaps at the end of each major chapter.
%}
%
%\paul{
%I do not want to introduce all my citations here (the way Liz has to
%due to the APA-formatting standard).  Only enough that the user
%understand the context for reading the research chapters.  So,
%closely related work should be introduced for the first time when
%it is compared with my work.
%}

\paul{I probably need to come back to this and add more references.}

\paul{I grouped work into the following categories (subsection headings),
Categories are presented in roughly the order in which they solve the
multicore programming problem.}

% Parallelism and concurrency.
It is important to distinguish between parallelism and concurrency.
Concurrency is defined as multiple computations executing simultaneously,
and may communicate with one another.
On a single processor concurrency can be achieved by switching between
alternative processes quickly.
Parallelism occurs when multiple computations \textbf{actually} execute in parallel,
speeding up a computation.
Often parallelism is achieved through concurrency,
for example, multiple concurrent processes execution on a multicore processor,
this leads to a confusion between concurrency and parallelism.
Concurrency can be extremely useful when it matches the programmer's intentions.
For example,
when programming a web server that may service multiple clients at once;
it is natural to model each client's connection as a separate concurrent process.
Not all programs can be easily modeled with concurrent tasks,
in these cases it is best to introduce parallelism without concurrency.
Therefore, distinguishing these concepts is important.

\subsection{Concurrency}
\label{sec:backgnd_concurrency}

% Threading.
There are numerous imperative languages that support parallelism through
concurrency.
Those that do not support concurrency nativly can add support
with the use of a library.
For example, 
C and C++ use the POSIX Threads~\citep{butenhof1997:pthreads} or Windows
Threads~\citep{winthreads} libraries that are provided with operating systems.
These libraries tend to be low level;
they simply provide an interface to the operating system.
These examples use the \emph{threading} model of concurrency:
Different threads of execution run independently,
sharing a heap and static data.
Threads may read and write the same data,
however,
concurrent read and write, and write and write operations are unsafe.
Code that uses shared data is known as a critical section,
Only one thread may execute a critical section for a given piece of shared data
at once,
this is known as \emph{mutual exclusion}~\citep{Dijkstra:Mutex}.
Critical sections therefore need to be protected by locks.
Upon entering a critical section a thread must acquire the lock associated with
the data,
it must release the lock when it leaves the critical section and the lock
ensures that only one thread may hold the lock at a time.
The programmer is responsible for defining critical sections and using locks
correctly.

Many other languages and libraries support the threading model of concurrency.
Many such languages support threading directly, buy building support into the
language or its standard library such as Java~\citep{java-threads}.

Mutual exclusion locking is problematic,
it is too easy to make mistakes when using locks.
Frequent mistakes include: forgetting to synchronise access to a critical
section,
making the critical section too small, or too large,
these can lead to incorrect program behaviour, inconsistent memory states and
crashes. 
When using multiple locks in a single critical section,
because that critical section uses multiple resources,
the locks must be acquired in the same order in each such critical section
otherwise deadlocks can occur.
This also prevents critical sections from being nestable.
All of these problems are difficult to debug because they are intermittent.
Often so intermittent that introducing tracing code or compiling with debugging
support can prevent the problem from occurring,
this is humorously known as a \emph{heisenbug} ---
a testament to just how frustrating it can be to find.

% Message passing.
Alternative models of concurrency exist,
another popular model is \emph{message passing}.
Notable examples of message passing libraries are MPI\citep{mpi} and
PVM~\citep{pvm}.
Unlike threads, message passing's \emph{processes} do not share static or heap data.
They communicate by sending messages to one another.
In MPI and PVM the programmer is responsible for encoding and decoding messages,
which can be tedious.
Languages such as Erlang~\citep{erlang} support message passing natively and
encoding and decoding is done for the programmer.
The major benefit of message passing is that it does not make assumptions about where processes
are executing:
they can be running all on one machine or spread-out across a network.
Therefore, message passing is very popular for high performance computing,
where shared memory systems are not feasible.
Message passing also avoids the problems with threading described above.
However,
it is often slower on shared memory systems since the same memory cannot easily
be referenced.
There are exceptions to this,
Go~\citep{balbaert:2012:go} for example uses message passing but also allows
access to shared memory,
and therefore has all the problems of threading.
Message passing does not avoid the problem of deadlocks,
Consider two processes,
they both attempt to send a message to their counterpart but only after
receiving such a message;
neither can proceed and therefore a deadlock occurs.
Most deadlocks are much more subtle and harder to find and fix.

% STM
A better way to avoid all these problems in concurrent programing is to use
software transactional memory (STM)~%
\citep{harris:2005:haskell-stm,mika:mercury-stm}.
STM allows programmers to mark different variables as transactional
and to define critical sections.
Any changes to a transactional variable made during a critical section must
be sequentially consistent with any other operations and critical
sections executed by other threads.
If an in consistency is detected one or more critical sections must
\emph{roll back} --- undoing any changes made to transactional
variables.
It is important that a transaction has no side-effects, otherwise
consistency may not be guaranteed.


\subsection{Explicit Parallelism}
\label{sec:back_par_explicit}

% Parallelism w/o concurrency in imperative languages.
Parallelism can also be achieved without concurrency,
a popular language and library for this is OpenMP~\citep{openmp},
which allows programmers to annotate parts of their programs for parallel
execution without describing \textbf{how} this parallelism should be achieved;
making it much easier for programmers to use.
We will refer to this as \emph{explicit parallelism}
since parallelism is expressed directly by the programmer by the explicit
annotations.
In OpenMP most parallelism is achieved by annotating loops.
However,
the programmer is responsible for guaranteeing that the iterations of the loops
are independent,
that is to say, that no iteration depends on the results of affects of any of
the previous iterations.
If an iteration makes a function call, then this guarantee must also be true for
the callee all transitive callees.
Explicit parallelism otherwise avoids the problems with threading and message
passing.

% Declarative languages.
One of the easiest ways to make explicit parallelism easier is to prevent
side-effects from occurring, usually by designing them out of the language.
Pure declarative languages like Mercury~\citep{mercury_jlp},
Haskell~\citep{haskell98} and Clean~\citep{1991:concurrent-clean} do not
allow side-effects.
\paul{I should be able to find a more useful citation for Clean.}
This is done by ensuring that a function's declaration declares all of the
functions effects.
Therefore,
the programmer or the compiler can read a function's declaration to see if it
is safe to parallelise.
Mercury and Haskell both support explicit parallelism,
the programmer need only annotate their code, indicating that it should be
executed in parallel.

Glasgow Parallel Haskell (GpH) allows programmers to request parallel
evaluation of certain expressions by annotating them with a function
whose operational semantics cause parallel
evaluation~\citep{gph,loidi:2008:gph-semiexplicit-parallelism}.
Unfortunately Haskell's lazy evaluation strategy interacts poorly with
parallelism.
A parallelised computation will be evaluated to weak-head-normal-form,
that is to say that if it is a data term with parameters,
the parameters will not be evaluated.
This means that any parallelised work usually does not do enough work to
make parallelisation worthwhile.
The GpH implementors have tried to solve this problem by allowing
the programmer to express how deep evaluation should
continue~\citep{trinder:98:strategies}.
An alternative Haskell library called Monad.Par~\citep{marlow:monadpar}
fully evaluates the result of any spawned-off computation.
However, the programmer must still describe how any of their own
datatypes can be fully evaluated.
These solutions do not really solve the problem,
they still require the programmer to introduce strictness.
\paul{I can add support to my argument by describing problems with
space leaks, for example foldl, foldr, and foldl'.
I want to avoid a rant or a religious war, but I do want to
objectively assess the literature.}
It is better to avoid these and many other problems completely by using
a strict language.

\label{ref:parallel_conjunction}
Mercury allows programmers to request parallel evaluation of
conjunctions by replacing the normal conjunction operator with the
parallel conjunction operator introduced by Thomas Conway in
\citep{conway:2002:par,wang:2006:hons,wang:2011:dep-par}.
Since Mercury is a strict language it does not have Haskell's
lazyness problems,
programmers simply annotate where parallel execution
should be used.
A more detailed description of parallelism in Mercury can be found in
section \ref{sec:backgnd_merpar}

When a problem is not naturally a concurrent problem 
explicit parallelism such as in Mercury and Haskell
is preferable as the programmer does not need to force their program
into a concurrent model.
Especially since in all but the STM cases of concurrency programmers
must also describe how parallel computations communicate and
synchronise.

\paul{I would like a citation here, I think I found something a while ago
about profiling}
However, explicit parallelism is a drawback because it requires the
programmer to know where their program spends most of its execution
time, it is understood that most programmers are poor at this.
Parallel execution has additional overheads such as:
spawning parallel tasks,
cleaning up completed parallel tasks,
operating system scheduling and
hardware behaviour such as cache effects.
The speedup gained when parallelising a computation will depend upon
these costs.
Programmers must therefore know whether parallelising a particular
computation is going to be an improvement in spite of the additional
costs of parallel execution.
Furthermore programmers must know whether there will be enough
processors free at runtime to execute the parallelised computation:
the additional costs of parallel execution will have an effect even
if there is not a processor available to execute the parallel work.
Parallelisation is another optimisation such as inlining or efficient
register allocation.
Therefore,
it would be better for an optimising compiler to handle parallelisation
automatically;
the programmer will not need to worry about parallel evaluation any more
than they currently worry about inlining and register allocation.

\subsection{Implicit Parallelism}
\label{sec:lit_implicit-parallelism}

Some computer languages support implicit parallelism,
in these languages many parts of programs are executed in parallel.
Parallel execution is the normal mode of execution,
it is used in most places within the program.

% Implicitly parallel prologs. (and OR-parallelism)
A number of parallel Prolog-like languages that were developed during the
1980's are classified as implicitly parallel languages.
These included Concurrent
Prolog~\citep{saraswat85:probl_with_concur_prolog,saraswat86:concurrent_prolog_definition,shapiro:flat_concur_prolog},
Parlog~\citep{clark:84:parlog_sys_prog,clark:86:parlog} and GHC~\citep{ueda:ghc}.
Nearly all tasks in these languages were carried out in
parallel,
as a result the overheads of parallel execution are typically
greater than the benefit of running most small tasks in parallel.
Furthermore implicitly parallel programs have an \emph{embarrassingly
  parallel} workload,
this occurs when much more parallel work is available than the parallel
processing capacity of the machine;
thereby dramatically reducing the benefit of parallel execution while
the cost remained the same.
Both these effects often caused very poor performance.

% Granularity control
Granularity control was introduced in order to solve these
problems~\citep{lopez96:distance_granularity,shen_98_granularity-control}.
It attempts to reduce the amount of work being executed in parallel.
There are a number of different methods, some incur a
runtime cost in order to determine if there is already ample parallel
work available while other static methods do not.
All methods help improve the performance of parallel programs and are
quite valuable, especially in recursive procedures.
\paul{I need to point out cases where GC does not help or is not good
enough,
I am going to have to find this in the literature and come back to this
paragraph and possibly the previous one.}

Some languages allow for the parallelisation of data parallel
tasks such as NESL~\citep{blelloch:95:nesl} and Data Parallel
Haskell (DpH)~\citep{dph:2007:status_report,dph:2008:harnessing_the_multicores}.
These languages use special datatypes to denote parallelism,
sequences and parallel arrays in NESL and DpH respectively.
Only operations on elements of these collections are parallelised.
We have none-the-less classified these systems as implicitly parallel,
since every action on these data types is executed in parallel.
Because operations are data-parallel they are independent and suitable
\paul{Define SMP in the introduction}
for execution on vector machines as well as SMP machines.
By transforming code and arranging for one thread to work on more than
one data item at a time granularity can be improved.
The drawback of these data-parallel approaches is that they can
parallelise data parallel programs
--- only a small subset of computer programs.

With the exception of DpH, implicit parallelism often performs worse
than explicit parallelism.
It is understood that carefully adding a few explicit parallelism annotations
to a program with the aid of a profiler will produce a faster-running
program than implicitly parallelising most independent computations.

\subsection{Automatic Parallelisation}
\label{sec:lit_automatic-parallelisation}

% Look at automatic parallelisation in Haskell.
\citet{harris_07_feedback_imp_par} developed a profiler
feedback directed automatic parallelisation approach for Haskell programs.
They have reported speed ups of up to 80\% compared to the sequential
execution of their test programs on a four core machine.
However they were not able to improve the performance of some
programs, they attributed this to a lack of parallelism
available in these programs.
They have shown that automatic parallelisation is a promising idea for
improving the performance of software.

We believe that more can be done to improve the effectiveness of
automatic parallelisation,
In some cases it may be possible to transform common programming
pasterns that lack parallelism into equivalent patterns with available
parallelism.
Furthermore, an advanced profiler --- such as Mercury's deep
profiler~\citep{conway:2001:mercury-deep} --- can provide information
that enables a compiler to make good parallelisation choices.
However there are a number of challenges facing automatic
parallelisation.
When using profiler feedback the profiled execution of the program may
not be a typical execution of the program, or there may be several
typical executions of the program.
We cannot control whether users profile typical executions of their
programs, but we may be able to allow users to merge execution
profiles to create a composite profile that is more representative of
their program's usage.
Another challenge can occur when a program has very little parallelism
available in it, it may be difficult to parallelise effectively.
We hope that in some cases the compiler can transform such a program
into an equivalent program with more available parallelism.

% Jerome's work.
\citet{tannier:2007:parallel_mercury} previously attempted to automatically
parallelise Mercury programs using profiler feedback
information to automatically parallelise a program.
\citet{tannier:2007:parallel_mercury} approach selects the most expensive predicates
of a program and attempts to parallelise conjunctions within them.
Tannier also makes use of compile-time granularity
control to reduce the over-parallelisation that can occur in recursive
code.
Unfortunately, he estimated the costs and benefits of parallelising
dependant conjunctions based on the number of dependant variables that
they shared.
In practice most producers produce dependant variables late in their
execution and most consumers consume them early.
Therefore Tannier's calculation is na\"ive: the time that these
variables are produced by one conjunct and consumed by the other may
not correlate with the number of dependant variables.
We believe that Tannier's algorithm is, in general, too optimistic
about the parallelism available in dependant conjunctions.

% My honours thesis.
\citet{bone:2008:hons} improved on this approach by using
information from a modification of Mercury's deep profiler to
calculate when the producing conjunct is most likely to produce the
dependant values and when the consuming conjunct is likely to need
them.
This information can be used to estimate the parallel speedup of
dependant conjunctions.
The effectiveness of this approach is not yet clear.

Mercury's deep profiler~\citep{conway:2001:mercury-deep} provides
detailed and accurate profiling information,
among other things the deep profiler records separate profiling
information for separate uses of the same code.
This will make it easier to implement optimisations such as
parallel specialisation --- generating sequential and parallel
versions of one procedure and using the sequential version
in situations where parallelism is not an optimisation.
Extracting information from the deep profiler to guide compiler
optimisations is supported by the feedback framework developed by
\citet{bone:2008:hons}.
These are examples of the flexibility that the deep profiler provides,
describing other ideas is outside the context of a literature review.
No equivalent profiler exists for Haskell or Clean, making Mercury an
important choice for our implementation.

There is another challenge with automatic parallelism: a lot of
information about the execution of a program will not be recorded by the
profiler, often recording information in infeasible.
In these cases we must be careful to make safe, conservative
assumptions when calculating estimates of this information.

% Write about haskell's call centre stacks, maybe they provide enough
% information to perform similar optimisations.

% How does clean compare?

We expect that automatic parallelisation will more easily and
effectively parallelise declarative programs.
Furthermore, it will be easier to maintain such programs, as
characteristics of the program that are used to explicitly parallelise
a program will not necessarily be true in future versions or uses of that
program.
Automatic parallelisation allows the programmer to re-parallelise
their program quickly, based on a current execution profile of the
program.



\section{General approach}
\label{sec:intro_general_approach}

Unfortunately automatic parallelisation technology is yet to be developed to the
point where it is generally useable.
Our aim is that automatic parallelisation will be easy to use and
will parallelise programs more effectively that most programmers can by
hand.
Most significantly,
automatic parallelism will be very simple to use compared with the
difficulty of manual parallelisation.
Furthermore as programs change,
costs of computations within them will change,
and this may make manual parallelisations (using explicit parallelism) less
effective.
An automatic parallelisation system will therefore make it easier to
maintain programs as the automatic parallelisation analysis can simply be
redone to re-parallelise the programs.
We are looking forward to a future where programmers think about
parallelism no more than they currently think about traditional compiler
optimisations.

In this dissertation we have solved several of the critical issues with
automatic parallelism.
Our work is targeted towards Mercury.
We choose to use Mercury because
it already supports explicit parallelism of dependent conjunctions,
and it provides powerful profiling tools which generate data for our profile
feedback analyses.
Mercury's support for parallel execution and the previous
auto-parallelisation system \citep{bone:2008:hons} is described in
Chapter~\ref{chap:backgnd}.
In this dissertation we make a number of improvements to Mercury's runtime
system that improve the performance of parallel Mercury programs
(Chapter~\ref{chap:rts}).
In Chapter~\ref{chap:overlap} we describe our
automatic parallelism analysis tool and its algorithms,
and show how it can speedup several programs.
In Chapter~\ref{chap:loop_control} we introduce a new transformation that
improves the performance of 
some types of recursive code and achieve almost perfect linear speedups on
several benchmarks.
The transformation also allows recursive code within parallel conjunctions
to take advantage of tail recursion optimisation.
Chapter~\ref{chap:tscope} describes a proposal to add support for Mercury to
the \tscope parallel profile visualisation tool.
We expect that the proposed features will very useful for programmers and
researchers alike.
Finally in Chapter~\ref{chap:conc} we conclude the dissertation,
tieing together the various contributions.
We believe that our work could also be adapted for other systems;
this will be easier in similar languages and more difficult in less similar
languages.



\section{Parallelism in programming languages}
\label{sec:literature_review}

\section{Parallelism in Programming Languages}
\label{sec:literature_review}

\status{Mostly notes}

%\paul{
%This should just 'set the stage' for my work.  I would
%like to leave the contrasting discussions until later in the
%thesis, perhaps at the end of each major chapter.
%}
%
%\paul{
%I do not want to introduce all my citations here (the way Liz has to
%due to the APA-formatting standard).  Only enough that the user
%understand the context for reading the research chapters.  So,
%closely related work should be introduced for the first time when
%it is compared with my work.
%}

\paul{I probably need to come back to this and add more references.}

\paul{I grouped work into the following categories (subsection headings),
Categories are presented in roughly the order in which they solve the
multicore programming problem.}

% Parallelism and concurrency.
It is important to distinguish between parallelism and concurrency.
Concurrency is defined as multiple computations executing simultaneously,
and may communicate with one another.
On a single processor concurrency can be achieved by switching between
alternative processes quickly.
Parallelism occurs when multiple computations \textbf{actually} execute in parallel,
speeding up a computation.
Often parallelism is achieved through concurrency,
for example, multiple concurrent processes execution on a multicore processor,
this leads to a confusion between concurrency and parallelism.
Concurrency can be extremely useful when it matches the programmer's intentions.
For example,
when programming a web server that may service multiple clients at once;
it is natural to model each client's connection as a separate concurrent process.
Not all programs can be easily modeled with concurrent tasks,
in these cases it is best to introduce parallelism without concurrency.
Therefore, distinguishing these concepts is important.

\subsection{Concurrency}
\label{sec:backgnd_concurrency}

% Threading.
There are numerous imperative languages that support parallelism through
concurrency.
Those that do not support concurrency nativly can add support
with the use of a library.
For example, 
C and C++ use the POSIX Threads~\citep{butenhof1997:pthreads} or Windows
Threads~\citep{winthreads} libraries that are provided with operating systems.
These libraries tend to be low level;
they simply provide an interface to the operating system.
These examples use the \emph{threading} model of concurrency:
Different threads of execution run independently,
sharing a heap and static data.
Threads may read and write the same data,
however,
concurrent read and write, and write and write operations are unsafe.
Code that uses shared data is known as a critical section,
Only one thread may execute a critical section for a given piece of shared data
at once,
this is known as \emph{mutual exclusion}~\citep{Dijkstra:Mutex}.
Critical sections therefore need to be protected by locks.
Upon entering a critical section a thread must acquire the lock associated with
the data,
it must release the lock when it leaves the critical section and the lock
ensures that only one thread may hold the lock at a time.
The programmer is responsible for defining critical sections and using locks
correctly.

Many other languages and libraries support the threading model of concurrency.
Many such languages support threading directly, buy building support into the
language or its standard library such as Java~\citep{java-threads}.

Mutual exclusion locking is problematic,
it is too easy to make mistakes when using locks.
Frequent mistakes include: forgetting to synchronise access to a critical
section,
making the critical section too small, or too large,
these can lead to incorrect program behaviour, inconsistent memory states and
crashes. 
When using multiple locks in a single critical section,
because that critical section uses multiple resources,
the locks must be acquired in the same order in each such critical section
otherwise deadlocks can occur.
This also prevents critical sections from being nestable.
All of these problems are difficult to debug because they are intermittent.
Often so intermittent that introducing tracing code or compiling with debugging
support can prevent the problem from occurring,
this is humorously known as a \emph{heisenbug} ---
a testament to just how frustrating it can be to find.

% Message passing.
Alternative models of concurrency exist,
another popular model is \emph{message passing}.
Notable examples of message passing libraries are MPI\citep{mpi} and
PVM~\citep{pvm}.
Unlike threads, message passing's \emph{processes} do not share static or heap data.
They communicate by sending messages to one another.
In MPI and PVM the programmer is responsible for encoding and decoding messages,
which can be tedious.
Languages such as Erlang~\citep{erlang} support message passing natively and
encoding and decoding is done for the programmer.
The major benefit of message passing is that it does not make assumptions about where processes
are executing:
they can be running all on one machine or spread-out across a network.
Therefore, message passing is very popular for high performance computing,
where shared memory systems are not feasible.
Message passing also avoids the problems with threading described above.
However,
it is often slower on shared memory systems since the same memory cannot easily
be referenced.
There are exceptions to this,
Go~\citep{balbaert:2012:go} for example uses message passing but also allows
access to shared memory,
and therefore has all the problems of threading.
Message passing does not avoid the problem of deadlocks,
Consider two processes,
they both attempt to send a message to their counterpart but only after
receiving such a message;
neither can proceed and therefore a deadlock occurs.
Most deadlocks are much more subtle and harder to find and fix.

% STM
A better way to avoid all these problems in concurrent programing is to use
software transactional memory (STM)~%
\citep{harris:2005:haskell-stm,mika:mercury-stm}.
STM allows programmers to mark different variables as transactional
and to define critical sections.
Any changes to a transactional variable made during a critical section must
be sequentially consistent with any other operations and critical
sections executed by other threads.
If an in consistency is detected one or more critical sections must
\emph{roll back} --- undoing any changes made to transactional
variables.
It is important that a transaction has no side-effects, otherwise
consistency may not be guaranteed.


\subsection{Explicit Parallelism}
\label{sec:back_par_explicit}

% Parallelism w/o concurrency in imperative languages.
Parallelism can also be achieved without concurrency,
a popular language and library for this is OpenMP~\citep{openmp},
which allows programmers to annotate parts of their programs for parallel
execution without describing \textbf{how} this parallelism should be achieved;
making it much easier for programmers to use.
We will refer to this as \emph{explicit parallelism}
since parallelism is expressed directly by the programmer by the explicit
annotations.
In OpenMP most parallelism is achieved by annotating loops.
However,
the programmer is responsible for guaranteeing that the iterations of the loops
are independent,
that is to say, that no iteration depends on the results of affects of any of
the previous iterations.
If an iteration makes a function call, then this guarantee must also be true for
the callee all transitive callees.
Explicit parallelism otherwise avoids the problems with threading and message
passing.

% Declarative languages.
One of the easiest ways to make explicit parallelism easier is to prevent
side-effects from occurring, usually by designing them out of the language.
Pure declarative languages like Mercury~\citep{mercury_jlp},
Haskell~\citep{haskell98} and Clean~\citep{1991:concurrent-clean} do not
allow side-effects.
\paul{I should be able to find a more useful citation for Clean.}
This is done by ensuring that a function's declaration declares all of the
functions effects.
Therefore,
the programmer or the compiler can read a function's declaration to see if it
is safe to parallelise.
Mercury and Haskell both support explicit parallelism,
the programmer need only annotate their code, indicating that it should be
executed in parallel.

Glasgow Parallel Haskell (GpH) allows programmers to request parallel
evaluation of certain expressions by annotating them with a function
whose operational semantics cause parallel
evaluation~\citep{gph,loidi:2008:gph-semiexplicit-parallelism}.
Unfortunately Haskell's lazy evaluation strategy interacts poorly with
parallelism.
A parallelised computation will be evaluated to weak-head-normal-form,
that is to say that if it is a data term with parameters,
the parameters will not be evaluated.
This means that any parallelised work usually does not do enough work to
make parallelisation worthwhile.
The GpH implementors have tried to solve this problem by allowing
the programmer to express how deep evaluation should
continue~\citep{trinder:98:strategies}.
An alternative Haskell library called Monad.Par~\citep{marlow:monadpar}
fully evaluates the result of any spawned-off computation.
However, the programmer must still describe how any of their own
datatypes can be fully evaluated.
These solutions do not really solve the problem,
they still require the programmer to introduce strictness.
\paul{I can add support to my argument by describing problems with
space leaks, for example foldl, foldr, and foldl'.
I want to avoid a rant or a religious war, but I do want to
objectively assess the literature.}
It is better to avoid these and many other problems completely by using
a strict language.

\label{ref:parallel_conjunction}
Mercury allows programmers to request parallel evaluation of
conjunctions by replacing the normal conjunction operator with the
parallel conjunction operator introduced by Thomas Conway in
\citep{conway:2002:par,wang:2006:hons,wang:2011:dep-par}.
Since Mercury is a strict language it does not have Haskell's
lazyness problems,
programmers simply annotate where parallel execution
should be used.
A more detailed description of parallelism in Mercury can be found in
section \ref{sec:backgnd_merpar}

When a problem is not naturally a concurrent problem 
explicit parallelism such as in Mercury and Haskell
is preferable as the programmer does not need to force their program
into a concurrent model.
Especially since in all but the STM cases of concurrency programmers
must also describe how parallel computations communicate and
synchronise.

\paul{I would like a citation here, I think I found something a while ago
about profiling}
However, explicit parallelism is a drawback because it requires the
programmer to know where their program spends most of its execution
time, it is understood that most programmers are poor at this.
Parallel execution has additional overheads such as:
spawning parallel tasks,
cleaning up completed parallel tasks,
operating system scheduling and
hardware behaviour such as cache effects.
The speedup gained when parallelising a computation will depend upon
these costs.
Programmers must therefore know whether parallelising a particular
computation is going to be an improvement in spite of the additional
costs of parallel execution.
Furthermore programmers must know whether there will be enough
processors free at runtime to execute the parallelised computation:
the additional costs of parallel execution will have an effect even
if there is not a processor available to execute the parallel work.
Parallelisation is another optimisation such as inlining or efficient
register allocation.
Therefore,
it would be better for an optimising compiler to handle parallelisation
automatically;
the programmer will not need to worry about parallel evaluation any more
than they currently worry about inlining and register allocation.

\subsection{Implicit Parallelism}
\label{sec:lit_implicit-parallelism}

Some computer languages support implicit parallelism,
in these languages many parts of programs are executed in parallel.
Parallel execution is the normal mode of execution,
it is used in most places within the program.

% Implicitly parallel prologs. (and OR-parallelism)
A number of parallel Prolog-like languages that were developed during the
1980's are classified as implicitly parallel languages.
These included Concurrent
Prolog~\citep{saraswat85:probl_with_concur_prolog,saraswat86:concurrent_prolog_definition,shapiro:flat_concur_prolog},
Parlog~\citep{clark:84:parlog_sys_prog,clark:86:parlog} and GHC~\citep{ueda:ghc}.
Nearly all tasks in these languages were carried out in
parallel,
as a result the overheads of parallel execution are typically
greater than the benefit of running most small tasks in parallel.
Furthermore implicitly parallel programs have an \emph{embarrassingly
  parallel} workload,
this occurs when much more parallel work is available than the parallel
processing capacity of the machine;
thereby dramatically reducing the benefit of parallel execution while
the cost remained the same.
Both these effects often caused very poor performance.

% Granularity control
Granularity control was introduced in order to solve these
problems~\citep{lopez96:distance_granularity,shen_98_granularity-control}.
It attempts to reduce the amount of work being executed in parallel.
There are a number of different methods, some incur a
runtime cost in order to determine if there is already ample parallel
work available while other static methods do not.
All methods help improve the performance of parallel programs and are
quite valuable, especially in recursive procedures.
\paul{I need to point out cases where GC does not help or is not good
enough,
I am going to have to find this in the literature and come back to this
paragraph and possibly the previous one.}

Some languages allow for the parallelisation of data parallel
tasks such as NESL~\citep{blelloch:95:nesl} and Data Parallel
Haskell (DpH)~\citep{dph:2007:status_report,dph:2008:harnessing_the_multicores}.
These languages use special datatypes to denote parallelism,
sequences and parallel arrays in NESL and DpH respectively.
Only operations on elements of these collections are parallelised.
We have none-the-less classified these systems as implicitly parallel,
since every action on these data types is executed in parallel.
Because operations are data-parallel they are independent and suitable
\paul{Define SMP in the introduction}
for execution on vector machines as well as SMP machines.
By transforming code and arranging for one thread to work on more than
one data item at a time granularity can be improved.
The drawback of these data-parallel approaches is that they can
parallelise data parallel programs
--- only a small subset of computer programs.

With the exception of DpH, implicit parallelism often performs worse
than explicit parallelism.
It is understood that carefully adding a few explicit parallelism annotations
to a program with the aid of a profiler will produce a faster-running
program than implicitly parallelising most independent computations.

\subsection{Automatic Parallelisation}
\label{sec:lit_automatic-parallelisation}

% Look at automatic parallelisation in Haskell.
\citet{harris_07_feedback_imp_par} developed a profiler
feedback directed automatic parallelisation approach for Haskell programs.
They have reported speed ups of up to 80\% compared to the sequential
execution of their test programs on a four core machine.
However they were not able to improve the performance of some
programs, they attributed this to a lack of parallelism
available in these programs.
They have shown that automatic parallelisation is a promising idea for
improving the performance of software.

We believe that more can be done to improve the effectiveness of
automatic parallelisation,
In some cases it may be possible to transform common programming
pasterns that lack parallelism into equivalent patterns with available
parallelism.
Furthermore, an advanced profiler --- such as Mercury's deep
profiler~\citep{conway:2001:mercury-deep} --- can provide information
that enables a compiler to make good parallelisation choices.
However there are a number of challenges facing automatic
parallelisation.
When using profiler feedback the profiled execution of the program may
not be a typical execution of the program, or there may be several
typical executions of the program.
We cannot control whether users profile typical executions of their
programs, but we may be able to allow users to merge execution
profiles to create a composite profile that is more representative of
their program's usage.
Another challenge can occur when a program has very little parallelism
available in it, it may be difficult to parallelise effectively.
We hope that in some cases the compiler can transform such a program
into an equivalent program with more available parallelism.

% Jerome's work.
\citet{tannier:2007:parallel_mercury} previously attempted to automatically
parallelise Mercury programs using profiler feedback
information to automatically parallelise a program.
\citet{tannier:2007:parallel_mercury} approach selects the most expensive predicates
of a program and attempts to parallelise conjunctions within them.
Tannier also makes use of compile-time granularity
control to reduce the over-parallelisation that can occur in recursive
code.
Unfortunately, he estimated the costs and benefits of parallelising
dependant conjunctions based on the number of dependant variables that
they shared.
In practice most producers produce dependant variables late in their
execution and most consumers consume them early.
Therefore Tannier's calculation is na\"ive: the time that these
variables are produced by one conjunct and consumed by the other may
not correlate with the number of dependant variables.
We believe that Tannier's algorithm is, in general, too optimistic
about the parallelism available in dependant conjunctions.

% My honours thesis.
\citet{bone:2008:hons} improved on this approach by using
information from a modification of Mercury's deep profiler to
calculate when the producing conjunct is most likely to produce the
dependant values and when the consuming conjunct is likely to need
them.
This information can be used to estimate the parallel speedup of
dependant conjunctions.
The effectiveness of this approach is not yet clear.

Mercury's deep profiler~\citep{conway:2001:mercury-deep} provides
detailed and accurate profiling information,
among other things the deep profiler records separate profiling
information for separate uses of the same code.
This will make it easier to implement optimisations such as
parallel specialisation --- generating sequential and parallel
versions of one procedure and using the sequential version
in situations where parallelism is not an optimisation.
Extracting information from the deep profiler to guide compiler
optimisations is supported by the feedback framework developed by
\citet{bone:2008:hons}.
These are examples of the flexibility that the deep profiler provides,
describing other ideas is outside the context of a literature review.
No equivalent profiler exists for Haskell or Clean, making Mercury an
important choice for our implementation.

There is another challenge with automatic parallelism: a lot of
information about the execution of a program will not be recorded by the
profiler, often recording information in infeasible.
In these cases we must be careful to make safe, conservative
assumptions when calculating estimates of this information.

% Write about haskell's call centre stacks, maybe they provide enough
% information to perform similar optimisations.

% How does clean compare?

We expect that automatic parallelisation will more easily and
effectively parallelise declarative programs.
Furthermore, it will be easier to maintain such programs, as
characteristics of the program that are used to explicitly parallelise
a program will not necessarily be true in future versions or uses of that
program.
Automatic parallelisation allows the programmer to re-parallelise
their program quickly, based on a current execution profile of the
program.



\chapter{Background}

\section{Mercury}
\label{sec:back_mercury}

\status{This section is complete.
}

Mercury is a pure logic/functional programming language
intended for the creation of large, fast, reliable programs.
While the syntax of Mercury is based on the syntax of Prolog,
semantically the two languages are very different
due to Mercury's purity and its type, mode, determinism and module systems.

Mercury programs consist of modules,
each has a seperate namespace and sperate compilation is used making it easy
to create large programs.
Each module contains predicates and functions.
Functions are a syntactic sugar for predicates with an extra (result)
argument.
The intention is that
predicates define relationships between their arguments and
functions define a mapping from their arguments to their result.
in the remainder of this dissertation we will use the word predicate to refer to
either a predicate or a function.

A predicate or function $P$ is defined in terms of a goal $G$:

%\vspace{-1\baselineskip}
% \begin{figure}[htb]
$$
\begin{array}{lll}
~P
    & :~ p(X_1, \ldots, X_n)~\leftarrow~G
        & \hbox{predicates} \\
    & |~ f(X_1, \ldots, X_n)=X_{n+1}~\leftarrow~G
        & \hbox{functions} \\
\end{array}
$$

\paul{Peter suggested using a more grammer-like description of the language.
I think that his motivation was that terminals and non-terminals should look
more distinct.
I've capitalized the variable names since that's closer to mercury syntax,
it's not the same as showing terminals and non-terminals differently but
perhaps it's clearer.}

\noindent
Atomic goals do not refer to other goals;
compisite goals are defined in terms of other goals.
Mercury's goal types are:

$$
\begin{array}{lll}
G
    & :~ X = Y ~|~ X = f(Y_1,~\ldots,~Y_n)
        & \hbox{unifications}\\
    & |~ p(X_1,~\ldots,~X_n)
        & \hbox{predicate calls} \\
    & |~ X_{n+1} = f(X_1,~\ldots,~X_n)
        & \hbox{function calls} \\
    & |~ X_0(X_1,~\ldots,~X_n)
        & \hbox{higher order calls} \\
    & |~ m(X_1,~\ldots,~X_n)
        & \hbox{method calls} \\
    & |~ \hbox{foreign}(p,
        [X_1:~Y_1,~\ldots,X_n:~Y_n],
        \hbox{\emph{foreign code}}),
        & \hbox{foreign code} \\
    & |~ (G_1,~\ldots,~G_n)
        & \hbox{sequential conjunctions}\\
    & |~ (G_1~\&~\ldots~\&~G_n)
        & \hbox{parallel conjunctions}\\
    & |~ (G_1 ; \ldots ; G_n)
        & \hbox{disjunctions}\\
    & |~ \hbox{switch}~X~(f_1:~G_1~;~\ldots~,f_n:~G_n)
        & \hbox{switches}\\
    & |~ (if~G_{cond}~then~G_{then}~else~G_{else})
        & \hbox{if-then-elses}\\
    & |~ not~G
        & \hbox{negations}\\
    & |~ some~[X_1,\ldots,X_n]~G
        & \hbox{existential quantification}\\
    & |~ all~[X_1,\ldots,X_n]~G
        & \hbox{universal quantification} \\
    & |~ promise\_pure~G
        & \hbox{purity promise}\\
    & |~ promise\_semipure~G
        & \hbox{purity promise}\\
\end{array}
$$
% \caption{The abstract syntax of Mercury}
% \label{fig:abstractsyntax}
% \end{figure}
%\vspace{-1mm}

\noindent
The atomic goals are unifications
(which the compiler breaks down until they contain
at most one function symbol each),
plain first-order calls,
higher-order calls,
method calls,
and calls to predicates defined by code in a foreign language (usually C).
\peter{You only need to cover enough about Mercury for the purposes of
  your thesis.  I don't think how inlining of foreign code works is important.}
\paul{We use it to implement primatives on which our transformations are
built,
I may remove it later if the detail of how this is done is unimportant.}
The compiler can inline foreign code definitions when generating code
for a Mercury predicate.
To allow this, the representation of a foreign code construct includes
not just the name of the predicate being called
but also the foreign code that is the predicate's definition
and the mapping from the Mercury variables that are the call's arguments
to the names of the variables that stand for them in the foreign code.
The composite goals include
sequential and parallel conjunctions,
disjunctions, if-then-elses, negations and existential quantifications.
A detailed description of parallel conjunctions can be found in Section
\ref{sec:back_mer_par}.
The abstract syntax does not include universal quantifications:
they are allowed at the source level,
but are transformed into combinations of negations and existential quantification:
$all~[X_1,~\ldots,~X_n]~G~\rightarrow~not~(~some~[X_1,~\ldots,~X_n]~(~not~G~))$.
Similarly,
a negation can be simplified using the constants $true$ and $false$ and an
if-then-else:
$not~G~\rightarrow~(if~G~then~false~else~true)$.
A switch is a disjunction in which
each disjunct unifies the same bound variable
with a different function symbol.
Switches in Mercury are thus analogous to switches in languages like C.
If there is a case in the switch for each function symbol in the
switched-on value's type, then the switch is said to be complete.
Otherwise the switch is incomplete.
That is to say,
there are function symbols in the variable's type that are not covered
by a case in the switch.
The purity promise will be described on page \pageref{page:purity}.

\paul{XXX: Need to describe semidet disjunctions somewhere}

Mercury has a strong Hindley-Milner~\citep{hindley69:types,milner78:types} type
system that was inspired by that of Hope~\citep{hope_types}
and similar to Haskell's~\citep{haskell98}.
Mercury programs are statically typed; the compiler knows the type of every
argument of every predicate (from declarations or inference) and every local
variable (from inference).
Types may be parametric,
type parameters are written in uppercase while
concrete types are written in lower case.
The following are references to types.
Type declrations are not covered here,
more information can be found in \citet{mercury_refman}.

\begin{description}

    \item[\code{int}] is a concrete type describing integers.

    \item[\code{T}] is an abstract type, \code{T} may be substituted for any other
    type.

    \item[\code{list(T)}] is a type describing a list, the list is defined
    in terms of some type \code{T}.
    In this case, \code{T} is used as the type of the elements in the list.

    \item[\code{list(int)}] is a concrete type describing a list of
    integers.

\end{description}

\noindent
The definition and type signature of \code{append/3} is:

\begin{verbatim}
:- pred append(list(T), list(T), list(T)).

append([], Ys, Ys).
append([X | Xs], Ys, [X | Zs]) :-
    append(Xs, Ys, Zs).
\end{verbatim}

\noindent
Mercury also has a strong mode system.
The mode of a predicate describes the \emph{instantiation state} changes,
if any, of the predicate's arguments.
A variable may be free, ground or clobbered.
Free variables currently have no value,
ground variables have a fixed value,
clobbered variables once had a value, but it is no longer available.
Partial instantiation can also occur,
however, the compiler does not fully support partial instantiaton
and therefore it is rarely used.
Ground and bound instantiation states may also be described as unique,
meaning they are not aliased with any other value.
When variables are used, their instantiation state may change;
such a change is described by a transition between two instantiation states,
or the same instantiation state if nothing changed.
The symbol \code{>>} denotes a state change.
This is also known as a mode (of an argument),
this is not the same as the mode of a predicate.
When a variable is used as an argument or in a unification it can only
become \emph{more instantiated};
that is to say `free to ground' is legal, but `ground to free' is not.
Similarly, uniqueness cannot be added to a value that is already
bound or ground.
Commonly used modes such as input (\code{in}), output (\code{out}),
destructive input (\code{di}) and unique output (\code{uo}) are
defined in the standard library:

\begin{verbatim}
:- mode in  == ground >> ground.
:- mode out == free >> ground.
:- mode di == unique >> clobbered.
:- mode uo == free >> unique.
\end{verbatim}

\noindent
There may be multiple modes for any given predicate.
Here are two mode declarations for the append predicate above;
these are usually written immediately after the type declaration.
When a predicate has only a single mode,
the mode declration can be combined with the type signature as a single
declration.

\begin{verbatim}
:- mode append(in, in, out) is det.
:- mode append(out, out, in) is multi.
\end{verbatim}

\noindent
The compiler enforces the instantiation state constraints on variables passed
as arguments:
Variables passed as input arguments must be ground before and after the call
while variables passed as output arguments must be free before the call and
ground after the call.
Similarly, a variable passed in a destructive input argument will be destroyed
before the end of the call;
it must be unique before the call and clobbered afterwords.
It is illegal to reference a clobbered variable.
A variable passed in a unique output argument must initially be free and
will become unique.
These two modes are often used together to allow the compiler to
destructively update data structures such as arrays.
The compiler will infer the instantiation states of all variables before and
after every goal in a predicate or function's body.

The Mercury compiler generates separate code
for each mode of a predicate or function,
which we call a \emph{procedure}.
In fact, individual procedures are handled as separate entities by
the compiler after mode checking.
This allows the mode checking pass of the compiler to minimally
reorder conjuncts (in both sequential and parallel conjunctions)
so that the producer of a variable (the code that makes the variable ground)
occurs before all consumers of that variable (any code that expects the
variable to be ground).
This means that for each variable in each procedure,
the compiler knows exactly where that variable becomes ground.

% Mode invariants
The mode system both enforces three invariants
that we need to refer to later in the dissertation:

\begin{description}

  \item[Conjunction invariant:]
  In any set of conjoined goals,
  which includes not just conjuncts in conjunctions
  but also the condition and then-part of an if-then-else,
  each variable that is consumed by any one of the goals
  is produced by exactly one earlier goal.
  This does not effect the else part of an if-then-else, it is not conjoined
  with the cond part.

  \item[Branched goal invariant:]
  In disjunctions, switches or if-then-elses,
  the goal types that contain alternative branches of execution,
  each branch of execution must produce
  the exact same set of variables
  that are consumed from outside the branched goal,
  with one exception:
  a branch of execution that cannot succeed (see determinisms below)
  may produce a subset of this set of variables.
  
  \item[Negated goal invariant:]
  A negated goal may not bind
  any variable that is visible to goals outside it,
  and the condition of an if-then-else may not bind a variable
  that is visible anywhere except in
  the then-part of that if-then-else.

\end{description}

\noindent
Each procedure and goal has a determinism,
which may put upper and lower bounds on the number of its possible solutions
(in the absence of infinite loops and exceptions).
Mercury's determinisms are:

\begin{description}
    \item[\ddet] procedures succeed exactly once
    (upper bound is one, lower bound is one).
    \item[\dsemidet] procedures succeed at most once
    (upper bound is one, no lower bound).
    \item[\dmulti] procedures succeed at least once
    (lower bound is one, no upper bound).
    \item[\dnondet] procedures may succeed any number of times
    (no bound of either kind).
    \item[\dfailure] procedures can never succeed
    (upper bound is zero, no lower bound).
    \item[\derroneous] procedures have an upper bound of zero and a lower
    bound of one, which means they can neither succeed nor fail.
    They must either throw an exception or loop forever.
    % We use the cc detisms a little bit, since parallelism may be used in
    % cc_multi predicates.
    \item[\dccmulti] procedures may have more than one solutions, like \dmulti
    but they commit to the first solultion.
    In practice, they succeed exactly once.
    \item[\dccnondet] procedures may have any number of solutions, like
    \dnondet
    but they commit to the first solution.
    In practice, they succeed at most once.
\end{description}

\noindent
In practice, most parts of a Mercury program are deterministic (\ddet).
Each procedure's mode declaration
typically declares its determinism (see the mode declarations for append above).
If this is omitted, the compiler can infer the missing information.

% Switch detection.

\begin{figure}
\parbox{0.5\textwidth}{
$$
\begin{array}{ll}
(\\
& x~=~f_1(\ldots),~G_1 \\
; \\
& x~=~f_2(\ldots),~G_2 \\
; \\
& x~=~f_n(\ldots),~G_n \\
)
\end{array}
$$}%
\parbox{0.5\textwidth}{
$$
\begin{array}{ll}
\hbox{switch}~x~( \\
f_1: \\
& x~=~f_1(\ldots),~G_1 \\
; \\
f_2: \\
& x~=~f_2(\ldots),~G_2 \\
; \\
f_n: \\
& x~=~f_n(\ldots),~G_n \\
)
\end{array}
$$}
\caption{Switch detection example}
\label{fig:switch_detect}
\end{figure}

Before the compiler attempts to check or infer
the determinism of each procedure,
it runs a switch detection algorithm that looks for disjunctions
in which each disjunct unifies the same input variable
(a variable that is already bound when the disjunction is entered)
with any number of different function symbols.
Figure \ref{fig:switch_detect} shows an example.
When the same variable is used with multiple levels of unification in a
disjunction switch detection will rely on
common subexpression elimination (CSE)
to generate nested switches.

The point of this is that it allows determinism analysis
to infer much tighter bounds on the number of solutions of the goal.
For example, if each of the $G_i$ is deterministic
(i.e. it has determinism \code{det})
and the various $f_i$ comprise all the function symbols in $x$'s type,
then the switch can be inferred to be deterministic as well,
whereas a disjunction that is \emph{not} a switch and produces at least one
variable
cannot be deterministic,
since each disjunct may generate a solution.

Mercury has a module system.
Calls may be qualified by the name of the module
that defines the predicate or function being called,
with the module qualifier and the predicate or function name
separated by a dot.
The \io (Input/Output) module of the Mercury standard library
defines an abstract type called the \io state,
which represents the entire state of the world outside the program.
The \io module also defines a large set of predicates that perform I/O.
These predicates all have determinism \code{det}.
and besides other arguments,
they all take a pair of \io states
whose modes are respectively \di and \uo,
(As discussed above, \di being shorthand for \emph{destructive input}
and \uo for \emph{unique output}.)
The \code{main/2} predicate that represents the entire program
(like \code{main()} in C)
also has two arguments, a \code{di,uo} pair of \io states.
A program is thus given
a unique reference to the initial state of the world,
every I/O operation conceptually destroys the current state of the world
and returns a unique reference to the new state of the world,
and the program must return the final state of the world
as the output of \code{main/2}.
Thus, the program (\code{main/2}) defines a relationship between the state
of the world before the program was executed,
and the state of the world when the program terminates.
These conditions guarantee that at each point in the execution,
there is exactly one current state of the world.

As an example, here is one version of ``Hello world'':

\begin{verbatim}
:- pred main(io::di, io::uo) is det.

main(S0, S) :-
    io.write_string("Hello ", S0, S1),
    io.write_string("world\n", S1, S).
\end{verbatim}

\noindent
% In this version,
\code{S0} is the initial state of the world,
\code{S1} is the state of the world after printing ``\code{Hello }'',
and \code{S} is the state of the world
after printing ``\code{world\n}'' as well.
Note that the difference e.g.\ between \code{S1} and \code{S}
represents not just the printing of ``\code{world\n}'',
but also all the changes made in the state of the world
by \emph{other} processes since the creation of \code{S1}.
This threaded state sequentialises I/O operations:
``\code{world\n}'' cannot be printed until the value for \code{S1}
is available.

%\peter{This isn't relevant to your work, unless you're intent on
%  including code with state variables in your discussion.  I think
%  it's a complication better left out.}
Numbering each version of the state of the world
(or any other state that a program may pass around) is cumbersome,
Mercury has syntactic sugar to avoid the need for this,
but this sugar does not affect
the compiler's internal representation of the program.
This syntactic sugar is shown below, along with the convention of naming
the I/O state \code{!IO}.
The example below will be transformed into the example above during
compilation.

\begin{verbatim}
:- pred main(io::di, io::uo) is det.

main(!IO) :-
    io.write_string("Hello ", !IO),
    io.write_string("world\n", !IO).
\end{verbatim}

\noindent
Any term beginning with a bang (\code{!})
will be expanded into pairs of automatically named
variables.
The variable names created sequence conjunctions from left to right in
conjunctions when used with pairs of \code{di/uo} or \code{in/out}
modes.
%\peter{End of stuff I think you should leave out.}

\label{page:purity}
Mercury divides goals into three purity categories:

\begin{description}

    \item[pure goals] have no side effects
    and their outputs do not depend on side effects;

    \item[semipure goals] have no side effects
    but their outputs may depend on other side effects;

    \item[impure goals] may have side effects, and may produce outputs
      that depend on other side effects.

\end{description}

\noindent
Semipure and impure predicates and functions
have to be declared as such,
and calls to them must be prefaced with either
\code{impure} or \code{semipure} (whichever is appropriate).
The vast majority of Mercury code is pure,
with impure and semipure code confined to a few places
where it is used to implement pure interfaces.
(For example, the implementation of the all-solutions predicates
in the Mercury standard library use impure and semipure code.)
Programmers can wrap
\code{promise\_pure} or \code{promise\_semipure} goals around a goal
to promise to the compiler (and to human readers) that
the goal as a whole is pure or semipure respectively,
even though some of the code inside the goal may be less pure.

The compiler keeps a lot of information associated with each goal,
whether atomic or not.
This includes:

\begin{itemize}
\item
the set of variables bound (or \emph{produced}) by the goal;
\item
the purity of the goal
\item
the \emph{nonlocal set} of the goal,
which means the set of variables
that occur both inside the goal and outside it; and
\item
the determinism of the goal.
\end{itemize}

\noindent
A complete description of Mercury
can be found in the language reference manual \citep{mercury_refman}.



\section{Explicit Parallelism in Mercury}
\label{sec:back_mer_par}

The low-level C backend of the Mercury implementation uses C as a
pesudo-assembler;
A Mercury \emph{engine} structure represents a CPU that can execute Mercury
Abstract Machine instructions.
An engine has fields that represent virual registers,
some of which are mapped (using GCC's~\citep{gcc} global register extension) to
physical machine registers - depending on the achitecture of the physical
machine.
These registers include 1024 general-purpose registers,
three stack pointers for the two stacks,
and a success continuation pointer for implementing tail-calls.
The structure contains many other fields for bookkeeping,
we will discuss them as they become relevant.
The abstract machine instructions are implemented as C preprocessor macros,
which use the engine structure.
More information about Mercury's execution algorithm can be found in
\citet{mercury_jlp}.

To make parallel execution possible \citet{conway:2002:par} allowed
multiple instances of the engine structure to be used at once.
The Operating System's threading library,
namely POSIX Threads~\citep{pthreads},
is used to expose parallelism to the OS.
Each POSIX Thread (pthread) owns a single engine.
The number of engines that a parallel Mercury program will allocate on startup
is configurable by the user,
but it defaults to the actual number of CPUs.
Users can also configure the runtime system to pin engines to CPUs
ensure there is an even distribution of engines across the machine.
\paul{Maybe discuss how thread pinning and SMT interact.  Although all
this thread pinning work is new work - it's not necessary part of the
background.  I think this discussion may belong in \ref{chap:rts}}

Eacn POSIX Thread must maintain a pointer to its engine;
this is kept in a physical machine register as it is accessed
frequently.
In sequential code,
there was only one engine structure,
so a compiled-in constant could easily be used.
Therefore, in parallel builds there are fewer physical registers that
virtual registers may be mapped onto.

\citet{conway:2002:par} also introduced a new structure called a
\emph{context}, also known as a green thread.
Contexts represent computations in progress.
An engine may be idle, or it may be executing a context;
a context can be running on an engine, or it may be suspended.
Each context has two stacks, a det stack and a nondet stack;
Procedures that can succeed more than once
store their frames on the nondet stack;
all other procedures use the det stack
(which behaves like a stack in an imperative language).
These stacks account for the majority of the context's memory usage.
When a context finishes execution,
it can either be retained by the engine or have its storage released to the
free context pool.
This decision depends on what the engine will do next,
if the engine will execute a different context or go to sleep
(because there's no work to do);
then the current context is released,
otherwise it is retained.
Contexts cannot be preempted.

Following \citet{simonmar_2009_multicore_rts},
we economize on memory by using \emph{sparks}
to represent goals that have been spawned off
but whose execution has not yet been started.
Sparks are very light weight being only three words long.
Therefore, compared to contexts,
they represent outstanding parallel work very efficiently.
When an engine executes a spark it will convert the spark into a context.
If the engine was already holding a context that context will be used
(the engine's current context is always free when the engine attempts to run
a spark),
or if it does not have a context a new one is allocated
(from the pool of free contexts if the pool is not empty,
and a newly created context otherwise).
Unlike \cite{simonmar_2009_multicore_rts},
Mercury's sparks cannot be garbage collected and must be executed.

The only parallel construct in Mercury is parallel conjunction,
which is denoted $(G_1~\&~\ldots~\&~G_n)$.
All the conjuncts must be \ddet or \dccmulti,
that is, they must all have exactly one solution,
or commit to exactly one solution.
This restriction greatly simplifies the implementation,
since it guarantees that there can never be any need
to execute $(G_2~\&~\ldots~\&~G_n)$ multiple times,
just because $G_1$ has succeeded multiple times.
(Any local backtracking inside $G_1$ will not be visible to the other conjuncts;
bindings made by \ddet code are never retracted.)
The current Mercury implementation supports parallelism only for \ddet and \dccmulti,
since supporting it for code that may have no solution
would represent speculative execution,
while supporting for code that may have more than one solution
would require significant new infrastructure for managing bindings.
However, this is not a significant limitation.
Since the design of Mercury strongly encourages deterministic code,
in our experience, about 75 to 85\% of all Mercury procedures are \ddet,
and most programs spend an even greater fraction of their time in \ddet code.
\peter{Do we have any statistics regarding what proportion of execution time
  is spent in det code?  That would be a more relevant statistic.}
Existing algorithms for executing nondeterministic code in parallel
have very significant overheads, generating slowdowns by integer factors.
Thus we have given priority to parallelizing deterministic code,
which we can do with \emph{much} lower overhead.
We think that avoiding such slowdowns is a good idea,
even if it does mean foregoing the parallelization of 10 to 25\% of a program.

\begin{figure}
\begin{verbatim}
    /*
    ** A SyncTerm. One syncterm is created for each parallel conjunction.
    */
struct MR_SyncTerm_Struct {
    MR_Context              *MR_st_orig_context;
    MR_Word                 *MR_st_parent_sp;
    volatile MR_Unsigned    MR_st_count;
};

    /*
    ** A Mercury Spark.  A spark is created for a spawned off parallel
    ** conjunct.
    */
struct MR_Spark {
    MR_SyncTerm             *MR_spark_sync_term;
    MR_Code                 *MR_spark_resume;
    MR_ThreadLocalMuts      *MR_spark_thread_local_mutables;
}
\end{verbatim}
\caption{Syncterms and sparks}
\label{fig:spark_and_syncterm}
\end{figure}

The Mercury compiler implements $(G_1~\&~G_2~\&~\ldots~\&~G_n)$
by creating a \emph{syncterm} (synchronisation term), a data structure
representing a barrier for the whole conjunction.
The syncterm contains:
a pointer to the context that begun executing the parallel conjunction
(the parent context),
a pointer to the stack frame for the procedure containing this parallel
conjunction,
and a thread safe counter that tracks the number of conjuncts that have not
yet executed their barrier.
After initializing this syncterm, the engine then
spanws off $(G_2~\&~\ldots~\&~G_n)$ as a spark and continues by executing
$G_1$ itsself.
The spark contains a pointer to the syncterm,
as well as a pointer to the code it must execute
and another pointer to any thread local mutables that should be available to the
spawned off code.
Figure \ref{fig:spark_and_syncterm} shows these two datastructures.
The spark is added to a global run queue of sparks, or if that queue is too
full, because there's already enough parallelism,
then the spark is added to a queue owned by the current context.
\citet{wang_hons_thesis} intended to use the local queues for work stealing
but had not completed his implementation,
The work-stealing dequeue structure that \citet{wang_hons_thesis} 
is described in \cite{Chase_2005_wsdeque}.
see Chapter \ref{chap:rts} for details.

If the context finishes the execution of the first conjunct ($G_1$)
and finds that spark is still at the head of its queue,
it will pick it up and run it itself.
This is a useful optimization,
% it's also really well-known.
since it avoids using a separate context in the relatively common case
that all the other CPUs are busy with their own work.
This optimization is also useful since it can avoid the creation of supurfluous
contexts and their stacks.

\begin{figure}
\begin{center}
\begin{tabular}{ll}
\multicolumn{1}{c}{\textbf{Source code}} &
\multicolumn{1}{c}{\textbf{Transformed pseudo-code}} \\
\hline
                    & \code{~~MR\_SyncTerm ST;} \\
\code{~~}$($        & \code{~~spawn\_off(\&ST, Spawn\_Label);} \\
\code{~~~~}$G_1$    & \code{~~}$G_1$ \\
\code{~~}$\&$       & \code{~~join\_and\_continue(\&ST, Cont\_Label);} \\
                    & \code{Spawn\_Label:} \\
\code{~~~~}$G_2$    & \code{~~}$G_2$ \\
\code{~~}$)$        & \code{~~join\_and\_terminate(\&ST);} \\
                    & \code{Cont\_Label:} \\
\end{tabular}
\end{center}
\caption{The implementation of a parallel conjunction}
\label{fig:par_conj}
\end{figure}

Since $(G_2~\&~\ldots~\&~G_n)$ is itself a conjunction,
it is handled in a similar way:
the context executing it
first spawns off $(G_3~\&~\ldots~\&~G_n)$ as a spark that points to the barrier
created earlier,
and then executes $G_2$ itself.
Eventually, the spawned-off remainder of the conjunction
consists only of the final conjunct, $G_n$,
and the context just executes it.
The code of each conjunct synchronizes on the barrier once it has
completed its job.
When all conjuncts have done so,
the original context will continue execution after the parallel conjunction.
An example of the instrumentation necessary to implement parallel
conjunctions is shown in Figure \ref{fig:par_conj}.

When an engine becomes idle, it will first try
to resume a suspended but runnable context if there is one.
If not, it will attempt to run a spark from the global spark queue.
If it successfully finds a spark, it will allocate a context,
and start running the spark in that context.

Barrier code is placed at the end of each conjunct,
this is named \code{join\_and\_continue} (Figure \ref{fig:par_conj}).
This code starts by atomically decrementing the number of outstanding
conjuncts in the conjunction's syncterm and checking the result for zero
(the whole operation is atomic, not just the decrement).
If there are no remaining conjuncts and the current context is the parent
context,
then execution jumps to the label after the parallel conjunction.
If the current context is not the parent context then
we can infer that the parent context is suspended
(or in the process of being suspended),
therefore, 
the engine will schedule the parent context
(the engine may busy-wait until the context is suspended before scheduling it). 
Alternativly, if there are outstanding conjuncts and
this is the parent context then it is suspended.
Finally,
if an engine with a context other than the original one finds that
there are still outstanding jobs,
they check their local spark queue for any such work,
otherwise they look for global work.

% XXX: revising this WRT the old code.

An engine looks for global work first by checking the local spark
queue of its context.
In some cases this check can be optimized out, for example after there
are no outstanding conjuncts in a parallel conjunction.
It will then check for runnable but suspended contexts,
If it finds a runnable context and is still holding a context from a
previous execution, it saves the old context onto the free context list.
If there are no runnable contexts,
it will then attempt to steal work from other contexts.
If unsuccessful, it will become idle and sleep for a period of time
before it looks for work again
or is woken up because a context has become runnable.

\begin{figure}
\begin{verbatim}
map_foldl(M, F, L, Acc0, Acc) :-
    (
        L = [],
        Acc = Acc0
    ;
        L = [H | T],
        new_future(FutureAcc1),
        (
            M(H, MappedH),
            F(MappedH, Acc0, Acc1),
            signal_future(FutureAcc1, Acc1)
        &
            map_foldl_par(M, F, T, FutureAcc1, Acc)
        )
    ).

map_foldl_par(M, F, L, FutureAcc0, Acc) :-
    (
        L = [],
        wait_future(FutureAcc0, Acc0),
        Acc = Acc0
    ;
        L = [H | T],
        new_future(FutureAcc1),
        (
            M(H, MappedH),
            wait_future(FutureAcc0, Acc0),
            F(MappedH, Acc0, Acc1),
            signal_future(FutureAcc1, Acc1)
        &
            map_foldl_par(M, F, T, FutureAcc1, Acc)
        )
    ).
\end{verbatim}
%\vspace{2mm}
\caption{\mapfoldl{} with synchronization}
\label{fig:map_foldl_sync}
%\vspace{-1\baselineskip}
\end{figure}

% XXX: This paragraph raises an interesting point and leads to another
% justification for the use of sparks.  However, it is out of place here.
%
% The simplest way to implement a parallel conjunction with $n$ conjuncts
% is to spawn off $n-1$ of the conjuncts,
% letting other CPUs pick them up when they are free,
% and have the original CPU continue executing the last conjunct,
% with all conjuncts meeting at a barrier at the end,
% and only the original CPU continuing from the barrier.
% In practice, this approach has unnecessarily high overhead.
% The reason is that since current CPU chips
% have only a few cores (usually only 2-4),
% the probability is quite high that
% there will be no free CPU to pick up most of the spawned-off conjuncts,
% which means that

Mercury's mode system allows a parallel conjunct to consume variables
that are produced by conjuncts to its left, but not to its right.
This guarantees the absence of circular dependencies
and hence the absence of deadlocks between the conjuncts,
but it does allow a conjunct to depend on data that is yet to be computed
by a conjunct running in parallel.
We handle these dependencies through a source-to-source transformation
\cite{wang_dep_par_conj}.
The compiler knows which variables
are produced by one parallel conjunct and consumed by another.
For each of these shared variables,
it creates a data structure called a \emph{future} \cite{multilisp}.
When the producer has finished computing the value of the variable,
it puts the value in the future and signals its availability.
When a consumer needs the value of the variable,
it waits for this signal (if it has not yet happened),
and then retrieves the value from the future.
% Both operations are protected by mutexes.

For each of these shared variables,
the compiler creates a data structure called a \emph{future} \cite{multilisp},
which contains room for the value of the variable,
a flag indicating whether the variable has been produced yet,
a queue of consumer contexts waiting for the value, and a mutex.
The initial value of the future has the flag set to `not yet produced'.
The signal operation on the future sets the value of the variable,
sets the flag to `produced',
and wakes up all the waiting consumers,
all under the protection of the mutex.
The wait operation on the future is also protected by the mutex:
it checks the value of the flag,
and if it says `not yet produced',
the engine will put its context on the queue and suspend it before
looking for other work.
When it wakes up,
or if the flag said that the value was already `produced',
the wait operation simply gets the value of the variable.

To minimize waiting,
the compiler pushes signal operations on each future
as far to the left into the producer conjunct as possible,
and it pushes wait operations
as far to the right into each of its consumer conjuncts as possible.
This means not only pushing them
into the body of the predicate called by the conjunct,
but also into the bodies of the predicates they call,
with the intention being that
each signal is put immediately after
the primitive goal that produces the value of the variable,
and each wait is put immediately before
the leftmost primitive goal that consumes the value of the variable.
Since the compiler has complete information
about which goals produce and consume which variables,
the only things that can stop the pushing process from placing the
wait immediately before the value is to be used and the signal
immediately after it is produced are
higher order calls and module boundaries:
the compiler cannot push a wait or signal operation
into code it cannot identify or cannot access.
% into the body of a predicate
% if it does not have access to the identity or to the body of the predicate.

% Work stealing --- Most things execute in sequence, parallel
% execution occurs when an engine has no work of it's own.

Given the \mapfoldl predicate in Figure~\ref{fig:mapfoldl},
this synchronization transformation
generates the code in Figure~\ref{fig:map_foldl_sync}.



\section{Profiling and Feedback in Mercury}
\label{sec:back_deep}

\status{
    This section and 2.5 have been merged, as they provide motivation for one
    another.
    This will likly be a large section, so I've broken it into sub-sections
}

% Declaration of work done that contributed to a previous award,
Some of the work described in this section contributed towards my honours research project,
% Note that my degree is not '... with Honours', it's just 'Honours', despite
% what my degree actually says.  Basically, I've been erroneously awarded an
% extra bachelor's degree.
which was in partial forfillment of the Degree of Batchelor of Computer Science
Honours.
Most parts of the implementation were incomplete and did not work correctly or
at all.
Therefore,
their evaluation could not be included in my honours report.
They have since been completed and corrected during my Ph.D.\ candidature.
There are some exceptions,
most sagnificantly the coverage profiling transformation (Section \ref{sec:coverage})
developed for my honours research project was complete, and is used, largly
unchanged in my Ph.D.\ research.
The related analysis described in Section \ref{sec:var_use_analysis}
and the feedback framework in Section \ref{sec:feedback}
were also writen as part of my honours research project.
Still other parts of this section are not my own work,
namely the work described in
\citet{conway:2001:mercury-deep} and
\citet{tannier:2007:parallel_mercury}.

\status{Plan for this section:}

\begin{itemize}
\item Why profile directed parallelisation.
\item the deep profiler,
\item including a description of why deep profiling provides us with
      more useful data than conventional profiling.
\item Prior auto-parallelism work.
\begin{itemize}
    \item My honours thesis
\end{itemize}
\item Coverage profiling.
\item Variable use analysis
\item profiling feedback framework.
\end{itemize}

\paul{XXX: These items aren't background.}
\begin{itemize}
\item also explain the limitations of deep profiling,
      for example, that levels within recorsive code don't have seperate
      data recorded for them.
      (The solution to this will probably be in the overlap chapter)
      XXX: This problem didn't exist in my honours work, it
\item Callgraph search
\end{itemize}

Most compiler optimizations operate only on the representation of the program
in the compiler's memory.
For most optimizations this is sufficient.
However,
automatic parallelisation is sensative to variations in the runtime cost of
parallelised tasks.
This sensativity increases when dependent parallelisation is used.
For example,
a search operation on a small list is cheap, compared to the same operation on
a large list.
It may not be useful to parallelise the search on the small list against some
other computation,
but it will usually be useful to parallelise the search on the large list
against another computation.
It's important not to create too much parallelism;
overheads of parallelisation for which there is no speed up will slow the
program down.
Therefore, not just sub-optimal to parallelise the search of the small list,
but detrimental.
The only way we can know the actual cost of most peices of code,
is by understanding their typical inputs,
or measuring their runtime cost while operating on typical inputs.
Therefore,
profiling data can be used in auto-parallelisation;
it allows us to predict the runtime cost of computations whose
code is the same but whose inputs are different.
These predictions are used to drive parallelisation decisions,
which are, in turn, used to select where and how parallelism is introduced.

\picfigure{prof_fb}{Profiler feedback loop}

To use profiling data for optimizations the usual workflow for compiling
software must change.
Figure \ref{fig:prof_fb} shows a profiler-feedback workflow.
The source code is compiled, with profling enabled and the resulting program is
executed on a representative input.
As the program terminates it writes its profiling data to disk,
The profling data includes a bytecode representation of the program
similar to the compiler's representation.
The analysis tool reads this file,
and writes out a description of how to parallelise the program in the form of a
feedback file.
The program is compiled for a second time,
this time with auto-parallelism enabled.
The compiler reads the feedback file and introduces parallelism into the
resulting program.
Figure \ref{fig:prof_fb} shows this workflow.

A common criticism of profile directed optimisation is that the programmer will
have to compile the program twice,
and run it at least once to generate profiling data.
"If I have to run the program in order to optimize it, then the program has
already
done its job and there's no point continuing with the optimisation?"
The answer to this is that a program's life time is often far more than a
single execution.
A program will usually be used many times, and by many people.
Each time the optimization will have a benifit,
this benifit will pay off the cost of the feedback directed optimisation.
We expect that feedback-directed optimisations would normally be used
when the programmer is building a \emph{release candidate} version of their
program,
after they have performed testing and fixed any bugs.

Another criticism is that if the program is profiled with one set of input and
used with another that the profile will not necessarly represent the actual use
of the program,
and that the optimisation may not be as good as it could be.
There are many different types of programs, and therefore many different
program inputs;
we cannot hope to address this particular problem directly.
Selecting a typical input for a given program must be done by the programmer.
Other stratergies exist that avoid this problem
such as dynamic recompilation.

\subsection{The deep profiler}

Typical Mercury programs make heavy use of code reuse in the form of
parametric polymorphism and higher order calls.
This is also true for other declarative languages.
For example, while a C program may have
separate tables for different kinds of entities,
for whose access functions
the profiler would gather separate performance data,
most Mercury programs would use
the same polymorphic code to handle all those tables,
making the task of disentangling the characteristics of the different tables
infeasibly hard.

This problem has been solved for Mercury by introducing deep profiling
\citep{conway:2001:mercury-deep}.
Mercury's deep profiler gathers much more information about the context of
each measurement than traditional profilers like \code{gprof}\cite{gprof}.
When it records the occurrence of a call,
a memory allocation or a profiling clock interrupt,
it records with it the chain of ancestor calls,
all the way from the current call to the entry point of the program,
a procedure named \code{main}.
To make this tractable,
recursive and mutually recursive calls,
known as \emph{strongly connected components} (SCCs),
must be folded into a single memory structure.
Therefore, the call graph of the program is a tree (there are no cycles)
and SCCs are represented as single nodes in this tree.

Deep profiling allows the profiler to find and present to the user
not just information such as the total number of calls to a procedure
and the average cost of a call,
or even information such as the total number of calls to a procedure
from a particular call site and the average cost of a call from that call site,
but also information such as the total number of calls to a procedure
\emph{from a particular call site
when invoked from a particular chain of ancestor SCCs}
and the average cost of a call \emph{in that context}.
For example, it could tell that
procedure $h$ called procedure $i$ ten times
when $h$'s chain of ancestors was $main \calls f \calls h$,
while $h$ called $i$ only seven times
when $h$'s chain of ancestors was $main \calls g \calls h$,
the calls from $h$ to $i$ took on average twice as long
from the $main \calls g \calls h$ context as from $main \calls f \calls h$,
so that despite the fewer calls,
$main \calls g \calls h \calls i$ took more time than $main \calls f \calls h \calls i$.
\paul{Draw a figure}

Profilers have traditionally measured time
by sampling the program counter at clock interrupts.
Unfortunately, even on modern machines
the usual, and portable infrastructure for clock interrupts
(\emph{e.g}., SIGPROF on Unix)
supports only one frequency for such interrupts,
which is usually 60 or 100Hz.
This frequency is far too low for the kind of detailed measurements
the Mercury deep profiler wants to make,
since for typical program runs of few seconds,
it results in almost all calls having a recorded time of zero,
with the calls recording a nonzero time
(signifying the occurrence of an interrupt during their execution)
being selected almost at random.

We have therefore implemented a finer-grained measure of time
that turned out to be very useful
even though it is inherently approximate.
This measure is call sequence counts (CSCs):
the profiled program basically behaves
as if the occurrence of a call signified
the occurrence of a new kind of profiling interrupt.
In imperative programs, this would be a horrible measure,
since calls to different functions can have hugely different runtimes.
However, in declarative language like Mercury there are no explicit loops;
what a programmer would do with a loop in an imperative language
must be done by a recursive call.
This means that the only thing that the program can execute between two calls
is a sequence of primitive operations such as unifications and arithmetic.
For any given program,
there is a strict upper bound on the maximum length of such sequences,
and the distribution of the length of such sequences
is very strongly biased towards very short sequences
of half-a-dozen to a dozen operations.
In practice, we have found that
the fluctuations between the lengths of different such sequences
can be ignored for any measurement
that covers any significant number of call sequence counts,
say more than a hundred.
The only drawback of this scheme that we have found
is that on 32 bit platforms,
its usability is limited to short program runs (a few seconds)
by the wraparound of the global CSC counter;
on 64 bit platforms, the problem would occur
only on a profiling run that lasts for years.

\subsection{Prior Auto-parallelism Work}

\citet{tannier:2007:parallel_mercury} describes a prior attempted at automatic
parallelism in Mercury.
In this approch, the parallelisation feedback data was gathered by an analysis
tool,
this feedback was a list of the procedures with the highest cost in the program,
these were choosen by a configurable measurement and threshold.
The analysis did not use a representation of the program,
as at that time the profiler did not support that capability.
Nor did it use any advanced features of the profiler.
Parallelisation decisions were made in the compiler:
each procedure was searched for calls that appeared in the list of top procedures
and calls to these where parallelised against similar calls if they
where independent or had fewer than some $N$ shared variables.

\paul{XXX:  'my' or 'our' in this work.  I know we usually use plural
first person in papers,
Is this true for dissertations as well?}
For my honours project I attempted automatic parallelism of Mercury
\citep{paul_hons}.
There where a number of differences between my work and Tannier's.
The first differences is that at the time of my work the deep profiler
was able to export a representation of the program
(my work was the motivation for this feature).
This allowed my analysis to access profiling data and a representation
of the program at the same time.
This allows the analysis to measure when variables are produced and
consumed within parallel conjuncts,
and therefore calculate,
for conjunctions with a single shared variable,
how much parallelism is available.
The formula used was:

\begin{eqnarray*}
T_{Sequential} & = & T_A + T_B \\
T_{DependantB} & = & max(T_{BeforeProduceA}, T_{BeforeConsumeB}) +
T_{AfterConsumeB} \\
T_{Parallel} & = & max(T_A, T_{DependantB}) + T_{Overheads} \\
Speedup & = & \frac{T_{Sequntial}}{T_{Parallel}}
%\label{eqn:time_deppar}
\end{eqnarray*}

This describes the speedup due to parallelism of two conjuncts, $A$ and $B$.
whose secution times are $T_A$ and $T_B$.
In the parallel case $B$'s execution time is $T_{DependantB}$ since it
accounts for the $B$'s dependency on $A$.
Except for this small difference,
the calculation of parallel execution time and speedup are the commonly
used formulas for many parallel execution cost models.

This cost model requires information about when a shared variable is
produced by the first conjunct and consumed by the second.
To provide this information we introduced coverage profiling to
Mercury's deep profiler,
and used the resulting coverage data in a variable use time analysis.

\subsection{Coverage Profiling}
\label{sec:coverage}

The deep profiler records port counts for predicates.
The traditional port model~\cite{port_model} describes each predicate as
having four ports through which control may flow.
Control can flow into a predicate through either the call or redo ports,
control flows out of a predicate through either the exit or fail ports.
These describe:
    a call being made to the predicate,
    a call to the predicate being re-entered because there may be more
    solutions,
    the predicate exiting with a solution,
    or the predicate failing because there are no more solutions.
Mercury adds an extra port to these semantics known as the exception port,
which a predicate uses to return control to its caller when it throws an
exception.
The following invarient must hold,
control must leave a predicate the same number of times that it enters.

\begin{equation*}
Calls + Redos = Exits + Fails + Exceptions
\end{equation*}

\noindent
Mercury adds additional invariants for many determinism types.
For example, determinisitic code may not Redo or Fail.
The deep profiler tracks the number of times each port is used for a given
call.
This data can be uesd to measure code coverage in a Mercury program.

However, port counts alone will not provide complete coverge information.
We introduced coverage points to address this problem.
When the compiler instruments the program for deep profiling it will make
a forward pass through the goal tree of each predicate.
During this pass it will track whether or not it will be possible to
calculate the coverage of the current goal based on those that came before
it.
For example, a unifcation (which does not have any port counts of its own)
that follows a call will be called the same number of times as the call
exits.
If the unification is determinstic then it will also have the same number of
exits.
Coverage points are introduced often at the beginning of braches such as
switches --- but never in the last case of a swich whose coverage can be
inferred from the others.
They're also inserted after any goal whose exit port count is unknown.
Each coverage point referrs to a counter which is incremented each time the
coverage point is executed.
These counters are not associated with context the way other performance
data in Mercury is.
Therefore,
coverage data of a particular predicate covers the general use of that
predicate,
The auto-parallelisation work described in my honours report
\citep{pbone_hons}
would not have benifited from context specific coverage data.

\begin{figure}
\begin{verbatim}
\code{map(P, Xs0, Ys) :-} \\
\code{~~~~(} \\
\code{~~~~~~~~}\instr{coverage\_point(ProcStatic, 0);} \\
\code{~~~~~~~~Xs0 = [],} \\
\code{~~~~~~~~Ys = []} \\
\code{~~~~;
\code{~~~~~~~~Xs0 = [X $|$ Xs],} \\
\code{~~~~~~~~P(X, Y),} \\
\code{~~~~~~~~map(P, Xs, Ys0),} \\
\code{~~~~~~~~Ys = [Y $|$ Ys0]} \\
\code{~~~~).} \\
\end{verbatim}
\caption{Coverage annotated \code{map/3}.}
\label{fig:map_coverage}
\end{figure}

\paul{Explain changes to data structures.}

Figure \ref{fig:map_coverage} shows \code{map/3} instrumented for coverage
profiling.
There is a single coverage point inserted in the beginning of the first
case of the switch on the list.
The call to the instrumetnation code 
refers to the ProcStatic structure,
and an index into the coverage point array contained in it.

a structure from the deep profiler where information about this procedure is
stored.
\paul{XXX Write about ProcStatic and other structures in the deep profiler
section.}


\paul{XXX: Add example}

\status{These two subsections are the only ones not written in this chapter
now.}

\subsection{Variable Use Time Analysis}
\label{sec:var_use_analysis}


% Incorrect background, some of this prose can be used later.
%This approch also used profiling feedback data similar to Tannier's
%work.
%However, it also made use of and some of the advanced
%features of the deep profiler.
%The deep profiler was modified to export a representation of the profiled
%program.
%This representation, along with the profiling data could both be accessed by
%the analysis.
%My approch differed from Tannier's in that the analysis was done outside of the
%compiler in an analysis tool.
%The feedback given to the compiler was the result of this analysis:
%a list of candidate parallel conjunctions that the compiler would attempt to parallelise.
%Each candidate parallel conjunction describes its location in the origianal
%program and how its conjuncts should be constructed of smaller goals.
%However, at the completion of my honours project the implementation was incomplete
%and no feedback data was actually given to the compiler.
%Figure \ref{fig:prof_fb} closely describes the workflow for our feedback
%analysis.
%
%% call graph traversal.
%
%The auto-parallelism analysis traverses the call graph looking for parallelism
%opertunities.
%The traversal begins at the root of the call graph,
%namely \code{main/2}.
%Each node in the graph is an SCC, and each edge is a call from a call site to a
%distinct callee.
%Each procedure in an SCC is searched for parallelisation candidates before any
%child-nodes are checked (which are also SCCs).
%The search continues in this fassion until it finds an edge
%representing a call whose cost is too low such that it is not worth parallelising
%anything within the callee, and therefore, in the callee's callees.
%In practice, only a small part of the callgraph is traversed,
%since most parts of a program have a very low runtime cost and are not worth
%parallelising.
%The search will also track the number of processors available for spawned-off work
%at any point in the call graph,
%as parallelisation candidates are found, this number will decrease.
%Once it reaches zero, the search will stop as parallelising anything below this
%point in the call graph will not create any useful parallelism.
%This traversal could only be done with access to both a representation of the
%program being profiled and profiling data of the program.
%Therefore, implementing the analysis in a seperate tool was the obvious choice.
%
%Because each node in the graph is an SCC, and the graph is a tree
%the traversal does not need to worry about following cycles infinitly.
%However,
%determining the cost of recursive code is problematic.
%If a predicate calls itsself, we cannot know the cost of the recursive
%call,
%this cost is attributed to the cost of the call of that predicate which
%is already active.
%The automatic parallelism analysis described in \citet{paul_hons}
%treats recursive calls naively,
%A solution to this is presented in Section
%\ref{sec:recursive_call_costs}.


\subsection{Feedback framework}
\label{sec:feedback}

Automatic parallelism is just one use of profiler feedback.
Other optimisations
such as inlining,
branch hints
and type specialisation
can also benefit from profiling feedback.

We have designed and implemented a generic feedback framework that allows
tools to create feedback information for the compiler.
These tools may include the profiler, the compiler its self,
or any other tool that is able to link to the feedback library code.
Any Mercury type can be used as feedback information,
making the feedback framework very flexible.
New feedback-directed optimisations may require feedback information
from new analyses.
We expect that many new feedback information types will be added to
support these optimisations.
We also expect that an optimisation may use more than one type of
feedback information and that more than one optimisation may use the
same feedback information.
These where considered requirements of the feedback framework.

% Describe on-disk format,
The on-disc format for the feedback information is very simple,
it contains a header that identifies the file format,
including a version number.
What follows is a list of feedback information items.
When the file is read in the file format identifier and revision
number are checked,
the list of information items is checked to ensure that no two items
describe the same feedback information,
for example,
at most one item in this list can describe how to automatically parallelise
a program.

\paul{Should I explain the API in more detail.}



\section{Historic Automatic Parallelism in Mercury}

\paul{Jerome's work}

\paul{My honours thesis}

\paul{Should this section occur before profiling in Mercury, since it uses the
framework established by the deep profiler?
Or is this section a montivation for a discussion of the deep profiler?}

\chapter{Overlap}
\label{chap:overlap}
% vim: ts=4 sw=4 et ft=tex

\chapter{Overlap Analysis for Dependent AND-parallelism}
\label{chap:overlap}

\status{This has not been revised since the paper. I have only removed parts
of it to place elsewhere, for example, in the background section.}

\begin{itemize}
\item \paul{
    Add discussion about how our data-flow analysis does not recognise loops
    since they need to be treated specially.}

\item \paul{Explain the limitations of deep profiling,
      for example, that levels within recursive code do not have separate
      data recorded for them.}
\item \paul{Describe Callgraph search}
\item \paul{Minor changes to var use algorithm.}
\item \paul{Coverage (and var use) now work on deep profiling data.}
\end{itemize}

Introducing parallelism into a Mercury program is easy.
A programmer can use the parallel conjunction operator (an ampersand)
instead of the plain conjunction operator (a comma) to tell the compiler
that the conjuncts of the conjunction should be executed in parallel with
one another.
However,
in many places where parallelism can be introduced it will not be
profitable as it is not worth while parallelising small computations.
Making a task available to another CPU
may take thousands of instructions,
spawning off a task that takes only a hundred instructions is clearly a loss.
Even spawning off a task of a few thousand instructions is not a win;
it should only be done for computations
that take long enough to benefit from parallelism.
It is difficult for a programmer to determine if parallelisation of any
particular computation is worth while.
Researchers have therefore worked towards automatic parallelisation.
Autoparallelising compilers have long tried to use granularity analysis to
ensure that they only spawn off computations whose cost will probably exceed
the spawn-off cost by a comfortable margin.
However, this is not enough to yield good results,
because data dependencies may \emph{also} limit
the usefulness of running computations in parallel.
If a spawned off computation blocks almost immediately
and can resume only after another has completed its work,
then the cost of parallelisation again exceeds the benefit.

%This chapter is based on and extends our paper:
%
%\begin{quote}
%\pubauthor{Paul Bone, Zoltan Somogyi and Peter Schachte.}
%% This spelling of parallelisation is okay.
%\pubtitle{Estimating the overlap between dependent computations for automatic
%parallelization}
%\pubhow{Theory and Practice of Logic Programming,}{11(4--5):575--591, 2011.}
%\end{quote}

%\noindent
This chapter presents a set of algorithms for recognising places in a
program where it is worthwhile to execute two or more computations in
parallel,
algorithms that pay attention to the second of these issues as well as the
first.
Our system uses profiling information to compute
times at which a procedure call is expected to consume the values of its
input arguments
and the times at which it is expected to produce the values of its output
arguments.
Given two calls that may be executed in parallel,
our system uses the times of production and consumption
of the variables they share
to determine how much their executions are likely to overlap
when run in parallel,
and therefore whether executing them in parallel is a good idea or not.

We have implemented this technique for Mercury
in the form of a tool
that uses data from Mercury's deep profiler
to generate recommendations about what to parallelise.
The compiler applies these recommendations the next time the program is
compiled.
We present preliminary results that show that
this technique can yield useful parallelisation speedups,
while requiring nothing more from the programmer
than representative input data for the profiling run.

The structure of this chapter is as follows.
Section \ref{sec:overlap_aims} states our two aims for this chapter.
Then Section \ref{sec:overlap_approach} outlines our general approach
including information about the callgraph search for parallelisation
opportunities.
Section \ref{sec:overlap_reccalls}
describes how we calculate information about recursive calls missing from
the profiling data.
Section \ref{sec:overlap_coverage}
describes our change to coverage profiling which provides more accurate
coverage data for the new callgraph based search for parallelisation
opportunities.
Section \ref{sec:overlap_overlap_alg} describes our algorithm for
calculating the execution overlap between two or more dependent conjuncts.
A conjunction with more than two conjuncts can be parallelised
in several different ways;
section \ref{sec:overlap_howto} shows how we choose the best way.
Section \ref{sec:overlap_pragmatic} discusses some pragmatic issues.
Section \ref{sec:overlap_perf} evaluates
how our system works in practice on some example programs, and
section \ref{sec:overlap_conc} concludes
with comparisons to related work.

\section{Aims}
\label{sec:overlap_aims}

When parallelising Mercury programs,
the best parallelisation opportunities occur
where two goals take a significant and roughly similar time to execute.
Their execution time should be as large as possible
so that the relative costs of parallel execution are small,
and they should be independent to minimise synchronisation costs.
Unfortunately, goals expensive enough to be worth executing in parallel
are rarely independent.
For example, in the Mercury compiler itself,
there are 53 conjunctions containing two or more expensive goals,
but in only one of those conjunctions are the expensive goals independent.
This is why Mercury supports the parallel execution of dependent conjunctions
through the use of futures and a compiler transformation
\citep{wang:2011:dep-par,wang:2006:hons} (Section \ref{sec:backgnd_deppar}).
If the \emph{consumer} of the variable attempts to retrieve the variable's value
before it has been produced, then its execution is blocked
until the \emph{producer} makes the variable available.

\picfigure{overlap_compare}{Ample vs smaller parallel overlap between \code{p} and \code{q}}

Dependent parallel conjunctions differ widely
in the amount of parallelism they have available.
Consider a parallel conjunction with two similarly-sized conjuncts,
\code{p} and \code{q}, that share a single variable \code{A}.
If \code{p} produces \code{A} late but \code{q} consumes it early,
as shown on the right side of figure \ref{fig:overlap_compare},
there will be little parallelism,
since \code{q} will be blocked soon after it starts,
and will be unblocked only when \code{p} is about to finish.
Alternatively, if \code{p} produces \code{A} early
and \code{q} consumes it late,
as shown on the left side of in figure \ref{fig:overlap_compare},
we would get much more parallelism.
The top part of each scenario
shows the execution of the sequential form of the conjunction.

Unfortunately, in real Mercury programs,
almost all conjunctions are dependent conjunctions,
and in most of them,
shared variables are produced very late and consumed very early.
Parallelising them would therefore yield slowdowns instead of speedups,
because the overheads of parallel execution would far outweigh the
benefit of the small amount of parallelism that is available.
We want to parallelise only conjunctions
in which any shared variables are produced early, consumed late,
or (preferably) both;
such computations expose more parallelism.
The first purpose of this chapter is to show how one can find these conjunctions.

\begin{figure}
\begin{center}
\begin{minipage}[b]{0.49\textwidth}
\subfigure[Sequential \mapfoldl]{%
\label{fig:map_foldl_seq}
{\small
\begin{tabular}{l}
\code{map\_foldl(\_, \_, [], Acc, Acc).} \\
\code{map\_foldl(M, F, [X $|$ Xs], Acc0, Acc) :-} \\
\code{~~~~M(X, Y),} \\
\code{~~~~F(Y, Acc0, Acc1),} \\
\code{~~~~map\_foldl(M, F, Xs, Acc1, Acc).} \\
\end{tabular}}
}
\end{minipage}
%
\begin{minipage}[b]{0.49\textwidth}
\subfigure[Parallel \mapfoldl with overlap]{%
\label{fig:map_foldl_par}
{\small
\begin{tabular}{l}
\code{map\_foldl(\_, \_, [], Acc, Acc).} \\
\code{map\_foldl(M, F, [X $|$ Xs], Acc0, Acc) :-} \\
\code{~~~~(} \\
\code{~~~~~~~~M(X, Y),} \\
\code{~~~~~~~~F(Y, Acc0, Acc1)} \\
\code{~~~~) \&} \\
\code{~~~~map\_foldl(M, F, Xs, Acc1, Acc).} \\
\end{tabular}}
}
\end{minipage}

\end{center}
\caption{Sequential and parallel \mapfoldl}
% the recursive call is less dependent
% on the conjunction of the first two calls.
\label{fig:map_foldl}
%\vspace{-2\baselineskip}
\end{figure}

\picfigure{mapfoldl-overlap}{Overlap of \mapfoldl (figure
\ref{fig:map_foldl_par})}

The second purpose of this chapter is to find the best way to parallelise
these conjunctions.
Consider the \mapfoldl predicate in figure~\ref{fig:map_foldl_seq}.
The body of the recursive clause has three conjuncts.
We could make each conjunct execute in parallel,
or we could execute two conjuncts in sequence
(either the first and second, or the second and third),
and execute that sequential conjunction in parallel with the remaining conjunct.
In this case, there is little point in executing
the higher order calls to the \M and \F predicates
%(herein map and fold respectivly)
in parallel with one another,
since in virtually all cases,
\M will generate \code{Y} very late and
\F will consume \code{Y} very early.
However, executing the sequential conjunction of the calls to \M and \F
in parallel with the recursive call \emph{will} be worthwhile
if \M is time-consuming,
because this implies that
a typical recursive call will consume its fourth argument late.
The recursive call processing the second element of the list
will have significant execution overlap (mainly the cost of \M)
with its parent processing the first element of the list
even if (as is typical) the fold predicate generates \code{Acc1} very late.
This parallel version of \mapfoldl is shown in figure
\ref{fig:map_foldl_par}.
A representation of the first three iterations of it 
is shown in figure \ref{fig:mapfoldl-overlap}.
(This is the kind of computation that
Reform Prolog \citep{bevemyr:reform} was designed to parallelise.)

\section{Our general approach}
\label{sec:overlap_approach}

\plan{Goal}
We want to find the conjunctions in the program
whose parallelisation would be the most profitable.
This means finding the conjunctions with conjuncts
whose execution cost exceeds the spawning-off cost by the highest margin,
and whose interdependencies, if any,
allow their executions to overlap the most.
It is better to spawn off a medium-sized computation
whose execution can overlap almost completely
with the execution of another medium-sized computation,
than it is to spawn off a big computation
whose execution can overlap only slightly
with the execution of another big computation,
but it is better still to spawn off a big computation
whose execution can overlap almost completely
with the execution of another big computation.
Essentially, the more the tasks' executions can overlap with one another,
the greater the margin by which
the likely runtime of the parallel version of a conjunction beats
the likely runtime of the sequential version (speedup),
and the more beneficial parallelising that conjunction will be.

\plan{Profiler feedback}
To compute this likely benefit,
we need information
both about the likely cost of calls
and the execution overlap allowed by their dependencies.
A compiler may be able to estimate some cost information from static
analysis.
However, this will not be accurate;
static analysis cannot take into account sizes of data terms or other
values that are only available at runtime.
Therefore, we use profiler feedback information in our implementation.
This was introduced in section \ref{sec:backgnd_autopar}
which also includes a description of Mercury's deep profiler.
To generate the profiler feedback data,
we require programmers to follow this sequence of actions after they have
tested and debugged the program.

\begin{enumerate}
\item
Compile the program
with options asking for profiling
for automatic parallelisation.
\item
Run the program on a representative set of input data.
This will generate a profiling data file.
\item
Invoke our feedback tool on the profiling data file.
This will generate a parallelisation feedback file.
\item
Compile the program for parallel execution,
specifying the feedback file.
The file tells the compiler
\emph{which} sequential conjunctions to convert to parallel conjunctions,
and exactly \emph{how}.
For example, \code{c1, c2, c3} can be converted
into \code{c1 \& (c2, c3)},
into \code{(c1, c2) \& c3}, or
into \code{c1 \& c2 \& c3},
and as the \code{map\_foldl} example shows,
the speedups you get from them can be strikingly different.
\end{enumerate}

\noindent
A visual representation of such a workflow is shown in figure
\ref{fig:prof_fb} on page \pageref{fig:prof_fb}.
It is up to the programmer using our system
to select training input for the profiling run in step 2.
Obviously, programmers should pick input that is as representative as possible,
but the recommended parallelisation can be useful
even for input data that is quite different from the training input.
The main focus of this chapter is on step 3;
we give the main algorithms used by the feedback tool.
However, we will also touch on steps 1 and 4.
We believe that step 2 can only be addressed by the programmer,
as they understand what input is representative for their program.

\plan{DFS \& limits}
Our feedback tool looks for parallelisation opportunities
by doing a depth-first search of the call tree of the profling run,
each node of which is an SCC (strongly connected component) of procedures.
It explores the subtree below a node in the tree
only if the per-call cost of the subtree is greater than a configurable
threshold,
and if the amount of parallelism it has found at and above that node
is below another configurable threshold.
The first test lets us avoid looking at code
that would take more work to spawn off than to execute,
while the second test lets us avoid creating
more parallel work than the target machine can handle.
Together these tests dramatically reduce the portions of a program that need
analysis,
reducing the time required to search for parallelisation opportunities.

For each procedure in the call tree,
we search its body for conjunctions that contain two or more calls with
execution times above yet another configurable threshold.
This test also reduces the parts of the program that will be analysed
further;
it quickly rejects procedures that cannot contain any profitable
parallelism.
Parallelising a conjunction
requires partitioning the original conjuncts into two or more groups,
with the conjuncts in each group being executed sequentially
but different groups being executed in parallel.
Each group represents a hypothetical sequential conjunction,
and the set of groups represents a hypothetical parallel conjunction.
As this parallel conjunction represents a possible parallelisation of the
original conjunction, we call it a \emph{candidate parallelisation}.
Most conjunctions can be partitioned into several alternative candidate
parallelisations,
for example, we showed above that \mapfoldl has three alternative
parallelisations of its recursive branch.
We use the algorithms of section \ref{sec:overlap_overlap_alg}
to compute the expected parallel execution time of each parallelisation.
These algorithms take into account the runtime overheads of parallel execution.
However,
large conjunctions can have a very large number of
parallelisations ($2^{n-1}$ for $n$ conjuncts).
Therefore,
we use the algorithms of section \ref{sec:overlap_howto}
to heuristically reduce the number of parallelisations whose expected
execution time we calculate.
If the best-performing parallelisation we find
shows a nontrivial speedup over sequential execution,
we remember that we want to perform that parallelisation on this conjunction.
A procedure can contain several conjunctions with two or more goals that we
consider parallelising,
therefore multiple candidate parallelisations may be generated for different
conjunctions in a procedure.
The same procedure may also appear more than once in the call graph,
and therefore multiple parallelisations may be generated for the same
conjunctions within the procedure.
We discuss how we resolve conflicting recomentations for the same procedure
in section \ref{sec:overlap_pragmatic}.

\paul{Move this later when we need to talk about program representation and
coverage}
Our feedback tool is an extension of the Mercury deep profiler.
The profiler is able to access the relevant parts of the compiler's
representation of the program.
This includes a representation of each procedure body,
and for each atomic subgoal (call or unification) within each body,
the set of variables bound or referenced by that subgoal.
We also use coverage profiling (Section \ref{sec:backgnd_coverage})
which records how many times execution reaches each point in the program,
this number is the \emph{coverage} of that point in the program.
It is possible to figure this out for \emph{most} program points using
only the call counts associated with call sites.
However, we require complete coverage information;
coverage profiling provides this using a minimum of extra profiling
instrumentation.
As we will see in section~\ref{sec:overlap_overlap_alg},
we need the complete coverage information
to calculate the likely speedup from parallelising a conjunction.


\paul{Move this paragraph elsewhere, conclusion or introduction?}
An important benefit of profile-directed parallelisation is that
since programmers do not annotate the source program,
it can be re-parallelised easily after a change to the program
obsoletes some old parallelisation opportunities and creates others.
Nevertheless, if programmers want to parallelise some conjunctions manually,
they can do so: our system will not override the programmer.

% \section{Traversing the callgraph}
% \label{sec:overlap_dfs}
% 
% % XXX Further work.
% \paul{TODO: This feature is not yet implemented.}
% If the depth first search later finds
% some of the conjuncts to have parallelisable code inside them,
% we revisit this conjunction,
% this time using updated data about the cost of those conjuncts.
% Otherwise,
% we add a recommendation to perform the selected parallelisation
% to the feedback advice we generate for the compiler.



% \paul{This is not yet implemented and will not be for this version of the paper.}
% \peter{Then you need to say that.}

% GREEDY_SEARCH The top level algorithm of the feedback tool
% GREEDY_SEARCH is a traversal of the tree of cliques
% GREEDY_SEARCH recorded in the deep profiling data file.
% GREEDY_SEARCH Each clique has its own unique entry point,
% GREEDY_SEARCH which will be a call site in a higher clique;
% GREEDY_SEARCH this higher clique is the parent node of this clique.
% GREEDY_SEARCH Likewise, every call site
% GREEDY_SEARCH in every procedure in the clique
% GREEDY_SEARCH will be the entry point of another clique,
% GREEDY_SEARCH provided that
% GREEDY_SEARCH (a) it is actually executed and (b) the callee is not in this clique.
% GREEDY_SEARCH These lower cliques are the children of this clique.

% GREEDY_SEARCH % We will describe our traversal algorithm in detail
% GREEDY_SEARCH % in section \ref{sec:bestfirst},
% GREEDY_SEARCH % but for now, consider this traversal
% GREEDY_SEARCH % as operating on a \emph{candidates list},
% GREEDY_SEARCH % a list of cliques sorted on total cost.
% GREEDY_SEARCH Our traversal algorithm operates on a \emph{candidates list},
% GREEDY_SEARCH which contains a list of cliques sorted on total cost.
% GREEDY_SEARCH We start with the list containing only
% GREEDY_SEARCH the clique of the top level call to \code{main},
% GREEDY_SEARCH the predicate where every Mercury program starts execution.
% GREEDY_SEARCH Then, at each step,
% GREEDY_SEARCH \begin{itemize}
% GREEDY_SEARCH \item
% GREEDY_SEARCH we remove the clique at the start of the candidates list;
% GREEDY_SEARCH \item
% GREEDY_SEARCH we process this clique
% GREEDY_SEARCH by looking at the conjunctions in the clique's procedures
% GREEDY_SEARCH to see whether they should be parallelised; and then
% GREEDY_SEARCH \item
% GREEDY_SEARCH we insert the child cliques (if any) of this clique into the candidates list.
% GREEDY_SEARCH \end{itemize}
% GREEDY_SEARCH We stop when either even the highest cost candidate
% GREEDY_SEARCH is too cheap to be worth parallelising,
% GREEDY_SEARCH or we have achieved our target CPU utilisation
% GREEDY_SEARCH for all phases of the program's execution.

% GREEDY_SEARCH This is only an outline of our traversal algorithm.
% GREEDY_SEARCH In section \ref{sec:pragmatic}, we will describe it in detail,
% GREEDY_SEARCH together with our solutions to several issues that come up in practice.


% \zoltan{this is wrong: the overheads should be PART OF the parallel time}
% \begin{equation*}
% Speedup = \frac{Time_{Seq}}{Time_{Par} + ParOverheads}
% \end{equation*}

\section{The cost of recursive calls}
\label{sec:overlap_reccalls}

% leave discussion of granularity estimation by static analysis
% for the related work section;
% mention that this work has not extended to large programs.

The Mercury deep profiler gives us directly
the costs of all non-recursive call sites in a clique.
For recursive calls,
the costs of the callee are mingled together
with the costs of the caller,
which is either the same procedure as the callee,
or is mutually recursive with it.
Therefore if we want to know the cost of a recursive call site (and we do),
we have to infer this
from the cost of the clique as a whole,
the cost of each call site within the procedures of the clique,
the structures of the bodies of those procedures,
and the frequency of execution of each path through those bodies.

For now, we will restrict our attention to SCCs
that contain only a single procedure and where that procdure matches one of
the three recursion patterns below.
These are among the most commonly used recursion patterns and the
inference processes for them are also among the simplest.
Later, we will discuss how partial support could be added for mutually
recursive procedures.

{\bf Pattern 1: no recursion at all.}
This is not a very interesting pattern, but we support it completely.
We do not need to compute the costs of recursive calls if a procedure is not
recursive.

{\bf Pattern 2: simply recursive procedures.}
The first pattern consists of procedures whose bodies
have just two types of execution path through them:
base cases, and recursive cases containing a single recursive call site.
Our example for this category is \code{map\_foldl},
whose code is shown in figure \ref{fig:map_foldl}.

Let us say that of the 100 calls to the procedure,
90 were from the recursive call site
and 10 were from a call site in the parent SCC.
Then we would calculate
that each non-recursive call
(from the parent SCC)
would on average yield nine recursive calls
(from within the SCC).
We call this the average deepest recursion:

\begin{equation*}
AvgMaxDepth = Calls_{RecCallSites} / Calls_{ParentCallSite}
\end{equation*}

The deepest call executes only the non-recursive path,
and incurs only its costs.
The next deepest would take the recursive path,
and incur one copy of the costs of the non-recursive calls along that path,
plus the cost of the last call.
The third last would incur two copies of the costs
of the non-recursive calls along the recursive path,
plus the cost of the last call.
By induction,
the cost of a recursive call at depth $D$ (0 being the deepest)
is:

\begin{equation*}
cost(D) = CostNonRec + D(CostRec + 1)
\end{equation*}

We can now calculate the average cost of a call at any level of the
recursion.
Simply recursive procedures have a uniform number of calls at each depth of
the recursion.
The depth representing the typical use of such a procedure is half of
$AvgMaxDepth$.
This allows us to calculate the typical cost of a recursive call from this
call site.
For example, if \mapfoldl's non-recursive path cost is 10 call sequences
counts (a unit defined in section \ref{sec:backgnd_deep}),
its recursive path cost is 100csc,
and its typical depth is $9/2 = 4.5$.
Then its typical cost is $10 + 4.5(100 + 1) = 464.5$ call sequence counts.

\begin{figure}[tb]
\begin{center}
\begin{minipage}[b][1.9in]{0.49\textwidth}
\subfigure[Accumulator quicksort]{%
\label{fig:quicksort_acc}
\begin{tabular}{l}
\code{quicksort([], Acc, Acc).} \\
\code{quicksort([Pivot $|$ Xs], Acc0, Acc) :-} \\
\code{~~~~partition(Pivot, Xs, Lows, Highs),} \\
\code{~~~~quicksort(Lows, Acc0, Acc1),} \\
\code{~~~~quicksort(Highs, [Pivot $|$ Acc1], Acc).} \\
% Add whitespace to shift the table upwards without also moving the caption.
\\
\\
\\
\end{tabular}
}
\hfill
\end{minipage}
\begin{minipage}[b][1.9in]{0.49\textwidth}
\subfigure[Callgraph]{%
\includegraphics[width=0.98\textwidth]{pics/call_tree_dc}
\label{fig:quicksort_acc_callgraph}
}
\hfill
\end{minipage}
\end{center}
\vspace{-2ex}
\caption{Accumulator quicksort, definition and callgraph}
\end{figure}

{\bf Pattern 3: Divide-and-conquer procedures.}
The third pattern consists of procedures whose bodies
that also have just two types of execution path through them:
base cases, and recursive cases containing \emph{two} recursive call sites.
Our example for this category is an accumulator version of \quicksortacc,
whose code is shown in figure~\ref{fig:quicksort_acc}.

Calculating the recursion depth of \quicksortacc can be more
problematic.
We know that if a good value for \code{Pivot} is chosen \quicksortacc runs
optimally,
dividing the list in half with each recursion.
Figure \ref{fig:quicksort_acc_callgraph} shows the callgraph for such an
invocation of \quicksortacc.
This graph has 15 nodes excluding ``main'',
each node has exactly one call leading to it;
therefore there is one call from outside \quicksortacc's SCC
(the call from ``main''),
and 14 calls within the SCC
(the calls from ``qs'' nodes).
By inspection, there are four complete levels of recursion.
There are always $\log_2(N+1)$ levels in a divide and conqure call graph
with $N$ nodes,
and since there are always $N-1$ calls from within the SCC for a graph with
$N$ nodes then
it follows that,
there are $\log_2(C+2)$ levels for a divide and conqure call graph with $C$
recursive calls.
In this exaple $C = 14$ and therefore there are four levels of recursion
as we noded above by inspection.
If there were two invocations of \quicksortacc from \code{main/2} then there
would be two calls from outside the SCC and 28 calls from within,
in this case the depth is still four.
The average maximum recursion depth in terms of call counts for divide and
conqure code is therefore:

\begin{equation*}
AvgMaxDepth = \log_2 
	\left(\frac{Calls_{RecCallSites}}{Calls_{ParentCallSite}} + 2\right)
\end{equation*}

If consistently worst-case pivots are chosen then \quicksortacc fails to
divide the list at each recursion passing an empty list to one recursive call
and a list whose length is one item shorter than the input list to the other
call.
Pathologically bad cases are rare and are considered bugs;
programmers will usually want to remove such bugs in order to improve
sequential execution performance.
Slight imbalance in a callgraph is a more common occurance,
such situtations fall into the same class as those where the profiling data
is not quite representative of the optimised program's future use,
or our estimates of parallel execution overheads do not match the hardware's
actual performance.
Such variations in our calcuations inputs rarely change a ``should
parallelise'' decision into a ``should not parallelise'' decision and
\emph{vice-versa}.
Therefore, we assume that all divide and conqure code is, on average,
evenlly balanced.

As before, the deepest call executes only the non-recursive path,
and incurs only its costs.
The next deepest takes the recursive path,
it incurs the costs of the goals along that path,
plus twice the base case's cost, plus the costs of calls to the base case.
By induction,
the cost of a recursive call in a divide and conquer procedure at depth $D$ is:

\begin{equation*}
cost(D) = 2^D{CostNonRec} + 2^{D-1}(CostRec + 1)
\end{equation*}

\plan{interesting depth of d\&c}
Most of the execution of a divide and conquer algorithm occurs at deep
recursion levels as there are many more calls made at these levels than higher
levels.
However, for parallelism the high recursion levels are more interesting:
we know that parallelising the top of the algorithm's call graph can provide
ample coarse-grained parallelism.
\plan{quicksort example}
For example, lets compute the costs of the recursive calls at the top of
\quicksortacc's call graph.
In this example, we gathered data using Mercury's deep profiler on 10
executions of \quicksortacc sorting a list of 32,768 elements.
the profiler reports that there are 655,370 calls into this SCC,
10 of which come from the parent SCC, leaving 655,360 from the two call sites
within \quicksortacc's SCC.
With each recursive call the size of the list is roughly halved,
thus on average there are $\log_{2}(655,360/10) = 16$ levels of recursion.
This is not quite accurate for the case of \quicksortacc;
there are two reasons why:
it is possible to choose a bad value for \code{Pivot} and,
at each level \code{Pivot} is removed from the list before partitioning it.
It is more accurate for mergesort for example.
The total per-call cost of the call site to partition reported by the profiler
is approximately 35.5csc,
it is the only other goal in either the base case or recursive case with a
non-zero cost.
The call at the 16\textsuperscript{th} level of recursion is the call from the
parent SCC into \quicksortacc.
We want to calculate the cost of the calls in the next level down from the
parent,
the 15\textsuperscript{th} level, which is:
$0 + 2^{15 - 1}(35.5 + 1) \approx 591,462$.
This is the average cost of both of the two recursive calls;
depending on how the list was partitioned,
either sub-list could be larger, making the other one
smaller and affecting the costs of the recursive calls similarly.
However the deep profiler does not provide this type of information and we
believe that on average the costs computed here will be good enough.
Since the deep profiler reports that the total per-call cost of the
\quicksortacc SCC is 1,229,106csc, we can see that the above figure makes
sense.

%\begin{algorithm}
%\begin{algorithmic}
%\Procedure{classify}{$Goal$}
%    \Switch{$Goal$}
%        \Case unify or builtin call
%            \State $Res \gets [\reccallres{0}{1}{0}]$
%        \EndCase
%        \Case non-recursive call
%            \State $Res \gets [\reccallres{0}{1}{\cost{Goal}}]$
%        \EndCase
%        \Case recursive call
%            \State $Res \gets [\reccallres{1}{1}{0}]$
%        \EndCase
%        \Case conjunction
%            \State $Res \gets $classify\_conj$($conjuncts$(Goal))$
%        \EndCase
%        
%    \EndSwitch
%\EndProcedure
%\end{algorithmic}
%\end{algorithm}

\begin{figure}
\begin{center}
\begin{tabular}{rlrr}
\C{Line} & \C{Code}         & \C{Coverage}  & \C{Cost}  \\
 1  & \code{p(X, Y, ...) :-}&         100\% &           \\
 2  & \code{~~~~(}          &               &           \\
 3  & \code{~~~~~~~~X = a,} &          20\% &         0 \\
 4  & \code{~~~~~~~~p(...)} &          20\% &           \\
 5  & \code{~~~~;}          &               &           \\
 6  & \code{~~~~~~~~X = b,} &          60\% &         0 \\
 7  & \code{~~~~~~~~q(...)} &          60\% &     1,000 \\
 8  & \code{~~~~;}          &               &           \\
 9  & \code{~~~~~~~~X = c,} &          20\& &         0 \\
10  & \code{~~~~~~~~r(...)} &          20\% &     2,000 \\
11  & \code{~~~~),}         &               &           \\
12  & \code{~~~~(}          &               &           \\
13  & \code{~~~~~~~~Y = d,} &          10\% &         0 \\
14  & \code{~~~~~~~~p(...)} &          10\% &           \\
15  & \code{~~~~;}          &               &           \\
16  & \code{~~~~~~~~Y = e,} &          90\% &         0 \\
17  & \code{~~~~~~~~s(...)} &          90\% &    10,000 \\
18  & \code{~~~~).}         &               &           \\
\end{tabular}
\end{center}
\caption{Two recursive calls and six code paths.}
\label{fig:2_reccalls_4_paths}
\end{figure}

\plan{How we classify recursion type}
We classify recursion types with an algorithm that walks over the structure
of a procedure.
As it traverses the procedure,
it counts the number of recursive calls along each path,
the path's cost and the number of times the path is executed.
When a branching structure like an if-then-else or switch is found,
the algorithm processes each branch independently and then merges its
results at the end of the branch.
If several branches have the same number of recursive calls (including zero)
they can be merged.
If several branches have different numbers of recursive calls they are all
added to the result set.
This means that the result of traversing a goal might include data for
several different recursion counts.
Consider the example in figure \ref{fig:2_reccalls_4_paths}.
The example code has been annotated with coverage information (in the
third column) and with cost information where it is available (forth
column). 
The conjunction on lines six and seven does not contain a recursive call.
The result of processing it is a list containing a single tuple:
\code{[(reccalls: 0, coverage: 60\%, cost: 1,000)]}.
The other conjunction in the same switch (lines three and four) does contain
a recursive call.
The result of processing it is a list containing the tuple:
\code{[(reccalls: 1, coverage: 20\%, cost: 0)]}.
The result for the remaining switch arm (lines 9 and 10) is:
\code{[(reccalls: 0, coverage: 20\%, cost: 2,000)]}.
When the algorithm is finished processing all the cases in the switch it
adds them together;
When adding tuples, we can add tuples together with the same number of
recursive calls by adding their coverage and adding their costs weighted by
coverage (these are per-call costs).
This simplifies multiple code paths with the same number of recursive calls
into a single ``code path''.
The result of processing the switch from line 2--11 is:

\noindent
\begin{center}
\begin{tabular}{l}
\code{[(reccalls: 0, coverage: 80\%, cost: 1,250),}  \\
\code{~(reccalls: 1, coverage: 20\%, cost: ~~~~0)]}. \\
\end{tabular}
\end{center}

\noindent
In this way, the result of processing a goal represents all the possible
code paths through that goal.  In this case there are three code paths
through the switch,
and the result has two entries, one represents the two base case code paths,
the other represents the single recursive case.

The result of processing the other switch in the example,
lines 12--18, is:

\noindent
\begin{center}
\begin{tabular}{l}
\code{[(reccalls: 0, coverage: 90\%, cost: 10,000),}  \\
\code{~(reccalls: 1, coverage: 10\%, cost: ~~~~~0)]}. \\
\end{tabular}
\end{center}

\noindent
In order to compute the result for the whole procedure, we must compute the
product of these two results;
this computes all the possible paths through the two switches.
We do this by constructing pairs of tuples from the two lists.
Since each list has two entries there are four pairs:

\noindent
\begin{center}
\begin{tabular}{rcl}
\code{[   (rc: 0, cvg: 80\%, cost: ~1,250)} &
    $\times$&
    \code{(rc: 0, cvg: 90\%, cost: 10,000),}
    \\
\code{   ~(rc: 0, cvg: 80\%, cost: ~1,250)} &
    $\times$&
    \code{(rc: 1, cvg: 10\%, cost: ~~~~~0),}
    \\
\code{   ~(rc: 1, cvg: 20\%, cost: ~~~~~0)} &
    $\times$&
    \code{(rc: 0, cvg: 90\%, cost: 10,000),}
    \\
\code{   ~(rc: 1, cvg: 20\%, cost: ~~~~~0)} &
    $\times$&
    \code{(rc: 1, cvg: 10\%, cost: ~~~~~0)]}
    \\
\end{tabular}
\end{center}

\noindent
For each pair we compute a new tuple by adding the number of recursive
calls, averaging the coverage counts, and adding the costs.

\noindent
\begin{center}
\begin{tabular}{l}
\code{[   (rc: 0, cvg: 72\%, cost: 11,250),} \\
\code{   ~(rc: 1, cvg: ~8\%, cost: ~1,250),} \\ 
\code{   ~(rc: 1, cvg: 18\%, cost: 10,000),} \\
\code{   ~(rc: 2, cvg: ~2\%, cost: ~~~~~0),} \\
\end{tabular}
\end{center}

\noindent
Again, we can merge the cases with the sane numbers of recursive calls.

\noindent
\begin{center}
\begin{tabular}{l}
\code{[   (rc: 0, cvg: 72\%, cost: 11,250),} \\
\code{   ~(rc: 1, cvg: 26\%, cost: ~1,319),} \\ 
\code{   ~(rc: 2, cvg: ~2\%, cost: ~~~~~0)]} \\
\end{tabular}
\end{center}

\noindent
There are six paths through this procedure,
and two recursive calls.
However, we can represent all six paths using just three tuples,
one for each path type.
This allows us to conveniently handle procedures of different forms as
rather simple recursion types such as ``simply recursion'' we saw above:
a procedure with two recursive paths with one call each can be handled as
if it has just one recursive path and one base case.

We can determine the type of recursion for any list of recursion path
information.
If there is a single path entry with zero recursive calls
then the procedure is not recursive.
If there is an entry for a path with zero recursive calls,
and a path with one recursive call
then the procedure is ``simply recursive''.
Finally if there are two paths, one with zero recursive calls and one with
two recursive calls, then we know that the procedure
uses ``divide and conquer'' recursion.
It is possible to generalise further, if a procedure has two entries,
one with zero recursive calls and the other with some $N$ recursive calls.
Then the recursion pattern is similar to divide and conquer except that the
base in the formulas shown above is $N$ rather than 2.
We have not found it necessary to handle these cases.

\paul{Clarify if the table shows static or dynamic (deep) procedure counts.}

\begin{table}
\begin{center}
\begin{tabular}{l|rrr}
Recursion Type & No.\ of Procedures & Percent & Total cost \\
\hline
Not recursive  &            292,893 & 78.07\% & 2,320,270,385 \\
Simple recursion&            48,458 & 12.92\% &   402,430,967 \\
Divide and conquer&           1,917 &  0.51\% &     5,337,293 \\
Mutual recursion: 2 procs &   4,066 &  1.08\% &    27,099,504 \\
Mutual recursion: 3 procs &   9,393 &  2.50\% &    14,846,076 \\
Mutual recursion: 4 procs &   1,092 &  0.29\% &    12,542,308 \\
Mutual recursion: 5 procs &   1,035 &  0.28\% &     3,863,295 \\
Mutual recursion: other   &   3,707 &  0.99\% &   139,975.394 \\

Other recursion with rec-branches: 1, 2
                           &     44 &  0.01\% &       580,994 \\
Other recursion with rec-branches: 2, 3
                           &    281 &  0.07\% &       189,377 \\
Other recursion with rec-branches: 2, 3, 4
                           &    564 &  0.15\% &     3,902,188 \\
Other multi recursive-branch:
                           &    200 &  0.05\% &         5,908 \\
Unknown (built-in \& foreign language code) 
						   & 10,623 &  2.83\% &           364 \\
Unknown (error)            &  1,180 &  0.32\% &     8,838,291 \\
\end{tabular}
\end{center}
\caption{Survey of recursion types in an execution of the Mercury compiler}
\label{tab:recursion_types}
\end{table}

\plan{Recursion type survey}
There are many possible types of recursion,
and we wanted to limit our development effort to just those recursion types
that would occur often enough to be important.
Therefore,
we ran our analysis across all the procedures in the Mercury compiler,
the largest open source Mercury program,
to determine which types of recursion are most common.
Table \ref{tab:recursion_types} summarises the results.
We can see that most procedures are not recursive,
and the next biggest group is the simply recursive procedures,
having nearly 13\% of the execution.
If our analysis finds an SCC with more than one procedure,
it counts the number of procedures and marks the SCC's procedures as
mutually recursive and does no further analysis.
It may be possible to resolve some of these SCCs into simple recursion,
but we have not found it important to do this yet.
Mutual recursion as a whole accounts for a larger proportion of procedures
than divide and conquer.
The ``Other'' recursion types refer to cases where there are multiple
recursive paths through the procedure with different numbers of recursive
calls plus a base case,
however we have not classified these cases further.
Some of these ``Other'' cases are due to the implementation of the
\code{map} ADT and library module,
which uses a 2--3--4 tree implementation, and hence many of the
mutli-recursive path cases with 2, 3 or 4 recursive calls on a path
are often 2--3--4 tree algorithms.
These algorithms also account for some of the other multi recursive path
cases,
such as those with 2 or 3 recursive calls on a path,
these may be due to the same algorithms running on a tree without any
4-nodes.
The table row labelled ``Unknown (error)'' refers to cases that our algorithm
could not handle.
These are primarily procedures that may backtrack because they are either
\dnondet or \dmulti.

\plan{Mutual recursion}

\plan{What is unimplemented}
Our current implementation does not attempt to parallelise divide and conquer
code.

\section{Deep coverage information}

\paul{Describe 'deep' coverage}

\section{Calculating the overlap between dependent conjuncts}
\label{sec:overlap_overlap_alg}

As we can see from the difference between the two sides of
figure~\ref{fig:dep_conj_overlap1},
figuring out the overlap
in the parallel executions of two dependent conjuncts
requires knowing, for each of the variables they share,
when that variable is generated by the first conjunct and
when it is first consumed by the second conjunct.
Our algorithms for computing these times are considerably simplified
by the Mercury mode system
and by the fact that we only parallelise deterministic goals.

The profiling data gives us both
the total execution time of each conjunct
and its number of invocations;
the ratio of the two is the expected execution time for each invocation.


Suppose the first appearance of the variable (call it $X$)
in a conjunction $G_1, \ldots, G_n$ is in $G_k$, and $G_k$ is a switch.
If $X$ is consumed by some switch arms and not others,
then on some execution paths,
the first consumption of the variable may be in $G_k$ (a),
on some others it may be in $G_{k_1}, \ldots, G_n$ (b),
% XXX: We do not handle c - but we should.
and on some others it may not be consumed at all (c).
For case (a),
we compute the average time of first consumption by the consuming arms,
and then compute the weighted average of these times,
with the weights being the probability of entry into each arm, as before.
For case (b), we compute the probability of entry
into arms which do \emph{not} consume the variable,
and multiply the sum of those probabilities
by the weighted average of those arms' execution time
\emph{plus} the expected consumption time of the variable
in $G_{k+1},~\ldots,~G_n$.
For case (c)
we pretend $X$ is consumed at the very end of the goal,
and then handle it in the same way as (b).
This is because for our overlap calculations,
a goal that does not consume a variable is equivalent to
a goal that consumes it at the end of its execution.

% Show formulas for calculating the overlap in a simple case, one
% shared variable between two conjuncts.

Suppose a candidate parallel conjunction has two conjuncts $p$ and $q$,
and their execution times in the original, sequential conjunction $p, q$,
are ${SeqTime}_p$ and ${SeqTime}_q$.
Suppose ${SV}_i$ are the variables shared between them,
and for each ${SV}_i$,
the time at which $p$ produces it is ${ProdTime}_{pi}$, and
the time at which $q$ consumes it is ${ConsTime}_{qi}$.

If we denote the execution times of the conjuncts
in the parallel conjunction $p~\&~q$
as ${ParTime}_p$ and ${ParTime}_q$,
then the expected speedup
from parallelising the original sequential conjunction
is ${Speedup} = {SeqTime} / {ParTime}$,
where ${SeqTime} = {SeqTime}_p + {SeqTime}_q$,
and ${ParTime} = {SpawnOverhead} + {max}({ParTime}_p, {ParTime}_q)$.
The profile gives us ${SeqTime}_p$ and ${SeqTime}_q$,
and if we ignore overheads for now (we will come back to them later),
then ${ParTime}_p$ will always be equal to ${SeqTime}_p$.
The main task of computing the speedup
therefore consists of computing ${ParTime}_q$;
as we saw in figure~\ref{fig:dep_conj_overlap1},
this will differ from ${SeqTime}_q$
whenever $q$ needs to wait for $p$ to produce a shared variable.

\begin{figure}[tb]
\begin{center}
\begin{verbatim}
find_par_time(Conjs) returns TotalParTime:
  N := length(Conjs)
  ProdTimeMap := empty
  TotalParTime := 0
  for i in 1 to N:
    CurSeqTime := 0
    CurParTime := 0
    sort ProdConsList_i on Time_ij
    forall (Var_ij, Time_ij) in ProdConsList_i:
      Duration_ij := Time_ij - CurSeqTime
      CurSeqTime := CurSeqTime + Duration_ij
      if Conj_i produces Var_ij:
        CurParTime := CurParTime + Duration_ij
        ProdTimeMap[Var_ij] := CurParTime
      else Conj_i must consume Var_ij:
        ParWantTime := CurParTime + Duration_ij
        CurParTime := max(ParWantTime, ProdTimeMap[Var])
    DurationRest_i := SeqTime_i - CurSeqTime
    CurParTime := CurParTime + DurationRest_i
    TotalParTime := max(TotalParTime, CurParTime)
\end{verbatim}
\end{center}
\caption{Dependent parallel conjunction algorithm}
\label{fig:dep_par_conj_overlap_middle}
%\vspace{-2\baselineskip}
\end{figure}

% XXX \iclp{
Figure~\ref{fig:dep_par_conj_overlap_middle} shows
a simplified version of the algorithm we use to compute
the expected execution time of a conjunction
when its conjuncts are executed in parallel,
assuming an unlimited number of CPUs.
The inputs of the algorithm are \verb|Conjs|, the conjuncts themselves,
and \verb|ProdConsList|,
which gives, for each conjunct,
the list of its input and output variables,
together with the times at which,
% (during the profiling run, which executes the conjuncts in sequence)
in a sequential execution,
they are respectively first consumed or produced.
The times are relative to the start of the execution of the relevant conjunct.

The main task of the algorithm is
to divide the execution times of all the conjuncts into chunks
and keep track of when those chunks can execute.
The execution time of \verb|Conj_i|
has one chunk (\verb|Duration_ij|) for each of \verb|Conj_i|'s shared variables
that ends at the time at which that variable is produced or first consumed,
and there is one chunk (\verb|DurationRest_i|) at the end,
during which the call may produce non-shared variables.
Figure~\ref{fig:dep_conj_overlap1} shows that
the production of $A$ divides $p$ into two chunks, ${pA}$ and ${pR}$,
while the consumption of $A$ divides $q$ into ${qA}$ and ${qR}$.

The algorithm processes the chunks in order, and keeps track
of the sequential and parallel execution times of the chunks so far.
When a chunk of \verb|Conj_i| ends with the production of a variable,
we record when that variable is produced,
and the next chunk can start executing immediately.
When a chunk ends with the consumption of a variable,
then in the \emph{sequential} version of \verb|Conj_i|
the next chunk can also execute immediately,
since the values of all the input variables will be available when it starts,
but in the \emph{parallel} version,
the variable may not have been produced yet.
If it has, then \verb|Conj_i| does not need to wait for it;
the left side of figure~\ref{fig:dep_conj_overlap1} shows this case.
However, it is also possible that it has not.
In that case, \verb|Conj_i| will suspend on the variable,
and will resume only when its producer signals that it is available;
the right side of figure~\ref{fig:dep_conj_overlap1} shows this case.
Note that \verb|Var_ij|
will always be in \verb|ProdTimeMap| when we look for it,
because the Mercury mode system reorders conjunctions
to put the producer of each variable before all its consumers.
% in a two-conjunct conjunction,
% the left conjunct can only produce
% the variables it shares with the right conjunct
% and the right conjunct can only consume
% those variables.
% However, in longer conjunctions,
% the conjuncts in the middle
% can both consume variables produced by conjuncts on their left
% and produce variables consumed by conjuncts on their right.
% This is why our algorithm associates with \code{Conj\_i}, the $i$th conjunct,
% \code{ProdConsList\_i}:
% the list of shared variables that \code{Conj\_i} either produces or consumes,
% together with their times of production and first consumption respectively.

The version of this algorithm we have actually implemented is
a bit longer than the one in figure~\ref{fig:dep_par_conj_overlap_middle},
because it also accounts for several forms of overhead:

\begin{itemize}
\item
Creating a spark and adding it to a work queue has a cost.
Every conjunct but the last conjunct incurs this cost
to create the spark for the rest of the conjunction.
\item
It takes some time to take a spark off a spark queue,
create or reuse a context for it, and start its execution.
Every parallel conjunct that is not the first incurs this delay
before it starts running.
\item
The signal and wait operations have a cost.
\item
It takes some time to wake up a context when its wait operation succeeds.
\item
It takes time for each conjunct to synchronise on the barrier
when it has finished its job.
\end{itemize}

\noindent
We can account for every one of these overheads
by adding the estimated cost of the relevant operation to \verb|CurParTime|
at the right point in the algorithm.

In many cases,
the conjunction given to the algorithm shown in figure~\ref{fig:dep_par_conj_overlap_middle}
will contain a recursive call.
In such cases, the speedup computed by the algorithm
reflects the speedup we can expect to get when the recursive call
calls the \emph{original, sequential} version of the predicate.
When the recursive call calls the parallelised version,
we can expect a similar saving (absolute time, not ratio)
on \emph{every} recursive invocation.
How this affects the expected speedup of the top level call
depends on the structure of the recursion.
For the most common recursion structure,
singly recursive predicates like \verb|map_foldl|,
calculating the expected speedup of the top level call is easy,
since we can compute the average depth of recursion
from the relative execution counts of the base and recursive cases.
For some less common structures,
such as doubly recursive predicates like \verb|quicksort|, it is a bit harder,
and for irregular structures in which different execution paths
contain different numbers of recursive calls,
the profiling data gathered by the current version of the Mercury profiler
contains insufficient information to allow our system to determine the
expected speedup.
However, an automated survey of the programs handled by our feedback tool
shows that such predicates are rare;
our system can compute
the expected recursion depth and therefore the expected speedup
for virtually all candidates for parallelisation.

So far, we have assumed an unlimited number of CPUs,
which is of course unrealistic.
If the machine has e.g.\ four CPUs,
then the prediction of any speedup higher than four is obviously invalid.
Less obviously,
even a predicted overall speedup of less than four may depend
on more than four conjuncts executing all at once at \emph{some} point.
We have not found this to be a problem yet.
If and when we do,
we intend to extend our algorithm to keep track
of the number of active conjuncts in all active time periods.
Then if a chunk of a conjunct wants to run in a time period
when all CPUs are predicted to be already busy executing previous conjuncts,
we assume that the start of that chunk is delayed until a CPU becomes free.

The limited number of CPUs also means that
there is a limit to how much parallelism we actually \emph{want}.
The spawning off of every conjunct incurs overhead,
but these overheads do not buy us anything if all CPUs are already busy.
% If the machine has e.g.\ four CPUs,
% then we do not actually want to spawn off
% hundreds of iterations for parallel execution,
% since parallel execution actually has several forms of overhead:
That is why our system supports \emph{throttling}.
If a conjunction being parallelised contains a recursive call,
then the compiler can be asked to replace the original sequential conjunction
not with the parallel form of the conjunction,
but with an if-then-else.
The condition of this if-then-else
will test at runtime
whether spawning off a new job is a good idea or not.
If it is, we execute the parallelised conjunction, but
if it is not, we execute the original sequential conjunction.
The condition is obviously a heuristic.
If the heuristic allows the list of runnable jobs to become empty,
then we will not have any work to give to a CPU
that finishes its task and becomes available.
On the other hand,
if the heuristic allows the list of runnable jobs to become too long,
then we incur the overheads of spawning off some jobs unnecessarily.
Currently, on machines with $N$ CPUs,
we prefer to have a total of $M$ running and runnable jobs where $M > N$,
so our heuristic stops spawning attempts
if and only if the queue already has $M$ entries.
Our current system by default sets $M$ to be $32$ for $N = 4$,
though users can easily override this.

% The overheads of parallel execution can also affect conjunctions
% that do not contain recursive calls:
% a conjunction that looks worth parallelising if you ignore overheads
% may look not worth parallelising if you take them into account.
% This is why our system actually uses
% a version of algorithm~\ref{alg:dep_par_conj_overlap_middle}
% that accounts for overheads.

% Algorithm~\ref{alg:dep_par_conj_overlap_complete}
% can also handle $n$-way conjunctions for $n>2$.
% Since the Mercury mode system reorders conjunctions
% to ensure that data flows only to the right,
% in a two-conjunct conjunction,
% the left conjunct can only produce
% the variables it shares with the right conjunct
% and the right conjunct can only consume
% those variables.
% However, in longer conjunctions,
% the conjuncts in the middle
% can both consume variables produced by conjuncts on their left
% and produce variables consumed by conjuncts on their right.
% This is why our algorithm associates with \code{Conj\_i}, the $i$th conjunct,
% \code{ProdConsList\_i}:
% the list of shared variables that \code{Conj\_i} either produces or consumes,
% together with their times of production and first consumption respectively.
% This is a generalisation of \code{ConsList\_q} in
% algorithm~\ref{alg:dep_par_conj_overlap_simple}.
% We also need to generalise \code{Prod\_pi},
% because the time at which a non-first conjunct produces a variable
% can and usually will be affected
% by the overheads and/or synchronisation delays suffered by that conjunct.
% This is why we use \code{ProdTimeMap},
% which maps each shared variable to its time of production.

% The main body of the algorithm consists of two nested loops.
% The outer loop loops over all the conjuncts from left to right,
% because the execution of a conjunct can be affected
% by the conjuncts to its left
% (through the time at which they produce the shared variables it consumes),
% but not by the conjuncts to its right.
% The first few lines of the outer loop body
% (the first two assignments to \code{CurParTime})
% compute for each conjunct
% the time at which that conjunct can start execution.

% The inner loop loops over all the components of the current conjunct,
% such as \code{pA} and \code{pR}
% from figure~\ref{fig:dep_conj_overlap1}.
% Just as our previous algorithm did,
% this loop updates the simulated current time
% in both the original sequential execution of the conjunct (\code{CurSeqTime})
% and in its modified parallelised execution (\code{CurParTime}).
% For time components that end in the consumption of a variable,
% we do what we did before,
% but also reflect the cost of the wait operation needed for the consumption.
% For time components that end in the production of a variable,
% we record the time at which
% that variable would be available in the parallel execution;
% this will be when the producer finishes executing the signal operation on it.

% We use \code{TotalParTime} to keep track of the ending time
% of the parallel conjunct that ends last.
% We also remember, in \code{FirstConjTime},
% the time at which the first conjunct finishes.
% The reason we do this is because
% our runtime system requires that
% when the parallel conjunction finishes,
% execution must continue in the context
% that entered the parallel conjunction in the first place.
% In our implementation, this context will execute the first conjunct.
% If the last conjunct to finish is the first conjunct,
% it can continue on without delay;
% if the last conjunct to finish is some other conjunct,
% then we need to free its context,
% and switch to executing the original context,
% which became idle when the first conjunct finished.
% The last two lines reflect this cost.

% XXX end of iclp }

\begin{algorithm}
\begin{verbatim}
CurSeqTime_q := 0
CurParTime_q := 0
sort ConsList_q on ConsTime_qi
forall (Var_i, ConsTime_qi) in ConsList_q:
    Duration_qi := ConsTime_qi - CurSeqTime_q
    CurSeqTime_q := CurSeqTime_q + Duration_qi
    ParWantTime_qi := CurParTime_q + Duration_qi
    CurParTime_q := max(ParWantTime_q, Prod_pi)
DurationRest_q := SeqTime_q - CurSeqTime_q
SeqTime_q := CurSeqTime_q + DurationRest_q
ParTime_q := CurParTime_q + DurationRest_q
\end{verbatim}
\caption{Dependent parallel conjunction overlap calculation}
\label{alg:dep_par_conj_overlap_simple}
\end{algorithm}

Algorithm~\ref{alg:dep_par_conj_overlap_simple} shows
a simple version of the algorithm we use to compute ${ParTime}_q$.
Its main input is ${ConsList}_q$,
a list of the variables shared by $p$ and $q$,
together with their times of consumption by $q$.

The main task of the algorithm is to divide up ${SeqTime}_q$ into chunks,
and keep track of when those chunks can execute.
There is one chunk (${Duration}_{qi})$ for each shared variable
that ends at the time at which that variable is first consumed,
and there is one chunk ${DurationRest}$
after the consumption of the last shared variable.
(Figure~\ref{fig:dep_conj_overlap1}
shows the former as ${qA}$ and the latter as ${qR}$.)
The algorithm keeps track of the sequential and parallel execution times of $q$
up to the consumption of the current shared variable.
In the sequential version,
each chunk can execute immediately after the previous chunk,
since the values of the shared variables are all available when $q$ starts.
In the parallel version,
$p$ is producing the shared variables while $q$ is running.
If $p$ has produced the value of ${SV}_i$ by the time $q$ needs it,
there $q$ does not need to wait for it;
the left side of figure~\ref{fig:dep_conj_overlap1} shows this case.
However, it is also possible that $p$ will produce ${SV}_i$
only after the time at which $q$ would like to use it.
In that case, $q$ will suspend on ${SV}_i$,
and will resume only when $p$ signals that it is available;
the right side of figure~\ref{fig:dep_conj_overlap1} shows this case.

On both sides figure~\ref{fig:dep_conj_overlap1}
${SeqTime}_p = 5$ and ${SeqTime}_q = 4$.
On the left side, ${ConsTime}_{qA} = 2$,
and therefore ${Duration}_{qA} = 2$ and ${DurationRest}_{qA} = 2$,
Since ${ProdTime}_{pA} = 1$,
the first update of ${CurParTime}_q$ sets it to $2$,
and the second sets it to $4$, so ${ParTime}_q = 4$.
On the right side, ${ConsTime}_{qA} = 1$,
and therefore ${Duration}_{qA} = 1$ and ${DurationRest}_{qA} = 3$,
Since ${ProdTime}_{pA} = 4$,
the first update of ${CurParTime}_q$ sets it to ${max}(1, 4) = 4$,
and the second sets it to $4+3 = 7$, so ${ParTime}_q$ is 7.

\begin{table}
\begin{center}
\begin{tabular}{l|rr}
 & \multicolumn{1}{|c}{Cost}
 & \multicolumn{1}{|c}{Local use of \code{Acc1}} \\
\hline
\code{M}  &   1,625,050 & none \\
\code{F}  &           3 & ${Prod}_{Acc1}$ =         3 \\
\mapfoldl &   1,625,054 & ${Cons}_{Acc1}$ = 1,625,051 \\
% Note: The cost of the recursive call assumes that there is one
% recursive case and one base case remaining in the recursion.
\end{tabular}
\end{center}
\caption{Rounded profiling data for \mapfoldl}
\label{tab:prof_data_map_foldl}
\end{table}

% \begin{figure}[tb]
% \begin{verbatim}
% map_foldl_par(_, _, [], Acc, Acc).
% map_foldl_par(M, F, [X | Xs], Acc0, Acc) :-
%     (
%         M(X, Y),
%         F(Y, Acc0, Acc1)
%     ) &
%     map_foldl_par(M, Xs, Acc1, Acc).
% \end{verbatim}
% \caption{Parallel \mapfoldl}
% % the recursive call is less dependent
% % on the conjunction of the first two calls.
% \label{fig:map_foldl_par}
% \end{figure}

To see how the algorithm works on realistic data,
consider the \mapfoldl example in figure~\ref{fig:map_foldl}.
Table \ref{tab:prof_data_map_foldl} gives
the approximate costs of the calls in the recursive clause of \mapfoldl
when used in a Mandelbrot image generator.
Each call to $M$ draws a row,
while $F$ appends the new row
onto the list of the rows already drawn.
The table also shows when $F$ produces ${Acc1}$
and when the recursive call consumes ${Acc1}$.
The costs were collected from a real execution using Mercury's deep profiler
and then rounded to make mental arithmetic easier.

Figure~\ref{fig:map_foldl_par} shows the best parallelisation of
\mapfoldl.
When evaluating the speedup for this parallelisation,
${Cons}_{{F} {Acc1}} = 1,625,000 + 2 = 1,625,002$, and
${Cons}_{{map\_foldl} {Acc1}} = 1,625,000 + 2 + 1,620,000 = 3,245,002$.
\zoltan{Show the rest of the algorithm's execution when this question is answered.}
\paul{I would have fixed this but I do not know what these formulas mean,
Maybe I am confused by notation}

% \zoltan{Show real data, if we can,
% that shows that quicksort has very little overlap.}

The speedup computed by algorithm~\ref{alg:dep_par_conj_overlap_simple}
applies only when the recursive call
calls the original sequential version of the predicate.
When the recursive call calls the parallelised version,
the maximum speedup available (assuming an unlimited number of CPUs)
depends on the structure of the recursion.
The profiling data gives us \tr{$E$,}
the number of entry calls to the procedure from the higher clique,
and \tr{$R_i$,} the number of recursive calls at each recursive call site.
From these, and the structure of the procedure's code,
we can calculate the average depth of recursion in most cases.

For singly recursive predicates like \mapfoldl,
there is only one recursive call site,
and the depth of recursion is simply $R_1/E$.
For example, if $E = 2$ and $R_1 = 20$,
then the average call sequence to the procedure
has one entry call followed by the recursive calls.
It also means that the average call sequence
has ten calls that cause the procedure to execute its recursive clause,
the clause containing the conjunction being parallelised,
followed by one call that executes the base clause.
From this, we can deduce that if ${SeqSaving} = {SeqTime} - {ParTime}$
is the time saving we get from parallelising the top conjunction
if the recursive call calls the original sequential version of the procedure,
then making it call the parallelised version of the procedure
would yield a time saving of ${ParSaving} = {SeqSaving} * R_1/E$
if we have enough CPUs to execute all the $R_1/E$ iterations in parallel.
\peter{I think this needs more explanation.  It does not look right to me.
I would expect it to be ${ParSaving} = {SeqTime} - {ParTime} * R_1/E$.}
This assumes that we get the same savings at each iteration,
\peter{I find this sentence weakens the claim, rather than strengthing it.
Can we just add ``, assuming we get the same savings at each iteration'' at
the end of the previous sentence?}
but this is reasonable,
since what we are doing is essentially executing
the different iterations of a loop in parallel,
and we have no reason to believe that the savings
from executing iteration $k$ in parallel with iteration $k+1$
would vary systematically based on the value of $k$.

Some singly recursive predicates have more than one recursive clause,
each with one recursive call site.
Suppose there are $n$ call sites, with execution counts $R_1 \ldots R_n$.
The overall time saving from parallelising the conjunct
that contains the call site associated $R_i$
has to be multiplied by the fraction of recursive calls
that execute the conjunction being parallelised:
${ParSaving} = {SeqSaving} * R_i/E * R_i/\sum_{j=1}^n R_j$.

For doubly recursive predicates like \code{quicksort}, $R_1 = R_2$,
and just under half of their calls invoke the recursive clause,
so for them, ${ParSaving} = {SeqSaving} * R_1/(2 * E)$.
\zoltan{check the math}
\peter{I think it is a bit confusing to use $R_1$ and $R_2$ to name the
recursive calls to quicksort, when above they have named the sole recursive
call from different clauses.  Maybe use $R^i_j$ to indicate the $i^{th}$
recursive call from the $j^{th}$ clause.  Then you can use $R^1_i$ in the
previous paragraph, and $R_1^1$ and $R_1^2$ in this one.}

All these calculations show that the available parallelism
can be greater than the number of CPUs.
If the machine has e.g.\ four CPUs,
then we do not actually want to spawn off
hundreds of iterations for parallel execution,
since parallel execution actually has several forms of overhead:

\begin{description}
\item[SparkCost]
is the cost of creating a spark and adding it to the local spark stack.
In a parallel conjunction,
every conjunct that is not the last conjunct incurs this cost
to create the spark for the rest of the conjunction.
\item[SparkDelay]
is the estimated length of time between the creation of a spark
and the beginning of its execution on another engine.
Every parallel conjunct that is not the first incurs this delay
before it starts running.
\item[SignalCost]
is the cost of signalling a future.
\item[WaitCost]
is the cost of waiting on a future.
\item[ContextWakeupDelay]
is the estimated time that it takes for a context to resume execution
after being placed on the runnable queue,
assuming that the queue is empty and there is an idle engine.
\item[BarrierCost]
is the cost of executing the operation
that synchronises all the conjuncts at the barrier
at the end of the conjunction.
\end{description}

Because of these overheads, our system uses \emph{throttling}.
If a conjunction being parallelised contains a recursive call,
then the compiler will replace the original sequential conjunction
not with the parallel form of the conjunction,
but with an if-then-else.
The condition of this if-then-else
will test at runtime
whether spawning off a new job is a good idea or not.
If it is, we execute the parallelised conjunction,
if it is not, we execute the original sequential conjunction.
The condition is obviously a heuristic.
If the heuristic allows the list of runnable jobs to become empty,
then we will not have any work to give to a CPU
that finishes its task and becomes available.
On the other hand,
if the heuristic allows the list of runnable jobs to become too long,
then we incur the overheads of spawning off some jobs unnecessarily.
Currently, on machines with $N$ CPUs,
we prefer to have a total of $M$ running and runnable jobs where $M > N$,
so our heuristic stops spawning attempts
if and only if the queue already has $M$ entries.
Our current system by default sets $M$ to be $32$ for $N = 4$,
though users can easily override this.

The overheads of parallel execution can also affect conjunctions
that do not contain recursive calls:
a conjunction that looks worth parallelising if you ignore overheads
may look not worth parallelising if you take them into account.
This is why our system actually uses
algorithm~\ref{alg:dep_par_conj_overlap_complete},
a version of algorithm~\ref{alg:dep_par_conj_overlap_simple}
that accounts for overheads.

\begin{algorithm}
\begin{verbatim}
find_par_time(Conjs) returns TotalParTime:
N := length(Conjs)
ProdTimeMap := empty
FirstConjTime := 0
TotalParTime  := 0
for i in 1 to N:
  CurSeqTime := 0
  CurParTime := (SparkCost + SparkDelay) * (i-1)
  if i != N:
    CurParTime := CurParTime + SparkCost
  sort ProdConsList_i on Time_ij
  forall (Var_ij, Time_ij) in ProdConsList_i:
    Duration_ij := Time_ij - CurSeqTime
    CurSeqTime := CurSeqTime + Duration_ij
    if Conj_i produces Var_ij:
      CurParTime := CurParTime + Duration_ij + SignalCost
      ProdTimeMap[Var_ij] := CurParTime
    else Conj_i must consume Var_ij:
      ParWantTime := CurParTime + Duration_ij
      CurParTime := max(ParWantTime, ProdTimeMap[Var]) + WaitCost
      if ParWantTime < ProdTimeMap[Var_ij]:
        CurParTime := CurParTime + ContextWakeupDelay
  DurationRest := SeqTime_i - CurSeqTime
  CurParTime := CurParTime + DurationRest + BarrierCost
  if i == 1:
    FirstConjTime = CurParTime
  TotalParTime := max(TotalParTime, CurParTime)
if TotalParTime > FirstConjTime:
  TotalParTime := TotalParTime + ContextWakeupDelay
\end{verbatim}
\caption{Dependent parallel conjunction complete algorithm}
\label{alg:dep_par_conj_overlap_complete}
\end{algorithm}

Algorithm~\ref{alg:dep_par_conj_overlap_complete}
can also handle $n$-way conjunctions for $n>2$.
Since the Mercury mode system reorders conjunctions
to ensure that data flows only to the right,
in a two-conjunct conjunction,
the left conjunct can only produce
the variables it shares with the right conjunct
and the right conjunct can only consume
those variables.
However, in longer conjunctions,
the conjuncts in the middle
can both consume variables produced by conjuncts on their left
and produce variables consumed by conjuncts on their right.
This is why our algorithm associates with \code{Conj\_i}, the $i$th conjunct,
\code{ProdConsList\_i}:
the list of shared variables that \code{Conj\_i} either produces or consumes,
together with their times of production and first consumption respectively.
This is a generalisation of \code{ConsList\_q} in
algorithm~\ref{alg:dep_par_conj_overlap_simple}.
We also need to generalise \code{Prod\_pi},
because the time at which a non-first conjunct produces a variable
can and usually will be affected
by the overheads and/or synchronisation delays suffered by that conjunct.
This is why we use \code{ProdTimeMap},
which maps each shared variable to its time of production.

The main body of the algorithm consists of two nested loops.
The outer loop loops over all the conjuncts from left to right,
because the execution of a conjunct can be affected
by the conjuncts to its left
(through the time at which they produce the shared variables it consumes),
but not by the conjuncts to its right.
The first few lines of the outer loop body
(the first two assignments to \code{CurParTime})
compute for each conjunct
the time at which that conjunct can start execution.

The inner loop loops over all the components of the current conjunct,
such as \code{pA} and \code{pR}
from figure~\ref{fig:dep_conj_overlap1}.
Just as our previous algorithm did,
this loop updates the simulated current time
in both the original sequential execution of the conjunct (\code{CurSeqTime})
and in its modified parallelised execution (\code{CurParTime}).
For time components that end in the consumption of a variable,
we do what we did before,
but also reflect the cost of the wait operation needed for the consumption.
For time components that end in the production of a variable,
we record the time at which
that variable would be available in the parallel execution;
this will be when the producer finishes executing the signal operation on it.

We use \code{TotalParTime} to keep track of the ending time
of the parallel conjunct that ends last.
We also remember, in \code{FirstConjTime},
the time at which the first conjunct finishes.
The reason we do this is because
our runtime system requires that
when the parallel conjunction finishes,
execution must continue in the context
that entered the parallel conjunction in the first place.
In our implementation, this context will execute the first conjunct.
If the last conjunct to finish is the first conjunct,
it can continue on without delay;
if the last conjunct to finish is some other conjunct,
then we need to free its context,
and switch to executing the original context,
which became idle when the first conjunct finished.
The last two lines reflect this cost.

% Each conjunct's execution depends on
% when the variables it consumes are produced by other conjuncts.
% These must be conjuncts to its left,
% since the compiler reorders conjunctions as needed
% to ensure that data flows only from left to right.
% Therefore, we can calculate the execution time of
% $G_1 \& \ldots \& G_n$
% by computing the execution time
% first of $G_1$,
% then of $G_1 \& G_2$,
% then of $(G_1 \& G_2) \& G_3$,
% % then of $((G_1 \& G_2) \& G_3) \& G_4$,
% and so on.

% \zoltan{Discuss how the synchronisation at the end of the algorithm is
% similar to the synchronisation while waiting for another variable.}

\section{Choosing how to parallelise a conjunction}
\label{sec:overlap_howto}

A conjunction with $n > 2$ conjuncts
can be converted into several different parallel conjunctions.
Converting all the commas into ampersands
(e.g.\ \code{c1, c2, c3} into \code{c1 \& c2 \& c3})
yields the most parallelism.
Unfortunately, this will often be \emph{too} much parallelism,
because in practice many conjuncts are unifications
and arithmetic operations whose execution takes very few instructions.
Executing such conjuncts in their own threads
costs far more in overheads than they save by running in parallel.
Therefore in most cases,
we want to create parallel conjunctions with $k < n$ conjuncts,
each consisting of a contiguous sequence
of one or more of the original sequential conjuncts,
effectively partitioning the original conjuncts into groups.

\begin{figure}
\begin{center}
\begin{verbatim}
global NumEvals := 0
find_best_partition(InitPartition, InitTime, LaterConjs)
    returns <FinalTime, FinalPartitionSet>:
  switch on LaterConjs:
  when LaterConjs = []:
    return <InitTime, {InitPartition}>
  when LaterConjs = [Head | Tail]:
    Extend := all_but_last(InitPartition) ++ [last(InitPartition) ++ [Head]]
    AddNew := InitPartition ++ [Head]
    ExtendTime := find_par_time(Extend)
    AddNewTime := find_par_time(AddNew)
    NumEvals := NumEvals + 2
    if ExtendTime < AddNewTime:
      BestExtendSoln := find_best_partition(Extend, ExtendTime, Tail)
      let BestExtendSoln be <BextExTime, BestExPartSet>
      if NumEvals < PreferLinearEvals:
        BestAddNewSoln := find_best_partition(AddNew, AddNewTime, Tail)
        let BestAddNewSoln be <BestANTime, BestANPartSet>
        if BestExTime < BestANTime:
          return BestExtendSoln
        else if BestExTime = BestANTime:
          return <BextExTime, BestExPartSet union BestANPartSet>
        else:
          return BestAddNewSoln
      else:
        return BestExtendSoln
    else:
      <symmetric with the then case>
\end{verbatim}
\end{center}
\caption{Search for the best parallelisation}
\label{fig:best_par_search}
%\vspace{-2\baselineskip}
\end{figure}

For any conjunction to be worth parallelising,
it should contain two or more expensive goals.
Our main algorithm (figure \ref{fig:best_par_search} works on the list
of conjuncts
from the first expensive goal to the last.
This will be the middle of original conjunction,
with (possibly empty) lists of cheap goals before it and after it.
Our initial search assumes that
the set of conjuncts in the parallel conjunction we want to create
is exactly the set of conjuncts in the middle.
A post-processing step then removes that assumption.

% into \code{(c1 \& c2), c3},
% into \code{c1, (c2 \& c3)},
% into \code{c1 \& (c2, c3)},
% into \code{(c1, c2) \& c3}, or
% into \code{c1 \& c2 \& c3}.

% In the usual case where $n >> 2$,
% there will be a huge number ways to do this.
% Our parallelisation algorithm,
% algorithm~\ref{alg:best_par_search},
% therefore tries to find the partition
% that yields the lowest overall execution time.

% A middle sequence with $n > 2$ conjuncts
% can be converted into several different parallel conjunctions;
% for example, \code{c1, c2, c3} can be converted
% into \code{(c1 \& c2), c3},
% into \code{c1, (c2 \& c3)},
% into \code{c1 \& (c2, c3)},
% into \code{(c1, c2) \& c3}, or
% into \code{c1 \& c2 \& c3}.
% The first two do not make sense if \code{c1} and {c3} are expensive goals,
% so we consider only conjunctions in which all conjuncts in the middle sequence
% are part of one parallel conjunct or another.
% The last of these gives the finest grain parallelism.
% Unfortunately, this will often be \emph{too} fine-grained,
% because in practice many conjuncts are unifications
% or builtin operations such as arithmetic
% whose execution takes very few instructions.
% Executing such conjuncts in their own threads
% can cost far more in overheads than they save by running in parallel.
% Therefore in most cases,
% we want to create parallel conjunctions with $k < n$ conjuncts,
% each consisting of a contiguous sequence
% of one or more of the original sequential conjuncts,
% effectively partitioning the original conjuncts into groups.
% % In the usual case where $n >> 2$,
% % there will be a huge number ways to do this.
% % Our parallelisation algorithm,
% % algorithm~\ref{alg:best_par_search},
% % therefore tries to find the partition
% % that yields the lowest overall execution time.

% Therefore, it is best to break sequential conjunctions into a number of
% smaller sequential conjunctions that are conjuncts of a larger
% parallel conjunction, however there are a multiple possible ways to do
% this.

If the middle sequence has $n$ conjuncts,
then there are $n-1$ AND operations between them,
each of which can be either sequential or parallel.
There are then $2^{n-1}$ combinations,
all but one of which are parallelisations.
That is a large space to search for the \emph{best} parallelisation,
and it would be larger still if we allowed code reordering,
that is, parallel conjuncts consisting of
a \emph{non}contiguous sequence of the original conjuncts.
We explore this space with a search algorithm,
\code{find\-\_\-best\-\_\-par\-ti\-tion}, which
we invoke with the empty list as \code{InitPartition},
zero as \code{InitTime}, and the list of middle conjuncts as \code{LaterConj}.
\code{InitPartition} expresses a partition of an initial sequence
of the middle goals into parallel conjuncts
whose estimated execution time is \code{InitTime},
and considers whether it is better to add the next middle goal
to the last existing parallel conjunct (\code{Extend}),
or to put it into a new parallel conjunct (\code{AddNew}).
It explores extensions of the better of the resulting partitions first.
If the search is still under the limit on the number of evaluations,
it explores the worse partition as well,
which is an exponential search.
When it hits the limit,
it switches to a linear search;
we explore the more promising partition first
to make this search more effective.
(This limit ensures that the algorithm runs in reasonable time.)
The algorithm returns a set of equal best parallelisations so far,
``best'' being measured by
\iclp{a version of the algorithm in figure~\ref{fig:dep_par_conj_overlap_middle} that
computes the estimated parallel execution time \emph{including} overheads.}
\tr{algorithm \ref{alg:dep_par_conj_overlap_complete},
that is, the estimated parallel execution time including overheads.}

There are some simple ways to improve this algorithm.
%\vspace{-2mm}
\begin{itemize}
\item
Most invocations of \verb|find_par_time| specify a partition
that is an extension of a partition processed in the recent past.
In such cases, \verb|find_part_time| should do its task
incrementally, not from scratch.
\item
If the expected execution time
for the candidate partition currently being considered
is already greater than the fastest existing complete partition,
we can stop exploring that branch;
it cannot lead to a better solution.
\tr{
(This is the idea of branch-and-bound algorithms.)
}
\item
Sometimes consecutive conjuncts do things that are
obviously a bad idea to do in parallel, such as building a ground term.
The algorithm should treat these as a single conjunct.
% XXX we could make the third item tr only if we need space
\tr{
\item
graph of dependencies
\item
take total CPU utilisation into account,
at least by using it to break ties on overall CPU time
}
\end{itemize}
%\vspace{-2mm}

\noindent
At the completion of the search,
we select one of the equal best parallelisations,
and post-process it to adjust both edges.
Suppose the best parallel form of the middle goals is $P_1~\&~\ldots~\&~P_p$,
where each $P_i$ is a sequential conjunction.
We compare the execution time of $P_1~\&~\ldots~\&~P_p$
with that of $P_1,~(P_2~\&~\ldots~\&~P_p)$.
If the former is slower,
which can happen if $P_1$ produces its outputs at its very end
and the other $P_i$ consume those outputs at their start,
then we conceptually move $P_1$ out of the parallel conjunction
(from the ``middle'' part of the conjunction to the ``before'' part).
We keep doing this for $P_2$, $P_3$ etc until either
we find a goal worth keeping in the parallel conjunction,
or we run out of conjuncts.
We also do the same thing at the other end of the middle part.
This process can shrink the middle part.

In cases where we do not shrink an edge, we can consider expanding that edge.
Normally, we want to keep cheap goals out of parallel conjunctions,
since more conjuncts tends to mean
more shared variables and thus more synchronisation overhead,
but sometimes this consideration is overruled by others.
Suppose the goals before the conjuncts in $P_1~\&~\ldots~\&~P_p$
in the original conjunction were $B_1,~\ldots,~B_b$
and the goals after it $A_1,~\ldots,~A_a$,
and consider $A_1$ after $P_p$.
If $P_p$ finishes before the other parallel conjuncts,
then executing $A_1$ just after $P_p$ in $P_p$'s context
may be effectively free:
the last context could still arrive at the barrier at the same time,
but this way, $A_1$ would have been done by then.
Now consider $B_b$ before $P_1$.
If $P_1$ finishes before the other parallel conjuncts,
\emph{and} if none of the other conjuncts wait for variables produced by $P_1$,
then executing $B_b$ in the same context as $P_1$ can be similarly free.

We loop from $i=b$ down towards $i=1$, and check whether
including $B_i,~\ldots,~B_b$ at the start of $P_1$ is improvement.
If not, we stop; if it is, we keep going.
We do the same from the other end.
% If we end up with moving
% The second search loops from $j=1$ up towards $j=a$
% and checks whether including $A_1, \ldots, A_j$ at the end of $P_p$
% is improvement.
% Each loop stops when the answer becomes ``no'',
The stopping points of the loops of the contraction and expansion phases
dictate our preferred parallel form of the conjunction, which
(if we shrunk the middle at the left edge and expanded it at the right)
will look something like
$B_1,$ $\ldots,$ $B_{b},$ $P_1,$ $\ldots~P_k,$
$(P_{k+1}$ $\&$ $\ldots$ $\&$ $P_{p-1}$ $\&$ $(P_p,$ $A_1,$ $\ldots,$ $A_j)),
A_{j+1},$ $\ldots,$ $A_a$.
% $B_1, \ldots, B_{i-1},
% ((B_i, \ldots, B_b, P_1) \& P_2, \ldots \& P_{p-1} \& (P_p, A_1, \ldots, A_j)),
% A_{j+1}, \ldots, A_a$.
If this preferred parallelisation is better than
the original sequential version of the conjunction by at least 1% (a configurable threshold),
then we include a recommendation for its conversion to this form
in the feedback file we create for the compiler.

% These two loops are specifically designed
% to allow the inclusion of cheap goals in the parallel conjunction.
% Note that this algorithm always tries to arrange
% \emph{all} the conjuncts in the conjunction,
% not just the conjuncts from the first costly goal to the last.
% Normally, we want to keep cheap goals out of parallel conjunctions,
% since more conjuncts usually means more shared variables,
% which means more synchronisation overhead.
% The reason why we expanding the scope of the parallel conjunction
% is that sometimes this consideration is overruled by others.
% Consider $A_1$ after $P_p$.
% If $P_p$ finishes before the other parallel conjuncts,
% then executing $A_1$ just after it in $P_p$'s context may be effectively free:
% the last context could still arrive at the barrier at the same time,
% but this way, $A_1$ would have completed by then.
% Now consider $B_b$ before $P_1$ where $P_1$ is still in a parallel conjunct.
% If $P_1$ finishes before the other parallel conjuncts,
% \emph{and} if none of the other conjuncts
% wait for variables produced by $P_1$,
% then executing $B_b$ in the same context as $P_1$ can be similarly free.

% \begin{itemize}
% \item
% Currently no tie breaking is done and we have not explored
% using other formulas for the search's objective function.
% % \item
% % Also, the current implementation does not make an estimate of the
% % minimum cost of the work that could be scheduled after the current point.
% % This affects the amount of pruning that the branch and bound code
% % is able to achieve.
% \end{itemize}

% When explaining the algorithm, tell readers to first assume that
% GoalsBefore and GoalsAfter are the empty list,
% i.e. MaxBefore = 0 and MinAfter = N+1.
% Only after explaining the algorithm in that case

% Consider changing the loop structure so that instead of
% loop on Before;
%     loop on After,
%         loop on Arrangement,
% it is
% loop on Arrangement,
%     loop on Before (assuming a given After) find the best and commit to it,
%     loop on After, find the best and commit to it.

\section{Pragmatic issues}
\label{sec:overlap_pragmatic}

% \subsection{The effects of module boundaries}
% \label{sec:pragmamoduleboundary}

% pushing waits and signals into calles stops at module boundaries

% \subsection{Cliques vs procedures}
% \label{sec:pragmacliqueproc}

\emph{Dynamic context}
The algorithms in sections \ref{sec:overlap_overlap_alg}
and \ref{sec:overlap_howto}
work on profiling data that shows the behaviour of a procedure
in the context given by a particular chain of ancestors.
Many procedures are of course called from multiple ancestor contexts.
What happens when our analysis of the behaviour of the same procedure
yields different results for different ancestor contexts?

At the moment, for any procedure
that our analysis indicates is worth parallelising in any context,
we pick one particular parallelisation (usually there is only one anyway),
and transform the procedure accordingly.
This gets the benefit of parallelisation when it is worthwhile,
but incurs its costs even in contexts when it is not.
In the future, we plan to fix this using multi-version specialisation.
For every procedure with different parallelisation recommendations,
we intend to create a specialised version for each recommendation,
leaving the original sequential version.
This will of course require the creation of specialised versions
of its parent, grandparent etc procedures,
until we get to an ancestor procedure
which occurs in the common prefix of all the conflicting ancestor contexts.

% \subsection{Searching for parallelism opportunities}
% \label{sec:pragmabestfirst}
%
% Our candidates list actually contains
% both cliques and conjunctions within cliques.
%
% The candidates list should contain cliques, since
% \begin{itemize}
% \item
% the entry points of some child cliques are not in conjunctions
% (e.g.\ they can be switch arms), and
% \item
% we want to delay breaking a clique down into its constituent conjunctions,
% since this way if our traversal stops before getting to a clique,
% then we never have to break it down.
% \end{itemize}
%
% The candidates list should also contain conjunctions,
% since a clique can contain both cheap and expensive conjunctions,
% and we do not want to evaluate the cheap ones
% until we have processed all the more expensive conjunctions
% not just in this clique but in all other cliques;
% again, we expect that this way,
% our traversal will stop before it gets to the cheapest conjunctions.

% \subsection{Parallelising children vs ancestors}
% \label{sec:pragmachildancestor}

\emph{Parallelising children vs ancestors}
What happens when we decide that a conjunction that should be parallelised
has an ancestor that we decided should also be parallelised?
We can
(1) parallelise only the ancestor,
(2) parallelise only this conjunction, or
(3) parallelise both

% The first alternative (parallelise neither) has already been rejected twice,
% since we concluded that (2) was better (1)
% when we decided to parallelise the ancestor,
% and we concluded that (3) was better (1)
% when we decided to parallelise this conjunction.

\zoltan{Does the implementation actually do this now?}
We choose among the other three alternatives
by evaluating the speedup you get from each of them, and just pick the best.
This reevaluation must take into account
the fact that for each invocation of the ancestor conjunction,
we will invoke the current conjunction many times,
and that therefore we will incur both the overheads and the benefits
of parallelising the current conjunction many times.
The profile will give the actual number.

% \subsection{Disagreement among children}
% \label{sec:pragmachildchild}

% what if for some ``current'' clique,
% you want to parallelise the ancestor,
% but for some other current clique,
% you do not want to parallelise the same ancestor?

\emph{Parallelising branched goals}
Many programs have code that looks like this:
\begin{verbatim}
( if ... then
    ... expensive call 1 ...
else
    ... cheap goal ...
),
expensive call 2
\end{verbatim}
If the condition of the if-then-else succeeds only rarely,
then the average cost of the if-then-else
may be below the threshold of what we consider to be an expensive goal.
We therefore would not even consider
parallelising the top-level conjunction,
rightly considering that its overheads would probably outweigh its benefits.

What we want to do in such cases
is execute just the two expensive calls in parallel,
which would be equivalent to parallelising the conjunction
in the then part of this transformed goal:
\begin{verbatim}
( if ... then
    ... expensive call 1 ...
    expensive call 2
else
    ... cheap goal ...
    expensive call 2
)
\end{verbatim}
We intend to change our feedback tool to detect such situations,
and if found, to recommend
some equivalence-preserving transformations for the compiler to apply
before parallelising some of the resulting conjunctions.

% \subsection{Garbage collector issues}
% \label{sec:pragmagc}

\emph{Garbage collector issues}
The Mercury implementation uses the Boehm-Demers-Weiser
conservative collector for C \zoltan{add cite} to manage memory.
This system has worse overheads
for parallel programs than for sequential programs.
First, even though this collector uses
a separate memory pool for each mutator thread
(and hence, in our system, for each Mercury engine),
you still need synchronisation to access the global pool
when a local pool runs out.
Second, this collector
does not support incremental collection for parallel programs,
and a full collection stops all threads,
and thrashes the caches of their CPUs.
We therefore ran our benchmarks with the collector tuned
to use large local pool sizes
and to grow the size of the global pool more quickly than usual.
These settings significantly improved
the performance of the sequential programs as well.

% GC In our case, the worry is that
% GC the Boehm collector may scale significantly worse than
% GC the Mercury code that we choose to parallelise.

% GC From the perspective of a GC implementor,
% GC a program's runtime has two components:
% GC the execution time of the main part of the program (the \emph{mutator}),
% GC and the execution time of the collector itself.
% GC When the mutator is a Mercury program,
% GC there is a fixed (and usually small) limit
% GC on the number of instructions it can execute
% GC between increments of the call sequence count
% GC (though the limit is program-dependent).
% GC There is no such limit on the collector.
% GC This is can be a problem.
% GC Since a construction unification does not involve a call,
% GC our profiler considers its CSC cost to be zero.
% GC \peter{Haven't defined ``CSC \emph{cost}''.}
% GC Yet if the memory allocation required by a construction
% GC triggers a collection,
% GC then this nominally zero-cost action can actually take as much time
% GC as many calls in the mutator.
% GC
% GC The normal way to view the time taken by gc
% GC is to simply distribute it among the allocations,
% GC so that one CSC represents the average time taken
% GC by the mutator between two calls
% GC plus the average amortized cost of the collections
% GC triggered by the unifications between those calls.
% GC For a sequential program, this view works very well.
% GC For a parallel program, it works less well,
% GC because the performance of the mutator and the collector may scale differently.
% GC In our case, the worry is that
% GC the Boehm collector may scale significantly worse than
% GC the Mercury code that we choose to parallelise.
% GC
% GC For Mercury code,
% GC the limits on speedups from parallelism fall into two categories:
% GC those that our analysis takes into account, and those it does not.
% GC The former include the cost of spawning new contexts,
% GC the cost of the barrier synchronisation at the end of the conjunction,
% GC the cost of creating the synchronisation terms,
% GC the cost of the wait and signal operations on those terms,
% GC and idle time of the consumer while waiting on the producer.
% GC The latter include interference

% lock at allocation vs stop the world at collection

% memory
% heat
%
% is to simply consider the amortized cost
% In sequential programs,
% one can consider that the cost of the
%
% In sequential programs,
% it is not really feasible to separate
% the time cost of the collector from the cost of the mutator.
% On
%
% When parallelising the program, there is unfortunately
% a natural way to separate
%
% and CSCs can be considered to measure both parts of the program.
%
% CSCs not a true representation of time
%
% take heap allocation intensity (allocations per CSC) into account
% in the algorithms above:
% consider two high-intensity goals being executed in parallel
% to be another source of overhead.
%
% convert allocs to CSCs, add them to both sequential and parallel times
%
% involve the alloc ratios somehow?
%
% There is also the consideration that with Boehm gc,
% a collection stops the world,
% and the overheads of this stopping scale with the number of CPUs being used.
% The overheads of stopping include
% not just the direct costs of the interruption,
% but also indirect costs,
% such as having to refill the cache after the collector trashes it.

\section{Performance results}
\label{sec:overlap_perf}

% \begin{table*}
% \begin{center}
% \begin{tabular}{l|rrrrrrrrrr}
%  ~ & \multicolumn{1}{|c|}{Seq RT} &
%   \multicolumn{1}{|c|}{Par RT} &
%   \multicolumn{2}{|c|}{No Deps} &
%   \multicolumn{2}{|c|}{Na\"ive} &
%   \multicolumn{2}{|c|}{Num Vars} &
%   \multicolumn{2}{|c}{Overlap} \\
% \multicolumn{1}{c|}{Program} & \multicolumn{1}{|c|}{345Time} &
%   \multicolumn{1}{|c|}{Time} &
%   \multicolumn{1}{|c|}{Conjs} & \multicolumn{1}{|c|}{Time} &
%   \multicolumn{1}{|c|}{Conjs} & \multicolumn{1}{|c|}{Time} &
%   \multicolumn{1}{|c|}{Conjs} & \multicolumn{1}{|c|}{Time} &
%   \multicolumn{1}{|c|}{Conjs} & \multicolumn{1}{|c}{Time} \\ \hline
% quicksort acc &   &   & 0 &   & 1 &   &   &   & 0 &   \\
% quicksort app &   &   & 1 &   & 1 &   &   &   & 1 &   \\
% fibs & 1 & a & 2 & 3 & 4 & 5 & 6 & 7 & 8 & 9 \\
% icfp2000 & 1 & a & 2 & 3 & 4 & 5 & 6 & 7 & 8 & 9 \\
% mandelbrot & 1 & a & 2 & 3 & 4 & 5 & 6 & 7 & 8 & 9 \\
% mmc & 1 & a & 2 & 3 & 4 & 5 & 6 & 7 & 8 & 9
% \end{tabular}
% \end{center}
% \caption{Results}
% \label{tab:results_temp}
% \end{table*}

% Report analysis times as well as sequential and parallel execution times
% and CPU usage (as integral if possible, as well as peack CPU usage).

We tested our system on three benchmark programs:
matrix multiplication, a mandelbrot image generator and a raytracer.
Matrixmult has abundant independent AND-parallelism.
Mandelbrot uses the actual \code{map\_foldl} predicate
from figure~\ref{fig:map_foldl}
to iterate over rows of pixels.
Raytracer does not use \code{map\_foldl},
but does use a similar code structure to perform a similar task.
This is not an accident:
\emph{many} predicates use this kind of code structure,
partly because programmers in declarative languages
often use accumulators to make their loops tail recursive.
% All three programs do lots of floating point arithmetic,
% and the mandelbrot program does a lot of integer arithmetic as well.
% The Mercury backend we are using always boxes floating point numbers,
% so each floating point operation requires the creation of a new cell on the heap.
% This makes matrixmult and raytracer very memory allocation intensive.
% Since the garbage collector accounts for a large fraction of their runtimes,
% Amdahl's law dictates the maximum speedup we can get by speeding up their mutators
% will be correspondingly limited.

We ran all three programs
with one set of input parameters to collect profiling data,
and with a \emph{different} set of input parameters to produce
the timing results in the following table.
All tests were run on
% taura
a Dell Optiplex 755 PC with a 2.4~GHz Intel Core 2 Quad Q6600 CPU
running Linux 2.6.31.
Each test was run ten times;
we discarded the highest and lowest times, and averaged the rest.

% \begin{table*}[h]
% \begin{center}
% \begin{tabular}{l||r|r|r|r|r|r}
% Program     & Seq   & No autopar   & 1 CPU        & 2 CPUs      & 3 CPUs      &
% 4 CPUs \\
% \hline
% mandelbrot  & 33.4  &  35.3 (0.95) &  35.4 (0.94) & 18.0 (1.85) & 12.2 (2.74) &
 % 9.4 (3.55) \\
% raytracer   & 12.33 & 14.01 (0.88) & 14.77 (0.83) & 9.40 (1.31) & 7.59 (1.62) &
% 6.70 (1.84) \\
% \end{tabular}
% \end{center}
% % \caption{Results}
% % \label{tab:results_temp}
% \end{table*}

% \vspace{-2mm}
% \begin{table*}[h]
% \begin{center}
% \begin{tabular}{|l|l||r|r|r|r|r|}
% %\hhline{|-|-||-|-|-|-|-|}
% \hline
% \multicolumn{1}{|c|}{\textbf{Program}} &
% \multicolumn{1}{ c||}{\textbf{Par}}    &
% \multicolumn{1}{ c|}{\textbf{1 CPU}}   &
% \multicolumn{1}{ c|}{\textbf{2 CPUs}}  &
% \multicolumn{1}{ c|}{\textbf{3 CPUs}}  &
% \multicolumn{1}{ c|}{\textbf{4 CPUs}}  \\
% %\hhline{|-|-||-|-|-|-|-|}
% \hline
% matrixmult & indep & 14.60 (0.75) &  7.55 (1.46) &  6.07 (1.81) &  5.21 (2.11) \\
% seq 11.00  & naive & 14.61 (0.75) &  7.53 (1.46) &  6.75 (1.63) &  5.17 (2.12) \\
% par 14.60  & overlap  & 14.59 (0.75) &  7.57 (1.45) &  5.26 (2.09) &  5.37 (2.05) \\
% %\hhline{|-|-||-|-|-|-|-|}
% \hline
% mandelbrot & indep & 35.27 (0.95) & 35.31 (0.95) & 35.15 (0.95) & 35.31 (0.95) \\
% seq 33.47  & naive & 35.33 (0.95) & 17.87 (1.87) & 12.07 (2.77) &  9.17 (3.65) \\
% par 35.27  & overlap  & 35.16 (0.95) & 17.91 (1.87) & 12.02 (2.78) &  9.15 (3.65) \\
% %\hhline{|-|-||-|-|-|-|-|}
% \hline
% raytracer  & indep & 11.33 (0.87) & 11.37 (0.87) & 11.36 (0.87) & 11.36 (0.87) \\
% seq  9.85  & naive & 11.20 (0.88) &  7.48 (1.32) &  5.91 (1.66) &  5.39 (1.83) \\
% par 11.29  & overlap  & 11.28 (0.87) &  7.56 (1.30) &  5.94 (1.66) &  5.38 (1.83) \\
% %\hhline{|-|-||-|-|-|-|-|}
% \hline
% \end{tabular}
% \end{center}
% % \caption{Results}
% % \label{tab:results_temp}
% \end{table*}
% \vspace{-2mm}

% \vspace{-2mm}
\begin{table}[tb]
\begin{center}
\begin{tabular}{llrrrrr}
\hline \hline
\multicolumn{1}{c}{\textbf{Program}} &
\multicolumn{1}{c}{\textbf{Par}}    &
\multicolumn{1}{c}{\textbf{1 CPU}}   &
\multicolumn{1}{c}{\textbf{2 CPUs}}  &
\multicolumn{1}{c}{\textbf{3 CPUs}}  &
\multicolumn{1}{c}{\textbf{4 CPUs}}  \\
\hline
%\hhline{|-|-||-|-|-|-|-|}
% \hline
% matrixmult & indep & 14.7 (0.76) &  7.6 (1.47) &  5.2 (2.15) &  5.2 (2.15) \\
% seq 11.2   & naive & 14.7 (0.76) &  8.0 (1.40) &  5.7 (1.96) &  4.7 (2.38) \\
% par 14.6   & overlap  & 14.7 (0.76) &  7.6 (1.47) &  6.7 (1.67) &  5.2 (2.15) \\
matrixmult & indep    & 14.6 (0.75) &  7.5 (1.47) &  7.0 (1.66) &  5.2 (2.12) \\
seq 11.0   & naive    & 14.6 (0.75) &  7.6 (1.45) &  5.2 (2.12) &  5.2 (2.12) \\
par 14.6   & overlap  & 14.6 (0.75) &  7.5 (1.47) &  6.2 (1.83) &  5.2 (2.12) \\
%\hhline{|-|-||-|-|-|-|-|}
\hline
mandelbrot & indep    & 35.2 (0.95) & 35.1 (0.95) & 35.2 (0.95) & 35.3 (0.95) \\
seq 33.4   & naive    & 35.4 (0.94) & 18.0 (1.86) & 12.1 (2.76) &  9.1 (3.67) \\
par 35.2   & overlap  & 35.6 (0.94) & 17.9 (1.87) & 12.1 (2.76) &  9.1 (3.67) \\
%\hhline{|-|-||-|-|-|-|-|}
\hline
raytracer  & indep    & 26.2 (0.87) & 26.3 (0.86) & 26.1 (0.87) & 26.2 (0.87) \\
seq 22.7   & naive    & 25.3 (0.90) & 16.0 (1.42) & 11.2 (2.03) &  9.4 (2.42) \\
par 26.5   & overlap  & 25.1 (0.90) & 16.0 (1.42) & 11.2 (2.03) &  9.4 (2.42) \\
%\hhline{|-|-||-|-|-|-|-|}
\hline \hline
\end{tabular}
\end{center}
%\vspace{-2\baselineskip}
\end{table}

Each group of three rows reports the results for one benchmark.
The first column shows the benchmark name,
the runtime of the program when compiled for sequential execution, and
its runtime when compiled for parallel execution
but without enabling auto-parallelisation.
This shows the overhead of support for parallel execution
when it does not buy any benefits.
We auto-parallelised each program three different ways:
executing expensive goals in parallel
only when they are independent (``indep'');
even if they are dependent, regardless of overlap (``naive'');  and
even if they are dependent, but only if they have good overlap (``overlap'').
The last four columns give the runtime in seconds
of each of these versions of the program
on 1, 2, 3 and 4 CPUs,
with speedups compared to the sequential version.

The parallel version of the Mercury system
needs to use a real machine register
to point to thread-specific data,
such as each engine's abstract machine registers.
On x86s, this leaves only one real register for the Mercury abstract machine,
so compiling for parallelism but not using it
yields a slowdown ranging from 5\% on mandelbrot to 25\% on matrixmult.
(We observe such slowdowns for other programs as well.)
On one CPU, autoparallelisation gets only this slowdown,
plus the (small) additional overheads of all the parallel conjunctions
that cannot get any parallelism.
% However, when we move to 2, 3 or 4 CPUs,
% some of the autoparallelised programs do get speedups.

The parallelism in the main predicate of matrixmult is independent,
Overlap parallelises the program the same way as indep,
so it gets the same speedup.
The numbers look different for 3 CPUs,
but all the runs for both versions actually took either 5.2 or 7.5 seconds,
depending (we think) on which way
the OS arranged the engines across the two CPU die of the Q6600;
the indep version just happened to get the 7.5s arrangement fewer times.
For naive, all the runs just happened to take 5.2 seconds,
even though naive creates a worse parallelisation than either indep or overlap:
during the expansion phase we described in section~\ref{sec:overlap_howto},
it includes an extra goal in the first of the parallel conjuncts;
this makes the conjunction dependent, which adds some overhead.
Naive also executes the code that does the matrix multiplication
in parallel with the goals that create its inputs,
which also adds overhead without speedup.
These overheads are too small to affect the results.

In mandelbrot and raytracer, all the parallelism is dependent,
which is why indep gets no speedup for them.
For mandelbrot, naive and overlap get speedups
that are as good as one can reasonably expect:
$35.2/9.1 = 3.87$ on four CPUs over the one CPU case.
% (Perfect speedups of 4.0 on 4 CPUs are not attainable in practice
% due to bottlenecks such as CPU-memory buses and stop-the-world garbage
% collection.)
For matrixmult and raytracer, the speedups they get,
2.12 and 2.42 on four CPUs,
also turn out to be pretty good when one takes a closer look.

For matrixmult, the bottleneck is almost certainly CPU-memory bandwidth.
Each step in this program does only one multiply and one add (both integer)
before creating a new cell on the heap and filling it in.
On current CPUs, the arithmetic takes much less time than the memory writes,
and since the new cells are never accessed again, caches do not help,
which makes it easy to saturate the memory bus, even when using only three CPUs.

The raytracer is very memory-allocation-intensive,
because it does lots of FP arithmetic,
and the Mercury backend we are using always boxes floating point numbers,
so each floating point operation requires
the creation of a new cell on the heap.
Because of this, memory bandwidth may also be an issue for it,
but its bigger problem is GC;
while GC takes only about 5\% of the runtime when run on one CPU,
it takes almost 40\% of the runtime when run on four CPUs,
even though we used four marker threads.
(For fairness, we used four marker threads
regardless of how many CPUs the Mercury code used.)
Given this fact, the best speedup we can hope for is
$(4 \times 0.6 + 0.4)/(0.6 + 0.4) = 2.8$,
and we do come pretty close to that.

GC becomes more expensive with more CPUs
not only because of increased contention,
but also because the GC has more work to do:
with more contexts being spawned, there are more stacks for it to scan.
We have tested versions of the raytracer in which
each spawned-off goal computed the pixels for several rows, not just one,
and these versions yield speedups of about 3.3 on four CPUs.
These versions spawn many fewer contexts, thus putting much less load
on the GC.
This shows that
program transformations that cause more work to be done in each context
are likely to be a promising area for future work.
% We thus expect that applying throttling
% (as described in section~\ref{sec:overlap})
% should significantly improve these results.

Most small programs like these benchmarks
have only one loop that dominates their runtime.
In all three of these benchmarks, and in many others,
the naive and overlap methods will parallelise the same loops,
and usually the same way;
they tend to differ only in how they parallelise code
that executes much less often (typically only once)
whose effect is lost in the noise.
The raw timings show a great deal of variability:
we have seen two consecutive runs of the same program on the same data
differ in their runtime by as much as 15\%.
% (One possible cause of this is differences
% in whether the OS puts frequently-communicating engines
% on cores on the same die, or cores on two different dies.)
% As the table shows,
Some of this variability remains even after filtering and averaging.
% However, the raw times showed significant variability,
% and this process does not entirely eliminate that variability.

To see the difference between naive and overlap,
we need to look at larger programs.
Our standard large test program is the Mercury compiler, which contains
53 conjunctions with two or more expensive goals.
Of these, 52 are dependent,
and only 31 have an overlap
that leads to a predicted local speedup of more than 1\%,
our default threshold.
Our algorithms can thus prevent
the unproductive parallelisation of $53-31=22$ of these conjunctions.
Unfortunately, programs that are large and complex enough
to show a performance effect from this saving
also tend to have large components
that cannot be profitably parallelised with existing techniques,
which means that (due to Amdahl's law)
our autoparallelisation system cannot yield overall speedups for them yet.

On the bright side,
our feedback tool generates feedback files
in less than a second from the profiles of small programs like these benchmarks,
and in only a minute or two even from much larger profiles.
The extra time taken by the Mercury compiler
when it follows the recommendations in feedback files
is so small that it is not noticeable.

% Currently, the Mercury runtime system
% often continues execution, on completion of a parallel conjunction,
% on a CPU different from the one being used before that parallel conjunction.
% When our system finds a smattering of parallel conjunctions
% through a mostly sequential program,
% these switches from a CPU with a warm cache to a CPU with a cold cache
% severely degrade the program's performance.
% Right now, for most programs,
% this effect yields a slowdown significantly bigger
% than the speedups yielded by automatic parallelisation.
% Once this defect is fixed, we hope to report significantly better results
% for more and bigger programs.

% \footnote{
% For referees
% who read this paper together
% with the submission by Wang and Somogyi,
% the version of the raytracer used in that paper
% had manual granularity control;
% the version we are using in this paper is the original version of the program,
% which was not written for parallelism and has no manual granularity control.}

% \begin{itemize}
% \item
% ICFP2000 --- Raytracer
% \item
% Mandelbrot Image generator.
% \item
% Variations on the above two programs including varying degrees
% of dependence and a \mapfoldl version.
% \item
% ICFP2001 --- SGML optimizer
% \item
% Compiler --- We do not expect this to speed up.
% However it will be useful to ensure that it does not slow down too much.
% \item
% pic --- a NuFib benchmark by ported by Peter
% \item
% SWRL --- An inference engine provided by Mission Critical.
% % The mission critical benchmark is at:
% %
% %     taura:/home/taura/pbone/mcdemo/swrl-snapshot.tar.gz
% %
% % SWRL is an inference engine.  Building and running it is covered by:
% %
% %     taura:/home/taura/pbone/mcdemo/swrl.txt
% %
% % These files are owned by a new group, mcdemo, since they are MC's
% % closed source project.  Peter Ross is easy going and has let us use
% % them without any formal non-disclosure agreement.  That said, I am
% % respecting this IP as much as I would anything where I had signed a
% % formal NDA.
% \end{itemize}

% Two programs, a raytracer and a mandelbrot image generator showed
% strong speedups, see figure \ref{tab:results}.
% This confirms that our analysis is correctly identifying parallelism
% available within the main loops of these programs.
% The two programs have a similar structure, they both have a loop with
% an accumulator that contains the rows of the images already rendered.
% We believe that a lot of dependent parallelism has this form as
% programmers in declarative languages are trained to use accumulators
% in their loops to ensure that the loop is tail-recursive.
% The mandelbrot program's loop uses \mapfoldl example above.
% \paul{Should I include a back-reference for the mapfoldl figure?}
%
% We used different inputs for the profiling and benchmarking
% executions to ensure that the profile analysis would not over-fit the
% parallelisation to a particular input.

% Other programs tested included the Mercury compiler and pic, a program
% ported to Mercury from the nofib~\cite{nofib} benchmark suite.
% Our analysis found exploitable parallelism within these benchmarks,
% however,
% it appears to be too fine-grained for Mercury's runtime to handle.
% We hope to correct these problems fix before the camera ready deadline.

% I will need to test a na\"ive approach, for instance: assuming maximum
% overlap, or factoring in some fixed cost for each shared variable.

% \section{Related work}
% \label{sec:related_work}

\section{Related work and conclusion}
\label{sec:overlap_conc}

% Mercury's strong mode system
% greatly simplifies the parallel execution of logic programs,
% making the comparison of parallel Mercury with parallel Prolog difficult.
% For example, \cite{Hermenegildo1995} defines non-strict
% goal independence such that goals that are non-strictly independent can be
% run in parallel without leading to incorrect results.
% Because Mercury
% statically determines a single goal in a conjunction to bind each variable,
% and because Mercury does not permit variables to be aliased,
% the conditions of non-strict goal independence
% are not necessary for Mercury to guarantee correctness.
% Similarly, other existing work on AND-parallelism in Prolog
% is not closely related to the present work,
% because Mercury sidesteps the
% problems that work seeks to overcome.
% \peter{Is that too hand-wavey and dismissive?}

Mercury's strong mode and determinism systems
greatly simplify the parallel execution of logic programs.
The information gathered by semantic analysis in Mercury
makes it easy to solve most of the problems faced by the
designers of parallel versions of Prolog and Prolog-like languages.
These include testing the independence of goals
in systems that support only independent AND-parallelism
and discovering producer-consumer relationships
in systems that also support dependent AND-parallelism,
such as \cite{DBLP:journals/tcs/GrasH09}.
They also make it possible to \emph{avoid} having to solve some tough problems,
the main example being how to execute nondeterministic conjuncts in parallel
without excessive overhead.

% That is what they were \emph{designed} to do.
% The information gathered by semantic analysis in Mercury
% Many problems in the parallel execution of Prolog and Prolog-like languages,
% like testing the independence of goals
% in systems that support only independent AND-parallelism,
% discovering producer-consumer relationships at runtime
% in systems that also support dependent AND-parallelism,
% and having to handle nondeterministic conjuncts,
% disappear completely,
% with the answers to the problem being presented on a silver platter
% Our group designed Mercury specifically to ensure this.

Most research in parallel logic programming so far
has focused on trying to solve these problems
of getting parallel execution to \emph{work} well,
with only a small fraction trying to find
when parallel execution would actually be \emph{worthwhile}.
Almost all previous work on automatic parallelisation
has focused on granularity control:
parallelising only computations that are expensive enough
to make parallel execution
worthwhile \cite{harris_07_feedback_imp_par,lopez96:distance_granularity},
and properly accounting for the overheads
of parallelism itself \cite{shen_98_granularity-control}.
Most of the rest has tried to find opportunities
to exploit independent AND-parallelism
during the execution of otherwise-dependent conjunctions
\cite{DBLP:journals/jlp/MuthukumarBBH99,DBLP:conf/lopstr/CasasCH07}.

Our experience with our feedback tool shows that
for Mercury programs, this is far from enough.
For most programs,
it finds enough conjunctions with two or more expensive conjuncts,
but almost all are dependent,
and, as we mention in section~\ref{sec:perf},
many of these have too little overlap to be worth parallelising.
% For example, the Mercury compiler contains
% 50 conjunctions with two or more expensive goals.
% 49 of these are dependent.
% Of these, only 38 of these have any overlap,
% and only for 31 does the overlap
% lead to a predicted local speedup of more than 1\%.

We know of only three attempts to estimate the overlap
between parallel computations.
One was in the context of speculative execution in imperative programs.
Given two successive blocks of instructions,
\cite{von_Praun:2007:implicit_parallelism_with_ordered_transactions}
% estimates the likely speedup
% from executing the two blocks in parallel
% by using the difference between the addresses of two instructions
decides whether the second block should be executed speculatively
based on the difference between the addresses of two instructions,
one that writes a value to a register and one that reads from that register.
% This is effectively a binary metric.
This works if instructions take a bounded time to execute,
but in the presence of call instructions
this heuristic will not be at all accurate.

Another attempt was a previous auto-parallelisation project for
Mercury \cite{tannier:2007:parallel_mercury}.
% This did not use profiling data,
% and instead used the number of shared variables between conjuncts
This used the number of shared variables between conjuncts
as a measure of the dependency between goals,
and as a predictor of the likely overlap.
While two conjuncts are indeed less likely
to have useful parallel overlap if they have more shared variables,
we have found this heuristic too inaccurate to be useful.

The most closely related work to ours
generated parallelism annotations for the ACE and/or-parallel system
\cite{Pontelli97automaticcompile-time}.
This system used, much as we do,
estimates of the costs of calls
and of the times at which variables are produced and consumed.
However, it produced its estimates through static analysis of the program.
This can work for small programs,
where the call trees of the relevant calls can be quite small and regular.
In large programs, the call trees of the expensive calls
are almost certain to be both tall and wide,
with a huge gulf between best-case and worst-case behaviour.
Using profiling data is the only way
for an automatic parallelisation system to find out
what the \emph{typical} behaviour of such calls is.

% There is a risk that the program could have changed between the
% profiling build and the parallelised build,
% this makes it more difficult for the compiler to apply the profiling
% advice.
% To reduce this risk the profiling build should be built with the same
% optimizations that the parallelised build will be built with.
% In usual circumstances inlining should be disabled during profiling so
% that a programmer can more easily understand their program's profile.
% Our implementation re-enables inlining in profiling builds if a
% suitable optimization level is selected and
% \code{--profile-for-implicit-parallelism} is passed to the compiler.
% % XXX: These details may be unimportant, especially the name of this
% % compiler option,  But this is (for now) an easy way to describe
% % this.

Our system's predictions of the likely speedup from parallelising a conjunction
are also fallible, since they currently ignore several relevant issues,
including cache effects
and the effects of bottlenecks
such as CPU-memory buses and stop-the-world garbage collection.
However, our system seems to be a sound basis for such further refinements.
% However, they come much closer
% to predicting actual overlaps than previous attempts,
% and our system seems to be a sound basis for further refinements.
% \begin{itemize}
% \item
% It is hard to define what a typical workload is,
% and we do not yet implement profile merging.
% \item
% The feedback framework is general purpose
% and can be used for other optimizations.
% \item
% \zoltan{I haven't covered any technical details about the feedback framework.
% I guess there's not much to say.}
% \end{itemize}
In the future, we plan to support parallelisation as a specialisation:
applying a specific parallelisation only when a predicate is called
from a specific parent, grandparent or other ancestor.
% we will look at how best to resolve cases
% where our tool gives different parallelisation advice for the same conjunction
% due to the different behaviour of that conjunction in different contexts.
We also plan to modify our feedback tool
to accept several profiling data files,
with a priority scheme to resolve any conflicts.
% between their advice.

% \paragraph{Acknowledgements}
We thank the rest of the Mercury team,
and Tom Conway and Peter Wang in particular,
for creating the infrastructure we build upon,
and the anonymous referees for their suggestions.

\section{Related work}

\paul{Say how this work differs from my honours work.}

The previous version of the tool did not use the call graph.
Instead it looked at the most expensive procedures,
considering only those whose cost was above its own threshold.
The callgraph traversal is better because it allows us to avoid parallelising
a callee when the caller already provides enough parallelism.


% TODO Items.


% \begin{algorithm}
% \begin{verbatim}
% MaxBefore := 0
% N := num_conjuncts(Conjs)
% for i in 1 to N:
%     if conjunct i in Conjs is below threshold then
%         MaxBefore := i
%     else
%         break
%
% MinAfter := N+1
% for i in N downto 1:
%     if conjunct i in Conjs is below threshold then
%         MinAfter := i
%     else
%         break
%
% BestTime := infinity
% Arrangements := [[[conjunct MaxBefore+1]]]
% # each element in Arrangements is
% #   a list of parallel conjuncts
% # each parallel conjunct consists of
% #   a list of consecutive conjuncts
% for i in MaxBefore+2 to MinAfter-1:
%     NewArrangements := []
%     for Arrangement in Arrangements:
%         ExtendLast := all_but_last(Arrangement)
%             ++ [last(Arrangement) ++ conjunct i]
%         AddNewLast := Arrangement ++ [conjunct i]
%         NewArrangements := NewArrangements ++
%             [ExtendLast, AddNewLast]
%     Arrangements := NewArrangements
%
%     for Before in 0 to MaxBefore:
%         for After in MinAfter to N+1:
%             GoalsBefore := conjuncts 1 .. Before in Conjs
%             GoalsAfter  := conjuncts After .. N in Conjs
%             # GoalsBefore and/or GoalsAfter may be empty
%             ExtraGoalsBefore := conjuncts (Before+1) .. MaxBefore in
%                Conjs
%             ExtraGoalsAfter := conjuncts MinAfter .. (After-1) in
%                Conjs
%
%             for each Arrangement in Arrangements
%                 Arrangement := [ExtraGoalsBefore ++ first(Arrangement)] ++
%                    all_but_first_and_last(Arrangement) ++
%                     [last(Arrangement) ++ ExtraGoalsAfter]
%                 ParConj := par_conj(Arrangement)
%                 OverallGoal :=
%                     seq_conj(GoalsBefore ++ [ParConj] ++ GoalsAfter)
%                 Time := compute_par_exec_time(OverallGoal)
%                 if Time < BestTime:
%                     BestTime := Time
%                     BestGoal := OverallGoal
% \end{verbatim}
% \caption{Search for best parallelisation}
% \label{alg:branch_and_bound_search}
% \end{algorithm}



\paul{
    Add discussion about how our data-flow analysis doesn't recognize loops
    since they need to be treated specially.}

\chapter{Loop Control}
\label{chap:loop_control}
% vim: ts=6 sw=4 et ft=tex

\chapter{Loop Control}
\label{chap:lc}
\label{chap:loop_control}

\status{Peter has checked the first 3 sections and I will retrieve them from
him later.
The next three sections are now ready for someone to check.}

In the previous chapter we described our
system that uses profiling data
to automatically parallelise Mercury programs by
finding conjunctions with expensive conjuncts
that can run in parallel with minimal synchronisation delays.
This worked very well in some programs but not as well as we had hoped for
others,
including as the raytracer.
This is because the way Mercury must execute dependent conjunctions,
and the way programmers typically write logic programs are at odds.
We introduced this as
``the right recursion problem''
in Section~\ref{sec:rts_original_scheduling_performance}.

In this chapter we present a novel program transformation that eliminates
this problem in all situations.
The transformation has several benefits:
First, it reduces peak memory consumption
by putting a limit on how many stacks
a conjunction will need to be alive at the same time.
Second,
it reduces the number of synchronisation barriers needed
from one per loop iteration to one per loop.
Third, it allows recursive calls inside parallel conjunctions to take
advantage of tail recursion optimisation.
Finally, it obsoletes the conjunct reordering transformation.
Our benchmark results show that our new transformation
greatly increases the speedups we can get from parallel Mercury programs;
in one case, it changes no speedup into almost perfect speedup on four cores.

We have written about the problem elsewhere in the dissertation,
however we have found that this problem is sometimes difficult to
understand.
Therefore
the introduction section
(Section~\ref{sec:lc_intro})
briefly describes the problem,
providing only the details necessary to understand and evaluate the rest of
this chapter.
\paul{It also motivates the reader correctly, but I do not say this.}
For more details about the problem see
Sections~\ref{sec:rts_original_scheduling}
and~\ref{sec:rts_original_scheduling_performance},
see also \citet{bone:2012:loop_control}, the paper on which this chapter is
based.
The rest of the chapter is organised as follows.
The structure of the remainder of this chapter is as follows.
Section~\ref{sec:lc_transformation} describes
the program transformation we have developed
to control memory consumption by loops.
Section~\ref{sec:lc_perf} evaluates
how our system works in practice on some benchmarks.
Section~\ref{sec:lc_further_work}
describes potential further work,
and
Section~\ref{sec:lc_conc} concludes with discussion of related work.

\section{Introduction}
\label{sec:lc_intro}

\status{This section is ready for Peter Schachte to check.}

The implementation of a parallel conjunction
has to execute the first conjunct and spawn off the later conjuncts.
For dependent conjunctions, it cannot be done the other way around,
because only the first conjunct is guaranteed to be immediately executable:
later conjuncts may need to wait for data to be generated by earlier
conjuncts.
This poses a problem when the last conjunct contains a recursive call,
which happens often,
since in logic programming languages,
tail recursive code has long been the preferred way to write a loop.
In independent code,
we can workaround the issue by reordering the conjuncts
in a conjunction (see Section~\ref{sec:rts_reorder}),
but in dependent code this is not possible.
The problem is that
\begin{itemize}
\item
the state of the computation up to this iteration of the loop
is stored in the stack used by the original computation's context,
whereas
\item
the state of the computation after this iteration of the loop
is stored in the stack used by the spawned off context.
\end{itemize}
We can continue the computation after the parallel conjunction
only if we have both the original stack
and the results computed on the spawned-off stack.
This means the original stack must hang around
until the recursive call is done.
However, there is no way to distinguish
the first iteration of a loop from the second, third etc,
so we must preserve the original stack on \emph{every} iteration.

We call this the ``right recursion problem'' because its effects are at
their worst when the recursive call is on the right hand side of a
parallel conjunction operator.
Unfortunately,
this is a natural way to (manually or automatically) parallelise programs
that were originally written with tail recursion in mind.
Thus, parallelisation often
transforms tail recursive sequential computations,
which run in constant stack space,
into parallel computations
that allocate a complete stack for each recursive call
and do not free them until the recursive call returns.
This means that the each iteration effectively requires memory
to store an entire stack, not just a stack frame.

\picfigurelabel{linear_context_usage}{fig:linear_context_usage2}{Linear context usage in right recursion}

Figure~\ref{fig:linear_context_usage2} shows a visualisation of this stack
usage.
At the top left,
four contexts are created and they execute four iterations of a loop;
this execution is indicated by boxes.
Once each of these iterations finishes,
its context stays in memory but is suspended,
indicated by the long vertical lines.
Another four iterations of the loop create another four contexts, and so on.
Later, when all iterations of the loop have been executed,
each of the blocked contexts resumes execution and immediately exits,
indicated by the horizontal line at the bottom of each of the vertical
lines.

If we allow the number of contexts, and therefore their stacks, to grow
unbounded, then the program will very quickly run out of memory,
often bringing the operating system to its knees.
Which is why we introduced the context limit work around
(described in Section~\ref{sec:rts_original_scheduling_performance})
which can prevent a program from crashing,
but which limits the amount of parallel execution.

% The natural way to execute a parallel version of such a loop
% is to keep spawning off the task of performing each iteration
% until all the available cores are busy
%

Our transformation explicitly limits
the number of stacks allocated to recursive calls
to a small multiple of the number of available processors in the system.
This transformation can also be asked
to remove the dependency of a parallel loop iteration
on the parent stack frame from which it was spawned,
allowing the parent frame to be reclaimed
before the completion of the recursive call.
This allows parallel tail recursive computations
to run in constant stack space.
The transformation is applied
after the automatic parallelisation transformation,
so it benefits both manually and automatically parallelised Mercury code.

% Our benchmark results are very encouraging.
% Limiting the number of stacks
% not only permits deep tail recursions to take advantage of multiple cores,
% but it also significantly improves performance.
% For most of our benchmarks, we get near-optimal speedups.
% % The second
% % transformation does not produce a speed improvement; in some cases it
% % causes a slight slow-down.  However, this small price is well worth
% % paying to allow parallel tail recursive computations to run in
% % constant stack space.

% % This is the old explaination, which is now covered in Ch3.
% \section{The main problem}
% \label{sec:problem}
%
% As Mercury is a declarative programming language,
% Mercury programs make heavy use of recursion.
% Like the compilers for most declarative languages,
% the Mercury compiler optimises tail recursive procedures
% into code that can run in constant stack space.
% Since this generally makes tail recursive computations
% more efficient than code using other forms of recursion,
% typical Mercury code makes heavy use of tail recursion in particular.
%
% Unfortunately, tail recursive computations are not naturally handled well
% by Mercury's implementation of parallel conjunctions.
% Consider the \mapfoldl{} predicate in Figure~\ref{fig:mapfoldl}.
% This code applies the map predicate \code{M} to each element of an input list,
% and then uses the fold predicate \code{F}
% to accumulate (in a left-to-right order) all the results produced by \code{M}.
% The best parallelisation of \mapfoldl{} executes
% \code{M} and \code{F} in parallel with the recursive call.
% \tr{
% \code{M(H, MappedH), F(MappedH, Acc0, Acc1)} in parallel with
% Although all executions of \code{M} are independent
% and need not wait for anything to begin their computation,
% each call to \code{F} must wait until
% the call to \code{M} generates the value of \code{MappedH}.
% Thus there would be no point in executing \code{M} and \code{F} in parallel:
% \code{F} would immediately suspend until \code{M} had produced its result.
% However, the recursive call to \mapfoldl \emph{can} begin in parallel,
% allowing the next call to \code{M} to run in parallel with this iteration.
% }
% The programmer (or an automatic tool) can make this happen
% in the original sequential version of \mapfoldl
% by replacing the comma before the recursive call
% with the parallel conjunction operator \verb'&'.

% % Example.
% The problem is that the execution of a call to \mapfoldlpar{}
% has bad memory behaviour.
% When a context begins execution of a call to \mapfoldl{},
% it begins by creating a spark for the second conjunct
% (which contains the recursive call),
% and executes the first conjunct (which starts with the call to \code{M}).
% If another Mercury engine is available at that time,
% it will pick up and execute the spark for the recursive call,
% itself creating a spark for another recursive call
% and executing the \emph{next} call to \code{M}.
% This will continue until all Mercury engines are in use
% and the newest spark for a recursive call % to \mapfoldlpar{}
% must wait for an % available
% engine.
% When an engine completes execution of \code{M} and \code{F},
% it posts the value of \code{Acc1} into \code{FutureAcc1}.
% Any computations waiting for \code{Acc1} will then be woken up;
% these will be the calls that wait for \code{Acc0} in the next iteration.
% In this case, the woken code will resume execution
% immediately before the call to \code{F}
% in the recursive invocation of \mapfoldlpar{}.
%
% One might hope that after a spark for the recursive call has been created,
% and once \code{M} and \code{F} had completed execution
% and \code{Acc1} has been signalled,
% the context used to execute the first conjunct could be released.
% Unfortunately, it cannot because this context is the one that was running
% when execution entered the parallel conjunction,
% and therefore this is the context whose stacks
% contain the state of the computation outside the parallel conjunction.
% If we allowed this context to be reused,
% then all this state would be lost.
%
% This means that until the base case of the recursion is reached,
% \emph{every} recursive call must have its own complete execution context.
% Since each context contains two stacks,
% it can occupy a rather large amount of memory,
% so it is not practical to simultaneously preserve an execution context
% for each and every recursive call to a tail-recursive predicate.
% Originally, programs which bumped into this problem
% often ran themselves and the operating system out of memory rather quickly,
% because the default size of every det stack was several megabytes.
% To reduce the scope of the problem,
% we made stacks dynamically expandable,
% which allowed us to reduce their initial size,
% but programs with the problem can still run out of memory,
% it just takes more iterations to do so.
% \label{sec:context_limit}
% Our runtime system prevents such crashes
% by imposing a global limit on the number of contexts
% that can be running or suspended at any point:
% if a context is needed to execute a spark
% and allocating the context would breach this limit,
% then the spark will not be executed.
% Eventually, the context that created the spark will execute it on its
% own stack, but this limits the remainder of the recursive computation
% to use only that context, so parallelism is curtailed at that point.
%
% A much better solution is to swap the order of the conjuncts
% in the parallel conjunction
% so that the conjunct containing the recursive call is executed first.
% This means that we will spawn off the non-recursive conjuncts,
% whose contexts \emph{can} be freed when their execution is complete.
% However, since the Mercury mode system requires that
% the producer of a variable precede all its consumers,
% this is possible only if the conjuncts are independent.
% The approach we have taken in this paper
% is to spawn off the non-recursive conjuncts,
% and continue execution of the recursive call
% without swapping the order of the conjuncts.
% We also directly limit the number of contexts that are used in a loop
% to a small multiple of the number of available CPUs.
% Finally, we can arrange for
% the inputs and outputs of the non-recursive conjuncts
% to be stored outside the stack frame of a tail recursive procedure,
% which allows such procedures to run in fixed stack space
% even when executed in parallel.
% In the next section, we explain all of these improvements.
% % If they are dependent, this solution is not applicable,
% % so for such conjunctions we need a completely different solution.
% % We present one in the next section.

% \peter{Not sure how to cover this:}
% \paul{This is not as important as other issues,
% I am not sure it is worth confusing the reader.}
% zs: compiling with and without loop control both have their own unique
% overheads. There is not a clear advantage either way, so the issue is not
% important enough to mention.
%
% There is at most, only one spark on the spark queue,
% which means that parallel work is not abundant.
% There are likely to be more context switches.
% These problems are secondary to the memory consumption problems.
% However, loop control also fixes them ensuring that loop control has lower
% overhead than parallel conjunctions.

\section{The loop control transformation}
\label{sec:lc_transformation}

\status{This section is ready for someone to check.}

\begin{figure}[tb]
\begin{verbatim}
map_foldl_par(M, F, L, FutureAcc0, Acc) :-
    lc_create_loop_control(LC),
    map_foldl_par_lc(LC, M, F, L, FutureAcc0, Acc).

map_foldl_par_lc(LC, M, F, L, FutureAcc0, Acc) :-
    (
        L = [],
        % The base case.
        wait_future(FutureAcc0, Acc0),
        Acc = Acc0,
        lc_finish(LC)
    ;
        L = [H | T],
        new_future(FutureAcc1),
        lc_wait_free_slot(LC, LCslot),
        lc_spawn_off(LC, LCslot, (
            M(H, MappedH),
            wait_future(FutureAcc0, Acc0),
            F(MappedH, Acc0, Acc1),
            signal_future(FutureAcc1, Acc1),
            lc_join_and_terminate(LCslot, LC)
        )),
        map_foldl_par_lc(LC, M, F, T,
            FutureAcc1, Acc)
    ).
\end{verbatim}
%\vspace{2mm}
\caption{\mapfoldlpar after the loop control transformation}
\label{fig:map_foldl_transformed}
%\vspace{-1\baselineskip}
\end{figure}

The main aim of loop control is to set an upper bound
on the number of contexts that a loop may use,
regardless of how many iterations of the loop may be executed,
without limiting the amount of parallelism available.
The loops we are concerned about
are procedures that we call right recursive:
procedures in which the recursive execution path
ends in a parallel conjunction,
whose last conjunct contains the recursive call.
A right recursive procedure may be tail recursive, or it may not be:
the recursive call could be followed by other code
either within the last conjunct, or after the whole parallel conjunction.
Programmers have long tended to write loops whose last call is a recursive
call in order to benefit from tail recursion.
\paul{My recursion type analysis does not yet separate right and left
recursion,
so I cannot actually say how common it is.}
Therefore,
right recursion is very common;
most parallel conjunctions in recursive procedures are right recursive.

To guarantee the imposition of an upper bound
on the number of contexts created during one of these loops,
we associate with each loop a data structure
that has a fixed number of slots,
and require each iteration of the loop that would spawn off a goal
to reserve a slot for the context of each spawned-off computation.
This slot is marked as in-use until that spawned-off computation finishes,
at which time it becomes available for use by another iteration.

This scheme requires us to use two separate predicates:
the first sets up the data structure
(which we call the \emph{loop control} structure)
and the second actually performs the loop.
The rest of the program knows only about the first predicate;
the second predicate is only ever called from the first predicate
and from itself.
Figure~\ref{fig:map_foldl_transformed} shows what these predicates look like.
In Section~\ref{sec:lc_structs},
we describe the loop control structure and the operations on it;
in Section~\ref{sec:lc_trans},
we give the algorithm that does the transformation;
while in Section~\ref{sec:lc_tailrec},
we discuss its interaction with tail recursion optimisation.

\subsection{Loop control structures}
\label{sec:lc_structs}

\begin{figure}
\begin{verbatim}
typedef struct MR_LoopControl_Struct        MR_LoopControl;
typedef struct MR_LoopControlSlot_Struct    MR_LoopControlSlot;

struct MR_LoopControlSlot_Struct
{
    MR_Context                *MR_lcs_context;
    MR_bool                   MR_lcs_is_free;
};

struct MR_LoopControl_Struct
{
    volatile MR_Integer       MR_lc_outstanding_workers;

    MR_Context* volatile      MR_lc_master_context;
    volatile MR_Lock          MR_lc_master_context_lock;

    volatile MR_bool          MR_lc_finished;

    unsigned                  MR_lc_free_slot_hint;

    /*
    ** MR_lc_slots MUST be the last field, since in practice, we treat
    ** the array as having as many slots as we need, adding the size of
    ** all the elements except the first to sizeof(MR_LoopControl) when
    ** we allocate memory for the structure.
    */
    unsigned                  MR_lc_num_slots;
    MR_LoopControlSlot        MR_lc_slots[1];
};
\end{verbatim}
\caption{Loop control structure}
\label{fig:loop_control_structure}
\end{figure}

The loop control structure
(shown in Figure~\ref{fig:loop_control_structure})
contains the following fields:
\begin{description}

\item[\code{MR\_lc\_slots}]
is an array of slots, each of which contains a boolean and a pointer.
The boolean says whether the slot is free,
and if it is not,
the pointer points to the context that is currently occupying it.
When the occupying context finishes,
the slot is marked as free again,
but the pointer remains in the slot
to make it easier (and faster) for the next computation that uses that slot
to find a free context to reuse.
Therefore we cannot encode the boolean in the pointer being \NULL or non-\NULL.
Although this is the most significant field in the structure it is last so that
the array can be stored with the data structure, and an extra memory
dereference can be avoided.
This is the last field in the structure even though it is the most significant field.
% (The description of the \code{lc\_wait\_free\_slot(LC)} operation below
% will show why cannot we encode the boolean
% in the pointer being null/non-null.)

\item[\code{MR\_lc\_num\_slots}]
stores the number of slots in the array.

\item[\code{MR\_lc\_outstanding\_workers}]
is the count of the number of slots that are currently in use.

\item[\code{MR\_lc\_master\_context}]
is a possibly null pointer to the \emph{master} context,
the context that created this structure,
and the context that will spawn of all of the iterations.
This slot will point to the master context whenever it is sleeping,
and will be \NULL at all other times.

\item[\code{MR\_lc\_master\_context\_lock}]
a mutex that is used to protect access to MR\_lc\_master\_context.
The other fields are protected using atomic instructions;
we will describe them when we show the code for the loop control procedures.
% \zoltan{Should we explain how accesses to some fields
% can dispense with the mutex?}
% \paul{We decided not to explain this}

\item[\code{MR\_lc\_finished}]
A boolean flag that says whether the loop has finished or not.
It is initialised to false, and is set to true
as the first step of the \lcfinish operation.

\item[\code{MR\_lc\_free\_slot\_hint}]
contains an index into the array of free slots.
It indicates which slot may be free;
we use it to speed up the search for a free slot in the average case.
\zoltan{we need an argument for WHY this is a speedup,
but that argument does not belong here,
since it depends on the behaviour of operations we have not described yet.
Alternatively, we can completely avoid mentioning the hint field.}
\paul{I think it is easier to not mention it, it is optimal in some cases and
neither optimal nor pessimal in other cases.}

\end{description}

\noindent
The finished flag
is not strictly needed for the correctness of the following operations,
but it can help the loop control code cleanup at the end of a loop more quickly.
In our description of these operations,
\LC is a reference to the whole of a loop control structure,
while \LCS is an index into the array of slots stored within \LC.

\begin{description}
\item[\code{LC = lc\_create\_loop\_control()}]
This operation creates a new loop control structure,
and initialises its fields.
The number of slots in the array in the structure
will be a small multiple of the number of cores in the system.
The multiplier is configurable
by setting an environment variable when the program is run.

\begin{algorithm}[tbp]
\begin{algorithmic}
\Procedure{MR\_lc\_wait\_free\_slot}{$lc$, $retry\_label$}
    %unsigned    hint, offset, i;

    \If{$lc.MR\_lc\_outstanding\_workers = lc.MR\_lc\_num\_slots$}
        \State MR\_aquire\_lock($lc.MR\_lc\_master\_context\_lock$)
        \If{$lc.MR\_lc\_outstanding\_workers = lc.MR\_lc\_num\_slots$}
            %MR\_Context *ctxt;
            \State /* Only commit to sleeping while holding the lock. */
            %so retest the outstanding worker count.
            \State $ctxt \gets$ MR\_ENGINE($MR\_eng\_this\_context$)
            \State MR\_save\_context($ctxt$)
            \State $ctxt.MR\_ctxt\_resume \gets retry\_label$
            %\State $ctxt.MR\_ctxt\_resume\_owner\_engine \gets$ MR\_ENGINE($MR_eng_id$)
            \State $lc.MR\_lc\_master\_context \gets ctxt$
            %MR_CPU_SFENCE;
            \State MR\_release\_lock($lc.MR\_lc\_master\_context\_lock$)
            \State MR\_ENGINE($MR\_eng\_this\_context$) $\gets$ \NULL
            \State MR\_idle()
        \EndIf 
        \State MR\_release\_lock($lc.MR\_lc\_master\_context\_lock$)
    \EndIf

    \State $hint \gets lc.MR\_lc\_free\_slot\_hint$

    \For{$offset \gets 0$ to $lc.MR\_lc\_num\_slots$}
        \State $i \gets (hint + offset) \bmod lc.MR\_lc\_num\_slots$
        \If{$lc.MR\_lc\_slots[i].MR\_lcs\_is\_free$}
            \State $lc.MR\_lc\_slots[i].MR\_lcs\_is\_free \gets false$
            \State $lc.MR\_lc\_free\_slot\_hint \gets
                (i + 1) \bmod lc.MR\_lc\_num\_slots$
            \State MR\_atomic\_inc\_int($lc.MR\_lc\_outstanding\_workers$)
            \State $lcs\_idx \gets i$
            \State \Break
        \EndIf 
    \EndFor 

    \If{$lc.MR\_lc\_slots[i].MR\_lcs\_context = $\NULL}
        %\Comment Allocate a new context.
        \State $lc.MR\_lc\_slots[i].MR\_lcs\_context \gets$
            MR\_create\_context()
        %\State
        %    $lc.MR\_lc\_slots[i].MR\_lcs\_context.MR\_ctxt\_thread\_local\_mutables
        %    \gets MR\_THREAD\_LOCAL\_MUTABLES$
    \EndIf

    \State reset\_context\_stack\_ptr($lc.MR\_lc\_slots[i].MR\_lcs\_context$)

    \State \Return $lcs\_idx$
\EndProcedure
\end{algorithmic}
\caption{\lcwaitfreeslot}
\label{alg:lc_free_slot}
\end{algorithm}

\item[\code{LCslot = lc\_wait\_free\_slot(LC)}]
% This operation tests \linebreak[3] whether \LC{} has any free slots.
% This hyphenation improves this paragraph, I have swapped one evil for
% another.
% This operation tests whe\-ther \LC{} has any free slots.
This operation tests whether \LC{} has any free slots.
If it does not, the operation suspends until a slot becomes available.
When some slots are available, either immediately or after a wait,
the operation chooses one of the free slots, marks it in use,
fills in its context pointer and returns its index.
It can get the context to point to
from the last previous user of the slot,
from a global list of free contexts,
(in both cases it gets contexts which have been used previously
by computations that have terminated earlier),
or by allocating a new context
(which typically happens only soon after startup).
\paul{Keep the algorithm? Add algorithms for other operations?}

% I have edited this so that it does not refer to sparks, since they are not used
% with loop control.
\item[\code{lc\_spawn\_off(LC, LCslot, CodeLabel)}]
This operation sets up the context in the loop control slot,
and then puts it on the global runqueue,
where any engine looking for work can find it.
Setup of the context consists of initialising the context's parent stack
pointer to point to its master's stack frame,
and the context's resume instruction pointer to the value of \code{CodeLabel}.
% \paul{Possibly:
% Setup consists of creating a stack frame on the context's stack and copying
% the relevant values from the master's stack frame onto the worker-context's.
% Also, the context's resume instruction pointer will be initialised to
% the value of \code{CodeLabel}.
% }

\item[\code{lc\_join\_and\_terminate(LC, LCslot)}]
This operation marks the slot named by \LCS{} in \LC{} as available again.
It then terminates the context executing it,
allowing the engine that was running it to look for other work.

\item[\code{lc\_finish(LC)}]
This operation is executed by the master context when we know
that this loop will not spawn off any more work packages.
It suspends its executing context
until all the slots in \LC{} become free.
This will happen only when all the goals spawned off by the loop
have terminated.
This is necessary to ensure that
all variables produced by the recursive call
that are \emph{not} signalled via futures
have in fact had values generated for them.
A variable generated by a parallel conjunct
that is consumed by a later parallel conjunct will be signalled via a future,
but if the variable is consumed only by code after the parallel conjunction,
then it is made available by writing its value directly in its stack
slot.
Therefore such variables can exist
only if the original predicate had code after the parallel conjunction;
for example, map over lists
must perform a construction after the recursive call.
This barrier is the only barrier in the loop and it is executed just once;
in comparison, the normal parallel conjunction execution mechanism
executes one barrier in each iteration of the loop.
\end{description}

\zoltan{Should we give the pseudo-code of the operations?}
\paul{XXX: I have given the pseudo-code for one of these, if I have time I'll add
pseudo-code for the others.}

\noindent
See Figure~\ref{fig:map_foldl_transformed}
for an example of how we use these operations.
Note in particular that in this transformed version of \mapfoldl{},
the spawned-off computation contains the calls to \var{M} and \var{F},
with the main thread of execution making the recursive call.
This is the first step in preserving tail recursion optimisation.

% Some of these operations also perform some scheduling;
% we will discuss that later.

\picfigure{lc_context_usage}{Loop control context usage}

Figure~\ref{fig:lc_context_usage} shows a visual representation of context
usage when using loop control;
it should be compared with Figure~\ref{fig:linear_context_usage2}.
As before, this is how contexts are likely to be used on a four processor
system;
minor differences in the execution times of each task and similar variables
will mean that no execution will look as regular as in the figures.
In the Figure~\ref{fig:lc_context_usage},
we can see that a total of eight contexts are created and four are in use at
a time.
When the loop begins,
the master thread performs a few iterations, executing \lcwaitfreeslot and
\lcspawnoff.
This creates all the contexts and  adds them to the runqueue,
but only the first for contexts can be executed as there are only for
processors.
Once those contexts finish their work,
they execute \lcjoinandterminate.
Each call to \lcjoinandterminate marks the relevant slot in the loop control
structure as free,
allowing the master context to spawn off more work using the free slot.
Meanwhile, the other four contexts are now able to execute their work.
This continues until the loop is finished, at which point \lcfinish releases
all the contexts.

\subsection{The loop control transformation}
\label{sec:lc_trans}

Our algorithm for transforming procedures to use loop control
is shown in
Algorithms~\ref{alg:transform_alg},~\ref{alg:reccases_alg} and~\ref{alg:basecases_alg}.

\begin{algorithm}[tbp]
\begin{algorithmic}
\Procedure{loop\_control\_transform}{$OrigProc$}
  \State $OrigGoal \gets$ body($OrigProc$)
  \State $RecParConjs \gets$ set of parallel conjunctions
    in $OrigGoal$ that contain recursive calls
  \BigIf
    \algcondition{1}{$OrigProc$ is directly but not mutually recursive
        (HO calls are assumed not to create recursion), and}
    \algcondition{2}{$OrigGoal$ has at most one recursive call
      on all possible execution paths, and}
    \algcondition{3}{$OrigGoal$ has determinism \ddet, and}
    \algcondition{4}{no recursive call is within a disjunction,
      a scope that changes the determinism of a goal,
      a negation, or the condition of a if-then-else, and}
    \algcondition{5}{no member of $RecParConjs$ is within
      another parallel conjunction, and}
    \algcondition{6}{every recursive call is inside
      the last conjunct of a member of $RecParConjs$, and}
    \algcondition{7}{every execution path through
      one of these last conjuncts
      makes exactly one recursive call}
  \BigIfThen
    \State $LC \gets$ create\_new\_variable()
    \State $LCGoal \gets$ the call
      `lc\_create\_loop\_control($LC$)'
    \State $LoopProcName \gets$ a new unique predicate name
    \State $OrigArgs \gets$ arg\_list($OrigProc$)
    \State $LoopArgs \gets$ [$LC$] ++ $OrigArgs$
    \State $CallLoopGoal \gets$ the call
      `LoopProcName($LoopArgs$)'
    \State $NewProcBody \gets$ the conjunction `$LCGoal,~CallLoopGoal$'
    \State $NewProc \gets OrigProc$ with its body replaced
      by $NewProcBody$
    \State $LoopGoal \gets$ \parbox{0.7\textwidth}{create\_loop\_goal($OrigGoal$,
      $OrigProcName$, $LoopProcName$,\\
        $~~~~RecParConjs$, $LC$)}
    \State $LoopProc \gets$ new\_procedure(name: $LoopProcName$, args:
    $LoopArgs$, body: $LoopGoal$)
    \State $NewProcs \gets$ [$NewProc,~LoopProc$]
  \BigIfElse
    \State $NewProcs \gets$ [$OrigProc$]
  \EndBigIfElse
  \State \Return $NewProcs$
\EndProcedure
\end{algorithmic}
%\vspace{2mm}
\caption{The top level of the transformation algorithm}
\label{alg:transform_alg}
%\vspace{-1\baselineskip}
\end{algorithm}

\begin{algorithm}[tbp]
\begin{algorithmic}
\Procedure{create\_loop\_goal}{$OrigGoal$, $OrigProcName$, $LoopProcName$,
    $RecParConjs$, $LC$}
  \State $LoopGoal \gets OrigGoal$
  \For{$RecParConj \in RecParConjs$}
    \State $RecParConj$ has the form $`Conjunct_1~\&~\ldots~\&~Conjunct_n$'
        for some $n$
    \For{$i \gets 1 to n-1$}
      \Comment This does not visit the last goal in $RecParConj$
      \State $LCSlot_i \gets$ create\_new\_variable()
      \State $WaitGoal_i \gets$ the call
        `lc\_wait\_free\_slot($LC$, $LCSlot_i$)'
      \State $JoinGoal_i \gets$ the call
        `lc\_join\_and\_terminate($LC$, $LCSlot_i$)'
      \State $SpawnGoal_i \gets$ \parbox{0.7\textwidth}{a goal
        that spawns off the sequential conjunction \\
        `$Conjunct_i, JoinGoal_i$' as a work package}
      \State $Conjunct_i \gets$ the sequential conjunction
        `$WaitGoal_i, SpawnGoal_i$'
    \EndFor
    \State $Conjunct_n' \gets Conjunct_n$
    \For{each recursive call $RecCall$ in $Conjunct_n'$}:
      \State $RecCall$ has the form `$OrigProcName(Args)$'
      \State $RecCall' \gets$ the call
        `$LoopProcName$([$LC$] ++ $Args$)'
      \State \textbf{replace} $RecCall$ with $RecCall'$ in $Conjunct_n'$
    \EndFor
    \State $Replacement \gets$ \parbox{0.7\textwidth}{the flattened form
      of the sequential conjunction \\
      `$Conjunct_1',~\ldots,~Conjunct_n'$'}
    \State \textbf{replace} $RecParConj$ in $LoopGoal$ with $Replacement$
  \EndFor
  \State $LoopGoal \gets$ put\_barriers\_in\_base\_cases($LoopGoal$,
    $RecParConjs$, $LoopProcName$, $LC$)
  \State \Return $LoopGoal$
\EndProcedure
\end{algorithmic}
%\vspace{2mm}
\caption{Algorithm for transforming the recursive cases}
\label{alg:reccases_alg}
%\vspace{-1\baselineskip}
\end{algorithm}

Algorithm~\ref{alg:transform_alg} shows the top level of the algorithm,
which is mainly concerned with testing
whether the loop control transformation is applicable to a given procedure,
and creating the interface procedure if it is.

We impose conditions (1) and (2) because we need to ensure
that every loop we start for \var{OrigProc} is finished exactly once,
by the call to \lcfinish we insert into its base cases.
If \var{OrigProc} is mutually recursive with some other procedure,
then the recursion may terminate in a base case of the other procedure,
which our algorithm does not transform.
Additionally, if \var{OrigProc} has some execution path on which it calls
itself twice,
then the second call may continue executing loop iterations
after a base case reached through the first call has finished the loop.

We impose conditions (3) and (4) because the Mercury implementation
does not support the parallel execution of code that is not deterministic.
We do not want a recursive call to be called twice because
some code between the entry point of \var{OrigProc} and the recursive call
succeeded twice,
and we do not want a recursive call to be backtracked into because
some code between the recursive call and the exit point of \var{OrigProc}
has failed.
These conditions prevent both of those situations.

We impose condition (5) because we do not want another instance of loop control,
or an instance of the normal parallel conjunction execution mechanism,
to interfere with this instance of loop control.

We impose condition (6) for two reasons.
First, the structure of our transformation requires right recursive code:
we could not terminate the loop in base case code
if the call that lead to that code
was followed by any part of an earlier loop iteration.
Second, allowing recursion to sometimes occur
outside the parallel conjunctions we are trying to optimise
would unnecessarily complicate the algorithm.
(We do believe that it should possible to extend our algorithm
to handle recursive calls made outside of parallel conjunctions.)

% We impose condition (7) to simplify the construction of a correctness argument
% in favor of the proposition that the two algorithms
% that transform the recursive calls and the bases cases respectively
% (which are shown in Figures \ref{fig:reccases_alg}
% and \ref{fig:basecases_alg})
% do not interfere in each other's operation.

We impose condition (7) to ensure that
our algorithm for transforming base cases
(Algorithm~\ref{alg:basecases_alg})
does not have to process goals that have already been processed
by our algorithm for transforming recursive calls
(Algorithm~\ref{alg:reccases_alg}).

If the transformation is applicable, we apply it.
The transformed original procedure has only one purpose:
to initialise the loop control structure.
Once that is done, it passes a reference to that structure to \var{LoopProc},
the procedure that does the actual work.

The argument list of \var{LoopProc}
is the argument list of \var{OrigProc}
plus the \LC variable that holds the reference to the loop control structure.
The code of \var{LoopProc} is derived from the code of \var{OrigProc}.
Some execution paths in this code include a recursive call; some do not.
The execution paths that contain a recursive call
are transformed by Algorithm~\ref{alg:reccases_alg};
the execution paths that do not
are transformed by Algorithm~\ref{alg:basecases_alg}.

We start with Algorithm~\ref{alg:reccases_alg}.
Due to condition (6),
every recursive call in \var{OrigGoal}
will be inside the last conjunct a parallel conjunction,
and the main task of \createloopgoal
is to iterate over and transform these parallel conjunctions.
(It is possible that some parallel conjunctions do not contain recursive calls;
\createloopgoal will leave these untouched.)

% The twin aims of our transformation are to
% (a) spawn off each parallel conjunct before the final recursive conjunct,
% in order to generate work for other cores to do, and
% (b) limit the number of work packages spawned off by the loop at any one time,
% in order to limit memory consumption.

The main aim of the loop control transformation is
to limit the number of work packages spawned off by the loop at any one time,
in order to limit memory consumption.
The goals we want to spawn off
%as work packages that other cores can pick up and execute
are all the conjuncts before the final recursive conjunct.
(Without loop control, we would spawn off all the \emph{later} conjuncts.)
The first half of the main loop in \createloopgoal
therefore generates code that creates and makes available each work package
only after it obtains a slot for it in the loop control structure,
waiting for a slot to become available if necessary.
We make the spawned-off computation free that slot when it finishes.

To implement the spawning off process,
we extended the internal representation of Mercury goals
with a new kind of scope.
The only one shown in the abstract syntax in
Figure~\ref{fig:abstractsyntax}
%Section~\ref{sec:backgnd_mercury}
was the existential quantification scope
($some~[X_1,\ldots,X_n]~G$),
but the Mercury implementation had several other kinds of scopes already,
though none of those are relevant for the dissertation.
We call the new kind of scope the spawn-off scope,
and we make $SpawnGoal_i$ be a scope goal of this kind.
When the code generator processes such scopes,
it
\begin{itemize}
\item
generates code for the goal inside the scope
(which will end with a call to \lcjoinandterminate),
\item
allocates a new label,
\item
puts the new label in front of that code,
\item
puts this labelled code aside so that
later it can be added to the end of the current procedure's code, and
\item
inserts into the instruction stream a call to \lcspawnoff
that specifies that the spawned-off computation should start execution
at the label of the set-aside code.
The other arguments of \lcspawnoff come from the scope kind.
\end{itemize}

\noindent
Since we allocate a loop slot \var{LCSlot}
just before we spawn off this computation,
waiting for a slot to become available if needed,
and free the slot once this computation has finished executing,
the number of computations that have been spawned-off by this loop
and which have not yet been terminated
cannot exceed the number of slots in the loop control structure.

\begin{algorithm}[tbp]
\begin{algorithmic}
\Procedure{put\_barriers\_in\_base\_cases}{$LoopGoal$,
    $RecParConjs$, $LoopProcName$, $LC$}
  \If{$LoopGoal$ is a parallel conjunction in $RecParConjs$}
    \Comment{case 1}
    \State $LoopGoal' \gets LoopGoal$
  \ElsIf{there no call to $LoopProcName$ in $LoopGoal$}
    \Comment{case 2}
    \State $FinishGoal \gets$ the call `lc\_finish($LC$)'
    \State $LoopGoal' \gets$ the sequential conjunction
      `$LoopGoal,~FinishGoal$'
  \Else
    \Comment{case 3}
    \Switch{goal\_type($LoopGoal$)}
      \Case{`ite($C$, $T$, $E$)'}
        \State $T' \gets$ put\_barriers\_in\_base\_cases($T$,
          $RecParConjs$, $LoopProcName$, $LC$)
        \State $E' \gets$ put\_barriers\_in\_base\_cases($E$,
          $RecParConjs$, $LoopProcName$, $LC$)
        \State $LoopGoal' \gets$ `ite($C$, $T'$, $E'$)'
      \EndCase
      \Case{`switch($V$, [$Case_1$, \ldots, $Case_N$])'}
        \For{$i \gets 1 to N$}
          \State $Case_i \gets$ `case($FunctionSymbol_i$, $Goal_i$)'
          \State $Goal_i' \gets$ \parbox{0.7\textwidth}{put\_barriers\_in\_base\_cases($Goal_i$,
            $RecParConjs$, $LoopProcName$,\\$~~~~LC$)}
          \State $Case_i' \gets$ `case($FunctionSymbol_i$, $Goal_i'$)'
        \EndFor
        \State $LoopGoal' \gets$ `switch($V$, [$Case_1'$, \ldots, $Case_N'$])'
      \EndCase
      \Case{`$Conj_1$, \ldots $Conj_N$'}
        \Comment Sequential conjunction
        \State $i \gets 1$
        \While{$Conj_i$ does not contain a call to $LoopProcName$}
          \State $i \gets i + 1$
        \EndWhile
        \State $Conj_i' \gets$ put\_barriers\_in\_base\_cases($Conj_i$,
          $RecParConjs$, $LoopProcName$, $LC$)
        \State $LoopGoal' \gets LoopGoal$ with
          $Conj_i$ replaced with $Conj_i'$
      \EndCase
      \Case{`some($Vars$, $SubGoal$)'}
        \Comment Existential quantification
        \State $SubGoal' \gets$ \parbox{0.7\textwidth}{put\_barriers\_in\_base\_cases($SubGoal$,
          $RecParConjs$,\\$~~~~LoopProcName$, $LC$)}
        \State $LoopGoal' \gets$ `some($Vars$, $SubGoal'$)'
      \EndCase
      \Case{a call `$ProcName$($Args$)'}
        \If{$ProcName = OrigProcName$}
          \State $LoopGoal' \gets$ the call `$LoopProcName$([$LC$] ++ $Args$)'
        \Else
          \State $LoopGoal' \gets LoopGoal$
        \EndIf
      \EndCase
    \EndSwitch
  \EndIf
  \State \Return $LoopGoal'$
\EndProcedure
\end{algorithmic}
%\vspace{2mm}
\caption{Algorithm for transforming the base cases}
\label{alg:basecases_alg}
%\vspace{-1\baselineskip}
\end{algorithm}

The second half of the main loop in \createloopgoal
transforms the last conjunct in the parallel conjunction
by locating all the recursive calls inside it
and modifying them in two ways.
The first change is to make the call actually call the loop procedure,
not the original procedure, which after the transformation is non-recursive;
the second is to make the list of actual parameters match
the loop procedure's formal parameters
by adding the variable referring to the loop control structure
to the argument list.
Due to condition (6),
there can be no recursive call in \var{OrigGoal} that is left untransformed
when the main loop of \createloopgoal finishes.

In some cases, the last conjunct may simply \emph{be} a recursive call.
In some other cases, the last conjunct may be a sequential conjunction
consisting of some unifications and/or some non-recursive calls
as well as a recursive call,
with the unifications and non-recursive calls
usually constructing and computing some of the arguments of the recursive call.
And in yet other cases,
the last conjunct may be an if-then-else or a switch,
possibly with other if-then-elses and/or switches nested inside them.
In all these cases, due to condition (7),
the last parallel conjunct will execute exactly one recursive call
on all its possible execution paths.

The last task of \createloopgoal is to invoke the
\putbarriers function
that is shown in Algorithm~\ref{alg:basecases_alg}
to transform the base cases of the goal
that will later become the body of \var{LoopProc}.
This function recurses on the structure of \var{LoopGoal},
as updated by the main loop in Algorithm~\ref{alg:reccases_alg}.

When \putbarriers is called,
its caller knows that \var{LoopGoal} may contain
the already processed parallel conjunctions (those containing recursive calls),
it may contain base cases,
or it may contain both.
The main if-then-else in \putbarriers
handles each of these situations in turn.

If \var{LoopGoal} is a parallel conjunction that is in \var{RecParConjs},
then the main loop of \createloopgoal has already processed it,
and due to condition (7), this function does not need to touch it.
Our objective in imposing condition (7) was to make this possible.

If, on the other hand, \var{LoopGoal} contains no call to \var{LoopProc},
then it did not have any recursive calls in the first place,
since (due to condition (6))
they would all have been turned into calls to \var{LoopProc}
by the main loop of \createloopgoal.
Therefore this goal either \emph{is} a base case of \var{LoopProc},
or it is part of a base case.
In either case, we add a call to \code{lc\_finish($LC$)} after it.
In the middle of the correctness argument below,
we will discuss why this is the right thing to do.

If both those conditions fail,
then \var{LoopGoal}
definitely contains some execution paths that execute a recursive call,
and may also contain some execution paths that do not.
What we do in that case (case 3)
depends on what kind of goal \var{LoopGoal} is.

If \var{LoopGoal} is an if-then-else,
then we know from condition (4)
that any recursive calls in it
must be in the then part or the else part,
and by definition the last part of any base case code
in the if-then-else must be in one of those two places as well.
We therefore recursively process both the then part and the else part.
Likewise, if \var{LoopGoal} is a switch,
some arms of the switch may execute a recursive call and some may not,
and we therefore recursively process all the arms.
For both if-then-elses and switches,
if the possible execution paths inside them do not involve conjunctions,
then the recursive invocations of \putbarriers
will add a call to \lcfinish at
the end of each execution path that does not make recursive calls.

What if those execution paths do involve conjunctions?
If \var{LoopGoal} is a conjunction,
then we recursively transform the first conjunct that makes recursive calls,
and leave the conjuncts both before and after it (if any) untouched.
There is guaranteed to be at least one conjunct that makes a recursive call,
because if there were not, the second condition would have succeeded,
and we would never get to the switch on the goal type.
We also know at most one conjunct makes a recursive call.
If more than one did, then
there would be an execution path through those conjuncts
that would make more than one recursive call,
then condition (2) would have failed,
and the loop control transformation would not be applicable.

\emph{Correctness argument.}
One can view the procedure body, or indeed any goal,
as a set of execution paths that
diverge from each other
in if-then-elses and switches
(on entry to the then or else parts and the switch arms respectively)
and then converge again
(when execution continues after the if-then-else or switch).
Our algorithm inserts calls to \lcfinish into the procedure body
at all the places needed to ensure
that every non-recursive execution path executes such a call exactly once,
and does so after the last goal in the non-recursive execution path
that is not shared with a recursive execution path.
These places are
the ends of non-recursive then parts
whose corresponding else parts are recursive,
the ends of non-recursive else parts
whose corresponding then parts are recursive,
and the ends of non-recursive switch arms
where at least one other switch arm is recursive.
Condition (4) tests for recursive calls in the conditions of if-then-elses
(which are rare in any case)
specifically to make this correctness argument possible.

\zoltan{if we have room, add an example that contains
a branched goal with both recursive and non-recursive branches,
followed by a common suffix of code.}
\paul{I think this is easy enough to understand without such an example,
but perhaps because I have seem similar patterns all through my honours and
phd studies.
In particular, It did not even occur to me that I might need to
provide a correctness argument, it just seemed right from a structural
induction point-of-view.}
\paul{Addendum: This now means `if I have time'}

Note that for most kinds of goals, execution cannot reach case~3.
Unifications are not parallel conjunctions and cannot contain calls,
so if \var{LoopGoal} is a unification, we will execute case~2.
If \var{LoopGoal} is a first order call, we will also execute case~2,
because due to condition (6),
all recursive calls are inside parallel conjunctions;
since case 1 does not recurse,
we never get to those recursive calls.
\var{LoopGoal} might be a higher order call.
Although a higher order call might create mutual recursion;
any call to \var{OrigProc} will create a new loop control object and execute
its loop independently.
This may not be optimal, but it will not cause the transformation to create
invalid code or perform worse performance than Mercury's normal parallel
execution mechanics.
Therefore we treat higher order calls the same way that we treat plain calls
to procedures other than \var{OrigProc}.
If \var{LoopGoal} is a parallel conjunction,
then it is either in \var{RecParConj},
in which case we execute case~1,
or (due to condition (5))
it does not contain any recursive calls,
in which case we execute case~2.
Condition (4) also guarantees that we will execute case~2
if \var{LoopGoal} is a disjunction, negation,
or a quantification that changes the determinism of a goal
by cutting away (indistinguishable) solutions.
The only other goal type for which execution may get to case~3
are quantification scopes that have no effect on the subgoal they wrap,
whose handling is trivial.

% \paul{I was a little lost when you started explaining this,
% because I did not realize why it should be explained, but that becomes clear
% in the following two paragraphs.
% Is it possible to phrase this as "... because ..." rather than "... therefore
% ..."?}
We can view the execution of a procedure body that satisfies condition (2)
and therefore has at most one recursive call on every execution path
as a descent from a top level invocation from another procedure
to a base case, followed by ascent back to the top.
During the descent,
each invocation of the procedure executes
the part of a recursive execution path
up to the recursive call;
during the ascent,
after each return we execute
the part of the chosen recursive execution path after the recursive call.
At the bottom, we execute exactly one of the non-recursive execution paths.

In our case, conditions (5) and (6) guarantee
that all the goals we spawn off
will be spawned off during the descent phase.
When we get to the bottom and
commit to a non-recursive execution path through the procedure body,
we know that we will not spawn off any more goals,
which is why we can invoke \lcfinish at that point.
We can call \lcfinish at any point in \var{LoopGoal}
that is after the point
where we have committed to a non-recursive execution path,
and before the point where
that non-recursive execution path
joins back up with some recursive execution paths.

The code at case 2 puts the call to \lcfinish
at the last allowed point, not the first, or a point somewhere in the middle.
We chose to do this because after the code executing \var{LoopProc}
has spawned off one or more goals one level above the base case,
we expect that other processors will be busy executing those spawned off goals
for rather longer than it takes this processor to execute the base case.
% \paul{The next sentence does not explain to me how we learnt this through
% automatic parallelism, do you mean that it is the kind of decision that the
% automatic parallelism system would make?}
% We expect this because our automatic parallelisation system
% tries very hard not to parallelise (and thus to spawn off) cheap goals,
% while most base cases \emph{are} cheap.
By making this core do as much useful work as possible
before it must suspend to wait for the spawned-off goals to finish,
we expect to reduce the amount of work remaining to be done
after the call to \lcfinish by a small but possibly useful amount.
\lcfinish returns \emph{after} all the spawned-off goals have finished,
so any code placed after it
(such as if \lcfinish were placed at the first valid point)
would be executed sequentially after the loop;
where it would definitely add to the overall runtime.
Therefore, we prefer to place \lcfinish as late as possible,
so that this code occurs before \lcfinish
and is executed in parallel with the rest of the loop,
where it may have no effect on the overall runtime of the program;
it will just put to good use, what would otherwise be dead time.

We must of course be sure that every loop,
and therefore every execution of any base case of \var{LoopGoal},
will call \lcfinish exactly once: no more, no less.
(It should be clear that our transformation never puts that call
on an execution path that includes a recursive call.)
Now any non-recursive execution path through \var{LoopGoal}
will share a (possibly empty) initial part
and a (possibly empty) final part with some recursive execution paths.
On any non-recursive execution path,
\putbarriers will put the call \lcfinish
just before the first point where that path
rejoins a recursive execution path.
Since \var{LoopProc} is \ddet (condition (3)),
all recursive execution paths must consist
entirely of \ddet goals and the conditions of if-then-elses,
and (due to condition (4)) cannot go through disjunctions.
The difference between a non-recursive execution path
and the recursive path it rejoins
must be either that
one takes the then part of an if-then-else and the other takes the else part,
or that they take different arms of a switch.
Such an if-then-else or switch must be \ddet:
if it were \dsemidet, \var{LoopProc} would be too,
and if it were \dnondet or \dmulti,
then its extra solutions could be thrown away
only by an existential quantification that quantifies away
all the output variables of the goal inside it.
However by condition (4),
the part of the recursive execution path
that distinguishes it from a non-recursive path,
the recursive call itself,
cannot appear inside such scopes.
This guarantees that the middle part of the non-recursive execution path,
which is not part of either
a prefix or a suffix shared with some recursive paths,
must also be \ddet overall,
though it may have nondeterminism inside it.
Any code put after the second of these three parts of the execution path
(shared prefix, middle, shared suffix),
all three of which are \ddet,
is guaranteed to be executed exactly once.

% ite with nondet condition,
% recursive call in else part,
% no recursive call in then part:
% we may execute lc\_finish several times,
% once for each success of the condition.
% No: cannot happen: argument is above

\subsection{Loop control and tail recursion}
\label{sec:lc_tailrec}

When a parallel conjunction spawns off a conjunct
as a work package that other cores can pick up,
the code that executes that conjunct has to know
where it should pick up its inputs,
where it should put its outputs,
and where it should store its local variables.
All the inputs come from the stack frame of the procedure
that executes the parallel conjunction,
and all the outputs go there as well,
so the simplest solution, and the one used by the Mercury system,
is for the spawned-off conjunct to do all its work
in the exact same stack frame.
Normally, Mercury code accesses stack slots
via offsets from the standard stack pointer.
Spawned-off code accesses stack slots using
a special Mercury abstract machine register
called the parent stack pointer,
which the code that spawns off goals
sets up to point to the stack frame of the procedure doing the spawning.
That same spawning-off code sets up the normal stack pointer
to point to the start of the stack in the context executing the work package,
so any calls made by the spawned-off goal
will allocate their stack frames in that stack,
but the spawned-off conjunct will use
the original frame in the stack of the parent context.

This approach works, and is simple to implement:
the code generator generates code for spawned-off conjuncts normally,
and then just substitutes the base pointer in all references to stack slots.
However, it does have an obvious drawback:
until the spawned-off computation finishes execution,
it may make references to the stack frame of the parallel conjunction,
whose space therefore cannot be reused until then.
This means that even if
a recursive call in the last conjunct of the parallel conjunction
happens to be a tail call,
it cannot have the usual tail call optimisation applied to it.

Before this work, this did not matter, because
the barrier synchronisation needed at the end of the parallel conjunction
(\joinandcontinue),
which had to be executed at every level of recursion except the base case,
prevented tail recursion optimisation anyway.
However, the loop control transformation eliminates that barrier,
replacing it with the single call to \lcfinish in the base case.
So now this limitation \emph{does} matter in cases
where all of the recursive calls in the last conjunct of a parallel conjunction
are tail recursive.
% \zoltan{Does the implementation get the quantification right?}
% \paul{No,
% I thought that:
% if some are tail calls and some are not then the spawn\_off scope is
% marked with 'create\_new\_frame\_on\_worker\_stack'.
% Which is the safe option.
% This only extends to spawn off scopes in the same parallel conjunction.
% (since other spawn offs and their recursive calls will be on different code
% paths.
% This made sense to me because:
% If there's a chance that the parent stack frame could be clobbered due to (a
% single) tail recursion, then the implementation will allocate a frame on the
% worker's stack.
% However:
% what if there is a tail which means we \emph{must not} use the
% master's stack,
% AND there's a non tail call with a read of a variable that the spawned off
% code was supposed to write which means we \emph{must} use the master's stack.
% To solve this we must forbid the tail call and use the master's stack.
% The implementation is correct WRT the benchmarks, but this will need to be
% fixed.
% Note that this is only the wrong eay around if a scope has an output
% variable.}

If at least one call is not tail recursive,
then it prevents the reuse of the original stack frame,
so our system will still follow the scheme described above.
However, if they all are,
then our system can now be asked to follow a different approach.
The code that spawns off a conjunct
will allocate a frame at the start of the stack in the child context,
and will copy the input variables of the spawned-off conjunct into it.
The local variables of the spawned-off goal
will also be stored in this stack frame.
The question of where its output variables are stored is moot:
there cannot \emph{be} any output variables
whose stack slots would need to be assigned to.

The reason this is true has to do with the way
the Mercury compiler handles synchronisation between parallel conjuncts.
Any variable whose value is generated by one parallel conjunct
and consumed by one or more other conjuncts in that conjunction
will have a future created for it (Section~\ref{sec:backgnd_deppar}).
The generating conjunct, once it has computed the value of the variable,
will execute \signal on the variable's future
to wake up any consumers that may be waiting for the value of this variable.
Those consumers will get the value of the original variable from the future,
and will store that value in a variable that is local to each consumer.
Since futures are always stored on the heap,
the communication of bindings from one parallel conjunct to another
does \emph{not} go through the stack frame.

A variable whose value is generated by a parallel conjunct
and is consumed by code after the parallel conjunction
does need to have its value put into its stack slot,
so that the code after the parallel conjunction can find it.
However, if all the recursive calls in the last conjunct
are in fact tail calls, then by definition
there can be no code after the parallel conjunction.
Since neither code later \emph{in} the parallel conjunction,
nor code \emph{after} the parallel conjunction,
requires the values of variables generated by a conjunct
to be stored in the original stack frame,
storing it in the spawned-off goal's child stack frame is good enough.

In our current system,
the stack frame used by the spawned-off goal
has exactly the same layout as its parent, the spawning-off goal.
This means that in general,
both the parent and child stack frames will have some unused slots,
slots used only in the \emph{other} stack frame.
This is trivial to implement,
and we have not found the wasted space to be a problem.
This may be because we have mostly been working with
automatically parallelised programs,
and our automatic parallelisation tools
put much effort into granularity control (Chapter~\ref{chap:overlap}):
the rarer spawning-off a goal is,
the less the effect of any wasted space.
However, if we ever find this to be an issue,
squeezing the unused stack slots out of each stack frame
would not be difficult.

% \zoltan{This scheme does require us
% to reserve a context for a spawned-off goal when we create each work package,
% not when the work package starts executing.
% Discuss that tradeoff, and the possibility of staging the input vars
% in the work package itself, AFTER we have relevant performance numbers.}

% \section{Runtime support and optimisations}
% \label{sec:runtime}

\section{Performance evaluation}
\label{sec:lc_perf}

\status{This section is ready for review by someone.}

% We ran all our benchmarks on
% taura
% a Dell Optiplex 755 desktop PC with a 2.4~GHz Intel Core 2 Quad Q6600 CPU
% (four cores, no hyperthreading)
% running Linux 2.6.31.
% no hyperthreading
% apollo, and carlton
% a Dell Optiplex 980 desktop PC with a 2.8~GHz Intel i7 860 CPU
% (four cores, each with two hyperthreads)
% running Linux 2.6.35 in 64-bit mode.
% \zoltan{
% % goliath
% a SunFire X2250 server with two 3.0~GHz Intel Xeon X5472 CPUs
% (eight cores total, no hyperthreading)
% running Linux 2.6.26.
% }
% cabsav
% an AsRock Z68-Pro3 based PC with a 3.40GHz Intel i7-2600K CPU
% with frequency scaling (Speedstep and TurboBoost) disabled
% (four cores, each with two hyperthreads)
% running Linux 2.6.32.

We have benchmarked our system with four different programs,
three of which were used in earlier chapters.
All these programs use explicit parallelism.
\begin{description}
\item[mandelbrot] uses dependent parallelism using \mapfoldl from
Figure~\ref{fig:mapfoldl}.

\item[matrixmult]
multiplies two large matrices.
It computes the rows of the result in parallel.
%We have included both dependent and independent versions of matrixmult;
%neither is tail recursive without

\item[raytracer] uses dependent parallelism and is tail recursive.
Like mandelbrot, it renders the rows of the generated image in parallel,
but it does not use \mapfoldl{}.

\item[spectralnorm]
was donated by Chris King\footnote{Chris' version was published at
\url{http://adventuresinmercury.blogspot.com/search/label/parallelization}.}.
It computes the eigenvalue of a large matrix using the power method.
It has two parallel loops, both of which are executed multiple times.
Therefore, the parallelism available in spectralnorm is very fine-grained.
Chris' original version uses dependent parallelism.
%we have created a new independently parallel version from his original code.
\end{description}

\noindent
All these benchmarks have dependent AND-parallelism,
but for two of the benchmarks, matrixmult and spectralnorm,
we have created versions that use independent AND-parallelism as well.
The difference between the dependent and independent versions
is just the location of a unification that constructs a cell
from the results of two parallel conjuncts:
the unification is outside the parallel conjunction in the independent versions,
while it is in the last parallel conjunct in the dependent versions.
The sequential versions of mandelbrot and raytracer can both use tail call
optimisation to run in constant stack space.
This is not possible in the other programs without
the last call modulo constructor (LCMC)
optimisation \citep{ross:mercury-lcmc}.
LCMC is not supported with tail recursion so we have not used it.

We found that the independent spectralnorm benchmark performed poorly due
to its fine granularity and the overhead of notification.
Therefore, we compiled the runtime system in such a way so that
notifications were not used to communicate the availability of sparks
(Section~\ref{sec:rts_work_stealing2}).
Instead Mercury engines were configured to poll each other to find sparks
to execute.
This change only affects the non-loop control tests.
We ran each test of each program twenty times.

% Mandelbrot, raytracer and matrixmult
% all compute the rows of their result in parallel,
% mandelbrot uses the \mapfoldl abstraction, see Figure \ref{fig:map_foldl}
% Spectralnorm contains two parallel loops which are executed multiple times.
% Mandelbrot and raytracer both use dependant parallelism.
% Dependent and independent versions of matrixmult and spectralnorm are
% used.
% Only the dependent parallel tests use loop control.
% \paul{We might want to show independent code using loop control}

\begin{sidewaystable}[tbp]
\begin{center}
\input{mem_table}
\caption[Peak number of contexts used,
and peak memory usage for stacks]{Peak number of contexts used,
and peak memory usage for stacks, measured in megabytes}
\label{tab:lc_mem}
\end{center}
%\vspace{-1\baselineskip}
\end{sidewaystable}

\begin{sidewaystable}[tbp]
\begin{center}
\input{times_table}
%\vspace{3mm}
\caption{Execution times measured in seconds, and speedups}
\label{tab:lc_times}
\end{center}
%\vspace{-1\baselineskip}
\end{sidewaystable}

Tables~\ref{tab:lc_mem} and~\ref{tab:lc_times}
presents our memory consumption and timing results respectively.
In both tables,
the columns list the benchmark programs,
while the rows show the different ways
the programs can be compiled and executed.
% Due to space limits,
% each table shows only a subset of the rows,
% those with the most interesting results.
% These subsets are different for the two tables.

In Table~\ref{tab:lc_mem}, each box has two numbers.
The first reports the maximum number of contexts alive at the same time,
while the second reports the maximum number of megabytes
ever used to store the stacks of these contexts.
In Table~\ref{tab:lc_times}, each box has three numbers.
The first is the execution time of that benchmark in seconds
when it is compiled and executed in the manner prescribed by the row.
The second and third numbers (the ones in parentheses)
show respectively the speedup this time represents
over the sequential version of the benchmark (the first row),
and over the base parallel version (the second row).
Some of the numbers are affected by rounding.

In both tables, the first row
compiles the program without using any parallelism at all,
asking the compiler to automatically convert
all parallel conjunctions into sequential conjunctions.
Obviously, the resulting program will execute on one core.

The second row
compiles the program in a way that prepares it for parallel execution,
but it still asks the compiler to automatically convert
all parallel conjunctions into sequential conjunctions.
The resulting executables will differ from the versions in the first row
in two main ways.
First, they will incur some overheads
that the versions in the first row do not,
overheads that are needed to support the possibility of parallel execution.
The most important of these overheads is that,
as described in Chapters~\ref{chap:rts} and~\ref{chap:overlap},
the runtime reserves a hardware register to point to thread specific data and
% potentially-parallel code needs a way to access thread-specific data,
% and therefore when a program is compiled for parallel execution,
% the Mercury compiler has to reserve one machine register
% to hold a pointer to this data,
% making that machine register unavailable
% to the rest of the Mercury abstract machine.
% Given the dearth of callee-save machine registers on the x86\_64
% (we are not set up to use caller-save registers),
% this can lead to very significant slowdowns:
% for our benchmarks, as much as 30\%.
% The second difference is that,
compiling
the garbage collector and the rest of the runtime system for thread safety.
% must be thread safe,
These overheads lead to slowdowns in most cases,
even when using a single core for user code.
(The garbage collector uses one core in all of our tests.)
However, the mandelbrot program speeds up when thread safety is enabled;
it does not do very much memory allocation
and is therefore affected less by
the overheads of thread safety in the garbage collector.
Its slight speedup may be due to
its different code and data layouts
interacting with the cache system differently.

All the later rows
compile the program for parallel execution,
and leave the parallel conjunctions in the program intact.
They execute the program on 1 to 4 cores (`1c' to `4c').
The versions that execute on the same number of cores
differ from each other mainly in how they handle loops.
The rows marked `nolc' are the controls.
They do not use the loop control mechanism described in this chapter;
instead, they rely the context limit.
The actual limit is the number of engines
multiplied by a specified parameter,
which we have set to 128 in `c128' rows and to 512 in `c512' rows.
%A minor bug (where \code{<=} was used instead of \code{<}) allows this limit
%to be exceeded by a single context in some cases.
Although our code uses thread-safe mechanisms for counting the number of
contexts in use;
when it checks this value against the limit it does not use a thread-safe
mechanism.
Therefore, whilst the number of contexts in use can never be corrupted,
it can be exceed the limit by about one or two contexts.
Since different contexts can have different sized stacks,
the limit is only an approximate control over memory consumption anyway,
so this is an acceptable price to pay for reduced synchronisation overhead.
%\zoltan{Paul, have you checked this?}
%\paul{The cause is a case where I should have used < instead of <=,
%However the limit is also not-thread-safe, which can also cause
%similar problems.
%So leading the reader to believe that this is the cause is okay by me.}

The rows marked `lc$N$' do use our loop control mechanism,
with the value of $N$ indicating
the value of another parameter we specify when the program is run.
When the \code{lc\_create\_loop\_control} instruction
creates a loop control structure,
it computes the number of slots to create in it,
by multiplying the configured number of Mercury engines
(each of which can execute on its own core)
with this parameter.
We show memory consumption and timing results for $N=1$, 2 and 4.
The timing results for $N=1$ and $N=4$ are almost identical to those for
$N=2$.
It seems that as long as we put a reasonably small limit
on the number of contexts a loop control structure can use,
speed is not much affected by the precise value of the limit.

The rows marked `lc$N$, tr' are like the corresponding `lc$N$' rows,
but they also switch on tail recursion preservation
in the two benchmarks (mandelbrot and raytracer)
whose parallel loops are naturally tail recursive.
The implementation of parallelism without loop control
destroys this tail recursion,
and so does loop control unless we ask it to preserve tail recursion.
That means that mandelbrot and raytracer use tail recursion
in all the test setups except for
the parallel, non-loop control ones,
and loop control ones without tail recursion.
Since the other benchmarks are not naturally tail recursive,
they will not be tail recursive however they are compiled.
There are no such rows in Table~\ref{tab:lc_mem}
since the results in each `lc$N$, tr' row
would be identical to the corresponding `lc$N$' row.

There are several things to note in Table~\ref{tab:lc_mem}.
The most important is that when the programs are run on more than one core,
switching on loop control yields a dramatic reduction
in the maximum number of contexts used at any one time,
and therefore also in the maximum amount of memory used by stacks.
% \footnote{
(The total amount of memory used by these benchmarks
is approximately the maximum of this number
and the configured initial size of the heap.)
This shows that we have achieved our main objective.
Without loop control, the execution of
three of our four dependent benchmarks (mandelbrot, matrixmult and raytracer)
require the simultaneous existence of a context
for every parallel task that the program can spawn off.
For example, mandelbrot generates an image with 600 rows,
so the original context can never spawn off more than 600 other contexts.

On one core, the `nolc' versions spawn off sparks,
but since there is no other engine to pick them up,
the one engine eventually picks them up itself,
and executes them in the original context.
By contrast, the `lc$N$' versions directly spawn off new contexts, not sparks.
This avoids the overhead of converting a spark to a context,
but we can do this only because we know we will not create too many contexts.

When executing on two or more cores,
mandelbrot and raytracer use one more context that one would expect.
Before the compiler applies the loop control transformation,
it adds the synchronisation operations
needed by dependent parallel conjunctions.
As shown by Figure~\ref{fig:map_foldl_sync},
this duplicates the original procedure.
Only the inner procedure is recursive,
so the compiler performs the loop control transformation only on it.
The extra context is
the conjunct spawned off by the parallel conjunction in the outer procedure.

There are several things to note in Table~\ref{tab:lc_times} as well.
The first is that in the absence of loop control,
increasing the per-engine context limit from 128 to 512
yields significant speedups for three out of four the dependent benchmarks.
Nevertheless, the versions with loop control
significantly outperform the versions without, even `c512',
for all these benchmarks except for mandelbrot.
On mandelbrot, `c512' already gets a near-perfect speedup,
yet loop control still gets a small improvement.
Thus on all our dependent benchmarks,
switching on loop control yields a speedup
while greatly reducing memory consumption.

Overall, the versions with loop control
get excellent speedups on three of the benchmarks:
speedups of 3.95, 3.92 and 3.90 on four CPUs
for mandelbrot, matrixmult and spectralnorm respectively.
The one apparent exception, raytracer,
is very memory-allocation-intensive.
In Section~\ref{sec:rts_gc} we showed that this can significantly reduce the
benefit of parallel execution of Mercury code,
especially when the collector does not use parallelism itself.
In we have observed nearly 45\% of a the raytracer's
runtime being used in garbage collection (Table~\ref{tab:gc_amdahl}):
This means that parallel execution can only speed up the remaining 55\% of
the program.
Therefore the best speedup we can hope for is
$(4 \times 0.55 + 0.45)/(0.55 + 0.45) \approx 2.65$,
which our result of 2.65 reaches.
It is unusual to see results this close to their theoretical maximums;
in this case the parallelisation of the runtime may affect the garbage
collector's performance,
which can affect the figures that we have used with Amdahl's law.

Second,
% for two of the four benchmarks,
loop control is crucial for getting this kind of speedup,
unless you are willing to waste lots of memory.
On four cores, loop control raises the speedup compared to c128
from 3.36 to 3.95 for mandelbrot,
from 1.44 to 3.92 for matrixmult,
from 0.87 to 2.65 for raytracer,
and from 1.01 to 3.90 for spectralnorm.
Those are pretty impressive improvements.

Third, for the benchmarks that have versions using independent parallelism,
the independent versions are faster than
the dependent versions without loop control,
while there is no significant difference between
the speeds of the independent versions and the dependent loop control versions.
For matrix multiplication, the loop control dependent version is faster,
but the difference is small.
While for spectral-norm, the independent version result falls within the
    range of loop control results.
This shows that on these benchmarks, loop control completely avoids
the problems described at the beginning of this chapter.

Fourth, preserving tail recursion has a mixed effect on speed:
of the six relevant cases (mandelbrot and raytracer on 2, 3 and 4 cores),
one case gets a slight speedup, while the others get slight slowdowns.
Due to the extra copying required,
this tilt towards slowdowns is to be expected.
However, the effect is very small:
usually within 2\%.
%and usually in the noise.
%(For example, spectral-indep on three cores
%does everything exactly the same with c512 as with c128,
%so the difference between 6.44s and 6.41s is just noise.)
The possibility of such slight slowdowns is an acceptable price to pay
for allowing parallel code to recurse arbitrarily deeply
while using constant stack space.

\section{Further work}
\label{sec:lc_further_work}

\status{This section is ready for someone to check.}

There are a number of small improvements that could be made to improve loop
control.
As we mentioned in Section~\ref{sec:lc_tailrec} the allocation of stack
slots in tail recursive code is rather naive.
We could perform better stack slot allocation for both the parent stack
frame and the first stack frame on the stack slot's context.
We could also change how stack slots are allocated in non-recursive code.
By allocating the slots that are shared between the parent and child
computations into separate parts of the same stack frame,
we may be able to reduce \emph{false sharing}.
False sharing normally occurs when two parallel tasks contend for access to
a single cache-line sized area of memory \emph{without} actually using the
data in that memory to communicate;
this creates extra cache misses that can be avoided by allocating memory
differently.

Loop control allows the same small number of contexts to be used to execute
a parallel loop.
However, Futures are still allocated dynamically and are not re-used.
We could achieve further performance improvements by controlling the
allocation and use of futures.
This could reduce the number of allocations of futures from
$num\_futures\_per\_level \times num\_red\_levels$ to just a single
allocation,
depending on how futures where used in the loop.

These changes are minor optimisations, and might only increase
performance by a small fraction;
there are other areas of further work that could provide more benefit.
For example,
applying a similar transformation to other parallel code patterns,
such as right recursion or divide and conquer,
could improve performance of code using those patterns.
In some patterns this may require novel transformations such as we have
shown here,
in others it may simply be the application of loop control transformations
such as \citet{shen_98_granularity-control}


\section{Conclusion}
\label{sec:lc_conc}

\status{This section is ready for review by someone.}

Ever since the first parallel implementations
of declarative languages in the 1980s,
researchers have known that getting more parallelism out of a program
than the hardware could use can be a major problem,
because the excess parallelism brings no benefits, only overhead,
and these overheads could swamp
the speedups the system would otherwise have gotten.
Accordingly, they have devised systems to throttle parallelism,
keeping it at a reasonable level.

However, most throttling mechanisms we know of
have been general in nature,
such as granularity control systems
\citep{lopez96:granularity,
king:lower_bound_time_complexity,
shen_98_granularity-control}.
These have similar objectives,
but use totally different methods:
restricting the set of places in a program
\emph{where} they choose to exploit parallelism,
not changing \emph{how} they choose to exploit it.

We know of one system that tries to preserve tail recursion
even when the tail comes from a parallel conjunction.
The ACE system \citep{gupta01:optimization_for_parallel_nodet_code}
normally generates one parcall frame for each parallel conjunction,
but it will flatten two or more nested parcall frames into one
if runtime determinacy tests indicate it is safe to do so.
While these tests usually succeed for loops,
they can also succeed for other code,
and (unlike our system) the ACE compiler does not identify in advance
the places where the optimisation may apply.
The other main difference from our system is the motivation:
the main motivation of this mechanism in the ACE system is
neither throttling
nor the ability to handle unbounded input in constant stack space,
but reducing the overheads of backtracking.
This is totally irrelevant for us,
since our restrictions prevent any interaction
between AND-parallel code and code that can backtrack.

The only work on specially loop-oriented parallelism in logic languages
that we are aware of is Reform Prolog \citep{bevemyr:reform}.
This system was not designed for throttling either,
but it is more general than ours in one sense
(it can handle recursion in the middle of a clause)
and less general in other senses
(it cannot handle parallelism in any form other than loops,
and it cannot execute one parallel loop inside another).
It also has significantly higher overheads than our system:
it traverses the whole spine of the data structure being iterated over
(typically a list) \emph{before} starting parallel execution;
in some cases it synchronises computations by busy waiting;
and it requires variables stored on the heap to have a timestamp.
To avoid even higher overheads,
it imposes the same restriction we do:
it parallelises only deterministic code
(though the definition of ``deterministic'' it uses is a bit different).
Reform Prolog does not handle any other forms of parallelism,
whereas Mercury handles parallelism in various situations,
and the loop control transformation simply optimises one form of
parallelism.
% not throttled,
% no granularity control,

The only work on loop-oriented parallelism in functional languages
we know of is Sisal \citep{feo:1990:sisal-report}.
It shares two of Reform Prolog's limits:
no parallelism anywhere except loops, and
no nesting of parallel computations inside one another.
Since it was designed for number crunching on supercomputers,
it had to have lower overheads than Reform Prolog,
but it achieved those low overheads
primarily by limiting the use of parallelism
to loops whose iterations are \emph{independent} of each other,
which makes the problem much easier.
Similarly, while ACE Prolog supports both AND- and OR-parallelism,
the only form of AND-parallelism it supports is independent.

Our system is designed to throttle loops with dependent iterations,
and it seems to be quite effective.
By placing a hard bound on the number of contexts
that may be needed to handle a single loop,
our transformation allows parallel Mercury programs
to do their work in a reasonable amount of memory,
and since it does so without adding significant overhead,
permits them to live up to their full potential.
For one of our benchmarks,
loop control makes a huge difference:
on four cores, it turns a speedup of 1.01 into a speedup of 3.90.
It significantly improves speedups on two other benchmarks,
and it even helps the fourth and last benchmark,
even though that was already close to the maximum possible speedup.

The other main advantage of our system
is that it allows procedures to keep exploiting tail recursion optimisation
(TRO).
If TRO is applicable to the sequential version of a procedure,
then it will stay applicable to its parallel version.
Many programs cannot handle large inputs without TRO,
so they cannot be parallelised at all without this capability.
The previous advantage may be specific
to systems that resemble the Mercury implementation,
but this should apply to the implementation
of every eager declarative language.


%\section{Notes/To use}
%
%\status{This section will te removed.}
%
%Currently, the Mercury runtime system
%often continues execution, on completion of a parallel conjunction,
%on a CPU different from the one being used before that parallel conjunction.
%When our system finds a smattering of parallel conjunctions
%through a mostly sequential program,
%these switches from a CPU with a warm cache to a CPU with a cold cache
%severely degrade the program's performance.
%Right now, for most programs,
%this effect yields a slowdown significantly bigger
%than the speedups yielded by automatic parallelisation.
%Once this defect is fixed, we hope to report significantly better results
%for more and bigger programs.
%
%

%We would like to thank
%% for their support,
%Chris King for allowing us to use his spectralnorm benchmark.


% LocalWords:  parallelise parallelisation parallelised quantifications nondet
% LocalWords:  det parallelising syncterm mutex signalled nonecursive labelled
% LocalWords:  parallelises Prolog's


\paul{
    Add something about updating overlap's cost model, then we need to
    re-evaluate it.}

\chapter{Visualisation}
\label{chap:tscope}

\chapter{Visualisation}
\label{chap:tscope}

\status{This section needs a lot more work.  But it depends on unfinished
implementation}

%\begin{abstract}
%The behavior of parallel programs is even harder to understand
%than the behavior of sequential programs.
%Parallel programs may suffer from
%any of the performance problems afflicting sequential programs,
%as well as from several problems unique to parallel systems.
%Many of these problems are quite hard
%(or even practically impossible)
%to diagnose without help from specialized tools.
%We present a proposal for a tool
%for profiling the parallel execution of Mercury programs,
%a proposal whose implementation we have already started.
%This tool is an adaptation and extension of the \tscope profiler
%that was first built to help programmers visualize
%the execution of parallel Haskell programs.
%\end{abstract}

\section{Introduction}

When programmers need to improve the performance of their program,
they must first understand it.
The standard way to do this is to use a profiler.
Profilers record performance data from executions of the program,
and give this data to programmers
to help them understand how their program behaves.
This enhanced understanding then makes it easier
for programmers to speed up the program.

Profilers are needed because the actual behavior of a program
often differs from the behavior imagined by the programmer.
This is true even for sequential programs.
For parallel programs,
the gulf between human estimates and machine realities
is usually even wider.
This is because every cause
that can cause such differences for sequential programs
is present in parallel programs as well,
while parallel programs also have several causes unique to them.
% When writing software that uses parallelism,
% understanding how the program runs is more difficult.
% Parallelism may be used in different parts of the program,
% causing some parts to speed up while other parts remain sequential.

Consider a program containing a loop with many iterations
in which the iterations do not depend on one another
and can thus be done in parallel.
% is an independent parallel task.
Actually executing every iteration in parallel
may generate an overwhelming number of parallel tasks.
It may be that the average amount of computation done by one of these tasks
is less than the amount of work it takes to spawn one of them off.
In that case, executing all the iterations in parallel
will increase overheads to the point of actually slowing the program down,
perhaps quite substantially.
The usual way to correct this is to use granularity control
to make each spawned-off task execute
not one but several iterations of the loop,
thus making fewer but larger parallel tasks.
However, if this is done too aggressively,
performance may once again suffer.
For example,
there may be fewer tasks created than the computer has processors,
causing some processors to be idle when they could be used.
More commonly, the iterations of the loop and thus the tasks
may each need a different amount of CPU time.
If there are eight processors and eight unevenly sized tasks,
then some processors will finish their work early,
and be idle while waiting for the others.

All these problems can arise in programs with independent parallelism:
programs in which parallel tasks do not need to communicate.
Communication between tasks makes this situation even more complicated,
especially when a task may block waiting for information from another task.
A task that blocks may in turn further delay
other tasks that depend on data \emph{it} produces.
Chains of tasks that produce and consume values from one another are common,
Programmers need tools to help them
identify and understand performance problems,
whether they result from such dependencies or other causes.

Profiling tools that can help application programmers
can also be very useful for the implementors of runtime systems.
While optimizing Mercury's parallel runtime system,
we have needed to measure the costs of certain operations
and the frequency of certain behaviors.
Some examples are:
when a piece of work is made available,
how quickly can a sleeping worker-thread respond to this request?
When a task is made runnable after being blocked,
how often will it be executed on the same CPU that was previously executing it,
so that its cache has a chance to be warm?
Such information from profiles of typical parallel programs
can be used to improve the runtime system,
which can help improve the performance of \emph{all} parallel programs.

A third category of people who can use profiling tools for parallel programs
is researchers working on automatic parallelization tools.
Doing a good job of automatically parallelizing a program
requires a cost-benefit analysis for each parallelism opportunity,
which requires estimates of both the cost and the benefit of each opportunity.
Profiling tools can help researchers calibrate
the algorithms they use to generate estimates of both costs and benefits.
% For example, if it is more expensive to spawn off a computation
% than it is to execute it locally, we prefer to execute it locally.
% To answer this we need to know how expensive
% it is to spawn off the computation as a new task.
% We want to use profiling data from the parallel runtime to anwser this.

As researchers working on the parallel implementation of Mercury \citep{mercury_jlp},
a pure logic programming language designed for
the creation of large, reliable and efficient application programs,
we fall into all three of the above categories.
We have long needed a tool
to help us understand the behavior of parallel Mercury programs,
of the parallel Mercury runtime system
and of our autoparallelization tool \citep{bone:2011:overlap},
but the cost of building such a tool seemed daunting.
We therefore looked around for alternative approaches.
The one we selected was to adapt to Mercury
an existing profiler for another parallel system.

The profiler we chose was \tscope \citep{threadscope},
a profiler built by the team behind the Glasgow Haskell Compiler (GHC)
for their parallel implementation of Haskell.
We chose \tscope because
the runtime system it was designed to record information from
has substantial similarities to the Mercury runtime system,
because it was \emph{designed} to be extensible,
and, like many visualization tools, it is extremely useful
for finding subtle issues that cause significant problems.

% This paper presents the adaption of \tscope \citep{threadscope}
% --- a tool for profiling the parallel execution of Haskell programs
% --- to Mercury \citep{jlp} --- a pure logic programming language.
% We will use the above three types of users to determine the usefulness of our
% work.

The structure of the paper is as follows.
Section~\ref{sec:background} gives the background
needed for the rest of the paper:
it introduces Mercury,
the machinery involved in the parallel execution of Mercury programs,
and the \tscope profiler.
Section~\ref{sec:newevents}
describes how we extended the \tscope system
to collect the kinds of data needed to describe
the parallel execution of Mercury programs,
while section~\ref{sec:analysis} describes how
one can analyze that data to yield insights
that can be useful to our three audiences:
application programmers,
runtime system implementors,
and the implementors of autoparallelization tools.
Section~\ref{sec:conc}
concludes with a discussion of related work.

\section{Background}
\label{sec:background}

\subsection{\tscope}

\tscope was originally built
to help programmers visualize the parallel execution of Haskell programs
compiled with the dominant implementation of Haskell,
the Glasgow Haskell Compiler or GHC.
The idea is that during the execution of a parallel Haskell program,
the Haskell runtime system writes time-stamped reports
about significant events to a log file.
The \tscope tool later reads this log file,
and shows the user graphically what each CPU was doing over time.
The diagrams it displays reveal to the programmer
the places where the program is getting the expected amount of parallelism,
as well as the places where it is not.

We could adapt the \tscope system to Mercury for two main reasons.
The first is that the parallel implementations of Haskell and Mercury
in GHC and mmc (the Melbourne Mercury compiler)
are quite similar in several important respects,
even though they use different terms for
(slightly different implementations of) the same concepts.
For example, what Mercury calls an engine GHC calls a \emph{capability},
and what Mercury calls a context GHC calls a \emph{thread}.
(We will use Mercury terminology in the rest of the paper.)
The second reason is that
the \tscope log file format was designed to be extensible.
Each log file starts with a description
of each kind of event that may occur in it.
This description lists the arguments of the event and their total size,
which may be variable.
%\zoltan{Is this true, or can only one argument be variable size?}
%\paul{Only one argument can have a variable size, unless extra
%  argument are added to the event to describe $N-1$ of the
%  variable sizes, but this is not important enough to include in the
%  paper}
This description is precise enough that tools 
can skip events, and even of events, that they do not understand,
and process the rest.

Mercury is able to make use of a number of event types already supported by
\tscope and GHC,
in other cases we are able to add support for Mercury-specific event types to
\tscope.
Here we introduce the events in the first category,
The next section describes those in the second category.
Events of all types have a timestamp
that records the time of their occurrence, measured in nanoseconds.
This unit illustrates the level of precision \tscope aims for,
although of course there is no guarantee
that the system clock is capable of achieving it.

Most events in the log file are associated with an engine.
To avoid having to include an engine id with every event in the log file,
the log file format
groups sequences of events that are all from the same engine into a block,
and writes out the engine id just once, in a block header pseudo-event.
Since tools reading the log file can trivially remember
the engine id in the last block header they have read,
this makes the current engine id
(the id of the engine in whose block an event appears)
an implicit parameter of most event types.

The event types supported by the original version of \tscope 
that are relevant to our work are the following.

\emph{STARTUP:
Marks the beginning of the execution of the program,
and records the number of engines that the program will use.}

\emph{SHUTDOWN:
Marks the end of the execution of the program.}

\emph{CREATE\_THREAD:
Records the act of the current engine creating a context,
and gives the id of the context being created.}
(What Mercury calls a context Haskell calls a thread;
the name of the event uses Haskell terminology.)
Context ids are allocated sequentially,
but the parameter cannot be omitted,
since the runtime system can and will reuse the storage of a context
after the termination of the computation that has previously used it.

\emph{RUN\_THREAD:
Records the scheduling event
of the current engine switching to execute a context,
and gives the id of the context being switched to.}

\emph{STOP\_THREAD:
Records the scheduling event
of the current engine switching away from executing a context.}
Gives the id of the context being switched from,
as well as the reason for the switch.
The possible reasons include:
(a) the heap is full, and so the engine must invoke the garbage collector;
(b) the context has blocked, and
(c) the context has finished.

\emph{THREAD\_RUNNABLE:
Records that the current engine has made a blocked context runnable,
and gives the id of the newly-runnable context.}

\emph{RUN\_SPARK:
Records that the current engine is starting to run a spark
that it retrieved from its own local spark queue.}
Gives the id of the context that will execute the spark,
although this can be inferred by context.
% \zoltan{It can be inferred from the context id of the next
% CREATE\_SPARK\_THREAD event in Mercury,
% is this true in Haskell as well?}

\emph{STEAL\_SPARK:
Records that the current engine will run a spark
that it stole from the spark queue of another engine.}
Gives the id of the context that will execute the spark,
although again, this can be inferred by context.
% \zoltan{It can be inferred from the context id of the next
% CREATE\_SPARK\_THREAD event in Mercury,
% is this true in Haskell as well?}
Also gives the id of the engine that the spark was stolen from.
% This should really be called STOLE_SPARK.

\emph{CREATE\_SPARK\_THREAD:
Records the creation of a new context or the reuse of an old context
in order to execute a spark.}
Gives the id of the context.
Does not say whether the context is new or reused.
In most cases, that information is simply not needed,
but if it is, it can be inferred from context.
% In both Mercury and Haskell,
% the context will probably be used to execute other sparks
% once it finishes the current one.

\emph{GC\_START:
The current engine has initiated garbage collection;
control of the engine has been taken away from the mutator.}

\emph{GC\_END:
Garbage collection has finished on the current engine;
control of the engine has been returned to the mutator.}
% \paul{I do not know the intended semantics of the GC events that Haskell uses}
% More information about how Mercury uses these events is in Section
% \ref{sec:gc_stats} below.

% The concepts of \emph{engine} and \emph{capability} are the same in
% Mercury and Haskell respectively.
% so are the concepts of Contexts and Threads.
% Mercury and Haskell both use sparks, although in Haskell sparks do not need to
% be evaluated, and may be garbage collected.
% Because Mercury is strict sparks do not represent thunks and therefore
% sparks must be executed in order to write their
% results into their parent's stack frame.

% Mercury also adds the concept of futures (Section \ref{sec:futures}),
% Haskell does not need futures since any unevaluated value is a thunk
% and can be evaluated directly by the code trying to read its value.
% If the thunk is already being evaluated, the thread will block;
% otherwise, the thread will evaluate the thunk itself.

\noindent
The longer a parallel program being profiled runs,
the more events it will generate.
Long running programs can generate enormous log files.
To keep the sizes of log files down as much as possible,
events include only the information they have to.
If some information about an event
can be inferred from information recorded for other events,
then the design principles of \tscope say
that information should be inferred at analysis time
rather than recorded at profiling runtime.
(The presence of context ids in RUN\_SPARK and STEAL\_SPARK events
violates this principle,
since they are guaranteed to be the same as the id of the context already
running on that engine or
the context id in the following CREATE\_SPARK\_THREAD event.
but their removal is prevented by backwards compatibility concerns.)
In most cases, the required inference algorithm is quite simple.
For example, answering the question of whether the context
mentioned by a CREATE\_SPARK\_THREAD event is new or reused
merely requires searching the part of the log up to that event
looking for mentions of the same context id.
And we have already seen how the engine id parameter
missing from most of the above events
can be deduced from block headers.

\section{New events}
\label{sec:newevents}

In order to support the profiling of parallel Mercury programs,
we had to add new arguments to two of the existing \tscope events.
Originally, the RUN\_SPARK and STEAL\_SPARK events
did \emph{not} specify the identity of the spark being run or stolen.
This is not a problem for the Haskell version of \tscope,
since it does not care about sparks' identities.
However, we do, since sparks correspond to conjuncts in parallel conjunctions,
and we want to give to the user not just general information
about the behavior of the program as a whole,
but also about the behavior of individual parallel conjunctions,
and of the conjuncts in them.
We have therefore added an id that uniquely identifies each spark
as an additional argument of the RUN\_SPARK and STEAL\_SPARK events.
Note that CREATE\_SPARK\_THREAD does not need a spark id,
since the only two ways it can get the spark it converts into a context
is by getting it from its own queue or from the queue of another engine.
A CREATE\_SPARK\_THREAD event will therefore always be preceded
by either a RUN\_SPARK event or a STEAL\_SPARK event,
and the id of the spark in that event
will be the id of the spark being converted.

% In Haskell, their identities can always be inferred
% by modelling the spark queues of all the capabilities (engines) in the system.

We have extended the set of \tscope event types
to include several new types of events.
Most of these record information about constructs that do not exist in Haskell:
parallel conjunctions, conjuncts in those conjunctions, and futures.
Some provide information about the behavior of Mercury engines
that \tscope for Haskell does not need,
either because that information is of no interest,
or because the information is of interest
but it can be deduced from other events.
Even though the Haskell and Mercury runtime systems
generate many of the same events,
the stream of these common events they generate
do not necessarily obey the same invariants.

However, most of the work we have done towards adapting \tscope to Mercury
has been in the modification of the parallel Mercury runtime system
to generate all of the existing, modified and new events when called for.

In the rest of this section,
we report the event types we have added to \tscope.
In the next section, section \ref{sec:analysis},
we will show how the old and new events can be used together
to infer interesting and useful information
about the behavior of parallel Mercury programs.

% \item[STRING] Record an association between a String and an integer so
% that a static string occurring many times within a log file can be
% replaced with an integer.
% Making the log files shorter and therefore not flushing them to disk
% as often.

\emph{START\_PAR\_CONJUNCTION: records the fact that
the current engine is about to start executing a parallel conjunction.}
It identifies the parallel conjunction in two different ways.
The static id identifies
the location of the conjunction in the source code of the program.
The dynamic id is the address of the barrier structure
that all the conjuncts in the conjunction will synchronize on when they finish.
Note that barrier structures may be garbage collected,
and their storage reused for other barriers used by later code,
so these dynamic ids are not unique across time,
but they \emph{do} uniquely identify a parallel conjunction
at any given moment in time.
See the end of this section for a discussion of why we chose this design.

\emph{END\_PAR\_CONJUNCTION:
records the end of a parallel conjunction}.
It gives the dynamic id of the finishing conjunction.
Its static id can be looked up in the matching START\_PAR\_CONJUNCTION event.

\emph{CREATE\_SPARK: records the creation of a spark
for a parallel conjunct by the current engine.}
% \paul{XXX We call this SPARKING in the source code but should not}
Gives the id of the spark itself,
and the dynamic id of the parallel conjunction the conjunct is from.
To keep the log file small,
it does \emph{not} give the position of the conjunct in the conjunction.
However, since in every parallel conjunction
the spark for conjunct $n+1$
will be created by the context executing conjunct $n$,
and unless $n=1$, that context will itself have been created
from the spark for conjunct $n$.
(The first conjunct of a parallel conjunction is always executed directly,
without ever being represented by a spark.)
This means that the sparks for the non-first conjuncts
will always be created in order,
which makes it quite easy to determine which spark represents which conjunct.

The id of a spark is an integer consisting of two parts.
% spark IDs are 32 bits wide, the 8 most significant bits are the engine ID,
The first part is the number of the engine that created the spark,
and the second is an engine-specific sequence number,
so that in successive sparks created by the same engine,
the second part will be 1, 2, 3 etc.
This design allows the runtime system
to allocate globally unique spark ids without synchronization.

\emph{END\_PAR\_CONJUNCT:
records that the current engine
has finished the execution of a conjunct in a parallel conjunction.}
Gives the dynamic id of the parallel conjunction.
Note that there is no START\_PAR\_CONJUNCT event
to mark the start of execution of any conjunct.
For the first conjunct, its execution will start
as soon as the engine has finished recording the START\_PAR\_CONJUNCTION event.
The first thing the conjunct will do is create a spark
representing the second and later conjuncts
(which above we informally referred to as the spark for the second conjunct).
The CREATE\_SPARK event records the id of the spark,
then, either the RUN\_SPARK or STEAL\_SPARK event records which engine and context
executes the spark,
If the engine does not yet have a context a CREATE\_SPARK\_THREAD event is posted
that identifies the newly created context.
RUN\_THREAD and STOP\_THREAD events
also tell us when that context is executing, and on which engine.
The first thing the second conjunct does is spawn off a spark
representing the third and later conjuncts.
By following these links,
a single forward traversal of the log file can find out,
for each engine, which conjunct of which dynamic parallel conjunction
it is running at any given time
(if in fact it is running any;
an engine can be idle,
or it may be running code that is outside of any parallel conjunction).
When an engine records an END\_PAR\_CONJUNCT event,
it can only be for the conjunct it is currently executing.

\emph{FUTURE\_CREATE: records the creation of a future by the current engine.}
Gives the name of the variable the future is for,
as well as the dynamic id of the future.
It does not give the dynamic id of the parallel conjunction
whose conjuncts the future is intended to synchronize,
but that can be found out quite easily:
just scan forward for the next START\_PAR\_CONJUNCTION event.
The reason why this inference works is that
the code that creates futures
is only ever put into programs by the Mercury compiler,
and mmc never puts that code anywhere
except just before the start of a parallel conjunction.
If a parallel conjunction uses $n$ futures,
then every one of its START\_PAR\_CONJUNCTION events
will be preceded by $n$ FUTURE\_CREATE events,
one for each future.

The dynamic id of the future is its address in memory.
This has the same issues with reuse
as the dynamic ids of parallel conjunctions.

\emph{FUTURE\_SIGNAL: records that code running on the current engine
has signalled the availability
of the value of the variable protected by a given future.}
Gives the id of the future.
If another conjunct is already waiting for the value stored in this future,
this event will unblock its context.

\emph{FUTURE\_WAIT\_NO\_SUSPEND: records that the current engine
retrieved the value of the future without blocking.}
Gives the id of the future.
The future's value was already available
when the engine tried to wait for the value of the future,
so the context running on the current engine was not blocked.

\emph{FUTURE\_WAIT\_SUSPEND: records that the current engine
tried to retrieve the value of a future, but was blocked.}
Gives the id of the future.
Since the future's value was not already available,
the current context has been be suspended until it is.

We have added a group of four event types
that record the actions of engines
that cannot continue to work on what they were working before,
either because the conjunct they were executing finished,
or because it suspended.

\emph{TRY\_GET\_RUNNABLE\_CONTEXT: This engine is checking the global
run queue for a context to execute.}

\emph{TRY\_GET\_LOCAL\_SPARK: This engine is attempting to get a spark
from its own stack.}

\emph{TRY\_STEAL\_SPARK: This engine is attempting to steal a spark
from another engine.}

\emph{ENGINE\_WILL\_SLEEP: The engine is about to sleep.}
The next event from this engine will be from when it is next awake.

The idea is that an idle engine can look for work in three different places:
(a) the global runnable context queue,
(b) its own local spark queue, and
(c) another engine's spark queue.
An idle engine will try these in some order,
the actual order depending on the scheduling algorithm.
The engine will post the try event for a queue
before it actually looks for work in that queue.
If one of the tests is successful, the engine will post
a START\_THREAD event, a RUN\_SPARK event or a STEAL\_SPARK event respectively.
If one of the tests fails, the engine will go on the next.
If it does not find work in any of the queues,
it will go to sleep after posing the ENGINE\_WILL\_SLEEP event.

This design uses two events to record
the successful search for work in a queue:
TRY\_GET\_RUNNABLE\_CONTEXT and START\_THREAD,
TRY\_GET\_LOCAL\_SPARK and RUN\_SPARK,
or TRY\_STEAL\_SPARK and STEAL\_SPARK.
This may seem wasteful compared to using one event,
but this design enables us to measure several interesting things:
\vspace{-2mm}
\begin{itemize}
\item
how quickly a sleeping engine can wake up and try to find work
when work is made available by a CREATE\_SPARK and or CONTEXT\_RUNNABLE event;
% This can remove the space between items, as well as above and below
% the environment, I like the look of -0.5
% \vspace{-1\baselineskip}
\item
how often an engine is successful at finding work;
% \vspace{-1\baselineskip}
\item
when it is successful,
how long it takes an engine to find work and begin its execution;
% \vspace{-1\baselineskip}
\item
when it is unsuccessful,
how long it takes to try to find work from other sources,
fail, and then to go to sleep.
\end{itemize}
\vspace{-2mm}

% I have removed the idle and working events.
% I spoke with Simon Marlow about this, I cannot construct a sound
% argument why I need them, and in all cases the state of an engine
% can be inferred by other events.  Although this inference feels
% dirty/clumsy it is the correct option since it keeps the log file
% shorter.
% The problem now is that Haskell and Mercury both behave differently,
% and therefore may have different rules of inference.

\noindent
As we mentioned above,
using the address of the barrier structure
as the dynamic id of the parallel conjunction whose end the barrier represents
and using the address of the future structure
as the dynamic id of the future
are both design choices.
The obvious alternative would be to give them both sequence numbers
using either global or engine-specific counters.
However, this would require
adding a field to both structures to hold the id, which costs memory, and
filling in the field, which costs time.
Both these costs would be incurred by the program execution
whose performance we are trying to measure,
interfering with the measurement itself.
While such interference cannot be avoided
(writing events out to the log file also takes time),
we nevertheless want to minimize it if at all possible.
In this case, it is possible:
the addresses of the structures are trivially available,
and require no extra time or space during the profiled run.
The tradeoff is that we now need a pre-pass over the log file
that consistently renames apart
the different incarnations of the same id for a parallel conjunction or future.

\tr{
The tradeoff is that if an analysis requires globally unique dynamic ids
for parallel conjunctions or for futures,
and most analyses do,
then it needs to ensure that a pre-pass has been run over the log file.
This pre-pass would maintain a map from future ids
to a pair containing an active/inactive flag, and a replacement id,
and another map from dynamic conjunction ids
to a tuple containing an active/inactive flag, a replacement id,
and a list of future ids.
When the pre-pass sees a dynamic conjunction id or future id in an event,
it looks it up in the relevant table.
If the flag says the id is active,
it replaces the id with the replacement.
If the flag says the id is inactive,
it gets a new replacement id (e.g.\ by incrementing a global counter),
and sets the flag to active.
If the event is a FUTURE\_CREATE event,
the pre-pass adds the future's original id to a list.
If the event is a START\_PAR\_CONJUNCTION event,
it copies this list to the conjunction's entry, and then clears the list.
If the event is an END\_PAR\_CONJUNCTION event,
which ends the lifetime not only of the parallel conjunction
but also of all futures created for that conjunction,
the pre-pass sets to inactive
that flags of both the conjunction itself
and the futures listed in its entry.

This algorithm consistently renames apart
the different incarnations of the same id,
replacing them with globally unique values.
Its complexity can be close to linear, provided
the maps are implemented using a suitable data structure, such as a hash table.
This transformation does not even need an extra traversal of the log file data.
The \tscope tool must traverse the log file anyway
when it gets ready to display its contents,
and this transformation can be done
as part of that traversal.
The extra cost is therefore quite small.
}

\section{Deriving metrics from events}
\label{sec:analysis}

\tscope is primarily used to visualize the execution of parallel programs.
However, we can also use it, along with our new events,
to calculate and to present to the user
a number of different metrics about the execution of parallel Mercury programs.
% \zoltan{needs discussion about how metrics are presented to users}
% \zoltan{needs discussion about audiences}
Some of the metrics are of interest
only to application programmers,
or only to runtime system implementors,
or only to autoparallelization tool implementors,
but several are useful to two or even all three of those audiences.
Also, some of these metrics say something about the whole program;
some say something about individual parallel conjunctions,
and some say something about individual conjuncts inside those conjunctions.
We discuss our proposed metrics in that order;
we then discuss how we intend to present them to users.

% Such as how much overlap we are \emph{really} getting,
% and how efficient the RTS is and how this can drive settings in our
% automatic parallelization analysis.
% 
% Discuss granularity and parallel slackness, show how we can measure these.

\subsection{Whole program metrics}

% \emph{PARCONJ\_CONTEXT\_SELF:
% The number of contexts that are actually used
% by a given parallel conjunction.}
% This measurement is computed similarly to PARCONJ\_RUNNABLE\_SELF,
% but it does not count the part of a conjunct's lifetime
% that it spends as a spark;
% it counts only the part that it spends as a context.
% The difference is important,
% because contexts occupy substantial amounts of memory,
% while sparks do not.

% For optimum memory efficiency,
% the ratio between the time-weighted averages of
% PARCONJ\_CONTEXT\_SELF and PARCONJ\_RUNNABLE\_SELF
% should be as low as possible for all parallel conjuncts.

\emph{CPUS\_OVER\_TIME:
The number of CPUs being used by the program at any given time.}
It is trivial to scan all the events in the trace,
keeping track of the number of engines currently being used,
each of which corresponds to a CPU.
The resulting curve tells programmers
which parts of their program's execution is already sufficiently parallelized,
which parts are parallelized but not yet enough to use all available CPUs,
and which parts still have no parallelism at all.
They can then focus their efforts on the latter.
The visualization of this curve has been implemented,
and is described by \citet{threadscope}.

% number of sparks available for execution at any given time
% number of runnable but not running contexts at any given time
% number of sparks plus runnable but not running contexts at any given time

% \label{sec:gc_stats}
\emph{GC\_STATS: The number of garbage collections,
and the average, minimum, maximum and variance
of the elapsed time taken by each collection.}
Mercury uses the Boehm-Demers-Weiser garbage collector \citep{boehm:1988:gc}.
This collector supports parallel marking using GC-specific helper threads
rather than the OS threads that run Mercury engines.
Therefore, even when parallel marking is used,
we can only calculate the elapsed time used by garbage collection,
and not the \emph{CPU time} used by garbage collection.
The elapsed time of each garbage collection
is the interval between pairs of GC\_START and GC\_END events.

\emph{MUTATOR\_VS\_GC\_TIME:
The fraction of the program's runtime used by the garbage collector.}
We calculate this by summing the times
between pairs of GC\_START and GC\_END events,
and dividing the result by the program's total runtime.
(Both sides of the division refer to elapsed time, not CPU time.)
Due to Amdahl's law \citep{amdahl:1967:law},
% XXX Explain amdahl's law since the paper dosn't describe the usual
% interpretation.
this fraction limits the best possible speedup we can get
for the program as a whole by parallelizing the mutator.
For example, if the program spends one third of its time doing GC
(which unfortunately actually happens for some parallel programs),
then no parallelization of the program can yield a speedup of more than three,
\emph{regardless} of the number of CPUs available.

\emph{NANOSECS\_PER\_CALL:
The average number of nanoseconds between successive procedure calls.}
The Mercury profiler for sequential programs \citep{conway:2001:mercury-deep}
(which has nothing to do with \tscope)
measures time in call sequence counts (CSCs);
the call sequence counter is incremented at every procedure call.
It does this because there is no portable, or even semi-portable way
to access any realtime clocks that may exist on the machine,
and even the non-portable method on x86 machines (the RDTSC instruction)
is too expensive for it.
(The Mercury sequential profiler needs to look up the time at every call,
which in typical programs will happen every couple of dozen instructions or so.
The Mercury parallel profiler needs to look up the time only at each event,
which occur \emph{much} more rarely.)
% RDTSC still has some overhead:
% a) It is a sequencing operation for the CPU.
% b) It requires some arithmetic and maybe a function call, I do not
%    know if we can afford that.

The final value of the call sequence counter
at the end of a sequential execution of a program 
gives its length in CSCs; say $n$ CSCs.
When a parallelized version of the same program is executed on the same data
under \tscope,
we can compute the total amount of \emph{user time}
taken by the program on all CPUs; say $m$ nanoseconds.
The ratio $n/m$ gives the average number of nanoseconds
in the parallel execution of the program
per CSC in its sequential execution.
% \paul{I am having trouble parsing the above sentence, I ave tried to
% read it 3 times now, I do not know what you're trying to say}
% \zoltan{is the new phrasing clear?}
% Clearer, but not great, I get stuck at ``in its sequential execution''
% but now I know what you mean and this phrase is somewhat implicit.
This is useful information,
because our automatic parallelization system
uses CSCs, as measured by the Mercury sequential profiler,
as its unit of measurement of the execution time 
of both program components and system overheads.
Using this scale, we can convert predictions about time made by the tool
from being expressed in CSCs to being expressed in nanoseconds,
which is an essential first step in comparing them to reality.

\subsection{Conjunction specific metrics}

\emph{PARCONJ\_TIME: The time taken by a given parallel conjunction.}
For each dynamic parallel conjunction id that occurs in the trace,
it is easy to compute the difference between
the times at which that parallel conjunction starts and ends,
and it is just as trivial to associate these time intervals
with the conjunction's static id.
From this, we can compute,
for each parallel conjunction in the program that was actually executed,
both the average time its execution took,
and the variance in that time.
This information can then be compared,
either by programmers or by automatic tools,
with the sequential execution time of that conjunction recorded by
Mercury's sequential profiler,
to see whether executing the conjunction was a good idea or not.
\emph{provided} that the two measurements are done in the same units.
At the moment, they are not, but as we discussed above,
CSCs can be converted to nanoseconds using the NANOSECS\_PER\_CALL metric.

\emph{PARCONJ\_RUNNABLE\_SELF:
The number of CPUs that can be used by a given parallel conjunction.}
For each dynamic parallel conjunction id that occurs in the trace,
we can divide the time between its start and end events into blocks,
with the blocks bounded by
the CREATE\_SPARK\_THREAD and STOP\_THREAD events for its conjuncts,
and the FUTURE\_WAIT\_SUSPEND and FUTURE\_SIGNAL events
of the futures used by those conjuncts.
We can then compute, for each of these blocks,
the number of runnable tasks that these conjuncts represent.
From this we can compute the history of the number
of runnable tasks made available by this dynamic conjunction.
We can derive the maximum and the time-weighted average of this number,
and we can summarize those
across all dynamic instances of a given static conjunction.

Consider a conjunction with $n$ conjuncts.
If it has a maximum number of runnable tasks
that is significantly less than $n$,
this is a sign that the parallelism the programmer aimed for
could not be achieved.
If instead the average but not the maximum number of runnable tasks
is significantly less than $n$,
this suggests that the parallelism the programmer aimed for was achieved,
but only briefly.
The number of runnable tasks can drop due to
either a wait operation that suspends or the completion of a conjunct.
Both dependencies among conjuncts
and differences in the execution times of the conjuncts
limit the amount of parallelism available in the conjunction.
The impact of the individual drops on the time-weighted average
shows which effects do the most limiting in any given parallel conjunction.

\emph{PARCONJ\_RUNNABLE\_SELF\_AND\_DESC:
The number of CPUs that can be used by a given parallel conjunction
and its descendants.}
This metric is almost the same as PARCONJ\_RUNNABLE\_SELF,
but it operates
not just on a given dynamic parallel conjunction,
but also on the parallel conjunctions
spawned by the call-trees of its conjuncts as well.
It takes their events into account when it divides time into blocks,
and it counts the number of runnable tasks they represent.

The two metrics,
PARCONJ\_RUNNABLE\_SELF and PARCONJ\_RUNNABLE\_SELF\_AND\_DESC,
can be used together to see whether
the amount of parallelism that can be exploited
by the two parallel conjunctions together is substantially greater than
the amount of parallelism that can be exploited
by just the outer parallel conjunction alone.
If it is, then executing the inner conjunction in parallel is a good idea;
if it is not, then it is a bad idea.

If the outer and inner conjunction in the dynamic execution
come from different places in the source code,
then acting on such conclusions is relatively straightforward.
If they represent different invocations of the same conjunction in the program,
which can happen if one of the conjuncts contains a recursive call,
then acting on such conclusions will typically require
the application of some form of runtime granularity control.

% \paul{The number of parallel conjunctions in this case is as deep as
% the recursion, not just two.
% Can you explain how considering the first two levels of the recursion
% bares on all the levels of recursion?
% Or is this a case where a parallel conjunction is nested within
% another and no code within either conjunction is recursive?  Thanks}

\emph{PARCONJ\_RUNNING\_SELF:
The number of CPUs that are actually used by a given parallel conjunction.}
This metric is computed similarly to PARCONJ\_RUNNABLE\_SELF,
but it does not count a runnable conjunct until gets to use a CPU,
and stops counting a conjunct when it stops using the CPU
(when it blocks on a future, and when it finishes).
% \zoltan{Does the CPU ever get taken away from a conjunct?}
% \paul{Only when it blocks on a future, finishes a conjunct or calls
% \code{sched.yield}}

Obviously, PARCONJ\_RUNNING\_SELF can never exceed PARCONJ\_RUNNABLE\_SELF
for any parallel conjunction at any given point of time.
However, one important difference between the two metrics
is that PARCONJ\_RUNNING\_SELF can never exceed
the number of CPUs on the system either.
If the maximum value of PARCONJ\_RUNNABLE\_SELF for a parallel conjunction
does not exceed the number of CPUS,
then its PARCONJ\_RUNNING\_SELF metric
can have the same value as its PARCONJ\_RUNNABLE\_SELF metric,
barring competition for CPUs by other parallel conjunctions (see later)
or by the garbage collector.

On the other hand, some conjunctions
do generate more runnable tasks than there are CPUs.
In such cases, we want the extra parallelism
that the system hardware cannot accommodate
in the period of peak demand for the CPU
to ``fill in'' later valleys,
periods of time when the conjunction demands
less than the available number of CPUs.
This will happen only to the extent that
the delayed execution of tasks that lost the competition for the CPU
does not lead to further delays in later conjuncts
through variable dependencies.

The best way to measure this effect
is to visually compare the PARCONJ\_RUNNING\_SELF curves for the conjunction
taken from two different systems,
e.g.\ one with four CPUs and one with eight.
However, given measurements taken from e.g.\ a four CPU system,
it should also be possible to predict with \emph{some} certainty
what the curve would look like on an eight CPU system,
by using the times and dependencies recorded in the four-CPU trace
to simulate how the scheduler would handle the conjunction
on an eight CPU machine.
Unfortunately, the simulation cannot be exact
unless it correctly accounts for \emph{everything},
including cache effects and the effects on the GC system.

The most obvious use of the PARCONJ\_RUNNING\_SELF curve of a conjunction
is to tell programmers whether and to what extent
that parallel conjunction can exploit the available CPUs.

\emph{PARCONJ\_RUNNING\_SELF\_AND\_DESC:
The number of CPUs that are actually used by a given parallel conjunction
and its descendants.}
This metric has the same relationship to PARCONJ\_RUNNING\_SELF
as PARCONJ\_RUNNABLE\_SELF\_AND\_DESC has to PARCONJ\_RUNNABLE\_SELF.
Its main use is similar to the main use of PARCONJ\_RUNNING\_SELF:
to tell programmers whether and to what extent
that parallel conjunction and its descendants can exploit the available CPUs.
This is important because a conjunction and its descendants
may have enough parallelism
even if the top-level conjunction by itself does not,
and in such cases the programmer can stop looking for more parallelism,
at least in that part of the program's execution timeline.

\emph{PARCONJ\_AVAIL\_CPUS:
The number of CPUs available to a given parallel conjunction.}
Scanning through the entire trace,
we can compute and record the number of CPUs being used
at any given time during the execution of the program.
For any given parallel conjunction,
we can also compute PARCONJ\_RUNNING\_SELF\_AND\_DESC,
the number of CPUs used by that conjunction and its descendants.
By taking the difference between the curves of those two numbers,
we can compute the curve of the number of CPUs
that execute some task \emph{outside} that conjunction,
Subtracting that difference curve
from the constant number of the available CPUs
gives the number of CPUs available for use by this conjunction.

This number's maximum, time-weighted average
and the shape of the curve of its value over time,
averaged over the different dynamic occurrences
of a given static parallel conjunction,
tell programmers the level of parallelism they should aim for.
% \paul{Should there be a new sentence here.  Reading the first half of
% the sentence it sounds like you're talking about increasing the
% amount of parallelism,
% but the second half of the sentence talks about decreasing it --- at
% least that's how I understood it on first impression.}
Generating parallelism in a conjunction
that consistently exceeds the number of CPUs available for that conjunction
is more likely to lead to slowdowns from overheads
than to speedups from parallelism.

\emph{PARCONJ\_CONTINUE\_ON\_BEFORE:
the probability, for any given parallel conjunction,
that the code after the conjunction
will continue executing on the same CPU
as the code before the conjunction.}
If the parallel conjunction as a whole took only a small amount of time,
then CPUs other than the original CPU will still have relatively cold caches,
even if they ran some part of the parallel conjunction.
We want to keep using the warm cache of the original CPU.
The better the scheduling strategy is at ensuring this,
the more effectively the system as a whole will exploit the cache system,
and the better overall system performance will be.

\emph{PARCONJ\_CONTINUE\_ON\_LAST:
the probability, for any given parallel conjunction,
that the code after the conjunction
will continue executing on the same CPU
as the last conjunct to finish.}
If the parallel conjunction as a whole took a large amount of time,
then how warm the cache of a CPU will be
for the code after the conjunction
depends on how recently that CPU executed
either a conjunct of this conjunction or the code before the conjunction.
Obviously, all the conjuncts execute after the code before the conjunction,
so if the conjunction takes long enough for most of the data accessed
before the conjunction to 
% \paul{Cached data does not become cold, a cache becomes cold after
% useful data has been removed}.
% become cold and therefore
be evicted from the cache,
then only the CPUs executing the conjuncts will have useful data in
their caches.
% \paul{what race?}
% are in the race.
In the absence of specific information
about the code after the conjunct preferentially accessing data
that was also accessed (read or written) by specific conjuncts,
the best guess is that the CPU with the warmest cache
will be the one that last executed a conjunct of this conjunction.
The more often the scheduling strategy executes
the code after the conjunction on that CPU,
the more effectively the system as a whole will exploit the cache system,
and the better overall system performance will be.

\subsection{Conjunct specific metrics}

There are several simple times
whose maximums, minimums, averages and variances
can be computed for each static parallel conjunct.

\emph{CONJUNCT\_TIME\_AS\_SPARK:
the time between the conjunct's creation
(which will be as a spark) and the start of its execution
(when the spark will be converted into a context).}

\emph{CONJUNCT\_TIME\_AS\_CONTEXT:
the time between the start of the conjunct's execution and its end.}

\emph{CONJUNCT\_TIME\_BLOCKED:
the total amount of time
between the start of the conjunct's execution and its end
that the conjunct spends blocked waiting for a future.}

% We also want a new metric,
% \item[CONJUNCT_TIME_BLOCKED_BARRIER] but discussing this means
% discussing right vs left recursion with Mercury's parallel runtime,
% which like to avoid for this paper.

\emph{CONJUNCT\_TIME\_RUNNABLE:
the total amount of time
between the start of the conjunct's execution and its end
that the conjunct spends runnable but not running.}

\emph{CONJUNCT\_TIME\_RUNNING:
the total amount of time
between the start of the conjunct's execution and its end
that the conjunct spends actually running on a CPU.}

Since every context is always either running, runnable or blocked,
the last three numbers must sum up to the second
(in absolute terms and on average, not in e.g.\ maximum or variance).

Programmers may wish to look at conjuncts
that spend a large percentage of their time blocked
to see whether the dependencies that cause those blocks can be eliminated.

\tr{
If such a dependency cannot be eliminated,
it may still be possible to improve
at least the memory impact of the conjunct
by converting some of the blocked time into spark time.
The scheduler should definitely prefer executing an existing runnable context
over taking a conjunct that is still a spark,
converting the spark into a context and running that context.
When there are no existing runnable contexts
and it must convert a spark into a context,
the scheduler should try to choose a spark whose consumed variables
(the variables it will wait for) are all currently available.
Of course, such a scheduling algorithm is not possible
without information about which variables conjuncts consume,
information that schedulers do not typically have access to,
but which it is easy to give them.

If all the sparks consume at least one variable
that is not currently available,
the scheduler should prefer to execute the one
whose consumed variables are the \emph{closest} to being available.
This requires knowledge of the expected behavior of the program,
to wit, the expected running times of the conjuncts generating those variables.
In some cases, that information may nevertheless be available,
derived from measured previous runs of the program,
although of course it can only ever be a guess.
% \paul{Should there be a comma after although?
% and maybe another after of course?}
% \zoltan{no}

Note also this is only one of several considerations
that an ideal scheduler should take into account.
For example, schedulers should also prefer to execute the conjunct
(whether it is a context or a spark)
that is currently next on the critical path to the end of the conjunction.
Knowledge of the critical path is also knowledge about the future,
and similarly must also be a guess.
Ideally, the scheduler should take into account
all these different considerations before coming to a decision
based on balancing their relevance in the current situation.
}

% When it chooses between several existing runnable contexts,
% ideally it should choose one XXX

% If such a dependency cannot be eliminated,
% the programmer may still be able to improve
% at least the memory impact of the conjunct
% by converting some of the blocked time into spark time.
% The idea is that if a conjunct starts but then blocks waiting for some data,
% and does not generate any of its own data in the meantime,
% then if possible its start should be delayed
% until a time closer to when the data it needs will be available.
% This reduces the contribution made by the conjunct
% to the integral of the program's aggregate memory demand over time.
% Such as delay is often possible
% by merging two parallel conjuncts in a larger parallel conjunction into one.

% As an example, consider the three-way parallel conjunction
% p(In1, X, W) & q(In2, X, Y) & r(In3, Y, Z),
% where In1, In2 and In3 are available at the start of the conjunction,
% p generates X and W, q generates Y and r generates Z.
% If r needs Y very soon after the start of its execution,
% then it maybe preferable to change that conjunction into the two-way
% p(In1, X, W) & ( q(In2, X, Y), r(In3, Y, Z) ).
% By executing q and r in sequence instead of in parallel, ...
% XXX this example illustrates the wrong point

\emph{CONJUNCT\_TIME\_AFTER:
the time, for each parallel conjunct,
between the end of its execution and the end of the conjunction as a whole.}
It is easy to compute this for every dynamic conjunct,
and to summarize it as a minimum, maximum, average and variance
for any given static conjunct.
If the average times after
for different conjuncts in a given parallel conjunction
are relatively stable (have low variance)
but differ substantially compared to the runtime of the conjunction as a whole,
then the speedup from the parallel execution of the conjunction
will be significantly limited by Amdahl's law.
In such cases, the programmer may wish to take
two conjuncts that are now executed in parallel
and execute them in sequence.
Provided the combined conjunct is not the last-ending conjunct,
and provided that delaying the execution of one of the original conjuncts
does not unduly delay any other conjuncts that consume the data it generates,
the resulting small loss of parallelism
may be more than compensated for
by the reduction in parallelism overhead.
It is of course much harder and maybe impossible
to select the two conjuncts to execute in sequence
if the times after for at least some of the conjuncts
are \emph{not} stable (i.e. they have high variance).

% \emph{FUTURE_BLOCK_PROBABILITY:
% the probability that a consuming conjunct
% will ever block on the future for a given variable.}
% The higher the value of this metric is for a variable,
% the more likely that variable is to be a bottleneck.

% \emph{FUTURE_BLOCK_TIME:
% the average, minimum, maximum and variance
% in the time that any consuming conjunct
% that blocks on a future for a given variable
% actually spends blocked on that future.}
% The more time that blocked conjuncts spend blocked waiting for a variable,
% the bigger the bottleneck created by read accesses to that variable.

% probability of blocking,
% time between signal and first wait in nonblocking case,
% time between first wait and signal (blocking time) in blocking case,
% total time between all waits and signal (blocking time) in blocking case:
% average, variance

\emph{FUTURE\_SUSPEND\_TIME\_FIRST\_WAITS:
for each shared variable in each parallel conjunction:
the minimum, average, maximum and variance of the time
of the first wait event on a future for this variable
by each consuming conjunct
minus the time of the corresponding signal event.}
If this value is consistently positive,
then contexts never or rarely suspend waiting for this future;
if this value is consistently negative,
then contexts often suspend on this future, and do so for a long time.
% but if the variance is low they are suspended for long.
Programmers should look at shared variables
that fit into the second category
and try to eliminate the dependencies they represent.
If that is not possible, they may nevertheless try to reduce them
by computing the variable earlier
or pushing the point of first consumption later.
% After optimizing these cases,
% programmer may wish to look at shared variables
% that fit into neither category.

\emph{FUTURE\_SUSPEND\_TIME\_ALL\_WAITS:
for each future in each dynamic parallel conjunction:
the minimum, average, maximum and variance of the time
of all wait events minus the time of the corresponding signal event,
and the count of all such wait events.}
This metric helps programmers understand how hard
delaying the point of first consumption of a shared variable in a conjunct
is likely to be.
Delaying the first consumption is easiest
when the variable is consumed in only a few places,
and the first consumption occurs a long time before later consumptions.
If the variable is consumed in many few places,
many of these points of consumption are just slightly after
the original point of first consumption,
then significantly delaying the point of first consumption
requires eliminating the consumption of the shared variable
in \emph{many} places in the code,
or at least significantly delaying the execution
of those many pieces of code.

% \emph{FUTURE_SUSPEND_TIME_ALL_WAITS:
% for each future in each dynamic parallel conjunction:
% the minimum, average, maximum and variance of
% the time of all wait events minus the time of the corresponding signal event.}
% When this value is consistently positive,
% then contexts never or rarely suspend waiting for this future;
% when this value is consistently negative,
% contexts often suspended on this future and for a long time;
% if the average value is close to zero,
% then contexts sometimes suspend on this future.
% % but if the variance is low they are suspended for long.
% Programmers should look at futures that fit into the second category
% and try to remove or reduce
% the dependencies they represent between parallel conjuncts;
% after optimizing these cases,
% they should look futures that fit into the third category.

% \emph{FUTURE_SUSPEND_TIME_FIRST_WAIT:
% for each future in each dynamic parallel conjunction:
% the minimum, average, maximum and variance of the time
% of the first wait event minus the time of the corresponding signal event.}
% Like FUTURE_SUSPEND_TIME_ALL_WAITS,
% this metric helps programmers understand
% how dependencies affect their program's performance.
% However, when compared to FUTURE_SUSPEND_TIME_ALL_WAITS,
% it can inform the user about cases where a dependency is a problem in
% only some of the parallel conjuncts of the parallel conjunction.
% \zoltan{what does this mean}

\emph{FUTURE\_SUSPEND\_PROBABILITY\_FIRST\_WAIT:
for the first wait event of each future:
the number of times the signal event occurs before the wait event versus
the number of times the wait event occurs before the signal event.}

\emph{FUTURE\_SUSPEND\_PROBABILITY\_ALL\_WAITS:
for each wait event of each future,
the number of times the signal event occurs before the wait event versus
the number of times the wait event occurs before the signal event.}
Like FUTURE\_SUSPEND\_TIME\_ALL\_WAITS and FUTURE\_SUSPEND\_TIME\_FIRST\_WAIT,
these two metrics can help programmers find out
how often contexts are suspended waiting for futures.

\emph{WAITED\_FUTURE\_SIGNAL\_TO\_CONJUNCT\_END\_TIME:
the average, maximum and variance of the time
between the signaling of a future on which another context is blocked
and the end of the parallel conjunct that signalled the future.}
This should be computed for every parallel conjunct
that signals at least one future.
If the average is below the time that it normally takes
for a context made runnable to begin executing on an engine,
and the variance is low, then it suggests that
it is better to execute the context made runnable by the signal
on the same engine immediately after the current conjunct finishes.

\emph{FUTURE\_SIGNAL\_TO\_CONJUNCT\_END\_TIME:
the average, maximum and variance of the time
between the signaling of a future
and the end of the parallel conjunct that signaled that future.}
As above, this should be computed for
every parallel conjunct that signals at least one future.
This value can be used to determine
how often the optimization described above will not be useful.

\emph{OUT\_OF\_ORDER\_PROBABILITY:
given two parallel conjuncts $A$ and $B$,
with $A$ coming logically before $B$
either by being to the left of $B$ in some conjunction,
or by some ancestor of $A$ being to the left of an ancestor of $B$
in some conjunction,
how much time does the system spend with $B$ running
while $A$ is runnable but not running,
compared to the time it spends with either $A$ or $B$ running?}
Ideally, when $A$ and $B$ are both runnable
but the system has only one available CPU,
it should choose to run $A$.
Since it comes logically earlier,
the consumers that depend on its outputs
will also tend to come logically earlier.
Delaying the running of $A$ has a substantial risk of delaying them,
thus delaying the tasks depending on \emph{their} outputs, and so on.
Any efficient scheduling algorithm
must take this effect into consideration.

\emph{OUT\_OF\_ORDER\_SCHEDULING\_BLOCKS\_ANOTHER\_CONTEXT:
when tasks are executed out of order, as described above,
how much of the time is another context $C$
blocked on a future produced by $A$?}
This metric describes how often
out of order execution has a direct impact on another task.

\emph{OUT\_OF\_ORDER\_SCHEDULING\_BLOCKS\_ANOTHER\_CONTEXT\_TC:
When tasks are executed out of order, as described above,
how much of the time do other contexts $D, E, F\ldots$ block
waiting on a future signaled by $C, D, E, \ldots$
which, eventually, depend on $A$?}
This metric considers the transitive closure of the dependency chain
measured by the previous metric.

\subsection{Presenting metrics to users}

The \tscope tool is a graphical program.
Its main display screen shows
a continuous timeline of the program's execution on the horizontal axis,
while along the vertical axis,
the display is divided into a discrete number of rows,
with one row per CPU (per engine in Mercury, per capability in Haskell).
The color of a display point for CPU $n$ at time $t$
shows what CPU $n$ was doing at time $t$:
whether it was idle, doing GC, or executing the program.
The time axis can be scaled to see an overview of the execution as a whole
or to focus on a specific period during the program run.
If the current scale allows for it,
the display also shows the individual events
that the on-screen picture is derived from.
Above the per-CPU rows \tscope shows a plot that is virtually identical
to the curve of the CPUS\_OVER\_TIME metric we defined above,
and now it can do so for Mercury programs as well as for Haskell programs.

The main \tscope GUI also has a mechanism that allows the user
to ask for simple scalar data about the profiled program run.
Originally, it was used only to display the overall time taken by the run.
It is trivial to modify this to let the user input
the number of call sequence counts (CSCs)
executed by a sequence version of the same program on the same data,
which would allow the tool to compute and display
the value of the NANOSECS\_PER\_CALL metric.
It is not much harder to modify it to compute and then output
the values of the GC\_STATS and MUTATOR\_VS\_GC\_TIME metrics.

We plan to modify the \tscope GUI so that
if the user hovers the pointer over the representation
of an event connected with a parallel conjunction as a whole
(such as a START\_PAR\_CONJUNCTION or END\_PAR\_CONJUNCTION event),
they get a menu allowing them to select
one of the conjunction-specific metrics listed above.
The system should then compute and print the selected metric.
Similarly if the user hovers the pointer over the representation
of an event connected with a specific parallel conjunct,
they should be able to ask for and get
the value of one of the conjunct-specific metrics.

We plan to add to \tscope a screen
that lists all the parallel conjunctions executed by the program.
(The information needed for that list is available in the strings
representing the static ids of the executed parallel conjunctions.)
By selecting items from this list,
the tool should be able to print summaries
from all the dynamic occurrences of the conjunction
for any of the conjunction-specific metrics.
It should also be able to take users
to all those occurrences in the main display.

We also plan to add to \tscope
a mechanism that allows the user to request
the generation of a plain text file in a machine-readable format
containing all the metrics that may be of interest
to our automatic parallelization tool.
We want our tool to be able to compare
its predictions of autoparallelized programs
with metrics derived from actual measurements of such programs,
so that we can help tune its performance to reduce the resulting discrepancies.

% \section{Time recording and synchronization}
%
% A critical part of profiling knowing how long certain things take,
% and at what time certain events occur.
% One of the most reliable and portable ways of getting the current time is the
% \code{gettimeofday} system call.
% However,
% system calls can be expensive due to the switch between privilege levels.
% Some operating systems have efficient solutions to these problems,
% such as Linux's VSO support \citep{linux:vso};
% however, such features are not standard.
% Therefore,
% on i386 and amd64 systems that support it we use the \RDTSCP and
% \RDTSC instructions \citep{intel:rdtsc}.
% These instructions read a time stamp counter (TSC) from the CPU's internal
% counter.
% In some processors there is no linear relationship between the time stamp
% counter and real time,
% id est, they do not have a \emph{constant TSC}.
% For example, the processor may increment its time stamp counter less when
% it a lower power state.
% This means that a different meaning of time is used,
% one in which time corresponds to the amount of work a processor can do
% rather than time as it pases in the world outside the processor.
% We do not know if this meaning of time is useful,
% or if programmers would find it confusing;
% therefore,
% if the implementation does not detect the \emph{constant TSC} feature then it
% falls back to \code{gettimeofday}.
% Support for \RDTSCP and \RDTSC is detected dynamically,
% if unavailable our implementation falls back to \code{gettimeofday}
%
% % XXX: Describe the alrogithm and cite it.
% When there are multiple cores each core may have its own time stamp counter.
% Assuming that time passes at the same rate for each core we can
% calculate the amount that the time on each core is different to the first core.
% This offset is subtracted from the times recorded by each of these cores.
% In practice we have not needed this algorithm,
% Whenever it was enabled it simply conformed that the TSCs in the different
% cores where already synchronized.

\section{Related work and conclusion}
\label{sec:conc}

% It is well established that a visualization tool is extremely useful
% for programmers debugging and tuning complex systems.
% \tscope definitely supports this when used either with
% Haskell \citep{threadscope},
% or, as we have proposed, when used with Mercury.
% Furthermore, extending \tscope to work with Mercury has been
% relatively straight forward,
% largely due to \tscope's extensible file format.
% 
% Other tools have existed for profiling the parallel execution of other
% languages such as Eden's trace viewer \citep{edentraceviewer}.
% This shows a similar timeline to that of threadscope,
% but since Eden is a distributed programming language it also shows
% communication.
% 
% GranSim \citep{loidl98:gransim} is a simulator that determines how much
% parallelism is in a Haskell program.
% Ben Lippmeier wrote a similar tool/paper, its on his website.
% 
% \citet{runciman93:profilingparfp} introduces another visualization tool.
% This also shows the amount of parallelism in a Haskell program, but it does it
% over time and also shows the number of blocked tasks.
% 
% Apparently GUM has some good profiling tools, but I did not read this
% paper \citep{trinder:1996:gum}

% Use this as a source of feedback.
% Discuss what this work implies for Haskell.
% Discuss how this can possibly be tied to perf(1).

Many parallel implementations of programming languages
come with visualization tools
since (as we argued in the introduction)
the people writing parallel programs
need such tools to understand the actual behavior of their programs,
and the people implementing those languages find them useful too.
Such tools have been created for all kinds of programming languages:
imperative (even Visual Studio has one),
functional
(\citet{edentraceviewer,loidl98:gransim,runciman93:profilingparfp} among others)
and logic (\citet{Foster96,vace}
and the systems cited by \citet{Gupta95parallelexecution}).

Many of these visualizers share common ideas and features.
For example, many systems (including \tscope)
use a horizontal bar with different parts colored differently
to display the behavior of one CPU,
while many other systems use trees
that display the structure of the computation.
(The latter approach is more common
among visualizers for sequential languages.)
Nevertheless, each visualizer is necessarily oriented
towards giving its users what they want,
and the users of different systems want different information,
since they are using different languages based on different concepts.
For example, functional languages have no concept of parallel conjunctions,
and their concept of data flow between computations
is quite different from Mercury's,
and users of concurrent logic programming languages
such as Concurrent Prolog, Parlog and Guarded Horn Clauses
have quite different concerns from users of Mercury
(where to disable parallelism, not where to enable it).

We know of only one visualizer
for a not-inherently-concurrent logic programming language
that does nevertheless support dependent AND-parallelism: VACE \citep{vace}.
However, the ACE system supports
OR-parallelism and independent AND-parallelism
as well as dependent AND-parallelism,
and its features do not seem designed
to help programmers exploit dependent AND-parallelism better.
As far as we know, noone has built a profiler
for a dependent AND-parallel logic programming language
with the capabilities that we propose.
We haven't yet built one either :-),
but we are working on it.
We expect to have at least a third of our proposed metrics implemented
by the time of the workshop,
and many more by the end of the year.

% I think our basic argument should be:
% - visualization tools have long proven to be useful
%   in improving the performance of parallel programs
%   and in helping the developers of parallel systems
% - such tools have been built for many different systems:
%   logic, functional, and imperative,
% - some features are common to many of these systems:
%   many show trees, while many others show CPU utilization
% - yet each visualizer is nevertheless specific to its own language
% - the only visualizer (VACE) we know of for a language
%   that supports dependent AND parallelism
%   does not focus on dependent AND parallelism,
%   and it does not gather the metrics we think we need

% XXX Acknowledgements.
We would like to thank everyone
who has contributed to the development of \tscope,
in particular Simon Marlow and Duncan Coutts,
who have answered many of our questions,
and helped us add Mercury specific features to \tscope.

% You can find these papers in ~pbone/papers/
% berthold_eden.pdf
% runciman_Profiling-par-fp.pdf
% I did not download the gransim or gum papers.

% zs: the relevant papers I found are in the files
% vace.ps
% hotpar10.pdf
% par_prolog_survey.pdf
% CRPC-TR93446.pdf
% in my home directory.



\chapter{Revising the RTS}
\label{chap:rts}

\paul{XXX}

\paul{Discuss thread pinning an SMT}

\paul{Discuss changes to work stealing (queues are now owned by engines, not
threads)}

\chapter{Order independent state update}
\label{ref:order_indep_state_update}

\paul{This chapter may be removed}

\chapter{Conclusion}

% Each research chapter contains a discussion of related work.

\bibliographystyle{plainnat}
\bibliography{bib}

\end{document}

