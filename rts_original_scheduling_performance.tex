
\section{Original spark scheduling performance}
\label{sec:rts_original_scheduling_performance}

\begin{figure}
\begin{center}
\subfigure[Right recursive]{%
\label{fig:map_right_recursive}
\begin{tabular}{l}
\code{map(\_, [], []).} \\
\code{map(P, [X $|$ Xs], [Y $|$ Ys]) :-} \\
\code{~~~~P(X, Y) \&} \\
\code{~~~~map(P, Xs, Ys).} \\
\end{tabular}
}
\subfigure[Left recursive]{%
\label{fig:map_left_recursive}
\begin{tabular}{l}
\code{map(\_, [], []).} \\
\code{map(P, [X $|$ Xs], [Y $|$ Ys]) :-} \\
\code{~~~~map(P, Xs, Ys) \&} \\
\code{~~~~P(X, Y).} \\
\end{tabular}
}%
\end{center}
\caption{Right and left recursive map/3}
\label{fig:map_right_and_left_recursive}
\end{figure}

In Section~\ref{sec:rts_gc} we ran our benchmarks with a recent version of the
runtime system.
In the rest of this chapter we describe many of the improvements to the
runtime system that led to the improved parallel performance reported in
that section.

\plan{Introduce right recursion.}
Figure~\ref{fig:map_right_and_left_recursive} shows two alternative, parallel
implementations of \code{map/3}.
While their declarative semantics are identical,
their operational semantics are very different.
In Section~\ref{sec:backgnd_merpar} we explained that parallel conjunctions
are implemented by spawning off the second and later conjuncts and executing
the first conjunct directly.
In the right recursive case (Figure~\ref{fig:map_right_recursive}),
the recursive call is spawned off as a spark,
and in the left recursive case (Figure~\ref{fig:map_left_recursive}),
the recursive call is executed directly, and the loop \emph{body} is
spawned off.
Declarative programmers are taught to prefer tail recursion,
and therefore tend to write right recursive code.

\begin{table}
\begin{center}
\begin{tabular}{r|rr|rrrr}
\multicolumn{1}{c|}{Max no.} &
\multicolumn{2}{c|}{Sequmential} &
\multicolumn{4}{c}{Parallel w/ $N$ Engines} \\
\Cbr{of contexts} & \C{not TS} & \Cbr{TS} & \C{1}& \C{2}& \C{3}& \C{4}\\
\hline
4        & 23.2 (0.93) & 21.5 (1.00)
         & 21.5 (1.00) & 21.5 (1.00) & 21.5 (1.00) & 21.5 (1.00) \\
64   &-&-& 21.5 (1.00) & 21.5 (1.00) & 21.5 (1.00) & 21.5 (1.00) \\
128  &-&-& 21.5 (1.00) & 19.8 (1.09) & 20.9 (1.03) & 21.2 (1.01) \\
256  &-&-& 21.5 (1.00) & 13.2 (1.63) & 15.5 (1.38) & 16.5 (1.30) \\
512  &-&-& 21.5 (1.00) & 11.9 (1.81) &  8.1 (2.66) &  6.1 (3.55) \\
1024 &-&-& 21.5 (1.00) & 11.8 (1.81) &  8.0 (2.67) &  6.1 (3.55) \\
2048 &-&-& 21.5 (1.00) & 11.9 (1.81) &  8.0 (2.67) &  6.0 (3.55) \\
\end{tabular}
\end{center}
\caption{Right recursion performance.}
\label{tab:right}
\end{table}

\plan{Show performance figures.}
Table~\ref{tab:right} shows average elapsed time in seconds for the
mandelbrot\_lowalloc program from 20 test runs.
We use the mandelbrot\_lowalloc program from Section~\ref{sec:rts_gc}.
Using this program we can easily observe the
speedup due to parallelism in Mercury without the effects of the garbage
collector.
The main loop of the renderer uses right recursion,
and is similar to \code{map/3}
(Figure~\ref{fig:map_right_recursive}),
this is the loop that iterates over the image's rows.
The leftmost column shows the maximum number of contexts permitted at
any time.
This is the limit that was mentioned in the previous section.
The next two columns give the elapsed execution time for a sequential
version of the program;
in this version of the program no parallel conjunctions were used.
These two columns give results without and with thread safety.
The following four columns give the elapsed execution times
using one to four Mercury engines.
The numbers in parentheses are the ratio between the time and the
sequential thread safe time.

\plan{Observations}
In general we achieve more parallelism when we use more contexts,
up to a threshold of 601 contexts,
as there are 600 rows in the image and a base case each one consumes a
context.
This is why the program does not benefit greatly from a high limit such as
1024 or 2048 contexts.
This program may use fewer than 600 contexts as any sequentially executed
sparks use their parent context.
This is possibly why a limit of 512 contexts also results in a good parallel
speedup.
When only 256 contexts are used,
the four core version achieves a speedup of 1.30,
compared to 3.55 for 512 or more contexts.
Given that mandelbrot uses independent parallelism there should never be any
need to suspend a context.
Therefore the program should parallelise well enough when restricted to
a small number of contexts (four to eight).
Too many contexts are needed to execute this program at the level of
parallelism that we want.
We noticed the same pattern in other programs with right recursive
parallelism.

\label{context_limit}
\plan{Describe the context limit problem.}
%Right recursion uses a context for each iteration of the loop,
%and then suspends that context.
To understand this problem we must consider how parallel conjunctions are
executed (see Section~\ref{sec:backgnd_merpar}).
The original context creates a spark for the second and later conjuncts and
puts it on the global spark queue.
It then executes the first conjunct \code{P/2}.
This takes much less time to execute than the spawned off call to
\code{map/3}.
After executing \code{P/2} the original context will call the
\joinandcontinue barrier.
It then attempts to execute a spark from its local spark stack,
which will fail because the only spark was placed on the global spark queue.
The original context cannot proceed until after the other context finishes,
but it will be needed then and must then be kept in memory.
Therefore It is suspended until all the other 599 iterations of the loop
have finished.
Meanwhile the spark that was placed on the global run queue is converted
into a new context.
This new context enters the recursive call and
becomes blocked within the recursive instance of the same parallel
conjunction;
it must wait for the remaining 598 iterations of the loop.
This process continues to repeat itself,
allocating more contexts.
Each context consumes a significant amount of memory,
much more than one stack frame.
Therefore
this problem makes programs that look tail recursive
\emph{consume much more memory than}
sequential programs that are not tail recursive.
We implement the limit on the number of contexts to prevent pathological
cases such as this one from crashing the program or the whole computer.
Once the limit is reached sparks cannot be placed on the global run queue
and sequential execution is used.
Therefore the context limit is a trade off between memory usage and
parallelism.

\picfigure{linear_context_usage}{Linear context usage in right recursion}

\plan{Diagram from the loop control talk}
An example of context usage over time using four engines is shown in
Figure~\ref{fig:linear_context_usage}.
Time is shown on the Y axis and contexts are drawn so that each additional
context is further along the X axis.
At the top left,
four contexts are created and they execute four iterations of a loop;
this execution is indicated by boxes.
Once each of these iterations finishes,
its context stays in memory but is suspended,
indicated by the long vertical lines.
Another four iterations of the loop create another four contexts, and so on.
Later, when all iterations of the loop have been executed,
each of the blocked contexts resumes execution and immediately exits,
indicated by the horizontal line at the bottom of each of the vertical
lines.
Later in the project we developed a solution to this problem
(see Chapter~\ref{chap:loop_control}).
Before developing the solution, we developed a work around
(see Section~\ref{sec:rts_reorder}.

\plan{Extra engines require more contexts}
When using either 128 or 256 contexts,
we noticed that when we used three or four Mercury engines
the program ran more slowly than with two Mercury engines.
A possible reason is that:
the more Mercury engines there are, the more often at least one engine is
idle.
When creating a spark, the runtime system checks for an idle engine,
if there is one then the spark may be placed on the global queue, subject to
    the context limit (see Section~\ref{sec:rts_original_scheduling}).
If there is no idle engine,
a spark is placed on the context's local queue regardless of the current
number of contexts.
When there are fewer Mercury engines being used then it is less likely that
one of them is idle.
Therefore more sparks are placed on the local queue and executed
sequentially and the program does not hit the context limit as quickly as
one using more Mercury engines.
The program can exploit more parallelism as it hits the context limit
later,
and achieves better speedups.

\plan{Suggest that left recursion might fix this,
(This is what we thought at the time).}
Only right recursion uses one context for each iteration of a loop. 
A left recursive predicate (Figure~\ref{fig:map_left_recursive}) has its
recursive call on the left of the parallel conjunction.
The context executing the conjunction sparks off the body of the loop
(in the figure this is \code{P}),
and executes the recursive call directly.
By the time the recursive call finishes,
the conjunct containing \code{P} should have also finished.
The context that created the parallel conjunction is less likely to be
blocked at the barrier,
and the context executing the spark is never blocked since it is not the
original context that executed the conjunction.

\begin{table}
\begin{center}
\begin{tabular}{r|rrrrrr}
\multicolumn{1}{c|}{Max no.} &
\multicolumn{2}{c|}{Sequmential} &
\multicolumn{4}{c}{Parallel w/ $N$ Engines} \\
\Cbr{of contexts} & \C{not TS} & \Cbr{TS}  & \C{1}& \C{2}& \C{3}& \C{4}\\
\hline
\hline
\multicolumn{7}{c}{Right recursion} \\
\hline
4        & 23.2 (0.00) & 21.5 (0.04)
         & 21.5 (0.02) & 21.5 (0.00) & 21.5 (0.01) & 21.5 (0.02) \\
64   &-&-& 21.5 (0.03) & 21.5 (0.01) & 21.5 (0.01) & 21.5 (0.02) \\
128  &-&-& 21.5 (0.02) & 19.8 (0.01) & 20.9 (0.03) & 21.2 (0.03) \\
256  &-&-& 21.5 (0.02) & 13.2 (0.05) & 15.5 (0.06) & 16.5 (0.07) \\
512  &-&-& 21.5 (0.02) & 11.9 (0.11) &  8.1 (0.09) &  6.1 (0.08) \\
1024 &-&-& 21.5 (0.03) & 11.8 (0.11) &  8.0 (0.06) &  6.1 (0.08) \\
2048 &-&-& 21.5 (0.03) & 11.9 (0.10) &  8.0 (0.08) &  6.0 (0.06) \\
\hline
\hline
\multicolumn{7}{c}{Left recursion (Sparks included in context limit)} \\
\hline
4        & 23.2 (0.01) & 21.5 (0.03)
         & 21.5 (0.02) & 21.5 (0.04) & 21.5 (0.04) & 21.5 (0.02) \\
64   &-&-& 21.5 (0.02) & 21.5 (0.02) & 21.4 (0.03) & 21.5 (0.03) \\
128  &-&-& 21.5 (0.04) & 21.5 (0.03) & 21.5 (0.03) & 21.5 (0.03) \\
256  &-&-& 21.5 (0.02) & 18.3 (0.75) & 18.2 (0.03) & 19.6 (1.37) \\
512  &-&-& 21.5 (0.02) & 17.9 (0.83) & 15.5 (1.29) & 16.4 (7.09) \\
1024 &-&-& 21.5 (0.03) & 18.0 (0.85) & 14.7 (2.18) & 16.1 (5.23) \\
2048 &-&-& 21.5 (0.02) & 18.0 (0.85) & 15.4 (2.15) & 17.8 (5.25) \\
\hline
\hline
\multicolumn{7}{c}{Left recursion (Sparks excluded from context limit)} \\
\hline
N/A      & 23.3 (0.01) & 21.5 (0.02)
         & 21.7 (0.78) & 17.9 (0.60) & 15.6 (2.49) & 14.2 (6.94) \\
\end{tabular}
\end{center}
\caption{
Right and Left recursion shown with standard deviation}
\label{tab:2009_left_nolimit}
\end{table}

\plan{Show performance figures for left recursion.}
Table~\ref{tab:2009_left_nolimit} shows benchmark results using left
recursion.
The table is broken into three sections:

\begin{enumerate}

\item
a copy of the right recursion data from Table~\ref{tab:right},
which is presented again for comparison;

\item
the left recursion data,
which we will discuss now;

\item
and left recursion data with a modified context limit,
which we will discuss in a moment.

\end{enumerate}

\noindent
The figures in parenthesis are the standard deviation of the samples.

The left recursive figures are underwhelming;
they are worse than right recursion.
Furthermore,
the standard deviations for left recursion results are much
higher.
In particular,
the more contexts and Mercury engines are used,
the higher the standard deviation.
Despite this,
we can see that the left recursion results are much slower than the right
recursive results.

\plan{Explain the premature scheduling problem that affects left-recursive
programs.}
Both left and right recursion are affected by the context limit,
however the cause for left recursion is different.
Consider a case of two Mercury engines and a context limit of eight.
The first Mercury engine is executing the original context,
which enters the left recursive parallel conjunction and spawns off a spark,
adding it to the global spark queue.
The original context then executes the recursive call and
continues to spawn off sparks.
As we described in Section~\ref{sec:rts_original_scheduling},
each spark on the global spark queue may be converted into a new
context,
and therefore sparks on the global queue contribute towards the context
limit.
With left recursion,
the first engine executes its tight loop which spawns off sparks.
Meanwhile the second engine takes sparks from the queue and executes them;
the execution of the work that a spark represents takes more time than each
recursive call executed by the first engine.
Therefore the first engine will put sparks on the global queue more quickly
than the second engine can remove and execute them.
After eight to twelve recursions,
it is very likely that the sparks on the global queue will exceed the
context limit,
and that new sparks are now placed on the original context's local spark
stack.
This happens sooner if the context limit is low.

\plan{Second problem, discuss the high variance}
The high variance in all the left recursive results is
indicative of nondeterminism.
The cause is going to be something that either varies between execution
runs or does not always effect execution runs.
A potential explanation is that in different runs, different numbers of
sparks are executed in parallel before the context limit is reached.
However this cannot be true, or at least is only partly true,
as the variance is higher when the context limit is higher,
including cases
where 2048 contexts are permitted and we know the program needs at most 601
(usually less).
Fortunately there is a better explanation.
The context limit is one of the two things used to make this scheduling
decision;
the other is: if there is no idle engine then the spark is always
placed on the context local spark stack
(Section~\ref{sec:rts_original_scheduling}).
At the time when the parallel conjunction in \code{map/3} is executed,
the other engines will initially be idle but will quickly become busy,
and once they are busy the original context will not place sparks on the
global spark queue.
If the other engines are slow to respond to the first sparks placed
on the global queue,
then more sparks are placed on this queue and more
parallel work is available.
If they are quick to respond,
then more sparks will be placed on the original contexts local queue,
where they will be executed sequentially.

\plan{Left recursion w/out limit results}
Mandelbrot uses only independent parallelism, which means no context can
ever be blocked by a future.
In a left recursive case,
a computation that is spawned off on a spark is
never blocked by a \joinandcontinue barrier;
only the original context executes a \joinandcontinue that may block.
Therefore the program never uses more than one context per Mercury engine
plus the original context.
For this program we can safely modify the runtime system so that globally
scheduled sparks do not count towards the context limit.
We did test 
that a low context limit prevents the majority of sparks from being placed on
the global queue and being eligible for parallel execution.
The final row group in
Table~\ref{tab:2009_left_nolimit}
shows results for the left recursive test
where sparks in the global queue do not count towards the context limit.
As expected, we got similar results for different values of the context
limit.
We have therefore shown only one row of results in the table.
This result is similar to those in the left recursive group with an
unmodified runtime system and a sufficiently high context limit.
Therefore we can say that the context limit is affecting left recursion in
this way.

\plan{Reinforce that these results support the idea that scheduling
decisions are made prematurely}
Both of the left recursive programs have the same cause:
the decision to execute a spark sequentially or in parallel is
made too early.
This decision is made when the spark is scheduled,
and placed either on the global spark queue or the context local spark
stack.
The data used to make the decision includes the number of contexts in
use,
the number of sparks on the global queue,
and if there is an idle engine.
These conditions will be different when the spark is scheduled compared
to when it may be executed,
and the conditions when it is scheduled are not a good indication of
the conditions when it may be executed.
We will refer to this problem as the \emph{premature spark scheduling problem}.
In the next section,
we solve this problem by delaying the decision whether to execute a spark in
parallel or in sequence.

%In the left recursive program scheduling is quite different.
%The parallel conjunction creates a spark for \code{P(X, Y)} and executes the
%recursive call directly.
%The spark is converted into a context,
%that context does not execute another parallel conjunction since it does not
%execute the recursive call.
%Therefore, it will not become blocked on the \joinandcontinue barrier in any
%nested parallel conjunction.
%It will execute the barrier after \code{P(X, Y)},
%this however does not block this context.
%The context is not the conjunction's original context and therefore once it
%reaches this barrier it is free,
%if it has any sparks on its local queue it may execute them,
%otherwise the engine executing it will look for global work,
%either another context or a spark from the global queue.
%If there is a spark on the global queue the engine will use this context to
%execute it since the context is otherwise unused.
%
%This led us to believe that the left recursion would be more efficient than
%right recursion,
%namely that since contexts are reused, the number of contexts wouldn't climb
%and prevent parallelism from being exploited.
%As Table~\ref{tab:right_v_left} shows, we were wrong:
%the context limit is affecting performance.
%As discussed, a left-recursive loop spawns of calls to \code{P} as sparks
%and executes its recursive call directly.
%It will, very quickly,
%make many recursive calls, spawn off many sparks.
%The context limit includes sparks on the global queue since
%executing them can require the creation of new contexts,
%Furthermore, if they were not included and the runtime system refused to
%convert a spark on the global queue into a context the system could become
%deadlocked.
%In the left recursive case,
%the context limit will be reached very quickly,
%often before engines have begun taking sparks from the queue and executing
%them.
%Once the limit is reached sparks are placed on the contexts local queues
%where they cannot be executed in parallel.
%The smaller the context limit,
%the more quickly the limit is reached and the fewer contexts are placed on
%the global queue.
%Additionally,
%the loop placing sparks on its context's local stack will execute very
%quickly.
%
%We concluded that
%in the left recursive case
%the scheduling decision for each spark is made much earlier than the spark's
%execution.
%Specifically,
%when the decision to place the spark on the global queue or local stack is
%made,
%often the context limit has already been reached:
%\paul{Need to decide how I communicate who the actor is for scheduling
%decisions.}
%the context will place the spark on its local stack.
%Later, when a different engine becomes idle,
%it cannot access the spark since it is on another engine's context's spark
%stack.
%At this point it is apparent that the scheduling decision made when the
%spark was placed on the local stack was incorrect,
%as there is an idle engine ready to execute the spark,
%and because contexts are re-used (in left recursion) there is either a free
%context or we can easily create one.

