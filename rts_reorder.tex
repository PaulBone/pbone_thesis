
\status{Reviewed}

\plan{Acknowledge results of previous section}
We have shown that right recursion suffers from a context limit problem.
Right recursive programs also get smaller speedups when compared with left
recursive problems.
The problem with right recursion is the large number of contexts needed,
this is memory intensive and becomes CPU intensive once we consider that
the Boehm GC must scan these context's stacks to check for pointers.
This can happen whenever a the second or later
conjunct of a parallel conjunction
takes longer to execute than the first conjunct;
the first conjunct becomes blocked on the later conjuncts.

\plan{Describe the intuition for the transformation.}
Meanwhile left recursion works well,
using work stealing we are able to get very good speedups (3.96 for
mandelbrot when using four Mercury engines).
When the conjunction is independent,
meaning there are no dependences between the conjuncts,
Mercury's purity allows us to re-order the conjuncts of the computation and
transform right recursion into left recursion.
Essentially transforming the procedure in Figure
\ref{fig:map_right_recursive} into the one in Figure
\ref{fig:map_left_recursive} (page \pageref{fig:map_right_recursive}.

\begin{algorithm}
\[
\begin{array}{l @{}l}
    \reorder SCC\,\epsilon      &= \epsilon \\
    \reorder SCC\,(conj:conjs)  &=
    \begin{cases}
        (conj:conjs') &
            \text{if } conj \text{ has a call to } SCC \\
        \trypushconjlater conj,conjs' &
            \text{otherwise} \\
    \end{cases} \\
                                &
    \begin{array}{l l @{}l}
        \text{where} & conjs'  &= \reorder SCC\,conjs \\
    \end{array} \\
\end{array}
\]
\[
\begin{array}{l @{}l}
    \trypushconjlater conj\,\epsilon    &= \epsilon \\
    \trypushconjlater conj\,(pivot:rest)&=
        \begin{cases}
            (conj:pivot:rest)   & \text{if } \neg \canswap conj\,pivot \\
            (pivot:pushed)      & \text{otherwise}
        \end{cases} \\
                                        &
    \begin{array}{l l @{}l}
        \text{where} & pushed &= \trypushconjlater conj\,rest \\
    \end{array} \\
\end{array}
\]
\caption{Reorder independent conjunctions}
\label{alg:reorder_conjunction}
\end{algorithm}

\plan{Describe the transformation}
At the moment we only attempt to re-order completely independent
conjunctions.
It may be possible to find independent sections of dependent conjunctions
and reorder them but we have not tried.
This transformation can occur as an alternative to the dependent conjunction
transformation described by \citep{wang:2011:dep-par}.
If the conjunction has no shared variables then a dependency analysis of the
current module is performed.
From this we gather the identify of the SCC that the current procedure
belongs to,
and the SCCs of any calls in each of the conjuncts.
The algorithm for \code{reorder} is shown in Algorithm
\ref{alg:reorder_conjunction}.
This code iterates over a conjunction, for each conjunct it tests if that
conjunct is recursive.
To reorder a conjunction so that the recursive conjuncts occur to the left,
the algorithm attempts to push non-recursive conjuncts to the right.
Each non-recursive conjunct found is swapped with the one to its until if
doing so is legal.
After each successful swap the algorithm attempts to swap the conjunct with
the one to the right of its new position.
There are a few cases where swaps are illegal, the only relevant example is
when one of the two conjuncts being swapped is impure.

\plan{Discuss reasoning for not showing results}.
We have benchmarked and tested this,
however the benchmarks are not interesting:
independent right recursion is transformed into independent left
recursion and therefore performs the same as independent left recursion.

