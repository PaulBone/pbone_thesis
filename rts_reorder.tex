
\section{Independent Conjunction Reordering}
\label{sec:rts_reorder}

\plan{Acknowledge results of previous section}
We have shown that right recursion suffers from a context limit problem.
Right recursive programs also get smaller speedups when compared with left
recursive programs.
The problem with right recursion is the large number of contexts needed.
This is memory intensive, and CPU intensive once we consider that
the Boehm GC must scan these context's stacks to check for pointers.
This can happen whenever the second or later
conjunct of a parallel conjunction
takes longer to execute than the first conjunct;
the first conjunct becomes blocked on the later conjunct(s).

\plan{Describe the intuition for the transformation.}
On the other hand left recursion works well,
using work stealing we are able to get very good speedups (3.96 for
mandelbrot when using four Mercury engines).
When the conjunction is independent,
Mercury's purity allows us to reorder the conjuncts of the computation and
transform right recursion into left recursion.
For example it would transform the procedure in figure
\ref{fig:map_right_recursive} into the one in figure
\ref{fig:map_left_recursive} (page \pageref{fig:map_right_recursive}).

\begin{algorithm}
\begin{algorithmic}[1]
\Procedure{reorder}{$SCC$, $Conjs$}
    \If{$Conjs = $ \nil}
        \State \Return \nil
    \Else
        \State \cons{Conj}{ConjsTail} $\gets Conjs$
        \State $ConjsTail \gets$ reorder($SCC$, $ConjsTail$)
        \If{$Conj$ contains a call to $SCC$}
            \State \Return \cons{Conj}{ConjsTail}
        \Else
            \State \Return try\_push\_conj\_later($Conj$, $ConjsTail$)
        \EndIf
    \EndIf
\EndProcedure
\Procedure{try\_push\_conj\_later}{$Goal$, $Conjs$}
    \If{$Conjs =$ \nil}
        \State \Return \single{Goal}
    \Else
        \State \cons{Pivot}{Rest} $\gets Conjs$
        \If{can\_swap($Goal$, $Pivot$)}
            \State $Pushed \gets$ try\_push\_conj\_later($Goal$, $Rest$)
            \State \Return \cons{Pivot}{Pushed}
        \Else
            \State \Return \cons{Goal}{Conjs}
        \EndIf
    \EndIf
\EndProcedure
\end{algorithmic}
\caption{Reorder independent conjunctions}
\label{alg:reorder_conjunction}
\end{algorithm}

\plan{Describe the transformation}
We only attempted to re-order completely independent conjunctions.
It may be possible to find independent sections of dependent conjunctions
and reorder them,
but we have not needed to.
\reorder is invoked on parallel conjunctions without shared variables,
its code is shown in algorithm \ref{alg:reorder_conjunction}.
Its arguments are a reference to the current SCC and a list of the parallel
conjunction's conjuncts.
\reorder iterates over these goals in reverse order by means of the
recursive call on line 6.
For each goal it tests if that goal makes a call to the current SCC,
if so it leaves the goal where it is,
if not \reorder will attempt to push the goal towards the end of the
list using \trypushconjlater.
\trypushconjlater is also very simple,
It takes the current goal and the list representing the \emph{rest} of the
conjunction,
It iterates down this list attempting to swap the goal being pushed
(\code{Goal}) with the goal at the front of the list (\code{Pivot}).
Not all swaps are legal,
for example a goal cannot be swapped with an impure goal,
therefore \trypushconjlater may not always push a goal all the way to the
end of the list.

\plan{Discuss reasoning for not showing results}.
We have benchmarked and tested this,
however the benchmark results are not interesting:
independent right recursion is transformed into independent left
recursion and therefore performs the same as independent left recursion.

